% see http://info.semprag.org/basics for a full description of this template
\documentclass[cm]{glossa}

% possible options:
% [times] for Times font (default if no option is chosen)
% [cm] for Computer Modern font
% [lucida] for Lucida font (not freely available)
% [brill] open type font, freely downloadable for non-commercial use from http://www.brill.com/about/brill-fonts; requires xetex
% [charis] for CharisSIL font, freely downloadable from http://software.sil.org/charis/
% for the Brill an CharisSIL fonts, you have to use the XeLatex typesetting engine (not pdfLatex)
% for headings, tables, captions, etc., Fira Sans is used: https://www.fontsquirrel.com/fonts/fira-sans
% [biblatex] for using biblatex (the default is natbib, do not load the natbib package in this file, it is loaded automatically via the document class glossa.cls)
% [linguex] loads the linguex example package
% !! a note on the use of linguex: in glossed examples, the third line of the example (the translation) needs to be prefixed with \glt. This is to allow a first line with the name of the language and the source of the example. See example (2) in the text for an illustration.
% !! a note on the use of bibtex: for PhD dissertations to typeset correctly in the references list, the Address field needs to contain the city (for US cities in the format "Santa Cruz, CA")

%\addbibresource{sample.bib}
% the above line is for use with biblatex
% replace this by the name of your bib-file (extension .bib is required)
% comment out if you use natbib/bibtex

\let\B\relax %to resolve a conflict in the definition of these commands between xyling and xunicode (the latter called by fontspec, called by charis)
\let\T\relax
\usepackage{xyling} %for trees; the use of xyling with the CharisSIL font produces poor results in the branches. This problem does not arise with the packages qtree or forest.
\usepackage[linguistics]{forest} %for nice trees!
\usepackage{longtable}

\providecommand{\tightlist}{%
  \setlength{\itemsep}{0pt}\setlength{\parskip}{0pt}}

\title[ReplicationLing]{Replication studies in experimental linguistic
journals}
% Optional short title inside square brackets, for the running headers.

% \author[Paul \& Vanden Wyngaerd]% short form of the author names for the running header. If no short author is given, no authors print in the headers.
% {%as many authors as you like, each separated by \AND.
%   \spauthor{Waltraud Paul\\
%   \institute{CNRS, CRLAO}\\
%   \small{105, Bd. Raspail, 75005 Paris\\
%   waltraud.paul@ehess.fr}
%   }
%   \AND
%   \spauthor{Guido Vanden Wyngaerd \\
%   \institute{KU Leuven}\\
%   \small{Warmoesberg 26, 1000 Brussel\\
%   guido.vandenwyngaerd@kuleuven.be}
%   }%
% }

\author[Kobrock \& Roettger]{
    \spauthor{Kristina Kobrock\\
  \institute{University of Osnabrück}\\
  \small{\href{mailto:kkobrock@uni-osnabrueck.de}{\nolinkurl{kkobrock@uni-osnabrueck.de}}}
  }%
  \AND  \spauthor{Timo B. Roettger\\
  \institute{University of Oslo}\\
  \small{}
  }%
  }

\usepackage{natbib}

% Pandoc syntax highlighting
\usepackage{color}
\usepackage{fancyvrb}
\newcommand{\VerbBar}{|}
\newcommand{\VERB}{\Verb[commandchars=\\\{\}]}
\DefineVerbatimEnvironment{Highlighting}{Verbatim}{commandchars=\\\{\}}
% Add ',fontsize=\small' for more characters per line
\usepackage{framed}
\definecolor{shadecolor}{RGB}{248,248,248}
\newenvironment{Shaded}{\begin{snugshade}}{\end{snugshade}}
\newcommand{\AlertTok}[1]{\textcolor[rgb]{0.94,0.16,0.16}{#1}}
\newcommand{\AnnotationTok}[1]{\textcolor[rgb]{0.56,0.35,0.01}{\textbf{\textit{#1}}}}
\newcommand{\AttributeTok}[1]{\textcolor[rgb]{0.77,0.63,0.00}{#1}}
\newcommand{\BaseNTok}[1]{\textcolor[rgb]{0.00,0.00,0.81}{#1}}
\newcommand{\BuiltInTok}[1]{#1}
\newcommand{\CharTok}[1]{\textcolor[rgb]{0.31,0.60,0.02}{#1}}
\newcommand{\CommentTok}[1]{\textcolor[rgb]{0.56,0.35,0.01}{\textit{#1}}}
\newcommand{\CommentVarTok}[1]{\textcolor[rgb]{0.56,0.35,0.01}{\textbf{\textit{#1}}}}
\newcommand{\ConstantTok}[1]{\textcolor[rgb]{0.00,0.00,0.00}{#1}}
\newcommand{\ControlFlowTok}[1]{\textcolor[rgb]{0.13,0.29,0.53}{\textbf{#1}}}
\newcommand{\DataTypeTok}[1]{\textcolor[rgb]{0.13,0.29,0.53}{#1}}
\newcommand{\DecValTok}[1]{\textcolor[rgb]{0.00,0.00,0.81}{#1}}
\newcommand{\DocumentationTok}[1]{\textcolor[rgb]{0.56,0.35,0.01}{\textbf{\textit{#1}}}}
\newcommand{\ErrorTok}[1]{\textcolor[rgb]{0.64,0.00,0.00}{\textbf{#1}}}
\newcommand{\ExtensionTok}[1]{#1}
\newcommand{\FloatTok}[1]{\textcolor[rgb]{0.00,0.00,0.81}{#1}}
\newcommand{\FunctionTok}[1]{\textcolor[rgb]{0.00,0.00,0.00}{#1}}
\newcommand{\ImportTok}[1]{#1}
\newcommand{\InformationTok}[1]{\textcolor[rgb]{0.56,0.35,0.01}{\textbf{\textit{#1}}}}
\newcommand{\KeywordTok}[1]{\textcolor[rgb]{0.13,0.29,0.53}{\textbf{#1}}}
\newcommand{\NormalTok}[1]{#1}
\newcommand{\OperatorTok}[1]{\textcolor[rgb]{0.81,0.36,0.00}{\textbf{#1}}}
\newcommand{\OtherTok}[1]{\textcolor[rgb]{0.56,0.35,0.01}{#1}}
\newcommand{\PreprocessorTok}[1]{\textcolor[rgb]{0.56,0.35,0.01}{\textit{#1}}}
\newcommand{\RegionMarkerTok}[1]{#1}
\newcommand{\SpecialCharTok}[1]{\textcolor[rgb]{0.00,0.00,0.00}{#1}}
\newcommand{\SpecialStringTok}[1]{\textcolor[rgb]{0.31,0.60,0.02}{#1}}
\newcommand{\StringTok}[1]{\textcolor[rgb]{0.31,0.60,0.02}{#1}}
\newcommand{\VariableTok}[1]{\textcolor[rgb]{0.00,0.00,0.00}{#1}}
\newcommand{\VerbatimStringTok}[1]{\textcolor[rgb]{0.31,0.60,0.02}{#1}}
\newcommand{\WarningTok}[1]{\textcolor[rgb]{0.56,0.35,0.01}{\textbf{\textit{#1}}}}

% Note that pandoc citeproc is not supported by the template
% and we added the following code here just in case it will be supported in
% the future




\begin{document}


\sffamily
\maketitle


\begin{keywords}
  replication, experimental linguistics, \ldots{}
\end{keywords}

\rmfamily

%  Body of the article
\hypertarget{introduction}{%
\section{Introduction}\label{introduction}}

\hypertarget{overview-analysis-rate-of-replication-mention}{%
\section{Overview analysis: Rate of replication
mention}\label{overview-analysis-rate-of-replication-mention}}

\hypertarget{research-questions}{%
\subsection{Research questions}\label{research-questions}}

\hypertarget{sample}{%
\subsection{Sample}\label{sample}}

\hypertarget{procedure}{%
\subsection{Procedure}\label{procedure}}

\hypertarget{data-analysis}{%
\subsection{Data Analysis}\label{data-analysis}}

\hypertarget{results}{%
\subsection{Results}\label{results}}

\hypertarget{detailed-analysis-types-and-contributing-factors}{%
\section{Detailed analysis: Types and contributing
factors}\label{detailed-analysis-types-and-contributing-factors}}

\hypertarget{research-questions-1}{%
\subsection{Research questions}\label{research-questions-1}}

\hypertarget{sample-1}{%
\subsection{Sample}\label{sample-1}}

\hypertarget{procedure-1}{%
\subsection{Procedure}\label{procedure-1}}

\hypertarget{data-analysis-1}{%
\subsection{Data Analysis}\label{data-analysis-1}}

\begin{Shaded}
\begin{Highlighting}[]
\CommentTok{\# loading the data and having a glimpse}
\NormalTok{coded\_articles }\OtherTok{=} \FunctionTok{read.csv}\NormalTok{(}\StringTok{"../data/Coding\_Articles.csv"}\NormalTok{, }\AttributeTok{sep=}\StringTok{"|"}\NormalTok{)}

\FunctionTok{head}\NormalTok{(coded\_articles)}
\end{Highlighting}
\end{Shaded}

\begin{verbatim}
##                                                          author
## 1              Baus, Cristina; Costa, Albert; Carreiras, Manuel
## 2                BRADLEY, DC; SANCHEZCASAS, RM; GARCIAALBEA, JE
## 3 Branigan, Holly P.; Catchpole, Ciara M.; Pickering, Martin J.
## 4                            Braun, Bettina; Tagliapietra, Lara
## 5             Brouwer, Susanne; Mitterer, Holger; Huettig, Falk
## 6                                 CARREIRAS, M; GERNSBACHER, MA
##                                                                                        title
## 1 Neighbourhood density and frequency effects in speech production: A case for interactivity
## 2                        THE STATUS OF THE SYLLABLE IN THE PERCEPTION OF SPANISH AND ENGLISH
## 3                                                   What makes dialogues easy to understand?
## 4                                       On-line interpretation of intonational meaning in L2
## 5        Speech reductions change the dynamics of competition during spoken word recognition
## 6                                               COMPREHENDING CONCEPTUAL ANAPHORS IN SPANISH
##                            journal
## 1 LANGUAGE AND COGNITIVE PROCESSES
## 2 LANGUAGE AND COGNITIVE PROCESSES
## 3 LANGUAGE AND COGNITIVE PROCESSES
## 4 LANGUAGE AND COGNITIVE PROCESSES
## 5 LANGUAGE AND COGNITIVE PROCESSES
## 6 LANGUAGE AND COGNITIVE PROCESSES
##                                                                                                                                                                                                                                                                                                                                                                                                                                                                                                                                                                                                                                                                                                                                                                                                                                                                                                                                                                                                                                                                                                                                                                                                                                                                                                                                                                                                                                                                                                                                                                                                                                                                                    abstract
## 1                                                                                                                                                                                                                                                                                                                                                                                                                               In three experiments, we explore the effects of phonological properties such as neighbourhood density and frequency on speech production in Spanish. Specifically, we assess the reliability of the recent observation made by Vitevitch and Stamer (2006), according to which the neighbourhood effect in Spanish has a reverse polarity to that observed in other languages. In Experiment 1, we replicate Vitevitch and Stamer's (2006) experiment, this time adding a control group. The same inhibitory neighbourhood effect found for both groups can not corroborate the hypothesis posited by Vitevitch and Stamer. In Experiment 2, our results show that native speakers of Spanish named pictures with words belonging to high density neighbourhoods faster than those belonging to low density neighbourhoods. In Experiment 3, we test for effects of neighbourhood frequency during lexical selection. Again, we find a facilitatory effect for words with a high-frequency neighbourhood. Together, the results of the present experiments suggest that lexical selection is facilitated by the number of neighbours and by neighbourhoods with higher frequency. These findings are consistent with the predictions of interactive models.
## 2 A series of monitoring studies is reported, in replication of the cross-language research of Cutler, Mehler, Norris and Segui (1983; 1986), which found evidence of language-specific perceptual routines. Monolingual speakers of Spanish and English detected CV and CVC target sequences in native and non-native materials. The replication succeeded only in the case of Spanish speakers and Spanish materials, where a cross-over interaction of target (CV vs CVC sequences) and carrier types (CV- vs CVC-syllabified words) gave evidence of a sensitivity to the input's syllabification; no such pattern emerged for Spanish speakers and English materials, nor for English speakers and materials in either language. For English speakers, the consistent finding was for faster performance with CVC targets, regardless of the structure of the carrier word. Whether or not this is to be interpreted as evidence of syllabified input representations is not clear. Analyses of English syllabification that are alternatives to that adopted by Cutler et al. exist, to weaken the original contrast drawn between syllable-favouring and syllable-disfavouring languages. A final experiment examines monitoring performance in Spanish speakers who have become bilingual as a consequence of emigration to an English-speaking country; these subjects showed no syllable sensitivity for Spanish language materials. We speculate that factors outside the perceptual system may determine the basis on which responses are made in the monitoring task, and therefore conclude that the case for language specificity in perceptual routines has yet to be made.
## 3                                                                                                                                                                                                                                                                                                                                                                                                                                        Two experiments investigate the question of why dialogues tend to be easier for anyone to understand than monologues. One possibility is that overhearers of dialogue have access to the different perspectives provided by the interlocutors, whereas overhearers of monologue have access to the speaker's perspective alone (Fox Tree, 1999). Directors first described a set of geometric shapes to matchers in monologue or dialogue eight times. Experiment 1 found that descriptions taken from dialogue were easier to understand than descriptions taken from monologue or descriptions taken from dialogue in which the matcher's contributions were excised. This advantage occurred on early trials (when the matcher made a considerable contribution) but also on late trials (when the matcher simply accepted a description). Experiment 2 replicated this finding and ruled out an explanation in which the advantage of dialogue is due to its use of discourse markers. We argue that the ease of dialogue occurs because interlocutors negotiate a perspective that they can agree on (Clark, 1996). This grounded perspective is likely to be objectively easier to understand than a perspective that has not been grounded.
## 4                                                                                                                                                                                                                                                                                                                                                                                                                                                                                                   Despite their relatedness, Dutch and German differ in the interpretation of a particular intonation contour, the hat pattern. In the literature, this contour has been described as neutral for Dutch, and as contrastive for German. A recent study supports the idea that Dutch listeners interpret this contour neutrally, compared to the contrastive interpretation of a lexically identical utterance realised with a double peak pattern. In particular, this study showed shorter lexical decision latencies to visual targets (e.g., PELIKAAN, opelicano) following a contrastively related prime (e.g., flamingo, oflamingoo) only when the primes were embedded in sentences with a contrastive double peak contour, not in sentences with a neutral hat pattern. The present study replicates Experiment 1a of Braun and Tagliapietra (2009) with German learners of Dutch. Highly proficient learners of Dutch differed from Dutch natives in that they showed reliable priming effects for both intonation contours. Thus, the interpretation of intonational meaning in L2 appears to be fast, automatic, and driven by the associations learned in the native language.
## 5                                                                                                                                                                                                                                                                                                                                                                                               Three eye-tracking experiments investigated how phonological reductions (e. g., puter for computer) modulate phonological competition. Participants listened to sentences extracted from a spontaneous speech corpus and saw four printed words: a target (e. g., computer), a competitor similar to the canonical form (e. g., companion), one similar to the reduced form (e. g., pupil), and an unrelated distractor. In Experiment 1, we presented canonical and reduced forms in a syllabic and in a sentence context. Listeners directed their attention to a similar degree to both competitors independent of the target's spoken form. In Experiment 2, we excluded reduced forms and presented canonical forms only. In such a listening situation, participants showed a clear preference for the canonical form competitor. In Experiment 3, we presented canonical forms intermixed with reduced forms in a sentence context and replicated the competition pattern of Experiment 1. These data suggest that listeners penalize acoustic mismatches less strongly when listening to reduced speech than when listening to fully articulated speech. We conclude that flexibility to adjust to speech-intrinsic factors is a key feature of the spoken word recognition system.
## 6                                                                                                                                                                                                            This paper examines the mechanisms involved in the assignment of an antecedent to an anaphoric element. In general, pronouns must match their antecedents at least with respect to number and gender. Sensitivity to such constraints has been shown in several experiments. But Gernsbacher (1991) has also shown that people have no difficulty comprehending a plural pronoun with an antecedent that is grammatically singular but conceptually plural. In the first three experiments, we tested whether such a conceptual effect was preserved with zero anaphors in Spanish. (The typical omission of pronouns in subject position in Spanish.) Verbs in a second clause were marked with plural or singular endings. Plural verbs were rated more natural than singular verbs when they followed three types of singular but conceptually plural antecedents (Experiment 1). Clauses containing plural verbs were read faster when they followed one type of singular but conceptually plural antecedents, i.e. collective sets (Experiments 2 and 3). In fact, clauses containing plural verbs were read equally fast when they followed literally singular collective sets or explicitly group nouns. Using pronominal anaphors, these reading time effects were replicated and extended to sentences that contained generic types as antecedents (Experiment 4). The results are discussed in terms of the use of information during the comprehension of anaphors.
##   cit_times pub_date pub_year vol   issue start_page end_page article_number
## 1        29              2008  23       6        866      888               
## 2        73      MAY     1993   8       2        197      233               
## 3        15              2011  26      10       1667     1686               
## 4         7              2011  26       2        224      235  PII 923379546
## 5        37              2012  27       4        539      571               
## 6        21  AUG-NOV     1992   7 03. Apr        281      299               
##                            doi experimental replication
## 1    10.1080/01690960801962372            1           1
## 2    10.1080/01690969308406954            1           1
## 3 10.1080/01690965.2010.524765            1           1
## 4 10.1080/01690965.2010.486209            1           1
## 5 10.1080/01690965.2011.555268            1           0
## 6    10.1080/01690969208409388            1           0
##                                                          comments oa_article
## 1                     one of several experiments is a replication          0
## 2                     one of several experiments is a replication          0
## 3                                         inner-paper replication          0
## 4          replication of previous experiment by the same authors          0
## 5 no clear communication of intent to replicate an original study          0
## 6 no clear communication of intent to replicate an original study          1
##                                                                                                                                                                                                                cit_init_study
## 1                                                                      Vitevitch, M. S., & Stamer, M. K. (2006). The curious case of competition in Spanish speech production. Language and Cognitive Processes, 21, 760 770.
## 2                                                                                                     Cutler, A., Mehler, J., Norris, D. & Segui, J. (1983). A language-spodfic comprehension strategy. Nature, 304, 159-160.
## 3                                                                                                                                                                                                                        same
## 4 Braun, B., & Tagliapietra, L. (2009). The role of contrastive intonation contours in the retrieval of contextual alternatives. Language and Cognitive Processes. Advance online publication.\ndoi:10.1080/01690960903036836
## 5                                                                                                                                                                                                                            
## 6                                                                                                                                                                                                                            
##                 journal_init_study auth_overlap year_init_study years_between
## 1 Language and Cognitive Processes            0            2006             2
## 2                           Nature            0            1983            10
## 3                             same            1            2011             0
## 4 Language and Cognitive Processes            1            2009             2
## 5                                            NA              NA            NA
## 6                                            NA              NA            NA
##   exp_paradigm sample materials_setup measurement manipulation control
## 1            0      0               0           0            0       1
## 2            0      1               1           0            0       0
## 3            0      0               0           0            1       0
## 4            0      1               0           0            0       0
## 5           NA     NA              NA          NA           NA      NA
## 6           NA     NA              NA          NA           NA      NA
##   type_replication
## 1          partial
## 2       conceptual
## 3          partial
## 4          partial
## 5             <NA>
## 6             <NA>
##                                                                        language
## 1                                                            Spanish (& German)
## 2 Spanish (Castillian) & Australian English (initial: French & British English)
## 3                                                                       English
## 4                                                                         Dutch
## 5                                                                              
## 6                                                                              
##   success rep_citation init_citation init_cit_til_rep
## 1      NA           29            61                3
## 2      NA           73            92               26
## 3      NA           15            NA               NA
## 4      NA            7            43                2
## 5      NA           37            NA               NA
## 6      NA           21            NA               NA
\end{verbatim}

\hypertarget{results-1}{%
\subsection{Results}\label{results-1}}

\hypertarget{discussion}{%
\section{Discussion}\label{discussion}}

\hypertarget{conclusion}{%
\section{Conclusion}\label{conclusion}}

\hypertarget{sec:refs}{%
\section*{References}\label{sec:refs}}
\addcontentsline{toc}{section}{References}

\bibliography{references.bib}


\end{document}
