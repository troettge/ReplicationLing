% Options for packages loaded elsewhere
\PassOptionsToPackage{unicode}{hyperref}
\PassOptionsToPackage{hyphens}{url}
%
\documentclass[
  english,
  man]{apa6}
\usepackage{amsmath,amssymb}
\usepackage{lmodern}
\usepackage{iftex}
\ifPDFTeX
  \usepackage[T1]{fontenc}
  \usepackage[utf8]{inputenc}
  \usepackage{textcomp} % provide euro and other symbols
\else % if luatex or xetex
  \usepackage{unicode-math}
  \defaultfontfeatures{Scale=MatchLowercase}
  \defaultfontfeatures[\rmfamily]{Ligatures=TeX,Scale=1}
\fi
% Use upquote if available, for straight quotes in verbatim environments
\IfFileExists{upquote.sty}{\usepackage{upquote}}{}
\IfFileExists{microtype.sty}{% use microtype if available
  \usepackage[]{microtype}
  \UseMicrotypeSet[protrusion]{basicmath} % disable protrusion for tt fonts
}{}
\makeatletter
\@ifundefined{KOMAClassName}{% if non-KOMA class
  \IfFileExists{parskip.sty}{%
    \usepackage{parskip}
  }{% else
    \setlength{\parindent}{0pt}
    \setlength{\parskip}{6pt plus 2pt minus 1pt}}
}{% if KOMA class
  \KOMAoptions{parskip=half}}
\makeatother
\usepackage{xcolor}
\IfFileExists{xurl.sty}{\usepackage{xurl}}{} % add URL line breaks if available
\IfFileExists{bookmark.sty}{\usepackage{bookmark}}{\usepackage{hyperref}}
\hypersetup{
  pdftitle={Assessing replication rates in experimental linguistics},
  pdfauthor={Kristina Kobrock1 \& Timo B. Roettger2},
  pdflang={en-EN},
  pdfkeywords={replication rate, experimental linguistics, \ldots{}},
  hidelinks,
  pdfcreator={LaTeX via pandoc}}
\urlstyle{same} % disable monospaced font for URLs
\usepackage{graphicx}
\makeatletter
\def\maxwidth{\ifdim\Gin@nat@width>\linewidth\linewidth\else\Gin@nat@width\fi}
\def\maxheight{\ifdim\Gin@nat@height>\textheight\textheight\else\Gin@nat@height\fi}
\makeatother
% Scale images if necessary, so that they will not overflow the page
% margins by default, and it is still possible to overwrite the defaults
% using explicit options in \includegraphics[width, height, ...]{}
\setkeys{Gin}{width=\maxwidth,height=\maxheight,keepaspectratio}
% Set default figure placement to htbp
\makeatletter
\def\fps@figure{htbp}
\makeatother
\setlength{\emergencystretch}{3em} % prevent overfull lines
\providecommand{\tightlist}{%
  \setlength{\itemsep}{0pt}\setlength{\parskip}{0pt}}
\setcounter{secnumdepth}{-\maxdimen} % remove section numbering
% Make \paragraph and \subparagraph free-standing
\ifx\paragraph\undefined\else
  \let\oldparagraph\paragraph
  \renewcommand{\paragraph}[1]{\oldparagraph{#1}\mbox{}}
\fi
\ifx\subparagraph\undefined\else
  \let\oldsubparagraph\subparagraph
  \renewcommand{\subparagraph}[1]{\oldsubparagraph{#1}\mbox{}}
\fi
% Manuscript styling
\usepackage{upgreek}
\captionsetup{font=singlespacing,justification=justified}

% Table formatting
\usepackage{longtable}
\usepackage{lscape}
% \usepackage[counterclockwise]{rotating}   % Landscape page setup for large tables
\usepackage{multirow}		% Table styling
\usepackage{tabularx}		% Control Column width
\usepackage[flushleft]{threeparttable}	% Allows for three part tables with a specified notes section
\usepackage{threeparttablex}            % Lets threeparttable work with longtable

% Create new environments so endfloat can handle them
% \newenvironment{ltable}
%   {\begin{landscape}\begin{center}\begin{threeparttable}}
%   {\end{threeparttable}\end{center}\end{landscape}}
\newenvironment{lltable}{\begin{landscape}\begin{center}\begin{ThreePartTable}}{\end{ThreePartTable}\end{center}\end{landscape}}

% Enables adjusting longtable caption width to table width
% Solution found at http://golatex.de/longtable-mit-caption-so-breit-wie-die-tabelle-t15767.html
\makeatletter
\newcommand\LastLTentrywidth{1em}
\newlength\longtablewidth
\setlength{\longtablewidth}{1in}
\newcommand{\getlongtablewidth}{\begingroup \ifcsname LT@\roman{LT@tables}\endcsname \global\longtablewidth=0pt \renewcommand{\LT@entry}[2]{\global\advance\longtablewidth by ##2\relax\gdef\LastLTentrywidth{##2}}\@nameuse{LT@\roman{LT@tables}} \fi \endgroup}

% \setlength{\parindent}{0.5in}
% \setlength{\parskip}{0pt plus 0pt minus 0pt}

% Overwrite redefinition of paragraph and subparagraph by the default LaTeX template
% See https://github.com/crsh/papaja/issues/292
\makeatletter
\renewcommand{\paragraph}{\@startsection{paragraph}{4}{\parindent}%
  {0\baselineskip \@plus 0.2ex \@minus 0.2ex}%
  {-1em}%
  {\normalfont\normalsize\bfseries\itshape\typesectitle}}

\renewcommand{\subparagraph}[1]{\@startsection{subparagraph}{5}{1em}%
  {0\baselineskip \@plus 0.2ex \@minus 0.2ex}%
  {-\z@\relax}%
  {\normalfont\normalsize\itshape\hspace{\parindent}{#1}\textit{\addperi}}{\relax}}
\makeatother

% \usepackage{etoolbox}
\makeatletter
\patchcmd{\HyOrg@maketitle}
  {\section{\normalfont\normalsize\abstractname}}
  {\section*{\normalfont\normalsize\abstractname}}
  {}{\typeout{Failed to patch abstract.}}
\patchcmd{\HyOrg@maketitle}
  {\section{\protect\normalfont{\@title}}}
  {\section*{\protect\normalfont{\@title}}}
  {}{\typeout{Failed to patch title.}}
\makeatother
\shorttitle{ReplicationLing}
\keywords{replication rate, experimental linguistics, ...\newline\indent Word count: X}
\DeclareDelayedFloatFlavor{ThreePartTable}{table}
\DeclareDelayedFloatFlavor{lltable}{table}
\DeclareDelayedFloatFlavor*{longtable}{table}
\makeatletter
\renewcommand{\efloat@iwrite}[1]{\immediate\expandafter\protected@write\csname efloat@post#1\endcsname{}}
\makeatother
\usepackage{lineno}

\linenumbers
\usepackage{csquotes}
\ifXeTeX
  % Load polyglossia as late as possible: uses bidi with RTL langages (e.g. Hebrew, Arabic)
  \usepackage{polyglossia}
  \setmainlanguage[]{english}
\else
  \usepackage[main=english]{babel}
% get rid of language-specific shorthands (see #6817):
\let\LanguageShortHands\languageshorthands
\def\languageshorthands#1{}
\fi
\ifLuaTeX
  \usepackage{selnolig}  % disable illegal ligatures
\fi
\newlength{\cslhangindent}
\setlength{\cslhangindent}{1.5em}
\newlength{\csllabelwidth}
\setlength{\csllabelwidth}{3em}
\newenvironment{CSLReferences}[2] % #1 hanging-ident, #2 entry spacing
 {% don't indent paragraphs
  \setlength{\parindent}{0pt}
  % turn on hanging indent if param 1 is 1
  \ifodd #1 \everypar{\setlength{\hangindent}{\cslhangindent}}\ignorespaces\fi
  % set entry spacing
  \ifnum #2 > 0
  \setlength{\parskip}{#2\baselineskip}
  \fi
 }%
 {}
\usepackage{calc}
\newcommand{\CSLBlock}[1]{#1\hfill\break}
\newcommand{\CSLLeftMargin}[1]{\parbox[t]{\csllabelwidth}{#1}}
\newcommand{\CSLRightInline}[1]{\parbox[t]{\linewidth - \csllabelwidth}{#1}\break}
\newcommand{\CSLIndent}[1]{\hspace{\cslhangindent}#1}

\title{Assessing replication rates in experimental linguistics}
\author{Kristina Kobrock\textsuperscript{1} \& Timo B. Roettger\textsuperscript{2}}
\date{}


\affiliation{\vspace{0.5cm}\textsuperscript{1} University of Osnabrück\\\textsuperscript{2} University of Oslo}

\abstract{
Abstract (to be written)
}



\begin{document}
\maketitle

\hypertarget{introduction}{%
\section{Introduction}\label{introduction}}

\hypertarget{overview-analysis-rate-of-replication-mention}{%
\section{Overview analysis: Rate of replication mention}\label{overview-analysis-rate-of-replication-mention}}

\hypertarget{research-questions}{%
\subsection{Research questions}\label{research-questions}}

\hypertarget{sample}{%
\subsection{Sample}\label{sample}}

\hypertarget{procedure}{%
\subsection{Procedure}\label{procedure}}

\hypertarget{data-analysis}{%
\subsection{Data Analysis}\label{data-analysis}}

\hypertarget{results}{%
\subsection{Results}\label{results}}

\hypertarget{detailed-analysis-types-and-contributing-factors}{%
\section{Detailed analysis: Types and contributing factors}\label{detailed-analysis-types-and-contributing-factors}}

\hypertarget{research-questions-1}{%
\subsection{Research questions}\label{research-questions-1}}

\hypertarget{sample-1}{%
\subsection{Sample}\label{sample-1}}

\hypertarget{procedure-1}{%
\subsection{Procedure}\label{procedure-1}}

\hypertarget{data-analysis-1}{%
\subsection{Data Analysis}\label{data-analysis-1}}

\begin{verbatim}
##                                                          author
## 1              Baus, Cristina; Costa, Albert; Carreiras, Manuel
## 2                BRADLEY, DC; SANCHEZCASAS, RM; GARCIAALBEA, JE
## 3 Branigan, Holly P.; Catchpole, Ciara M.; Pickering, Martin J.
## 4                            Braun, Bettina; Tagliapietra, Lara
## 5             Brouwer, Susanne; Mitterer, Holger; Huettig, Falk
## 6                                 CARREIRAS, M; GERNSBACHER, MA
##                                                                                        title
## 1 Neighbourhood density and frequency effects in speech production: A case for interactivity
## 2                        THE STATUS OF THE SYLLABLE IN THE PERCEPTION OF SPANISH AND ENGLISH
## 3                                                   What makes dialogues easy to understand?
## 4                                       On-line interpretation of intonational meaning in L2
## 5        Speech reductions change the dynamics of competition during spoken word recognition
## 6                                               COMPREHENDING CONCEPTUAL ANAPHORS IN SPANISH
##                            journal
## 1 LANGUAGE AND COGNITIVE PROCESSES
## 2 LANGUAGE AND COGNITIVE PROCESSES
## 3 LANGUAGE AND COGNITIVE PROCESSES
## 4 LANGUAGE AND COGNITIVE PROCESSES
## 5 LANGUAGE AND COGNITIVE PROCESSES
## 6 LANGUAGE AND COGNITIVE PROCESSES
##                                                                                                                                                                                                                                                                                                                                                                                                                                                                                                                                                                                                                                                                                                                                                                                                                                                                                                                                                                                                                                                                                                                                                                                                                                                                                                                                                                                                                                                                                                                                                                                                                                                                                    abstract
## 1                                                                                                                                                                                                                                                                                                                                                                                                                               In three experiments, we explore the effects of phonological properties such as neighbourhood density and frequency on speech production in Spanish. Specifically, we assess the reliability of the recent observation made by Vitevitch and Stamer (2006), according to which the neighbourhood effect in Spanish has a reverse polarity to that observed in other languages. In Experiment 1, we replicate Vitevitch and Stamer's (2006) experiment, this time adding a control group. The same inhibitory neighbourhood effect found for both groups can not corroborate the hypothesis posited by Vitevitch and Stamer. In Experiment 2, our results show that native speakers of Spanish named pictures with words belonging to high density neighbourhoods faster than those belonging to low density neighbourhoods. In Experiment 3, we test for effects of neighbourhood frequency during lexical selection. Again, we find a facilitatory effect for words with a high-frequency neighbourhood. Together, the results of the present experiments suggest that lexical selection is facilitated by the number of neighbours and by neighbourhoods with higher frequency. These findings are consistent with the predictions of interactive models.
## 2 A series of monitoring studies is reported, in replication of the cross-language research of Cutler, Mehler, Norris and Segui (1983; 1986), which found evidence of language-specific perceptual routines. Monolingual speakers of Spanish and English detected CV and CVC target sequences in native and non-native materials. The replication succeeded only in the case of Spanish speakers and Spanish materials, where a cross-over interaction of target (CV vs CVC sequences) and carrier types (CV- vs CVC-syllabified words) gave evidence of a sensitivity to the input's syllabification; no such pattern emerged for Spanish speakers and English materials, nor for English speakers and materials in either language. For English speakers, the consistent finding was for faster performance with CVC targets, regardless of the structure of the carrier word. Whether or not this is to be interpreted as evidence of syllabified input representations is not clear. Analyses of English syllabification that are alternatives to that adopted by Cutler et al. exist, to weaken the original contrast drawn between syllable-favouring and syllable-disfavouring languages. A final experiment examines monitoring performance in Spanish speakers who have become bilingual as a consequence of emigration to an English-speaking country; these subjects showed no syllable sensitivity for Spanish language materials. We speculate that factors outside the perceptual system may determine the basis on which responses are made in the monitoring task, and therefore conclude that the case for language specificity in perceptual routines has yet to be made.
## 3                                                                                                                                                                                                                                                                                                                                                                                                                                        Two experiments investigate the question of why dialogues tend to be easier for anyone to understand than monologues. One possibility is that overhearers of dialogue have access to the different perspectives provided by the interlocutors, whereas overhearers of monologue have access to the speaker's perspective alone (Fox Tree, 1999). Directors first described a set of geometric shapes to matchers in monologue or dialogue eight times. Experiment 1 found that descriptions taken from dialogue were easier to understand than descriptions taken from monologue or descriptions taken from dialogue in which the matcher's contributions were excised. This advantage occurred on early trials (when the matcher made a considerable contribution) but also on late trials (when the matcher simply accepted a description). Experiment 2 replicated this finding and ruled out an explanation in which the advantage of dialogue is due to its use of discourse markers. We argue that the ease of dialogue occurs because interlocutors negotiate a perspective that they can agree on (Clark, 1996). This grounded perspective is likely to be objectively easier to understand than a perspective that has not been grounded.
## 4                                                                                                                                                                                                                                                                                                                                                                                                                                                                                                   Despite their relatedness, Dutch and German differ in the interpretation of a particular intonation contour, the hat pattern. In the literature, this contour has been described as neutral for Dutch, and as contrastive for German. A recent study supports the idea that Dutch listeners interpret this contour neutrally, compared to the contrastive interpretation of a lexically identical utterance realised with a double peak pattern. In particular, this study showed shorter lexical decision latencies to visual targets (e.g., PELIKAAN, opelicano) following a contrastively related prime (e.g., flamingo, oflamingoo) only when the primes were embedded in sentences with a contrastive double peak contour, not in sentences with a neutral hat pattern. The present study replicates Experiment 1a of Braun and Tagliapietra (2009) with German learners of Dutch. Highly proficient learners of Dutch differed from Dutch natives in that they showed reliable priming effects for both intonation contours. Thus, the interpretation of intonational meaning in L2 appears to be fast, automatic, and driven by the associations learned in the native language.
## 5                                                                                                                                                                                                                                                                                                                                                                                               Three eye-tracking experiments investigated how phonological reductions (e. g., puter for computer) modulate phonological competition. Participants listened to sentences extracted from a spontaneous speech corpus and saw four printed words: a target (e. g., computer), a competitor similar to the canonical form (e. g., companion), one similar to the reduced form (e. g., pupil), and an unrelated distractor. In Experiment 1, we presented canonical and reduced forms in a syllabic and in a sentence context. Listeners directed their attention to a similar degree to both competitors independent of the target's spoken form. In Experiment 2, we excluded reduced forms and presented canonical forms only. In such a listening situation, participants showed a clear preference for the canonical form competitor. In Experiment 3, we presented canonical forms intermixed with reduced forms in a sentence context and replicated the competition pattern of Experiment 1. These data suggest that listeners penalize acoustic mismatches less strongly when listening to reduced speech than when listening to fully articulated speech. We conclude that flexibility to adjust to speech-intrinsic factors is a key feature of the spoken word recognition system.
## 6                                                                                                                                                                                                            This paper examines the mechanisms involved in the assignment of an antecedent to an anaphoric element. In general, pronouns must match their antecedents at least with respect to number and gender. Sensitivity to such constraints has been shown in several experiments. But Gernsbacher (1991) has also shown that people have no difficulty comprehending a plural pronoun with an antecedent that is grammatically singular but conceptually plural. In the first three experiments, we tested whether such a conceptual effect was preserved with zero anaphors in Spanish. (The typical omission of pronouns in subject position in Spanish.) Verbs in a second clause were marked with plural or singular endings. Plural verbs were rated more natural than singular verbs when they followed three types of singular but conceptually plural antecedents (Experiment 1). Clauses containing plural verbs were read faster when they followed one type of singular but conceptually plural antecedents, i.e. collective sets (Experiments 2 and 3). In fact, clauses containing plural verbs were read equally fast when they followed literally singular collective sets or explicitly group nouns. Using pronominal anaphors, these reading time effects were replicated and extended to sentences that contained generic types as antecedents (Experiment 4). The results are discussed in terms of the use of information during the comprehension of anaphors.
##   cit_times pub_date pub_year vol   issue start_page end_page article_number
## 1        29              2008  23       6        866      888               
## 2        73      MAY     1993   8       2        197      233               
## 3        15              2011  26      10       1667     1686               
## 4         7              2011  26       2        224      235  PII 923379546
## 5        37              2012  27       4        539      571               
## 6        21  AUG-NOV     1992   7 03. Apr        281      299               
##                            doi experimental replication
## 1    10.1080/01690960801962372            1           1
## 2    10.1080/01690969308406954            1           1
## 3 10.1080/01690965.2010.524765            1           1
## 4 10.1080/01690965.2010.486209            1           1
## 5 10.1080/01690965.2011.555268            1           0
## 6    10.1080/01690969208409388            1           0
##                                                          comments oa_article
## 1                     one of several experiments is a replication          0
## 2                     one of several experiments is a replication          0
## 3                                         inner-paper replication          0
## 4          replication of previous experiment by the same authors          0
## 5 no clear communication of intent to replicate an original study          0
## 6 no clear communication of intent to replicate an original study          1
##                                                                                                                                                                                                                cit_init_study
## 1                                                                      Vitevitch, M. S., & Stamer, M. K. (2006). The curious case of competition in Spanish speech production. Language and Cognitive Processes, 21, 760 770.
## 2                                                                                                     Cutler, A., Mehler, J., Norris, D. & Segui, J. (1983). A language-spodfic comprehension strategy. Nature, 304, 159-160.
## 3                                                                                                                                                                                                                        same
## 4 Braun, B., & Tagliapietra, L. (2009). The role of contrastive intonation contours in the retrieval of contextual alternatives. Language and Cognitive Processes. Advance online publication.\ndoi:10.1080/01690960903036836
## 5                                                                                                                                                                                                                            
## 6                                                                                                                                                                                                                            
##                 journal_init_study auth_overlap year_init_study years_between
## 1 Language and Cognitive Processes            0            2006             2
## 2                           Nature            0            1983            10
## 3                             same            1            2011             0
## 4 Language and Cognitive Processes            1            2009             2
## 5                                            NA              NA            NA
## 6                                            NA              NA            NA
##   exp_paradigm sample materials_setup measurement manipulation control
## 1            0      0               0           0            0       1
## 2            0      1               1           0            0       0
## 3            0      0               0           0            1       0
## 4            0      1               0           0            0       0
## 5           NA     NA              NA          NA           NA      NA
## 6           NA     NA              NA          NA           NA      NA
##   type_replication
## 1          partial
## 2       conceptual
## 3          partial
## 4          partial
## 5             <NA>
## 6             <NA>
##                                                                        language
## 1                                                            Spanish (& German)
## 2 Spanish (Castillian) & Australian English (initial: French & British English)
## 3                                                                       English
## 4                                                                         Dutch
## 5                                                                              
## 6                                                                              
##   success rep_citation init_citation init_cit_til_rep
## 1      NA           29            61                3
## 2      NA           73            92               26
## 3      NA           15            NA               NA
## 4      NA            7            43                2
## 5      NA           37            NA               NA
## 6      NA           21            NA               NA
\end{verbatim}

\begin{verbatim}
##                                                                                                                                                                                       author
## 1                                                                                                                                           Baus, Cristina; Costa, Albert; Carreiras, Manuel
## 2                                                                                                                                             BRADLEY, DC; SANCHEZCASAS, RM; GARCIAALBEA, JE
## 3                                                                                                                              Branigan, Holly P.; Catchpole, Ciara M.; Pickering, Martin J.
## 4                                                                                                                                                         Braun, Bettina; Tagliapietra, Lara
## 5                                                                                                                                          Brouwer, Susanne; Mitterer, Holger; Huettig, Falk
## 6                                                                                                                                                              CARREIRAS, M; GERNSBACHER, MA
## 7                                                                                                                                                                       Conrad, M; Jacobs, A
## 8                                                                                                                                       Content, A; Meunier, C; Kearns, RK; Frauenfelder, UH
## 9                                                                                                                                     Costa, A; Sebastian-Galles, N; Miozzo, M; Caramazza, A
## 10                                                                                                                                                        Dominguez, A; deVega, M; Cuetos, F
## 11                                                                                                             Andoni Dunabeitia, Jon; Kinoshita, Sachiko; Carreiras, Manuel; Norris, Dennis
## 12                                                                                                                                                                Emmorey, K; LilloMartin, D
## 13                                                                                                                                                                      Grosjean, F; Hirt, C
## 14                                                                                                                                         Gumnior, Heidi; Boelte, Jens; Zwitserlood, Pienie
## 15                                                                                                                                                     Jonsson, Martin L.; Hampton, James A.
## 16                                                                                                                                                 Kaan, E; Harris, A; Gibson, E; Holcomb, P
## 17                                                                                                                          Kennedy, Alan; Murray, Wayne S.; Jennings, Francis; Reid, Claire
## 18                                                                                                                                                         Kuipers, Jan-Rouke; La Heij, Wido
## 19                                                                                                                                        La Heij, Wido; Boelens, Harrie; Kuipers, Jan-Rouke
## 20                                                                                                                                                            Mathey, S; Robert, C; Zagar, D
## 21                                                                                                                                                                     PRASADA, S; PINKER, S
## 22                                                                                                                                                                  Robertson, C; Kirsner, K
## 23                                                                                                                                                                                Roelofs, A
## 24                                                                                                                                                 Santiago, J; MacKay, DG; Palma, A; Rho, C
## 25                                                                                                                                                                              Schiller, NO
## 26                                                                                                                                                    Schiller, Niels O.; Caramazza, Alfonso
## 27                                                                                                                                                          Spinelli, E; Segui, J; Radeau, M
## 28                                                                                                                                                                   Sullivan, MP; Riffel, B
## 29                                                                                                                                                          Tsang, Yiu-Kei; Chen, Hsuan-Chih
## 30                                                                                                                                       Ventura, P; Morais, J; Pattamadilok, C; Kolinsky, R
## 31                                                                                                                                                                  Vroomen, J; de Gelder, B
## 32                                                                                                                                                                  Wheeldon, LR; Morgan, JL
## 33                                                                                                         Wicha, NYY; Orozco-Figueroa, A; Reyes, I; Hernandez, A; de Barreto, LG; Bates, EA
## 34                                                                                                                                                                ZHOU, XL; MARSLENWILSON, W
## 35                                                                                                         Brown, Meredith; Salverda, Anne Pier; Gunlogson, Christine; Tanenhaus, Michael K.
## 36                                                                                            Croot, Karen; Lalas, George; Biedermann, Britta; Rastle, Kathleen; Jones, Kelly; Cholin, Joana
## 37                                                                                                                                                              Dupuis, Amanda; Berent, Iris
## 38                                                                                                                                        Heyselaar, Evelien; Peeters, David; Hagoort, Peter
## 39                                                                                                                                                                 Hien Pham; Baayen, Harald
## 40                                                                                                                                                      Lowder, Matthew W.; Gordon, Peter C.
## 41                                                                                                                               Lowder, Matthew W.; Maxfield, Nathan D.; Ferreira, Fernanda
## 42                                                                                                                                                       Nenadic, Filip; Tucker, Benjamin V.
## 43                                                                                                                                                                New, Boris; Nazzi, Thierry
## 44                                                                                                                                                          Patson, Nikole D.; Warren, Tessa
## 45                                                                                                                                   Reifegerste, Jana; Meyer, Antje S.; Zwitserlood, Pienie
## 46                                                                                                                                 Tomoschuk, Brendan; Ferreira, Victor S.; Gollan, Tamar H.
## 47                                                                                                                                                          Viebahn, Malte C.; Luce, Paul A.
## 48                                                                                                                                  Wang, Man; Chen, Yiya; Jiang, Minghu; Schiller, Niels O.
## 49                                                                                                                                    Yamasaki, Brianna L.; Stocco, Andrea; Prat, Chantel S.
## 50                                                                                       Yu, Lili; Cutter, Michael G.; Yan, Guoli; Bai, Xuejun; Fu, Yu; Drieghe, Denis; Liversedge, Simon P.
## 51                                                                                                                                                                         Barzilai, Maya L.
## 52                                                                                                                                              Ito, Kiwako; Turnbull, Rory; Speer, Shari R.
## 53                                                                                                                                    Tjuka, Annika; Huong Thi Thu Nguyen; Spalek, Katharina
## 54                                                                                                                                                        Bergelson, Elika; Swingley, Daniel
## 55                                                                                                                                Feiman, Roman; Mody, Shilpa; Sanborn, Sophia; Carey, Susan
## 56                                                                                                                                              Gerken, LouAnn; Quam, Carolyn; Goffman, Lisa
## 57                                                                                                                                      He, Angela Xiaoxue; Kon, Maxwell; Arunachalam, Sudha
## 58                                                                                                                                                     Nordmeyera, Ann E.; Frank, Michael C.
## 59                                                                                                                                                           Unal, Ercenur; Papafragou, Anna
## 60                                                                              Kurinec, Courtney A.; Wise, Ashleigh V. T.; Cavazos, Christina A.; Reyes, Elysse M.; Weaver, Charles A., III
## 61                                                                                                                                                         Lewis, Molly L.; Watson, Duane G.
## 62                                                                                                                                                       Shimada, Michihiro; Kanda, Takayuki
## 63                                                                                                               Thompson, Laura A.; Malloy, Daniel M.; Cone, John M.; Hendrickson, David L.
## 64                                                                                                                                                     Arnold, JE; Fagnano, M; Tanenhaus, MK
## 65                                                                                                                                                   Assink, EMH; Kattenberg, G; Wortmann, C
## 66                                                                                                                                                                  Burt, JS; Hutchinson, BJ
## 67                                                                                                                   Cummine, Jacqueline; Amyotte, Josee; Pancheshen, Brent; Chouinard, Brea
## 68                                                                                                                                                          Dufour, Sophie; Peereman, Ronald
## 69                                                                                                                                                                         HO, CSH; CHEN, HC
## 70                                                                                                                                                                   HUBBELL, JA; OBOYLE, MW
## 71                                                                                                                                                      Janssen, DP; Roelofs, A; Levelt, WJM
## 72                                                                                                                                                     Kamide, Y; Scheepers, C; Altmann, GTM
## 73                                                                                                                                                                       Kennison, Shelia M.
## 74                                                                                                                                                     Mazerolle, Erin L.; Marchand, Yannick
## 75                                                                                                                                    Miller, Krista A.; Raney, Gary E.; Demos, Alexander P.
## 76                                                                                                                                             Oettl, Birgit; Jaeger, Gerhard; Kaup, Barbara
## 77                                                                                                                       Pinango, Maria Mercedes; Aaron Winnick; Ullah, Rashad; Zurif, Edgar
## 78                                                                                                                                                                            Rakosi, Csilla
## 79                                                                                                                                                                            Rakosi, Csilla
## 80                                                                                                                                     Rigalleau, Francois; Guerry, Michele; Granjon, Lionel
## 81                                                                                                                                                                             Schmauder, AR
## 82                                                                                                                                          Schneider, Cosima; Bade, Nadine; Janczyk, Markus
## 83                                                                                                                                                                        Slioussar, Natalia
## 84                                                                                                                                                             Steinhauer, K; Friederici, AD
## 85                                                                                                                                                                    Tree, JEF; Meijer, PJA
## 86                                                                                                                                      Wallentin, Mikkel; Rocca, Roberta; Stroustrup, Sofia
## 87                                                                                                                                                                      Watt, SM; Murray, WS
## 88                                                                                                                                                                  Barcroft, J; Sommers, MS
## 89                                                                                                                                                            Ellis, Nick C.; Sagarra, Nuria
## 90                                                                                                                                          Henry, Nicholas; Culman, Hillah; VanPatten, Bill
## 91                                                                                                                                                                              Hui, Bronson
## 92                                                                                                                                                                               Robinson, P
## 93                                                                                                                                                                            Suzuki, Yuichi
## 94                                                                                                                                                                  Wong, Wynne; Ito, Kiwako
## 95                                                                                                                                                                    Campbell, JD; Katz, AN
## 96                                                                                                                     Flusberg, Stephen J.; Lauria, Mark; Balko, Samuel; Thibodeau, Paul H.
## 97                                                                                                                                     Walker, Esther J.; Bergen, Benjamin K.; Nunez, Rafael
## 98                                                                                                                                                            Islam, Farjana; Baggio, Giosue
## 99                                                                                                              Tieu, L.; Yatsushiro, K.; Cremers, A.; Romoli, J.; Sauerland, U.; Chemla, E.
## 100                                                                                                     Branzi, Francesca M.; Calabria, Marco; Gade, Miriam; Fuentes, Luis J.; Costa, Albert
## 101                                                                                                                                     Costa, Albert; Albareda, Barbara; Santesteban, Mikel
## 102                                                                                                                                                        Laxen, Jannika; Lavaur, Jean-Marc
## 103                                                                                                                       Qian, Zhiying; Lee, Eun-Kyung; Lu, Dora Hsin-Yi; Garnsey, Susan M.
## 104                                                                                                                                                           Runnqvist, Elin; Costa, Albert
## 105                                                                                                                                  Tremblay, Annie; Broersma, Mirjam; Coughlin, Caitlin E.
## 106                                                                                                                                                           Wang, Xin; Forster, Kenneth I.
## 107                                                                                                                                                          Yow, W. Quin; Markman, Ellen M.
## 108                                                                                                                                                                        Chan, Ka Long Roy
## 109                                                                                                                                               Kim, Euhee; Park, Myung-Kwan; Seo, Hye-Jin
## 110                                                                                                                                                                              De Jong, KJ
## 111                                                                                                                                                         Escudero, Paola; Wanrooij, Karin
## 112                                                                                                                                                                             Finley, Sara
## 113                                                                                                                                                        Kabak, Baris; Idsardi, William J.
## 114                                                                                                                                               Kazanas, Stephanie A.; Altarriba, Jeanette
## 115                                                                                                                                                       Pinnow, Eleni; Connine, Cynthia M.
## 116                                                                                                             Rimzhim, Anurag; Johri, Avantika; Kelty-Stephen, Damian G.; Fowler, Carol A.
## 117                                                                                                                                                  Stevenson, RJ; Nelson, AWR; Stenning, K
## 118                                                                                                                                                    STOWE, LA; TANENHAUS, MK; CARLSON, GN
## 119                                                                                                                                    Tomlinson, John M., Jr.; Gotzner, Nicole; Bott, Lewis
## 120                                                                                                                                Whalen, DH; Magen, HS; Pouplier, M; Kanq, AM; Iskarous, K
## 121                                                                                                                                    Flis, Gabriela; Sikorski, Adam; Szarkowska, Agnieszka
## 122 Orero, Pilar; Doherty, Stephen; Kruger, Jan-Louis; Matamala, Anna; Pedersen, Jan; Perego, Elisa; Romero-Fresco, Pablo; Rovira-Esteva, Sara; Soler-Vilageliu, Olga; Szarkowska, Agnieszka
## 123                                                                                                                                                               Avcu, Enes; Hestvik, Arild
## 124                                                                                                                                          Feiman, Roman; Maldonado, Mora; Snedeker, Jesse
## 125                                                                                                                                                   Hinterwimmer, Stefan; Brocher, Andreas
## 126                                                                                                                                                                Linzen, Tal; Oseki, Yohei
## 127                                                                                                                                                                              Pycha, Anne
## 128                                                                     Renans, Agata; Romoli, Jacopo; Makri, Maria Margarita; Tieu, Lyn; de Vries, Hanna; Folli, Raffaella; Tsoulas, George
## 129                                        Syrett, Kristen; Lingwall, Anne; Perez-Cortes, Silvia; Austin, Jennifer; Sanchez, Liliana; Baker, Hannah; Germak, Christina; Arias-Amaya, Anthony
## 130                                                                                                                                                         Arvaniti, A; Ladd, DR; Mennen, I
## 131                                                                                                                  Badin, Pierre; Boe, Louis-Jean; Sawallis, Thomas R.; Schwartz, Jean-Luc
## 132                                                                                                                                                          Janse, Esther; Ernestus, Mirjam
## 133                                                                                                                                                     Johnson, K; Strand, EA; D'Imperio, M
## 134                                                                                                                                                        Kirby, James; Sonderegger, Morgan
## 135                                                                                                                                                                               Moreton, E
## 136                                                                                                                                                          Reinisch, Eva; Mitterer, Holger
## 137                                                                                                                         Roettger, T. B.; Winter, B.; Grawunder, S.; Kirby, J.; Grice, M.
## 138                                                                                                                                                                         Shosted, Ryan K.
## 139                                                                                                                                                                 Sturm, Pavel; Volin, Jan
## 140                                                                                                                                                                    Basso, A; Caporali, A
## 141                                                                                                                                               Buchanan, L; Hildebrandt, N; MacKinnon, GE
## 142                                                                                                                                          Chang, Claire H. C.; Lin, Tzu-Hui; Kuo, Wen-Jui
## 143                                                                                                                                                  Dwivedi, Veena D.; Gibson, Raechelle M.
## 144                                                                                                                   Fraga, Isabel; Padron, Isabel; Acuna-Farina, Carlos; Diaz-Lago, Marcos
## 145                                                                                                                                     Phillips, NA; Segalowitz, N; O'Brien, I; Yamasaki, N
## 146                                                                                                                        Vieth, H. E.; McMahon, K. L.; Cunnington, R.; de Zubicaray, G. I.
## 147                                                                                                                                       Xiang, Ming; Grove, Julian; Giannakidou, Anastasia
## 148                                                                                                                                                                              Bird, Steve
## 149                                                                                                                                                                                 Byrne, B
## 150                                                                                                                                            Dufour, Sophie; Chuang, Yu-Ying; Nguyen, Noel
## 151                                                                                       Ellis, Nick C.; Hafeez, Kausar; Martin, Katherine I.; Chen, Lillian; Boland, Julie; Sagarra, Nuria
## 152                                                                                                                                                          Festman, Julia; Clahsen, Harald
## 153                                                                                                                                                                        Folk, JR; Rapp, B
## 154                                                                                                                                                                      Goswami, U; East, M
## 155                                                                                                                                         Zhao, Li-Ming; Alario, F. -Xavier; Yang, Yu-Fang
## 156                                                                                                                                                    Hall, Jeffrey A.; La France, Betty H.
## 157                                                                                                                                    Hansen, Karolina; Rakic, Tamara; Steffens, Melanie C.
## 158                                                                                                     Lagerwerf, Luuk; Boeynaems, Amber; van Egmond-Brussee, Charlotte; Burgers, Christian
## 159                                                                                                                             Malloch, Yining Zhou; Feng, Bo; Wang, Bingqing; Kim, Chelsea
## 160                                                                                                                                                                        Schroeder, Tobias
## 161                                                                                                                                                                               Andrews, S
## 162                                                                                                                                                          Arnett, Nathan; Wagers, Matthew
## 163                                                                                                                                                     Assfalg, Andre; Bernstein, Daniel M.
## 164                                                                                                                                                                 Bailey, KGD; Ferreira, F
## 165                                                                                                                                                                               Belke, Eva
## 166                                                                                                                                              Bell, Dane; Forster, Kenneth; Drake, Shiloh
## 167                                                                                                                                                                    Birdsong, D; Molis, M
## 168                                                                                                                                                            Brand, M; Rey, A; Peereman, R
## 169                                                                                                                                                                 BRENNAN, SE; WILLIAMS, M
## 170                                                                                                                                               Brown-Schmidt, S; Byron, DK; Tanenhaus, MK
## 171                                                                                                                                                                     Brown-Schmidt, Sarah
## 172                                                                                                                                                                     Chateau, D; Jared, D
## 173                                                                                                                   Chubala, Chrissy; Surprenant, Aimee M.; Neath, Ian; Quinlan, Philip T.
## 174                                                                                                                                                                             Dewhurst, SA
## 175                                                                                                                                          Dilley, Laura C.; Mattys, Sven L.; Vinke, Louis
## 176                                                                                                                                        Duyck, W; Szmalec, A; Kemps, E; Vandierendonck, A
## 177                                                                                                                                                        Evans, JSBT; Clibbens, J; Rood, B
## 178                                                                                                    Finley, Jason R.; Sungkhasettee, Victor W.; Roediger, Henry L., III; Balota, David A.
## 179                                                                                                                                                                         Fukumura, Kumiko
## 180                                                                                                                                      Garnsey, SM; Pearlmutter, NJ; Myers, E; Lotocky, MA
## 181                                                                                                                                                                  Hamilton, M; Rajaram, S
## 182                                                                                                                                                  Hartsuiker, RJ; Corley, M; Martensen, H
## 183                                                                                                                                          Heathcote, Andrew; Raymond, Frances; Dunn, John
## 184                                                                                                                   Hernandez, Mireia; Martin, Clara D.; Barcelo, Francisco; Costa, Albert
## 185                                                                                                                                     Hunt, R. Reed; Smith, Rebekah E.; Dunlap, Kathryn R.
## 186                                                                                                                 Jaeger, Lena A.; Mertzen, Daniela; Van Dyke, Julie A.; Vasishth, Shravan
## 187                                                                                                                                                                 Janssen, N; Caramazza, A
## 188                                                                                                                                                            Kwon, Nayoung; Sturt, Patrick
## 189                                                                                                                                   Lea, RB; Mason, RA; Albrecht, JE; Birch, SL; Myers, JL
## 190                                                                                                                                                       Lee, Chia-lin; Federmeier, Kara D.
## 191                                                                                                                                                                    Lipinski, J; Gupta, P
## 192                                                                                                                                                       Meyer, AS; Roelofs, A; Levelt, WJM
## 193                                                                                                                                        Nairne, James S.; Cogdill, Mindi; Lehman, Melissa
## 194                                                                                                                                                             Nishiyama, Ryoji; Ukita, Jun
## 195                                                                                                                                               Pecher, D; Zeelenberg, R; Raaijmakers, JGW
## 196                                                                                                                                 Perea, Manuel; Andoni Dunabeitia, Jon; Carreiras, Manuel
## 197                                                                                                                                                     Rawson, Katherine A.; Zamary, Amanda
## 198                                                                                                                                                  Rupprecht, Julia; Baeuml, Karl-Heinz T.
## 199                                                                                                                                                                      Saito, S; Miyake, A
## 200                                                                                                                             Salthouse, Timothy A.; Siedlecki, Karen L.; Krueger, Lacy E.
## 201                                                                                                                 Schotter, Elizabeth R.; Lee, Michelle; Reiderman, Michael; Rayner, Keith
## 202                                                                                                                                       Slowiaczek, LM; McQueen, JM; Soltano, EG; Lynch, M
## 203                                                                                                                                                   Tabor, W; Galantucci, B; Richardson, D
## 204                                                                                                                                                          Urbach, Thomas P.; Kutas, Marta
## 205                                                                                                                                                                 Velan, Hadas; Frost, Ram
## 206                                                                                                                                       Wallace, WP; Malone, CP; Swiergosz, MJ; Amberg, MD
## 207                                                                                                                                                                   Wheeldon, L; Lahiri, A
## 208                                                                                                                                              Yan, Ming; Luo, Yingyi; Inhoff, Albrecht W.
## 209                                                                                                                                                                  ZIEGLER, JC; JACOBS, AM
## 210                                                                                                                                                      Ziegler, JC; Montant, M; Jacobs, AM
##                                                                                                                                                                                                                                        title
## 1                                                                                                                                                 Neighbourhood density and frequency effects in speech production: A case for interactivity
## 2                                                                                                                                                                        THE STATUS OF THE SYLLABLE IN THE PERCEPTION OF SPANISH AND ENGLISH
## 3                                                                                                                                                                                                   What makes dialogues easy to understand?
## 4                                                                                                                                                                                       On-line interpretation of intonational meaning in L2
## 5                                                                                                                                                        Speech reductions change the dynamics of competition during spoken word recognition
## 6                                                                                                                                                                                               COMPREHENDING CONCEPTUAL ANAPHORS IN SPANISH
## 7                                                                                                         Replicating syllable frequency effects in Spanish in German: One more challenge to computational models of visual word recognition
## 8                                                                                                                                                                 Sequence detection in pseudowords in French: Where is the syllable effect?
## 9                                                                                                                                                                             The gender congruity effect: Evidence from Spanish and Catalan
## 10                                                                                                                                                                 Lexical inhibition from syllabic units in Spanish visual word recognition
## 11                                                                                                                         Is morpho-orthographic decomposition purely orthographic? Evidence from masked priming in the same-different task
## 12                                                                                                                                 Processing spatial anaphora: Referent reactivation with overt and null pronouns in American Sign Language
## 13                                                                                                                                    Using prosody to predict the end of sentences in English and French: Normal and brain-damaged subjects
## 14                                                                                                                                                                               A chatterbox is a box: Morphology in German word production
## 15                                                                                                                                             The modifier effect in within-category induction: Default inheritance in complex noun phrases
## 16                                                                                                                                                                                  The P600 as an index of syntactic integration difficulty
## 17                                                                                                                                                    Parsing Complements: Comments on the Generality of the Principle of Minimal Attachment
## 18                                                                                                                                                                              The limitations of cascading in the speech production system
## 19                                                                                                                               Object interference in children's colour and position naming: Lexical interference or task-set competition?
## 20                                                                                                                                               Neighbourhood distribution interacts with orthographic priming in the lexical decision task
## 21                                                                                                                                                                            GENERALIZATION OF REGULAR AND IRREGULAR MORPHOLOGICAL PATTERNS
## 22                                                                                                                                                          Indirect memory measures in spontaneous discourse in normal and amnesic subjects
## 23                                                                                                                                          Syllable structure effects turn out to be word length effects: Comment on Santiago et al. (2000)
## 24                                                                                                                                            Sequential activation processes in producing words and syllables: Evidence from picture naming
## 25                                                                                                                                                                                          Monitoring metrical stress in polysyllabic words
## 26                                                                                                                                     Grammatical gender selection and the representation of morphemes: The production of Dutch diminutives
## 27                                                                                                                                                                   Phonological priming in spoken word recognition with bisyllabic targets
## 28                                                                                                                                                                          The nature of phonological encoding during spoken word retrieval
## 29                                                                                                                                                        Do position-general radicals have a role to play in processing Chinese characters?
## 30                                                                                                                                                             The locus of the orthographic consistency effect in auditory word recognition
## 31                                                                                                                                                                              Lipreading and the compensation for coarticulation mechanism
## 32                                                                                                                                                                                        Phoneme monitoring in internal and external speech
## 33                                                                                              When zebras become painted donkeys: Grammatical gender and semantic priming interact during picture integration in a spoken Spanish sentence
## 34                                                                                                                                                                              WORDS, MORPHEMES AND SYLLABLES IN THE CHINESE MENTAL LEXICON
## 35                                                                                                                                                                                           Interpreting prosodic cues in discourse context
## 36                                                                                                                                                        Syllable frequency effects in immediate but not delayed syllable naming in English
## 37                                                                                                                                                                                          Signs are symbols: evidence from the Stroop task
## 38                                                                                                                                                                       Do we predict upcoming speech content in naturalistic environments?
## 39                                                                                                                                                             Vietnamese compounds show an anti-frequency effect in visual lexical decision
## 40                                                                                                                                            Eye-tracking and corpus-based analyses of syntax-semantics interactions in complement coercion
## 41                                                                                                                                                                          Processing of self-repairs in stuttered and non-stuttered speech
## 42                                                                                                                                                  Computational modelling of an auditory lexical decision experiment using jTRACE and TISK
## 43                                                                                                                                                                 The time course of consonant and vowel processing during word recognition
## 44                                                                                                                                                              Comparing the roles of referents and event structures in parsing preferences
## 45                                                                                                                          Inflectional complexity and experience affect plural processing in younger and older readers of Dutch and German
## 46                                                                                                                                                Translation distractors facilitate production in single- and mixed-language picture naming
## 47                                                             Where is the disadvantage for reduced pronunciation variants in spoken-word recognition? On the neglected role of the decision stage in the processing of word-form variation
## 48                                                                                                                                    The time course of speech production revisited: no early orthographic effect, even in Mandarin Chinese
## 49                                                                                                                                                  Relating individual differences in bilingual language experiences to executive attention
## 50                                                                                                                                                                    Word n+2 preview effects in three-character Chinese idioms and phrases
## 51                                                                                                                                                                  Templaticity Effects on Differential Processing of Consonants and Vowels
## 52                                                                                                                         Allophonic tunes of contrast: Lab and spontaneous speech lead to equivalent fixation responses in museum visitors
## 53                                                                                                                                                                Foxes, deer, and hedgehogs: The recall of focus alternatives in Vietnamese
## 54                                                                                                                                                                            Early Word Comprehension in Infants: Replication and Extension
## 55                                                                                                                                                                       What Do You Mean, No? Toddlers' Comprehension of Logical No and Not
## 56                                                                                                                                                      Adults Fail to Learn a Type of Linguistic Pattern that is Readily Learned by Infants
## 57                                                                                                                                                                               Linguistic Context in Verb Learning: Less is Sometimes More
## 58                                                                                                                                                 Early Understanding of Pragmatic Principles in Children's Judgments of Negative Sentences
## 59                                                                                                                                                                                       How Children Identify Events from Visual Experience
## 60                                                                                                                                    Clarity under cognitive constraint: Can a simple directive encourage busy speakers to avoid ambiguity?
## 61                                                                                                                                                                                       Effects of lexical semantics on acoustic prominence
## 62                                                                                                                                                                            What is the appropriate speech rate for a communication robot?
## 63                                                            The face-to-face light detection paradigm A new methodology for investigating visuospatial attention across different face regions in live face-to-face communication settings
## 64                                                                                                                                                                                            Disfluencies signal theee, um, new information
## 65                                                                                                                         Exploring the boundaries of sublexical word identification units: The use of onsets and rimes and reading ability
## 66                                                                                                                                                                Case-mixing effects on spelling recognition: The importance of test format
## 67                                                                                                       Evidence for the Modulation of Sub-Lexical Processing in Go No-Go Naming: The Elimination of the Frequency x Regularity Interaction
## 68                                                                                                                                     Competition Effects in Phonological Priming: The Role of Mismatch Position between Primes and Targets
## 69                                                                                                                                                   EFFECTS OF SYNTACTIC STRUCTURE IN THE MEMORY OF CONCRETE AND ABSTRACT CHINESE SENTENCES
## 70                                                                                                                        THE EFFECTS OF METAPHORICAL AND LITERAL COMPREHENSION PROCESSES ON LEXICAL DECISION LATENCY OF SENTENCE COMPONENTS
## 71                                                                                                                                                                          Stem complexity and inflectional encoding in language production
## 72                                                                                                             Integration of syntactic and semantic information in predictive processing: Cross-linguistic evidence from German and English
## 73                                                                                                                                  The Use of Verb Information in Parsing: Different Statistical Analyses Lead to Contradictory Conclusions
## 74                                                                                                                                         TypingSuite: Integrated Software for Presenting Stimuli, and Collecting and Analyzing Typing Data
## 75                                                                                                                                       Time toThrow in the Towel? No Evidence for Automatic Conceptual Metaphor Access in Idiom Processing
## 76                                                                                                                                                                The Role of Simple Semantics in the Process of Artificial Grammar Learning
## 77                                                                                                                                                                       Time-course of semantic composition: The case of aspectual coercion
## 78                                                                                                      Dealing with the Conflicting Results of Psycholinguistic Experiments: How to Resolve Them with the Help of Statistical Meta-analysis
## 79                                                                                                                                                         Replication of Psycholinguistic Experiments and the Resolution of Inconsistencies
## 80                                                                                                                            The Effect of Connectives on the Selection of Arguments: Implicit Consequentiality Bias for the Connective but
## 81                                                                                                                                      Ability to stand alone and processing of open-class and closed-class words: Isolation versus context
## 82                                                                                        Is Immediate Processing of Presupposition Triggers Automatic or Capacity-Limited? A Combination of the PRP Approach with a Self-Paced Reading Task
## 83                                                                                                                                                                  Processing of a Free Word Order Language: The Role of Syntax and Context
## 84                                                                            Prosodic boundaries, comma rules, and brain responses: The closure positive shift in ERPs as a universal marker for prosodic phrasing in listeners and readers
## 85                                                                                                                                                                                                  Building syntactic structure in speaking
## 86                                                                                                                                                                   Grammar, Gender and Demonstratives in Lateralized Imagery for Sentences
## 87                                                                                                                                                                                                     Prosodic form and parsing commitments
## 88                                                                                                                                                                    Effects of acoustic variability on second language vocabulary learning
## 89                                                                                                                                  LEARNED ATTENTION IN ADULT LANGUAGE ACQUISITION A Replication and Generalization Study and Meta-Analysis
## 90                                                                                                                    MORE ON THE EFFECTS OF EXPLICIT INFORMATION IN INSTRUCTED SLA A Partial Replication and a Response to Fernandez (2008)
## 91                                                                                                                          PROCESSING VARIABILITY IN INTENTIONAL AND INCIDENTAL WORD LEARNING AN EXTENSION OF SOLOVYEVA AND DEKEYSER (2018)
## 92  Cognitive abilities, chunk-strength, and frequency effects in implicit artificial grammar and incidental L2 learning: Replications of Reber, Walkenfeld, and Hernstadt (1991) and Knowlton and Squire (1996) relevance and their for SLA
## 93                                                                                                        THE ROLE OF PROCEDURAL LEARNING ABILITY IN AUTOMATIZATION OF L2 MORPHOLOGY UNDER DIFFERENT LEARNING SCHEDULES AN EXPLORATORY STUDY
## 94                                                                                  THE EFFECTS OF PROCESSING INSTRUCTION AND TRADITIONAL INSTRUCTION ON L2 ONLINE PROCESSING OF THE CAUSATIVE CONSTRUCTION IN FRENCH: AN EYE-TRACKING STUDY
## 95                                                                                                                                                                                          On reversing the topics and vehicles of metaphor
## 96                                                                                                                                                                Effects of Communication Modality and Speaker Identity on Metaphor Framing
## 97                                                                                                                                                 Disentangling Spatial Metaphors for Time Using Non-spatial Responses and Auditory Stimuli
## 98                                                                                                                                                                                                            Kripkeans of the world, unite!
## 99                                                                                                                                  On the Role of Alternatives in the Acquisition of Simple and Complex Disjunctions in French and Japanese
## 100                                                                                                                                                                                             On the bilingualism effect in task switching
## 101                                                                                                                                            Assessing the presence of lexical competition across languages: Evidence from the Stroop task
## 102                                                                         The role of semantics in translation recognition: effects of number of translations, dominance of translations and semantic relatedness of multiple translations
## 103                                                                                                                 Native and non-native (L1-Mandarin) speakers of English differ in online use of verb-based cues about sentence structure
## 104                                                                                                                                                    Is retrieval-induced forgetting behind the bilingual disadvantage in word production?
## 105                                                                                                         The functional weight of a prosodic cue in the native language predicts the learning of speech segmentation in a second language
## 106                                                                                                                                                         Masked translation priming with semantic categorization: Testing the Sense Model
## 107                                                                                                                               A bilingual advantage in how children integrate multiple cues to understand a speaker's referential intent
## 108                                                                                                                                          Being a 'purist' in trilingual Hong Kong: Code-switching among Cantonese, English and Putonghua
## 109                                                                                                                                                  L2ers' predictions of syntactic structure and reaction times during sentence processing
## 110                                                                                                                                                                                                 Rate-induced resyllabification revisited
## 111                                                                                                                                                                              The Effect of L1 Orthography on Non-native Vowel Perception
## 112                                                                                                                                                                                         Learning Exceptions in Phonological Alternations
## 113                                                                                                           Perceptual distortions in the adaptation of English consonant clusters: Syllable structure or consonantal contact constraints?
## 114                                                                                                                                                      Emotion Word Type and Affective Valence Priming at a Long Stimulus Onset Asynchrony
## 115                                                                                                                                                                              Phonological Variant Recognition: Representations and Rules
## 116                                                                                                                                                                   Transposition Effects in an Aksharic Writing System: The Case of Hindi
## 117                                                                                                                                                                           The role of parallelism in strategics of pronoun comprehension
## 118                                                                                                                                                     FILLING GAPS ONLINE - USE OF LEXICAL AND SEMANTIC INFORMATION IN SENTENCE PROCESSING
## 119                                                                                                                                                  Intonation and Pragmatic Enrichment: How Intonation Constrains Ad Hoc Scalar Inferences
## 120                                                                                                                                                           Vowel production and perception: Hyperarticulation without a hyperspace effect
## 121                                                                                                                            Does the dubbing effect apply to voice-over? A conceptual replication study on visual attention and immersion
## 122                                                                                                                                                      Conducting experimental research in audiovisual translation (AVT): A position paper
## 123                                                                                                                                                                                                                 Unlearnable phonotactics
## 124                                                                                                                                                               Priming quantifier scope: Reexamining the evidence against scope inversion
## 125                                                                                                                                                 An experimental investigation of the binding options of demonstrative pronouns in German
## 126                                                                                                                                                                              The reliability of acceptability judgments across Languages
## 127                                                                                                                                                              A new test for exemplar theory: Varying versus non-varying words in Spanish
## 128                                                                                                                                                  The abundance inference of pluralised mass nouns is an implicature: Evidence from Greek
## 129                                                                                              Differences between Spanish monolingual and Spanish-English bilingual children in their calculation of entailment-based scalar implicatures
## 130                                                                                                                                                                       Stability of tonal alignment: the case of Greek prenuclear accents
## 131                                                                                                                                                        Keep the lips to free the larynx: Comments on de Boer's articulatory model (2010)
## 132                                                                                           The roles of bottom-up and top-down information in the recognition of reduced speech: Evidence from listeners with normal and impaired hearing
## 133                                                                                                                                                                         Auditory-visual integration of talker gender in vowel perception
## 134                                                                                                                                                                                 Mixed-effects design analysis for experimental phonetics
## 135                                                                                                                                                                   Realization of the English postvocalic [voice] contrast in F-1 and F-2
## 136                                                                                                                                                      Exposure modality, input variability and the categories of perceptual recalibration
## 137                                                                                                                                                                         Assessing incomplete neutralization of final devoicing in German
## 138                                                                                                                                                                                 An articulatory-aerodynamic approach to stop excrescence
## 139                                                                                                                                 P-centres in natural disyllabic Czech words in a large-scale speech-metronome synchronization experiment
## 140                                                                                                                                                                      Targeted intervention for multiple language disorders: a case study
## 141                                                                                                                                                           Phonological processing of nonwords in deep dyslexia: Typical and independent?
## 142                                                                                                                                                                Does phonological rule of tone substitution modulate mismatch negativity?
## 143                                                                                                                                              An ERP investigation of quantifier scope ambiguous sentences: Evidence for number in events
## 144                                                                                                                                     Processing gender agreement and word emotionality: New electrophysiological and behavioural evidence
## 145                                                                                                              Semantic priming in a first and second language: evidence from reaction time variability and event-related brain potentials
## 146                                                                                                                                         Interference from object part relations in spoken word production: Behavioural and fMRI evidence
## 147                                                                                                     Semantic and pragmatic processes in the comprehension of negation: An event related potential study of negative polarity sensitivity
## 148                                                                                                                                                         Expert knowledge, distinctiveness, and levels of processing in language learning
## 149                                                                                                                   The learnability of the alphabetic principle: Children's initial hypotheses about how print represents spoken language
## 150                                                                                                                                                                        The processing of dialectal variants: Further insight from French
## 151                                                                                                                                                                An eye-tracking study of learned attention in second language acquisition
## 152                                                                                                                                          How Germans prepare for the English past tense: Silent production of inflected words during EEG
## 153                                                                                                                                             Interaction of lexical and sublexical information in spelling: Evidence from nonword priming
## 154                                                                                                                                                             Rhyme and analogy in beginning reading: Conceptual and methodological issues
## 155                                                                                                                                 Grammatical planning scope in sentence production: Further evidence for the functional phrase hypothesis
## 156                                                                                                                                                                               How Context Matters: Predicting Men's Homophobic Slang Use
## 157                                                                                                                                                  When Actions Speak Louder Than Words: Preventing Discrimination of Nonstandard Speakers
## 158                                                                                                        Immediate Attention for Public Speech: Differential Effects of Rhetorical Schemes and Valence Framing in Political Radio Speeches
## 159                                                                                                   Advice Padded With Encouragement: A Preliminary Assessment of an Extended Integrated Model of Advice Giving in Supportive Interactions
## 160                                                                                                                                                             A Model of Language-Based Impression Formation and Attribution Among Germans
## 161                                                                                                                                                    Lexical retrieval and selection processes: Effects of transposed-letter confusability
## 162                                                                                                                                                                                             Subject encodings and retrieval interference
## 163                                                                                                                                                              Puzzles produce strangers: A puzzling result for revelation-effect theories
## 164                                                                                                                                                                                 Disfluencies affect the parsing of garden-path sentences
## 165                                                                            Long-lasting inhibitory semantic context effects on object naming are necessarily conceptually mediated: Implications for models of lexical-semantic encoding
## 166                                                                                                                                         Early semantic activation in a semantic categorization task with masked primes: Cascaded or not?
## 167                                                                                                                                                              On the evidence for maturational constraints in second-language acquisition
## 168                                                                                                                                                                         Where is the syllable priming effect in visual word recognition?
## 169                                                                                                              THE FEELING OF ANOTHERS KNOWING - PROSODY AND FILLED PAUSES AS CUES TO LISTENERS ABOUT THE METACOGNITIVE STATES OF SPEAKERS
## 170                                                                                                                                                                   Beyond salience: Interpretation of personal and demonstrative pronouns
## 171                                                                                                                                           Partner-specific interpretation of maintained referential precedents during interactive dialog
## 172                                                                                                                                                                             Spelling-sound consistency effects in disyllabic word naming
## 173                                                                                                                                                           Does dynamic visual noise eliminate the concreteness effect in working memory?
## 174                                                                                                                                               Category repetition and false recognition: Effects of instance frequency and category size
## 175                                                                                                                     Potent prosody: Comparing the effects of distal prosody, proximal prosody, and semantic context on word segmentation
## 176                                                                                                                                         Verbal working memory is involved in associative word learning unless visual codes are available
## 177                                                                                                                                                                 The role of implicit and explicit negation in conditional reasoning bias
## 178                                                                                                                         Relative contributions of semantic and phonological associates to over-additive false recall in hybrid DRM lists
## 179                                                                                                                                                                                         Ordering adjectives in referential communication
## 180                                                                                                                                  The contributions of verb bias and plausibility to the comprehension of temporarily ambiguous sentences
## 181                                                                                                                                                                            The concreteness effect in implicit and explicit memory tests
## 182                                                                                                   The lexical bias effect is modulated by context, but the standard monitoring account doesn't fly: Related beply to Baars et al. (1975)
## 183                                                                                                                                                             Recollection and familiarity in recognition memory: Evidence from ROC curves
## 184                                                                                                                                                                                      Where is the bilingual advantage in task-switching?
## 185                                                                                                                                                                                     How does distinctive processing reduce false recall?
## 186                                                                                                                                           Interference patterns in subject-verb agreement and reflexives revisited: A large-sample study
## 187                                                                                                                                             The selection of closed-class words in noun phrase production: The case of Dutch determiners
## 188                                                                                                                                                            The use of control information in dependency formation: An eye-tracking study
## 189                                                                                                                                           Who knows what about whom: What role does common ground play in accessing distant information?
## 190                                                                                                                      Wave-ering: An ERP study of syntactic and semantic context effects on ambiguity resolution for noun/verb homographs
## 191                                                                                                                      Does neighborhood density influence repetition latency for nonwords? Separating the effects of density and duration
## 192                                                                                                                                                                   Word length effects in object naming: The role of a response criterion
## 193                                                                                                                                           Adaptive memory: Temporal, semantic, and rating-based clustering following survival processing
## 194                                                                                                                                                            Articulation of phonologically similar items disrupts free recall of nonwords
## 195                                                                                                                                                          Does pizza prime coin? Perceptual priming in lexical decision and pronunciation
## 196                                                                                                                                           Masked associative/semantic priming effects across languages with highly proficient bilinguals
## 197                                                                                               Why is free recall practice more effective than recognition practice for enhancing memory? Evaluating the relational processing hypothesis
## 198                                                                                                                                                        Retrieval-induced forgetting in item recognition: Retrieval specificity revisited
## 199                                                                                                                                          On the nature of forgetting and the processing-storage relationship in reading span performance
## 200                                                                                                                                                                                     An individual differences analysis of memory control
## 201                                                                                                                                                                  The effect of contextual constraint on parafoveal processing in reading
## 202                                                                                                                                           Phonological representations in prelexical speech processing: Evidence from form-based priming
## 203                                                                                                                                                                       Effects of merely local syntactic coherence on sentence processing
## 204                                                                                                                                           Quantifiers more or less quantify on-line: ERP evidence for partial incremental interpretation
## 205                                                                                                                                              Letter-transposition effects are not universal: The impact of transposing letters in Hebrew
## 206                                                                                                                                                                                          On the generality of false recognition reversal
## 207                                                                                                                                                                                                      Prosodic units in speech production
## 208                                                                                                                Syllable articulation influences foveal and parafoveal processing of words during the silent reading of Chinese sentences
## 209                                                                                                                                        PHONOLOGICAL INFORMATION PROVIDES EARLY SOURCES OF CONSTRAINT IN THE PROCESSING OF LETTER STRINGS
## 210                                                                                                                                                                           The feedback consistency effect in lexical decision and naming
##                                       journal
## 1            LANGUAGE AND COGNITIVE PROCESSES
## 2            LANGUAGE AND COGNITIVE PROCESSES
## 3            LANGUAGE AND COGNITIVE PROCESSES
## 4            LANGUAGE AND COGNITIVE PROCESSES
## 5            LANGUAGE AND COGNITIVE PROCESSES
## 6            LANGUAGE AND COGNITIVE PROCESSES
## 7            LANGUAGE AND COGNITIVE PROCESSES
## 8            LANGUAGE AND COGNITIVE PROCESSES
## 9            LANGUAGE AND COGNITIVE PROCESSES
## 10           LANGUAGE AND COGNITIVE PROCESSES
## 11           LANGUAGE AND COGNITIVE PROCESSES
## 12           LANGUAGE AND COGNITIVE PROCESSES
## 13           LANGUAGE AND COGNITIVE PROCESSES
## 14           LANGUAGE AND COGNITIVE PROCESSES
## 15           LANGUAGE AND COGNITIVE PROCESSES
## 16           LANGUAGE AND COGNITIVE PROCESSES
## 17           LANGUAGE AND COGNITIVE PROCESSES
## 18           LANGUAGE AND COGNITIVE PROCESSES
## 19           LANGUAGE AND COGNITIVE PROCESSES
## 20           LANGUAGE AND COGNITIVE PROCESSES
## 21           LANGUAGE AND COGNITIVE PROCESSES
## 22           LANGUAGE AND COGNITIVE PROCESSES
## 23           LANGUAGE AND COGNITIVE PROCESSES
## 24           LANGUAGE AND COGNITIVE PROCESSES
## 25           LANGUAGE AND COGNITIVE PROCESSES
## 26           LANGUAGE AND COGNITIVE PROCESSES
## 27           LANGUAGE AND COGNITIVE PROCESSES
## 28           LANGUAGE AND COGNITIVE PROCESSES
## 29           LANGUAGE AND COGNITIVE PROCESSES
## 30           LANGUAGE AND COGNITIVE PROCESSES
## 31           LANGUAGE AND COGNITIVE PROCESSES
## 32           LANGUAGE AND COGNITIVE PROCESSES
## 33           LANGUAGE AND COGNITIVE PROCESSES
## 34           LANGUAGE AND COGNITIVE PROCESSES
## 35        LANGUAGE COGNITION AND NEUROSCIENCE
## 36        LANGUAGE COGNITION AND NEUROSCIENCE
## 37        Language Cognition and Neuroscience
## 38        LANGUAGE COGNITION AND NEUROSCIENCE
## 39        LANGUAGE COGNITION AND NEUROSCIENCE
## 40        LANGUAGE COGNITION AND NEUROSCIENCE
## 41        LANGUAGE COGNITION AND NEUROSCIENCE
## 42        LANGUAGE COGNITION AND NEUROSCIENCE
## 43        LANGUAGE COGNITION AND NEUROSCIENCE
## 44        LANGUAGE COGNITION AND NEUROSCIENCE
## 45        LANGUAGE COGNITION AND NEUROSCIENCE
## 46        LANGUAGE COGNITION AND NEUROSCIENCE
## 47        LANGUAGE COGNITION AND NEUROSCIENCE
## 48        LANGUAGE COGNITION AND NEUROSCIENCE
## 49        LANGUAGE COGNITION AND NEUROSCIENCE
## 50        LANGUAGE COGNITION AND NEUROSCIENCE
## 51                       LABORATORY PHONOLOGY
## 52                       LABORATORY PHONOLOGY
## 53                       LABORATORY PHONOLOGY
## 54          LANGUAGE LEARNING AND DEVELOPMENT
## 55          LANGUAGE LEARNING AND DEVELOPMENT
## 56          LANGUAGE LEARNING AND DEVELOPMENT
## 57          LANGUAGE LEARNING AND DEVELOPMENT
## 58          LANGUAGE LEARNING AND DEVELOPMENT
## 59          LANGUAGE LEARNING AND DEVELOPMENT
## 60                     LANGUAGE AND COGNITION
## 61                     LANGUAGE AND COGNITION
## 62                        INTERACTION STUDIES
## 63                        INTERACTION STUDIES
## 64       JOURNAL OF PSYCHOLINGUISTIC RESEARCH
## 65       JOURNAL OF PSYCHOLINGUISTIC RESEARCH
## 66       JOURNAL OF PSYCHOLINGUISTIC RESEARCH
## 67       JOURNAL OF PSYCHOLINGUISTIC RESEARCH
## 68       JOURNAL OF PSYCHOLINGUISTIC RESEARCH
## 69       JOURNAL OF PSYCHOLINGUISTIC RESEARCH
## 70       JOURNAL OF PSYCHOLINGUISTIC RESEARCH
## 71       JOURNAL OF PSYCHOLINGUISTIC RESEARCH
## 72       JOURNAL OF PSYCHOLINGUISTIC RESEARCH
## 73       JOURNAL OF PSYCHOLINGUISTIC RESEARCH
## 74       JOURNAL OF PSYCHOLINGUISTIC RESEARCH
## 75       JOURNAL OF PSYCHOLINGUISTIC RESEARCH
## 76       JOURNAL OF PSYCHOLINGUISTIC RESEARCH
## 77       JOURNAL OF PSYCHOLINGUISTIC RESEARCH
## 78       JOURNAL OF PSYCHOLINGUISTIC RESEARCH
## 79       JOURNAL OF PSYCHOLINGUISTIC RESEARCH
## 80       JOURNAL OF PSYCHOLINGUISTIC RESEARCH
## 81       JOURNAL OF PSYCHOLINGUISTIC RESEARCH
## 82       JOURNAL OF PSYCHOLINGUISTIC RESEARCH
## 83       JOURNAL OF PSYCHOLINGUISTIC RESEARCH
## 84       JOURNAL OF PSYCHOLINGUISTIC RESEARCH
## 85       JOURNAL OF PSYCHOLINGUISTIC RESEARCH
## 86       JOURNAL OF PSYCHOLINGUISTIC RESEARCH
## 87       JOURNAL OF PSYCHOLINGUISTIC RESEARCH
## 88     STUDIES IN SECOND LANGUAGE ACQUISITION
## 89     STUDIES IN SECOND LANGUAGE ACQUISITION
## 90     STUDIES IN SECOND LANGUAGE ACQUISITION
## 91     STUDIES IN SECOND LANGUAGE ACQUISITION
## 92     STUDIES IN SECOND LANGUAGE ACQUISITION
## 93     STUDIES IN SECOND LANGUAGE ACQUISITION
## 94     STUDIES IN SECOND LANGUAGE ACQUISITION
## 95                        METAPHOR AND SYMBOL
## 96                        METAPHOR AND SYMBOL
## 97                        METAPHOR AND SYMBOL
## 98                       JOURNAL OF SEMANTICS
## 99                       JOURNAL OF SEMANTICS
## 100       BILINGUALISM-LANGUAGE AND COGNITION
## 101       BILINGUALISM-LANGUAGE AND COGNITION
## 102       BILINGUALISM-LANGUAGE AND COGNITION
## 103       BILINGUALISM-LANGUAGE AND COGNITION
## 104       BILINGUALISM-LANGUAGE AND COGNITION
## 105       BILINGUALISM-LANGUAGE AND COGNITION
## 106       BILINGUALISM-LANGUAGE AND COGNITION
## 107       BILINGUALISM-LANGUAGE AND COGNITION
## 108                       LINGUISTIC RESEARCH
## 109                       LINGUISTIC RESEARCH
## 110                       LANGUAGE AND SPEECH
## 111                       LANGUAGE AND SPEECH
## 112                       LANGUAGE AND SPEECH
## 113                       LANGUAGE AND SPEECH
## 114                       LANGUAGE AND SPEECH
## 115                       LANGUAGE AND SPEECH
## 116                       LANGUAGE AND SPEECH
## 117                       LANGUAGE AND SPEECH
## 118                       LANGUAGE AND SPEECH
## 119                       LANGUAGE AND SPEECH
## 120                       LANGUAGE AND SPEECH
## 121        JOURNAL OF SPECIALISED TRANSLATION
## 122        JOURNAL OF SPECIALISED TRANSLATION
## 123   GLOSSA-A JOURNAL OF GENERAL LINGUISTICS
## 124   GLOSSA-A JOURNAL OF GENERAL LINGUISTICS
## 125   GLOSSA-A JOURNAL OF GENERAL LINGUISTICS
## 126   GLOSSA-A JOURNAL OF GENERAL LINGUISTICS
## 127   GLOSSA-A JOURNAL OF GENERAL LINGUISTICS
## 128   GLOSSA-A JOURNAL OF GENERAL LINGUISTICS
## 129   GLOSSA-A JOURNAL OF GENERAL LINGUISTICS
## 130                      JOURNAL OF PHONETICS
## 131                      JOURNAL OF PHONETICS
## 132                      JOURNAL OF PHONETICS
## 133                      JOURNAL OF PHONETICS
## 134                      JOURNAL OF PHONETICS
## 135                      JOURNAL OF PHONETICS
## 136                      JOURNAL OF PHONETICS
## 137                      JOURNAL OF PHONETICS
## 138                      JOURNAL OF PHONETICS
## 139                      JOURNAL OF PHONETICS
## 140               JOURNAL OF NEUROLINGUISTICS
## 141               JOURNAL OF NEUROLINGUISTICS
## 142               JOURNAL OF NEUROLINGUISTICS
## 143               JOURNAL OF NEUROLINGUISTICS
## 144               JOURNAL OF NEUROLINGUISTICS
## 145               JOURNAL OF NEUROLINGUISTICS
## 146               JOURNAL OF NEUROLINGUISTICS
## 147               JOURNAL OF NEUROLINGUISTICS
## 148                 APPLIED PSYCHOLINGUISTICS
## 149                 APPLIED PSYCHOLINGUISTICS
## 150                 APPLIED PSYCHOLINGUISTICS
## 151                 APPLIED PSYCHOLINGUISTICS
## 152                 APPLIED PSYCHOLINGUISTICS
## 153                 APPLIED PSYCHOLINGUISTICS
## 154                 APPLIED PSYCHOLINGUISTICS
## 155                 APPLIED PSYCHOLINGUISTICS
## 156 JOURNAL OF LANGUAGE AND SOCIAL PSYCHOLOGY
## 157 JOURNAL OF LANGUAGE AND SOCIAL PSYCHOLOGY
## 158 JOURNAL OF LANGUAGE AND SOCIAL PSYCHOLOGY
## 159 JOURNAL OF LANGUAGE AND SOCIAL PSYCHOLOGY
## 160 JOURNAL OF LANGUAGE AND SOCIAL PSYCHOLOGY
## 161            JOURNAL OF MEMORY AND LANGUAGE
## 162            JOURNAL OF MEMORY AND LANGUAGE
## 163            JOURNAL OF MEMORY AND LANGUAGE
## 164            JOURNAL OF MEMORY AND LANGUAGE
## 165            JOURNAL OF MEMORY AND LANGUAGE
## 166            JOURNAL OF MEMORY AND LANGUAGE
## 167            JOURNAL OF MEMORY AND LANGUAGE
## 168            JOURNAL OF MEMORY AND LANGUAGE
## 169            JOURNAL OF MEMORY AND LANGUAGE
## 170            JOURNAL OF MEMORY AND LANGUAGE
## 171            JOURNAL OF MEMORY AND LANGUAGE
## 172            JOURNAL OF MEMORY AND LANGUAGE
## 173            JOURNAL OF MEMORY AND LANGUAGE
## 174            JOURNAL OF MEMORY AND LANGUAGE
## 175            JOURNAL OF MEMORY AND LANGUAGE
## 176            JOURNAL OF MEMORY AND LANGUAGE
## 177            JOURNAL OF MEMORY AND LANGUAGE
## 178            JOURNAL OF MEMORY AND LANGUAGE
## 179            JOURNAL OF MEMORY AND LANGUAGE
## 180            JOURNAL OF MEMORY AND LANGUAGE
## 181            JOURNAL OF MEMORY AND LANGUAGE
## 182            JOURNAL OF MEMORY AND LANGUAGE
## 183            JOURNAL OF MEMORY AND LANGUAGE
## 184            JOURNAL OF MEMORY AND LANGUAGE
## 185            JOURNAL OF MEMORY AND LANGUAGE
## 186            JOURNAL OF MEMORY AND LANGUAGE
## 187            JOURNAL OF MEMORY AND LANGUAGE
## 188            JOURNAL OF MEMORY AND LANGUAGE
## 189            JOURNAL OF MEMORY AND LANGUAGE
## 190            JOURNAL OF MEMORY AND LANGUAGE
## 191            JOURNAL OF MEMORY AND LANGUAGE
## 192            JOURNAL OF MEMORY AND LANGUAGE
## 193            JOURNAL OF MEMORY AND LANGUAGE
## 194            JOURNAL OF MEMORY AND LANGUAGE
## 195            JOURNAL OF MEMORY AND LANGUAGE
## 196            JOURNAL OF MEMORY AND LANGUAGE
## 197            JOURNAL OF MEMORY AND LANGUAGE
## 198            JOURNAL OF MEMORY AND LANGUAGE
## 199            JOURNAL OF MEMORY AND LANGUAGE
## 200            JOURNAL OF MEMORY AND LANGUAGE
## 201            JOURNAL OF MEMORY AND LANGUAGE
## 202            JOURNAL OF MEMORY AND LANGUAGE
## 203            JOURNAL OF MEMORY AND LANGUAGE
## 204            JOURNAL OF MEMORY AND LANGUAGE
## 205            JOURNAL OF MEMORY AND LANGUAGE
## 206            JOURNAL OF MEMORY AND LANGUAGE
## 207            JOURNAL OF MEMORY AND LANGUAGE
## 208            JOURNAL OF MEMORY AND LANGUAGE
## 209            JOURNAL OF MEMORY AND LANGUAGE
## 210            JOURNAL OF MEMORY AND LANGUAGE
##                                                                                                                                                                                                                                                                                                                                                                                                                                                                                                                                                                                                                                                                                                                                                                                                                                                                                                                                                                                                                                                                                                                                                                                                                                                                                                                                                                                                                                                                                                                                                                                                                                                                                                                                                                                                                                                                                                                                                                                                                                                                                                                                                                                                                                                                                                                                                                                                                                                                                                                                                                                                                                                                     abstract
## 1                                                                                                                                                                                                                                                                                                                                                                                                                                                                                                                                                                                                                                                                                                                                                                                                                                                                                                                                                                                                                                                                                                                                                                                                                                                                                                                                                                                                                In three experiments, we explore the effects of phonological properties such as neighbourhood density and frequency on speech production in Spanish. Specifically, we assess the reliability of the recent observation made by Vitevitch and Stamer (2006), according to which the neighbourhood effect in Spanish has a reverse polarity to that observed in other languages. In Experiment 1, we replicate Vitevitch and Stamer's (2006) experiment, this time adding a control group. The same inhibitory neighbourhood effect found for both groups can not corroborate the hypothesis posited by Vitevitch and Stamer. In Experiment 2, our results show that native speakers of Spanish named pictures with words belonging to high density neighbourhoods faster than those belonging to low density neighbourhoods. In Experiment 3, we test for effects of neighbourhood frequency during lexical selection. Again, we find a facilitatory effect for words with a high-frequency neighbourhood. Together, the results of the present experiments suggest that lexical selection is facilitated by the number of neighbours and by neighbourhoods with higher frequency. These findings are consistent with the predictions of interactive models.
## 2                                                                                                                                                                                                                                                                                                                                                                                                                                                                                                                                                                                                                                                                                                                                                                                                                                                                                                                                                                                  A series of monitoring studies is reported, in replication of the cross-language research of Cutler, Mehler, Norris and Segui (1983; 1986), which found evidence of language-specific perceptual routines. Monolingual speakers of Spanish and English detected CV and CVC target sequences in native and non-native materials. The replication succeeded only in the case of Spanish speakers and Spanish materials, where a cross-over interaction of target (CV vs CVC sequences) and carrier types (CV- vs CVC-syllabified words) gave evidence of a sensitivity to the input's syllabification; no such pattern emerged for Spanish speakers and English materials, nor for English speakers and materials in either language. For English speakers, the consistent finding was for faster performance with CVC targets, regardless of the structure of the carrier word. Whether or not this is to be interpreted as evidence of syllabified input representations is not clear. Analyses of English syllabification that are alternatives to that adopted by Cutler et al. exist, to weaken the original contrast drawn between syllable-favouring and syllable-disfavouring languages. A final experiment examines monitoring performance in Spanish speakers who have become bilingual as a consequence of emigration to an English-speaking country; these subjects showed no syllable sensitivity for Spanish language materials. We speculate that factors outside the perceptual system may determine the basis on which responses are made in the monitoring task, and therefore conclude that the case for language specificity in perceptual routines has yet to be made.
## 3                                                                                                                                                                                                                                                                                                                                                                                                                                                                                                                                                                                                                                                                                                                                                                                                                                                                                                                                                                                                                                                                                                                                                                                                                                                                                                                                                                                                                         Two experiments investigate the question of why dialogues tend to be easier for anyone to understand than monologues. One possibility is that overhearers of dialogue have access to the different perspectives provided by the interlocutors, whereas overhearers of monologue have access to the speaker's perspective alone (Fox Tree, 1999). Directors first described a set of geometric shapes to matchers in monologue or dialogue eight times. Experiment 1 found that descriptions taken from dialogue were easier to understand than descriptions taken from monologue or descriptions taken from dialogue in which the matcher's contributions were excised. This advantage occurred on early trials (when the matcher made a considerable contribution) but also on late trials (when the matcher simply accepted a description). Experiment 2 replicated this finding and ruled out an explanation in which the advantage of dialogue is due to its use of discourse markers. We argue that the ease of dialogue occurs because interlocutors negotiate a perspective that they can agree on (Clark, 1996). This grounded perspective is likely to be objectively easier to understand than a perspective that has not been grounded.
## 4                                                                                                                                                                                                                                                                                                                                                                                                                                                                                                                                                                                                                                                                                                                                                                                                                                                                                                                                                                                                                                                                                                                                                                                                                                                                                                                                                                                                                                                                                    Despite their relatedness, Dutch and German differ in the interpretation of a particular intonation contour, the hat pattern. In the literature, this contour has been described as neutral for Dutch, and as contrastive for German. A recent study supports the idea that Dutch listeners interpret this contour neutrally, compared to the contrastive interpretation of a lexically identical utterance realised with a double peak pattern. In particular, this study showed shorter lexical decision latencies to visual targets (e.g., PELIKAAN, opelicano) following a contrastively related prime (e.g., flamingo, oflamingoo) only when the primes were embedded in sentences with a contrastive double peak contour, not in sentences with a neutral hat pattern. The present study replicates Experiment 1a of Braun and Tagliapietra (2009) with German learners of Dutch. Highly proficient learners of Dutch differed from Dutch natives in that they showed reliable priming effects for both intonation contours. Thus, the interpretation of intonational meaning in L2 appears to be fast, automatic, and driven by the associations learned in the native language.
## 5                                                                                                                                                                                                                                                                                                                                                                                                                                                                                                                                                                                                                                                                                                                                                                                                                                                                                                                                                                                                                                                                                                                                                                                                                                                                                                                                                                                Three eye-tracking experiments investigated how phonological reductions (e. g., puter for computer) modulate phonological competition. Participants listened to sentences extracted from a spontaneous speech corpus and saw four printed words: a target (e. g., computer), a competitor similar to the canonical form (e. g., companion), one similar to the reduced form (e. g., pupil), and an unrelated distractor. In Experiment 1, we presented canonical and reduced forms in a syllabic and in a sentence context. Listeners directed their attention to a similar degree to both competitors independent of the target's spoken form. In Experiment 2, we excluded reduced forms and presented canonical forms only. In such a listening situation, participants showed a clear preference for the canonical form competitor. In Experiment 3, we presented canonical forms intermixed with reduced forms in a sentence context and replicated the competition pattern of Experiment 1. These data suggest that listeners penalize acoustic mismatches less strongly when listening to reduced speech than when listening to fully articulated speech. We conclude that flexibility to adjust to speech-intrinsic factors is a key feature of the spoken word recognition system.
## 6                                                                                                                                                                                                                                                                                                                                                                                                                                                                                                                                                                                                                                                                                                                                                                                                                                                                                                                                                                                                                                                                                                                                                                                             This paper examines the mechanisms involved in the assignment of an antecedent to an anaphoric element. In general, pronouns must match their antecedents at least with respect to number and gender. Sensitivity to such constraints has been shown in several experiments. But Gernsbacher (1991) has also shown that people have no difficulty comprehending a plural pronoun with an antecedent that is grammatically singular but conceptually plural. In the first three experiments, we tested whether such a conceptual effect was preserved with zero anaphors in Spanish. (The typical omission of pronouns in subject position in Spanish.) Verbs in a second clause were marked with plural or singular endings. Plural verbs were rated more natural than singular verbs when they followed three types of singular but conceptually plural antecedents (Experiment 1). Clauses containing plural verbs were read faster when they followed one type of singular but conceptually plural antecedents, i.e. collective sets (Experiments 2 and 3). In fact, clauses containing plural verbs were read equally fast when they followed literally singular collective sets or explicitly group nouns. Using pronominal anaphors, these reading time effects were replicated and extended to sentences that contained generic types as antecedents (Experiment 4). The results are discussed in terms of the use of information during the comprehension of anaphors.
## 7                                                                                                                                                                                                                                                                                                                                                                                                                                                                                                                                                                                                                                                                                                                                                                                                                                                                                                                                                                                                                                                                                                                                                                                                                                                                                                                                                                                                                                                                                                                                                                                                                                                                                                                                                                                                                                                                                                                                                                                                                                          Two experiments tested the role of syllable frequency in word recognition, recently suggested in Spanish, in another shallow orthography, German. Like in Spanish, word recognition performance was inhibited in a lexical decision and a perceptual identification task when the first syllable of a word was of high frequency. Given this replication of the inhibitory effect of syllable frequency in a second language, we discuss the issue whether and how computational models of word recognition would have to represent a word's syllabic structure in order to accurately describe processing of polysyllabic words.
## 8                                                                                                                                                                                                                                                                                                                                                                                                                                                                                                                                                                                                                                                                                                                                                                                                                                                                                                                                                                                                                                                                                                                                                                                                                                                                                                                                                                                                                                                                                                                            In two experiments, French speakers detected cv or cvc sequences at the beginning of disyllabic pseudowords varying in syllable structure and pivotal consonant. Overall, both studies failed to replicate the crossover interaction that has been previously observed in French by Mehler, Dommergues, Frauenfelder and Segui' (1981). In both experiments, latencies were shorter to cv than to cvc targets and this effect of target length was generally smaller for cvc. cv than for cv. cv carriers. However, a clear crossover interaction was observed for liquid pivotal consonants under target-blocking conditions, and especially for slow participants. A third experiment collected phonemegating data on the same pseudowords to obtain estimates of the duration of the initial phonemes. Regression analyses showed that phoneme duration accounted for a large proportion of the variance for cvc target detection, suggesting that participants were reacting rather directly to phonemic throughput. These findings argue against the hypothesis of an early syllabic classification mechanism in the perception of speech.
## 9                                                                                                                                                                                                                                                                                                                                                                                                                                                                                                                                                                                                                                                                                                                                                                                                                                                                                                                                                                                                                                                                                                                                                                                                                                                                                                                                                                                                                                                                                                                                                                                                                                                                                                                                                                                                                              In five picture-word interference experiments we explore the gender congruity effect observed in Dutch in two languages, Spanish and Catalan. Participants' performance was not affected by the relationship between the gender of the picture and the gender of the word. The results show that the gender congruity effect is not a universal effect, but varies from language to language, depending on crucial characteristics of the gender/determiner selection system used to process a given language. Consistent with the crosslinguistic hypothesis presented by Miozzo and Caramazza we argue that the retrieval of the noun's gender is enough to specify the determiner's phonological form in Dutch, but not in Catalan or Spanish, and this is the cause of the failure to replicate the Dutch results in these two languages.
## 10                                                                                                                                                                                                                                                                                                                                                                                                                                                                                                                                                                                                                                                                                                                                                                                                                                                                                                                                                                                                                                                                                                                                                                                                                                                                                                            In Spanish, there is some empirical support for the notion that, in visual word recognition, the syllables initially activate competing lexical candidates (Carreiras, Alvarez, & De Vega, 1993). This lexical activation, however, requires an inhibitory mechanism to select the appropriate lexical entry. The experiments presented here aimed to explore these inhibitory processes. A priming paradigm was used in which the prime and the target shared the first syllable (norma-norte), the initial letters but not the first syllable (noria-norte) or were unrelated (mando-norte or savia-norte). Subjects performed a lexical decision task on the targets and the results showed, in general, interference on the related pairs. When the frequency of mention of the prime was less than that of the target, an inhibitory effect was obtained for the same syllabic pairs but not for the same letter pairs (Experiment 1). When the frequency of mention of the prime was increased to promote direct processing of the target, the inhibitory effect extended to both conditions of related pairs (Experiment 2). Finally, the use of pseudoword primes (Experiment 3) replicated the basic pattern of results in Experiment 1. The applicability of the data to a dual-route model and the time course of syllabic processing is discussed.
## 11                                                                                                                                                                                                                                                                                                                                                                                                                                                                                                                                                                                                                                                                                                                                                                                                                                                                                                                                                                                                                                                                                                                                                                                                                                                                                                                                                                                                                                                                                                                                           Two experiments used the cross-case same-different task to test whether the orthographically driven morphological decomposition effects that have been found in the lexical decision task are obligatory. Experiment 1 replicated the manipulation used by Dunabeitia, Perea, and Carreiras (2007), testing transposed-letter (TL) priming effects spanning the boundary between the affix and the stem. In contrast to their finding observed with the lexical decision task, TL priming effect did not vanish with polymorphemic or pseudomorphemic words. Experiment 2 used the manipulation used by Rastle, Davis, and New (2004), comparing the effects of polymorphemic affixed words (e.g., walker), pseudo-affixed words (e.g., corner), and nonaffixed monomorphemic words (e.g., brothel) in target word recognition. Unlike the results observed in the original lexical decision study, equal priming effects were observed with all three types of words. These results suggest that the presence of an orthographically defined subunit (affix) is not sufficient to drive morphological decomposition processes.
## 12                                                                                                                                                                                                                                                                                                                                                                                                                                                                                                                                                                                                                                                                                                                                                                                                                                                                                                                                                                                                                  Unlike English, American Sign Language (ASL) permits phonologically null pronouns in tensed clauses. Null pronouns are licensed by morphological marking of ''agreeing'' verbs which agree with the spatial loci of the subject and object noun phrases of the sentence. We present two probe recognition experiments which investigated whether overt and null pronouns similarly reactivate their referents during on-line sentence comprehension. Experiment 1 revealed faster response times to probes that were referents of either an overt pronoun or a null pronoun compared to control probes, indicating that both overt pronouns and null pronouns associated with verb agreement reactivate their spatial referents. However, response times to non-referent probes were also faster than to control probes, and it was hypothesised that an end-of-sentence probe presentation may have tapped into a sentence integration process in which all possible referents were reactivated. In a second experiment, the same sentences were presented to a second group of subjects, but probe signs were presented before the end of the sentence, and 1000 msec after the anaphoric element. The main results of the first experiment were replicated (both overt and null pronouns reactivated their referents), and responses to non-referent probes were not significantly different from control probes. Both experiments indicated that (1) there is an important link between spatial verb agreement and the ASL pronominal system, and (2) non-referent inhibition does not have the same processing status in ASL as it does in English.
## 13                                                                                                                                                                                                                                                                                                                                                                                                                                                                                                                                                                                                                                                                                                                                                                                                                                                                                            In an earlier study (Grosjean, 1983), it was found that listeners of English were surprisingly accurate at predicting the temporal end of a sentence when only given the part up to the ''potentially last word'', that is a noun before an optional prepositional phrase of varying lengths. The present study investigated this phenomenon in four experiments. The first two experiments examined the prediction capabilities of listeners when presented with the whole sentence in segments of increasing duration and when presented with the potentially last word only. The results indicate that to be able to use prosody to predict the end of sentences correctly, subjects must have reached a point in the sentence where neither syntax nor semantics can contribute to the prediction process. The third experiment investigated whether the results obtained with English can be replicated in French, a language with a very different prosodic structure. It was found that unlike their English counterparts, French listeners were unable to differentiate between sentences that continued, although they could tell if a sentence ended or not. Finally, the fourth experiment examined whether left and right hemisphere brain-damaged (LHD, RHD) patients are equally proficient at estimating the length of a sentence. LHD patients behaved like their controls, but RHD patients experienced great difficulty doing the task. This confirms that sentence prosody may well involve the right hemisphere, especially when no other type of linguistic processing is involved. The extension of these studies to other types of linguistic material and to other languages is discussed, as is the on-line use of prediction in language processing.
## 14                                                                                                                                                                                                                                                                                                                                                                                                                                                                                                                                                                                                                                                                                                                                                                                                                                                                                                                                                                                                                                                                                                                                                                                                                                                                                                                                                                                                                                                                                                                                                                                                                                          Two experiments are reported in which university students translated visually presented English words into German, while German distractor words were simultaneously presented. Distractors were morphologically related, merely form-related or unrelated to the German translations (target words). The transparency of the semantic relation between target words and morphological distractors was also varied. Morphological distractors facilitated word-translation latencies irrespective of their semantic transparency, replicating results obtained with other tasks. Thus, in German word production, effects of morphological complexity seem to be largely independent of semantics. Morphological facilitation is also not due to mere form relatedness, since phonological distractors had no impact on translation latencies, relative to unrelated distractors. Our data corroborate the usefulness of word-translation for investigating spoken word production, in particular, for morphological processing.
## 15                                                                                                                                                                                                                                                                                                                                                                                                                                                                                                                                                                                                                                                                                                                                                                                                                                                                                                                                                                                                                                                                                                                                                                                                                                                                                                                                                                                                                                                                                                                                                                                                                                                                                                                                 Within-category induction is the projection of a generic property from a class (Apples are sweet) to a subtype of that class (Chinese apples are sweet). The modifier effect refers to the discovery reported by Connolly et al., that the subtype statement tends to be judged less likely to be true than the original unmodified sentence. The effect was replicated and shown to be moderated by the typicality of the modifier (Experiment 1). Likelihood judgements were also found to correlate between modified and unmodified versions of sentences. Experiment 2 elicited justifications, which suggested three types of reason for the effect-pragmatics, knowledge-based reasoning, and uncertainty about attribute inheritance. It is argued that the results provide clear evidence for the default inheritance of prototypical attributes in modified concepts, although a full account of the effect remains to be given.
## 16                                                                                                                                                                                                                                                                                                                                                                                                                                                                                                                                                                                                                                                                                                                                                                                                                                                                                                                                                                                                                                                                                                                                                                                                                                                                                                                                                                                                                                                                                                    The P600 component in Event Related Potential research has been hypothesised to be associated with syntactic reanalysis processes. We, however, propose that the P600 is not restricted to reanalysis processes, but reflects difficulty with syntactic integration processes in general. First we discuss this integration hypothesis in terms of a sentence processing model proposed elsewhere. Next, in Experiment 1, we show that the P600 is elicited in grammatical, non-garden path sentences in which integration is more difficult (i.e., who questions) relative to a control sentence (whether questions). This effect is replicated in Experiment 2. Furthermore, we directly compare the effect of difficult integration in grammatical sentences to the effect of agreement violations. The results suggest that the positivity elicited in who questions and the P600-effect elicited by agreement violations have partly overlapping neural generators. This supports the hypothesis that similar cognitive processes, i.e., integration, are involved in both first pass analysis of who questions and dealing with ungrammaticalities (reanalysis).
## 17                                                                                                                                                                                                                                                                                                                                                                                                                                                                                                                                                                                                                                                                                                                                                                                                                                                                                                                                                            Two experiments are described in this paper, which examine the processing of English sentences containing complement verbs, and which may be followed either by a nounphrase, as a direct object, or by a complement clause. It has been claimed by Frazier and Rayner (1982) that subjects are garden-pathed when reading reduced complements (lacking the overt complemetiser), and that this fact is strong evidence in support of the application of the principle of Minimal Attachment as a universal property of the human parser. Holmes, Kennedy, and Murray (1987) questioned this conclusion by providing evidence that full and reduced complements present equivalent problems, probably because of their greater structural complexity. Rayner and Frazier (1987) disputed this conclusion, ascribing it to an artefact resulting from the use of a self-paced reading task. The first experiment examines this controversy with a replication of the study by Holmes et al., measuring eye movements as the sentences are processed. The results replicate those of the original study and further show that the effects cannot be attributed to lexical preferences associated with the verb. The second experiment examines the format of the displayed text as one possible reason for the discrepancy between the two sets of results. When sentences are displayed across several lines, the line breaks are interpreted by subjects as signals for potential clause endings, garden-pathing the reader. However, misparsings induced in this way cannot be used to support the claim that readers are generally garden-pathed by the temporary ambiguity of reduced complement sentences.
## 18                                                                                                                                                                                                                                                                                                                                                                                                                                                                                                                                                                                                                                                                                                                                                                                                                                                                                                                                                                                                                                                                                                                                                                                                                                                                                                                                                                                                                                                                           Two competing views on how information flows in the speech production system are discussed. The full-cascading view holds that all activated concepts automatically activate their lexical and phonological representations. The limited-cascading view holds that a selection procedure interrupts the automatic flow of information through the speech production system. Recently, the full-cascading view has received support from the observation that ignored pictures activate their phonological representation. In two experiments the conditions to observe this finding were examined. Using coloured pictures to name, we replicated the finding that when the picture's name is phonologically related to the name of its colour, the colour-naming task is facilitated compared with when the name of the picture is unrelated. We also show that this effect is stronger when naming the picture has been practiced. By contrast, the colour's name has no effect on naming the picture, not even when colour naming is practiced. We conclude that strong versions of both the full-cascading view and the limited-cascading view cannot account for the complete set of data.
## 19                                                                                                                                                                                                                                                                                                                                                                                                                                                                                                                                                                                                                                                                                                                                                                                                                                                                                                                                                                                                                                                                                                                                                                                                                                                                                                                                                                                                                                                                                                                                                                 Cascade models of word production assume that during lexical access all activated concepts activate their names. In line with this view, it has been shown that naming an object's colour is facilitated when colour name and object name are phonologically related (e. g., 'blue' and 'blouse'). Prevor and Diamond's (2005) recent observation that children take longer to name the colour of real objects than of abstract forms could also be attributed to cascaded processing, resulting in competition between colour name and object name. Experiments 1 and 2 replicate this 'object-interference effect' in colour naming by children of 5-7 years of age and show that it generalises to position naming. Experiment 2 shows that the effect is also obtained with hard-to-name objects; a finding that is at variance with a lexical-competition account. The finding in Experiment 3 that the object-interference effect is absent in adults, is consistent with an alternative interpretation in terms of task-set competition. Implications for models of word production are discussed.
## 20                                                                                                                                                                                                                                                                                                                                                                                                                                                                                                                                                                                                                                                                                                                                                                                                                                                                                                                                                                                                                                                                                                                                                                                                                                                                                                                                                                                                                                                                                                                       Lexical decision tasks (LDTs) were used with a masked priming procedure to test whether neighbourhood distribution interacts with orthographic priming. Word targets had either 'single' neighbours when their two higher frequency orthographic neighbours were spread over letter positions (e.g., neighbours of LOBE : robe-loge ) or 'twin' neighbours when they were concentrated on a single letter position (e.g., neighbours of FARD : lard-tard ). All word targets were preceded by their highest frequency orthographic neighbour or by a control prime. An inhibitory priming effect was found for words with single neighbours, but not for words with twin neighbours, in both a yes/no LDT (Experiment 1a) and a go/no-go LDT (Experiment 1b). This interaction was replicated in a go/no-go LDT when the position of the letter yielding the neighbour prime was controlled (Experiment 2). Simulations run on the word materials revealed that the interactive activation model captures the inhibitory priming effect in the single-neighbour condition but fails to capture the loss of priming in the twin-neighbour condition.
## 21                                                                                                                                                                                                                                                                                                                           Both regular inflectional patterns (walk-walked) and irregular ones (swing-swung) can be applied productively to novel words (e.g. wug-wugged; spling -splung). Theories of generative phonology attribute both generalisations to rules; connectionist theories attribute both to analogies in a pattern associator network; hybrid theories attribute regular (fully predictable default) generalisations to a rule and irregular generalisations to a rote memory with pattern-associator properties. In three experiments and three simulations, we observe the process of generalising morphological patterns in humans and two-layer connectionist networks. Replicating Bybee and Moder (1983), we find that people's willingness to generalise from existing irregular verbs to novel ones depends on the global similarity between them (e.g. spling is readily inflectable as splung, but nist is not inflectable as nust). In contrast, generalisability of the regular suffix does not appear to depend on similarity to existing regular verbs: Regularly suffixed versions of both common-sounding plip and odd-sounding ploamph were reliably produced and highly rated, and the odd-sounding verbs were not rated as having worse past-tense forms, relative to the naturalness of their stems, than common-sounding ones. In contrast, Rumelhart and McClelland's connectionist past-tense model was found to vary strongly in its tendency to supply both irregular and regular inflections to these novel items as a function of their similarity to forms it was trained on, and for the dissimilar forms, successful regular inflection rarely occurred. We suggest that rule-only theories have trouble explaining patterns of irregular generalisations, whereas single-network theories have trouble explaining regular ones; the computational demands of the two kinds of verbs are different, so a modular system is optimal for handling both simultaneously. Evidence from linguistics and psycholinguistics independently calls for such a hybrid, where irregular pairs are stored in a memory system that superimposes phonological forms, fostering generalisation by analogy, and regulars are generated by a default suffix concatenation process capable of operating on any verb, regardless of its sound.
## 22                                                                                                                                                                                                                                                                                                                                                                                                                                                                                                                                                                                                                                                                                                                                                                                                                                                                                                                                                                                                                                                    It has been demonstrated that people reduce the duration of repeated words during spontaneous discourse. They do, this, presumably, to indicate that a word is 'old' or 'Given', and that the existing representation of the referent can therefore be used to simplify and facilitate comprehension. The experiment reported here was designed to replicate and extend a previous study by Fowler. The experiment extended Fowler's work in two ways; first, into reference, by comparing word repetitions that involved 'new' as distinct from Given information and, second, into memory processes, by comparing discourse from normal and amnesic subjects. Following Fowler, the experiment demonstrated that word duration is reduced for repeated words when the second utterance involves Given or old information provided that the repetition occurs before a topic change, a pattern that was observed for both normal and amnesic subjects. However, unlike Fowler, the experiment demonstrated that word duration is increased for the second utterance when that utterance involves New information, a pattern that was also observed for both normal and amnesic subjects. The experiment demonstrated that referential information is preserved and used during spontaneous discourse by amnesic subjects. The measurement of duration by acoustic analysis therefore offers a new approach to the analysis of memory processes in conversation, and the results suggest that Given/New marking involves records that create or maintain specific information about reference and/or Given/New status.
## 23                                                                                                                                                                                                                                                                                                                                                                                                                                                                                                                                                                                                                                                                                                                                                                                                                                                                                                                                                                                                                                                                                                                                                                                                                                                                                                                                                                                                                                                                                                                                                                     Santiago, MacKay, Palma, and Rho (2000) report two picture naming experiments examining the role of syllable onset complexity and number of syllables in spoken word production. Experiment 1 showed that naming latencies are longer for words with two syllables (e. g., demon) than one syllable (e. g., duck), and longer for words beginning with a consonant cluster (e. g., drill) than a single consonant (e. g., duck). Experiment 2 replicated these findings and showed that the complexity of the syllable nucleus and coda has no effect. These results are taken to support MacKay's (1987) Node Structure theory and to refute models such as WEAVER++ (Roelofs, 1997a) that predict effects of word length but not of onset complexity and number of syllables per se. In this comment, I show that a re-analysis of the data of Santiago et al. that takes word length into account leads to the opposite conclusion. The observed effects of onset complexity and number of syllables appear to be length effects, supporting WEAVER++ and contradicting the Node Structure theory.
## 24                                                                                                                                                                                                                                                                                                                                                                                                                                                                                                                                                                                                                                                                                                                                                                                                                                                                                                                                                                                                                                                                                                                                                                                                                                                                                                                                                                                                                                                                                                                                                                                                                                                  This study examines picture naming latencies for predicted effects of two word retrieval factors: onset complexity and number of syllables. In Experiment 1, naming latency was longer for depicted words with two syllables e.g.. demon, than one syllable, e.g., duck, and longer for words beginning with consonant clusters, e.g., drill, than single consonants, e.g., duck. Experiment 2 replicated these results and showed that vowel nucleus and coda complexity did not interact with onset complexity, and did not affect naming latency. Moreover, effects of onset complexity and number of syllables were additive, and unrelated to word frequency, articulatory factors, or picture complexity. These results comport with evidence from speech errors and metalinguistic tasks and with predictions of the Node Structure theory of language production, but do not support production theories that do not predict special processing difficulty for words with complex onsets and multiple syllables.
## 25                                                                                                                                                                                                                                                                                                                                                                                                                                                                                                                                                                                                                                                                                                                                                                                                                                                                                                                                                                                                                                                                                                                                                                                                                                                                                                                                                                                                                                                                                                                                                                                                                    This study investigated the monitoring of metrical stress information in internally generated speech. In Experiment 1, Dutch participants were asked to judge whether bisyllabic picture names had initial or final stress. Results showed significantly faster decision times for initially stressed targets (e.g., KAno canoe) than for targets with final stress (e.g., kaNON cannon; capital letters indicate stressed syllables). It was demonstrated that monitoring latencies are not a function of the picture naming or object recognition latencies to the same pictures. Experiments 2 and 3 replicated the outcome of the first experiment with trisyllabic picture names. These results are similar to the findings of Wheeldon and Levelt (1995) in a segment monitoring task. The outcome might be interpreted to demonstrate that phonological encoding in speech production is a rightward incremental process. Alternatively, the data might reflect the sequential nature of a perceptual mechanism used to monitor lexical stress.
## 26                                                                                                                                                                                                                                                                                                                                                                                                                                                                                                                                                                                                                                                                                                                                                                                                                                                                                                                                                                                                                                                                                                                                                                                                                                                                                                                             In this study, we investigated grammatical feature selection during noun phrase production in Dutch. More specifically, we studied the conditions under which different grammatical genders select either the same or different determiners. Pictures of simple objects paired with a gender-congruent or a gender-incongruent distractor word were presented. Participants named the pictures using a noun phrase with the appropriate gender-marked determiner. Auditory (Experiment 1) or visual cues (Experiment 2) indicated whether the noun was to be produced in its standard or diminutive form. Results revealed a cost in naming latencies when target and distractor take different determiner forms independent of whether or not they have the same gender. This replicates earlier results showing that congruency effects are due to competition during the selection of determiner forms rather than gender features. The overall pattern of results supports the view that grammatical feature selection is an automatic consequence of lexical node selection and therefore not subject to interference from incongruent grammatical features. Selection of the correct determiner form, however, is a competitive process, implying that lexical node and grammatical feature selection operate with distinct principles.
## 27                                                                                                                                                                                                                                                                                                                                                                                                                                                                                                                                                                                                                                                                                                                                                                                                                                                                                                                                                                                                                                                                                                                                                                                                                                                                                                                                                                                                                                                                                                                                                                                                                                                                                                        Four experiments were carried out to examine phonological priming effects on bisyllabic target words. In Experiments 1a and 1b, auditorily presented monosyllabic word and pseudoword primes facilitated lexical decisions to auditorily presented bisyllabic words. This facilitation was found for primes overlapping the targets' initial syllable (e.g., ver'' [worm in French] primed VERTIGE'' [VERTIGO]) and for primes overlapping the targets' final syllable (e.g., tige'' [stem] primed VERTIGE''). Experiment 2 replicated the initial-overlap effect for monosyllabic word primes using a crossmodal (auditory-visual) method; however no facilitation was observed for final-overlap nor for bisyllabic primes (e.g., verger'' [orchard] did not facilitate VERTIGE). In Experiment 3, the initial overlap facilitation effect was replicated in a naming task. These results are interpreted in terms of activation and deactivation of candidates.
## 28                                                                                                                                                                                                                                                                                                                                                                                                                                                                                                                                                                                                                                                                                                                                                                                                                                                                                                                                                                                                                                                                                                                                                                                                                                                                                                                                                In the present study, two priming experiments were performed to gain a better understanding of the planning processes that underlie phonological encoding. In Experiment 1, participants named picture primes and targets that were identical, shared the same onsets (e.g. coat-comb), or shared the same rhymes (e.g. mouse-house). In addition, the response stimulus interval (RSI) between the primes and targets was varied (650 and 1000 msec). The results revealed a facilitatory repetition priming effect that did not decrease with RSI. In addition, a reliable rhyme-related inhibitory effect was obtained, which was smaller in magnitude than the onset-related inhibitory effect. In Experiment 2, the identity condition was excluded to determine if a strategic comparison process may have produced the inhibitory effects. In addition, a new stimulus set was created and the RSI was held at 650 msec. The results replicated those of Experiment 1 by showing a reliably larger inhibitory effect for onset-related than rhyme-related targets. The results are consistent with Dell's (1988) two-stage sequential model of encoding in which there is an initial parallel activation within a lexical network followed by a sequential left-to-right selection of the intended word's phonemes.
## 29                                                                                                                                                                                                                                                                                                                                                                                                                                                                                                                                                                                                                                                                                                                                                                                                                                                                                                                                                                                                                                                                                                                                                                                                                                                                               Over 90% of Chinese characters are compounds, comprising two or more constituents called radicals. Two experiments employed a character matching task to examine the contribution of radical position in Chinese character processing. The task was to decide whether the target character had appeared in two briefly and sequentially presented preceding source characters. Experiment 1 discovered significantly more false matching when the target (e. g., '(sic)') shared radicals with the source stimuli ('(sic)', '(sic)') than when the target and the source shared no radical, indicating that radicals contribute to character processing. Experiment 2 replicated this finding and further demonstrated that sharing a single radical between target and source characters, regardless of its radical position, was sufficient to generate false matching. More importantly, participants took significantly longer time to correctly reject those target characters with two shared radicals (one at the same position and another at the changed position) relative to those with only a single shared radical at the same position. Furthermore, false matching rates were significantly affected by lexical variables such as character frequency. These results suggest that position-general radicals play a significant role in character recognition and processing.
## 30                                                                                                                                                                                                                                                                                                                                                                                                                                                                                                                                                                                                                                                                                                                                                                                                                                                                                                                                                                                                                                                                                                                                                                                                                                                                                                                                                                                                                                                                                                                                                                                                                                                                                                                        In Experiments 1-2, we replicated with two different Portuguese materials the consistency effect observed for French by Ziegler and Ferrand (1998). Words with rimes that can be spelled in two different ways (inconsistent) produced longer auditory lexical decision latencies and more errors than did consistent words. In Experiment 3, which used shadowing, no effect of orthographic consistency was found. This task difference could reflect the confinement of orthographic influences to either decisional or lexical processes. In Experiment 4, we tried to untangle these two interpretations by comparing two situations in which a shadowing response was made contingent upon either a lexical or a phonemic criterion. A significant effect of orthographic consistency was observed only in lexically contingent shadowing. We thus argue that lexical but not sublexical processes are affected by orthographic consistency.
## 31                                                                                                                                                                                                                                                                                                                                                                                                                                                                                                                                                                                                                                                                                                                                                                                                                                                                                                                                                                                                                                                                                                                                                                                                                                                                                                                                                                                                                                                                                                                                                                                                                                                                                                              Listeners compensate for coarticulatory influences of one speech sound on another. We examined whether lipread information penetrates this perceptual compensation mechanism. Experiment 1 replicated the finding that when an /as/ or /a integral/ sound preceded a /ta/-/ka/ continuum, more velar stops were perceived in the context of /as/. Experiments 2 and 3 investigated whether the same phoneme boundary shift would be obtained when the context was lipread instead of heard. An ambiguous sound between /as/ and /a integral/ was dubbed on the video of a speaker articulating /as/ or /a integral/. Subjects relied on the lipread information when identifying the ambiguous fricative sound as /s/ or /integral/, but there was no corresponding boundary shift in the following /ta/-/ka/ continuum. These results indicate that biasing of the fricative by lipread information and compensation for coarticulation can be dissociated.
## 32                                                                                                                                                                                                                                                                                                                                                                                                                                                                                                                                                                                                                                                                                                                                                                                                                                                                                                                                                                                                                                                                                                                                                                                                  Four experiments examine the time course of phoneme monitoring in internally and externally generated speech. The aim of this research was to replicate and extend previous findings of Wheeldon and Levelt (1995), who required their Dutch participants to monitor their own prearticulatory speech in order to investigate the generation of an abstract phonological code. Experiment 1 required British participants to silently generate English words with a CVCCVC structure and to monitor these words for the four constituent consonants. Similar to the Dutch study, a clear left-to-right monitoring effect was observed and a significant difference was seen between latencies for the consonants separated by the initial vowel and for the consecutive word medial consonants. Internal speech monitoring was also found to speed up across the word. A perception version of the task (Experiment 2) yielded a different pattern of monitoring latencies. In Experiments 3 and 4, all phonemes in the initial CVC syllable of bisyllabic words were monitored in internal and external speech respectively. A significant left-to-right pattern of monitoring latencies was again observed in the internal speech task. However, in the external speech task, monitoring latencies for the vowel and final consonant of the syllable did not differ. The conclusions of Wheeldon and Levelt are discussed and refined in light of the present results.
## 33                                                                                                                                                                                                                                                                                                                                                                                                                                                                                                                                                                                                                                                                                                                                                                                                                                                                              This study investigates the contribution of grammatical gender to integrating depicted nouns into sentences during on-line comprehension, and whether semantic congruity and gender agreement interact using two tasks: naming and semantic judgement of pictures. Native Spanish speakers comprehended spoken Spanish sentences with an embedded line drawing, which replaced a noun that either made sense or not with the preceding sentence context and either matched or mismatched the gender of the preceding article. In Experiment la (picture naming) slower naming times were found for gender mismatching pictures than matches, as well as for semantically incongruous pictures than congruous ones. In addition, the effects of gender agreement and semantic congruity interacted; specifically, pictures that were both semantically incongruous and gender mismatching were named slowest, but not as slow as if adding independent delays from both violations. Compared with a neutral baseline, with pictures embedded in simple command sentences like Now please say _, both facilitative and inhibitory effects were observed. Experiment 1b replicated these results with low-cloze gender-neutral sentences, more similar in structure and processing demands to the experimental sentences. In Experiment 2, participants judged a picture's semantic fit within a sentence by button-press; gender agreement and semantic congruity again interacted, with gender agreement having an effect on congruous but not incongruous pictures. Two distinct effects of gender are hypothesised: a global predictive effect (observed with and without overt noun production), and a local inhibitory effect (observed only with production of gender-discordant nouns).
## 34                                                                                                                                                                                                                                                                                                                                                                                                                                                                                                                                                                                                                                                                                                                                                                                                                                                                                                                                                                                                                                                                                                                                                                                                                                                                                                                                          This research uses the differential frequency effect as a diagnostic tool to investigate the mental representation of disyllabic compound words in Mandarin Chinese. In three experiments, subjects made lexical decision responses to spoken disyllabic words and nonwords. In Experiment 1, word frequency, morpheme frequency and syllable frequency were covaried, with either the first or second constituent of the compound held constant. Only word-frequency effects were found for real words, although responses were slower to nonwords with high-frequency initial syllables. The results for real words were replicated in Experiment 2, where syllable and morpheme frequency were varied for pairs of words sharing common morphemes in first or second position. Experiment 3, however, showed that when both word frequency and morpheme frequency were held constant, high-frequency first syllables slowed responses to real words. Experiment 3 also verified that syllable frequency effects for nonwords cannot be reliably obtained for second consituent contrasts. These effects were attributed to competition between homophonic morphemes. The overall results were interpreted in terms of a multi-level cluster model, with separate syllabic, morphemic and whole-word levels of representation.
## 35                                                                                                                                                                                                                                                                                                                                                                                                                                                                                                                                                                                                                                                                                                                                                                                                                                                                                                                                                                                                                                                                                                                                                                                                                                                                                                                                                                                                                                                        Two visual-world experiments investigated whether and how quickly discourse-based expectations about the prosodic realisation of spoken words modulate interpretation of acoustic-prosodic cues. Experiment 1 replicated the effects of segmental lengthening on activation of onset-embedded words (e.g. pumpkin) using resynthetic manipulation of duration and fundamental frequency (F0). In Experiment 2, the same materials were preceded by instructions establishing information-structural differences between competing lexical alternatives (i.e. repeated vs. newly assigned thematic roles) in critical instructions. Eye movements generated upon hearing the critical target word revealed a significant interaction between information structure and target-word realisation: Segmental lengthening and pitch excursion elicited more fixations to the onset-embedded competitor when the target word remained in the same thematic role, but not when its thematic role changed. These results suggest that information structure modulates the interpretation of acoustic-prosodic cues by influencing expectations about fine-grained acoustic-phonetic properties of the unfolding utterance.
## 36                                                                                                                                                                                                                                                                                                                                                                                                                                                                                                                                                                                                                                                                                                                                                                                                                                                                                                                                                                                                                                                                                                                                                                                                                                                                                                                                                                                                                                                                                                                                                Syllable frequency effects in production tasks are interpreted as evidence that speakers retrieve precompiled articulatory programs for high frequency syllables from a mental syllabary. They have not been found reliably in English, nor isolated to the phonetic encoding processes during which the syllabary is thought to be accessed. In this experiment, 48 participants produced matched high- and novel/low-frequency syllables in a near-replication of Laganaro and Alario's [(2006) On the locus of the syllable frequency effect in speech production. Journal of Memory and Language, 55(2), 198-196, http://dx.doi.org/10.1016/j.jml.2006.05.001] production conditions: immediate naming, naming following an unfilled delay, and naming after delay filled by concurrent articulation. Immediate naming was faster for high frequency syllables, demonstrating a robust syllable frequency effect in English. There was no high frequency advantage in either delayed naming condition, leaving open the question of whether syllable frequency effects arise during phonological or phonetic encoding.
## 37                                                                                                                                                                                                                                                                                                                                                                                                                                                                                                                                                                                                                                                                                                                                                                                                                                                                                                                                                                                                                                                                                                                                                                                                                                                                                                                                                                                                                                                                                                                                                              Most languages use spoken arbitrary symbols to access the conceptual system. Moreover, the link from spoken words to meaning is demonstrably automatic. Sign languages, by contrast, employ many iconic manual gestures. While some signs are arbitrary it is unclear whether such arbitrary signs automatically activate the conceptual system. To address this question, we examine the propensity of arbitrary colour signs in American Sign Language (ASL) to induce Stroop interference. Three experiments elicited colour naming of coloured videos depicting colour ASL signs - either congruent or incongruent with video colour - and an unrelated condition. Results showed that colour identification is modulated by its congruency with the ASL sign, and this finding replicated irrespective of response mode - signing vs. button-press - and the presence of congruent trials. These findings indicate that arbitrary signs automatically activate their meanings. We conclude that the capacity to link arbitrary phonological forms and meanings is an amodal design feature of language.
## 38                                                                                                                                                                                                                                                                                                                                                                                                                                                                                                                                                                                                                                                                                                                                                                                                                                                                                                                                                                                                                                                                                                                                                                                                                                                                                                                                                                                                                                                                                             The ability to predict upcoming actions is a hallmark of cognition. It remains unclear, however, whether the predictive behaviour observed in controlled lab environments generalises to rich, everyday settings. In four virtual reality experiments, we tested whether a well-established marker of linguistic prediction (anticipatory eye movements) replicated when increasing the naturalness of the paradigm by means of immersing participants in naturalistic scenes (Experiment 1), increasing the number of distractor objects (Experiment 2), modifying the proportion of predictable noun-referents (Experiment 3), and manipulating the location of referents relative to the joint attentional space (Experiment 4). Robust anticipatory eye movements were observed for Experiments 1-3. The anticipatory effect disappeared, however, in Experiment 4. Our findings suggest that predictive processing occurs in everyday communication if the referents are situated in the joint attentional space. Methodologically, our study confirms that ecological validity and experimental control may go hand-in-hand in the study of human predictive behaviour.
## 39                                                                                                                                                                                                                                                                                                                                                                                                                                                                                                                                                                                                                                                                                                                                                                                                                                                                                                                                                                                                                                                                                                                                                                                                                                                                                                                                                                                                                                                                                                                 Although Vietnamese has a long history of linguistic research, as yet no psycholinguistic studies addressing lexical processing in this language have been carried out. This paper is the first to investigate lexical processing in Vietnamese, and this addresses the reading of Vietnamese bi-syllabic compound words. A large single-subject experiment with 20,000 words was complemented by a smaller multiple-subject experiment with 550 words. We report the novel finding of an inhibitory, anti-frequency effect of Vietnamese compounds' constituents. We show that this anti-frequency effect is predicted by a computational model of lexical processing grounded in naive discrimination learning. We also show that predictors derived from this model provide a much better fit to the observed reaction times than traditional lexical-distributional predictors. Effects of the density of the compound graph, previously observed for English, were replicated for Vietnamese. Furthermore, tone diacritics were found to be important predictors of silent reading, providing further evidence for the role of phonology in reading.
## 40                                                                                                                                                                                                                                                                                                                                                                                                                                                                                                                                                                                                                                                                                                                                                                                 Previous work has shown that the difficulty associated with processing complex semantic expressions is reduced when the critical constituents appear in separate clauses as opposed to when they appear together in the same clause. We investigated this effect further, focusing in particular on complement coercion, in which an event-selecting verb (e.g. began) combines with a complement that represents an entity (e.g. began the memo). Experiment 1 compared reading times for coercion versus control expressions when the critical verb and complement appeared together in a subject-extracted relative clause (SRC) (e.g. The secretary that began/wrote the memo) compared to when they appeared together in a simple sentence. Readers spent more time processing coercion expressions than control expressions, replicating the typical coercion cost. In addition, readers spent less time processing the verb and complement in SRCs than in simple sentences; however, the magnitude of the coercion cost did not depend on sentence structure. In contrast, Experiment 2 showed that the coercion cost was reduced when the complement appeared as the head of an object-extracted relative clause (ORC) (e.g. The memo that the secretary began/wrote) compared to when the constituents appeared together in an SRC. Consistent with the eye-tracking results of Experiment 2, a corpus analysis showed that expressions requiring complement coercion are more frequent when the constituents are separated by the clause boundary of an ORC compared to when they are embedded together within an SRC. The results provide important information about the types of structural configurations that contribute to reduced difficulty with complex semantic expressions, as well as how these processing patterns are reflected in naturally occurring language.
## 41                                                                                                                                                                                                                                                                                                                                                                                                                                                                                                                                                                                                                                                                                                                                                                                                                                                                                                                                                                                                                                                                                                                                                                                                                                                                                                                                                                                                                                                                                                                                                                                Previous research suggests that listeners can use the presence of speech disfluencies to predict upcoming linguistic input. But how is the processing of typical disfluencies affected when the speaker also produces atypical disfluencies, as in the case of stuttering? We addressed this question in a visual-world eye-tracking experiment in which participants heard self-repair disfluencies while viewing displays that contained a predictable target entity. Half the participants heard the sentences spoken by a speaker who stuttered, and half heard the sentences spoken by the same speaker who produced the sentences without stuttering. Results replicated previous work in demonstrating that listeners engage in robust predictive processing when hearing self-repair disfluencies. Crucially, the magnitude of the prediction effect was reduced when the speaker stuttered compared to when the speaker did not stutter. Overall, the results suggest that listeners' ability to model the production system of a speaker is disrupted when the speaker stutters.
## 42                                                                                                                                                                                                                                                                                                                                                                                                                                                                                                                                                                                                                                                                                                                                                                                                                                                                                                                                                                                                                                                                                                                                                                                                                                                                                                                                                                                                                                                                                                                                                                                                                                                                                          We present a series of computational simulations of the auditory lexical decision task using the jTRACE and TISK models of spoken word recognition. Simulation 1 replicates high accuracy in word recognition and similar performance of these models using the small, default dictionary. Simulation 2 expands the set of words and phonemes, leading to issues in representing certain phonemes in jTRACE. Simulation 3 expands the lexicon of competitors and we find that TISK struggles to select the target word as the winner. Finally, Simulation 4 shows that the decision criteria employed leads to many false positives when pseudowords are presented to the model. None of the model estimates of the time cycle when the winner should be selected predicted participant response latency in the auditory lexical decision task. We discuss these findings and offer suggestions as to what a contemporary model of spoken word recognition should be able to do.
## 43                                                                                                                                                                                                                                                                                                                                                                                                                                                                                                                                                                                                                                                                                                                                                                                                                                                                                                                                                                                                                                                                                                                                                                                                                                                                                                                                                                                                                                                                                                                                                                                   In a recent study using a masked priming lexical decision task, New, Araujo, and Nazzi found priming of targets by primes sharing consonants (jalu-JOLI) but not by primes sharing vowels (vobi-JOLI). To examine the orthographic or phonological/lexical nature of this consonantal bias, and to determine whether vocalic priming can be obtained under different presentation conditions, we manipulated the duration of prime presentation. Experiment 1 examined masked priming effects for consonant-or vowel-related primes with 33 ms prime durations, while Experiments 2 and 3 examined masked priming effects with longer prime processing times (66 ms or 50 + 16 ms). Results replicated the relative advantage of consonant over vocalic priming, and established that the consonant bias was not at the orthographic level but rather at the phonological and lexical levels. Furthermore, primes sharing vowels with the target showed inhibition for longer processing times. The implications of these findings for models of visual word recognition are discussed.
## 44                                                                                                                                                                                                                                                                                                                                                                                                                                                                                                                                                                                                                                                                                                                                                                                                                                                                                                                                                                                                                                                                                                                                                                                                                                                    Recent work has shown that potentially reciprocal verbs are interpreted reciprocally when their subjects are complex reference objects (plural referents that incorporate multiple referents), but transitively when their subjects are undifferentiated plural sets. Four self-paced reading experiments investigated why this is the case. The experiments contrasted the hypothesis that the parser's behaviour is driven by the complexity or specificity of potential event structures with the hypothesis that what matters is simply the availability of multiple referents. First we replicated Patson & Ferreira's Experiment 1 with self-paced reading, and then in three studies we manipulated the conjuncts within a conjoined noun phrase (NP) subject (e. g. the men and the women vs. the man and the women) in order to vary event structures. The results indicated that potentially reciprocal verbs were interpreted as reciprocal with any conjoined subject, regardless of what type of NPs comprised it. This suggests that the parser is sensitive specifically to the presence of referents, not to the relative complexity or specificity of the event structure that could be built based on those referents. These findings are consistent with Patson and Ferreira's hypothesis that available referents matter because they immediately saturate the thematic roles of reciprocal verbs.
## 45                                                                                                                                                                                                                                                                                                                                                                                                                                                                                                                                                                                                                                                                                                                                                                                                                                                                                                                                                                                                                                                                                                                                                                                                                                                                                                                                                                                                                                                                                             According to dual-route models of morphological processing, regular inflections can be retrieved as whole-word forms or decomposed into morphemes. Baayen, Dijkstra, and Schreuder [(1997). Singulars and plurals in Dutch: Evidence for a parallel dual-route model. Journal of Memory and Language, 37, 94-117. doi:10.1006/jmla.1997.2509] proposed a dual-route model in which singular-dominant plurals (brides) are decomposed, while plural-dominant plurals (peas) are accessed as whole-word units. We report two lexical-decision experiments investigating how plural processing is influenced by participants' age and morphological complexity of the language (German/Dutch). For all Dutch participants and older German participants, we replicated the interaction between number and dominance reported by Baayen and colleagues. Younger German participants showed a main effect of number, indicating decomposition of all plurals. Access to stored forms seems to depend on morphological richness and experience with word forms. The data pattern fits neither full-decomposition nor full-storage models, but is compatible with dual-route models.
## 46                                                                                                                                                                                                                                                                                                                                                                                                                                                                                                                                                                                                                                                                                                                                                                                                                                                                                                                                                                                                                                                                                                                                                                                                                In the picture-word interference (PWI) task, semantically related distractors slow production, while translation-equivalent distractors speed it, possibly implying a language-specific bilingual production system [Costa, A., Miozzo, M., & Caramazza, A. (1999). Lexical selection in bilinguals: Do words in the bilingual's two lexicons compete for selection? Journal of Memory and Language, 41(3), 365-397. https://doi.org/10.1006/jmla.1999.2651]. However, in most previous PWI studies bilinguals responded in just one language, an artificial task restriction. We investigated translation facilitation effects in PWI with language switching. Spanish-English bilinguals named pictures in single- or mixed-language-response blocks, with superimposed distractors in the target language (Experiment 1), or in the non-target language (Experiment 2). Both experiments replicated previously reported translation facilitation effects in both single-language and mixed-language-response blocks. However, language dominance was reversed in mixed-language response blocks, implying inhibition of the dominant language and competition between languages. These results may be explained by a language non-specific selection model in which bilinguals do not restrict selection to one language, with translation facilitation being caused by facilitation at the semantic level offsetting competition at the lexical level.
## 47                                                                                                                                                                                                                                                                                                                                                                                                                                                                                                                                                                                                                                                                                                                                                                                                                                                                                                                                                                                                                                                                                                                                                                                                                                                                                                                                                                                                                                                                                                                                                                                   This study examines the role of the decision stage in the recognition of reduced pronunciation variants, specifically variants produced with nasal flaps (e.g. counter produced as couner). We investigate whether the previously reported disadvantage for nasally flapped variants is located at an early perceptual processing stage or at a post-perceptual decision stage. Experiment 1 replicates the variant effect with a lexical decision task. Experiment 2 uses the psychological refractory period paradigm, suggesting that the variant effect is located at a stage that requires central processing resources. Experiment 3 employed the shadowing task, indicating a significantly smaller variant effect compared to the lexical decision task. In concert, our results suggest that the disadvantage for nasally flapped variants has at least in part a decisional locus, which challenges representational and perceptual accounts of the effect. A mechanism is discussed that describes how decisional factors could cause the disadvantage for reduced variants.
## 48                                                                                                                                                                                                                                                                                                                                                                                                                                                                                                                                                                                                                                                                                                                                                                                                                                                                                                                                                                                                                                                                                                                                                                                                                                                                                                                      Most psycholinguistic models of speech production agree on an earlier semantic processing stage and a later word-form encoding stage. Using a logographic language, Mandarin Chinese, Zhang and Weekes [2009. Orthographic facilitation effects on spoken word production: Evidence from Chinese.Language and Cognitive Processes,24(7-8), 1082-1096.] reported an early effect of orthography in a picture-word-interference study and suggested orthography affects speech production via a lexical-semantic pathway at an early stage. This early orthographic effect without co-occurrence of phonological effect, however, was not replicated [Zhao, La Heij, & Schiller,2012. Orthographic and phonological facilitation in speech production: New evidence from picture naming in Chinese.Acta Psychologica,139(2), 272-280.]. The present study aimed to dissociate further the semantic and phonological representations from orthography by using simplex Chinese characters. The results of Experiment 1 and 2 revealed an orthographic effect but only at a similar point in time as the phonological effect, both of which followed the semantic effect. Our results thus raise further doubts about the role of orthography at the conceptual level of speech planning and lend new evidence to a two-step model of speech production.
## 49                                                                                                                                                                                                                                                                                                                                                                                                                                                                                                                                                                                                                                                                                                                                                                                                                                                                                                                                                                                                                                                                                                                                                                                                                                                                                                                                                                                                                                                                                                                                                                                                                                                   While bilinguals differ from monolinguals in brain structure and function, the extent to which these differences impact non-linguistic cognitive abilities remains debated. The current set of experiments was motivated by the view that all language experiences don't impact executive attention equally. Driven by contemporary hypotheses on the neurocognitive basis of bilingual language control, four characteristics of bilingualism (patterns of language use, similarity, proficiency, and age of acquisition) were related to performance on the Attentional Blink (AB) task. While not replicated in Experiment 2, results from Experiment 1 and a combined group analysis demonstrated that more balanced first- and second-language use and more distantly related languages were predictive of smaller ABs. In follow-up analyses, smaller ABs were associated with better Simon task performance across both experiments. These findings highlight the utility of an individual differences approach.
## 50                                                                                                                                                                                                                                                                                                                                                                                                                                                                                                                                                                                                                                                                                                                                                                                                                                                                                                                                                                                                                                                                                                                                                                                                                                                                                                                                                                                                                                                                                                                                                                                                                                            Prior research using the boundary paradigm suggests that Chinese readers only process word n+2 in the parafovea when word n+1 is a single character, high-frequency word. We attempted to replicate these findings (Experiment 1), and investigated whether greater n+2 preview effects are observed when word n+1 and n+2 form an idiom rather than a phrase (Experiment 2). Experiment 1 replicated prior findings, although additional analyses of word n+1 and n+2 as a single region revealed significant preview effects regardless of word n+1 frequency. In Experiment 2 there was a main effect of phrase type, such that idioms were read more quickly than phrases, and significant n+2 preview effects. There was no interaction between these variables, suggesting that idioms are not parafoveally processed to a greater extent than phrases. These results suggest that n+2 preview effects in Chinese occur under several circumstances. Factors influencing the observation of these effects are discussed.
## 51                                                                                                                                                                                                                                                                                                                                                                                                                                                                                                                                                                                                                                                                                                                                                                                                                                                                                                                                                                                                                                                                                                                                                                         The aim of this study is to investigate the role of morphophonology and written representations in speech processing. Results are presented from an immediate serial recall (ISR) study, designed to determine the relative effects of L1 morphology and orthography on the recall of consonants versus vowels. Forty-five speakers of English, Amharic, and Arabic completed an ISR experiment testing the differential recall of these two segment types. English speakers remembered sequences of syllables in which the consonant is held constant and the vowel changes (e.g., ma mi mu.) better than sequences in which the vowel is held constant and the consonant changes (e.g., ka ma za.), whereas Arabic speakers remembered both types of sequences equally well, replicating findings from Kissling (2012). Crucially, Amharic speakers in this study performed similarly to Arabic speakers, remembering both sequence types with equal accuracy. Given that Amharic and Arabic share a templatic morphological system, this new result suggests that the morphophonology of a listener's L1 impacts ISR. English and Amharic share similar orthographic systems; the mean accuracies of English and Amharic speakers were significantly different, and it therefore appears that orthography of the L1 does not affect recall accuracy. The results have implications for the role of the morphophonology of a given speaker's L1 in ISR and in speech processing more generally.
## 52                                                                                                                                                                                                                                                                                                                                                                                                                                                                                                                                                                                                                                                                                                                                                                                                      A prominent pitch accent is known to trigger immediate contrastive interpretation of the accented referential expression. Previous experimental demonstrations of this effect, where [L+H* unaccented] contours led to an increase in earlier responses than [H*!H*] contours in -contrastive context, may have benefited from the use of laboratory speech with stylized, -homogenous pitch contours as well as data collected from a uniform participant group-college students. The -present study tested visitors to a science museum, who better represent the - general public, comparing lab and spontaneous speech to replicate the contrast-evoking effect of prominent pitch accent. Across two eye-tracking experiments where participants followed spoken instructions to decorate Christmas trees, spontaneous two-word [L+H* unaccented] contours led to faster eye-movements to contrastive ornament sets than [H*!H*] contours with no delay as compared to lab speech. The differences in the fixation functions were overall smaller than those in a previous study that used clear lab speech in richer contexts. Detailed acoustic analyses indicated that the lab speech tune types were distinguishable by any of several independent F0 measures on the adjective and by F0 slope. In contrast, no single phonetic measure on the spontaneous speech adjective distinguished between tune types, which were best classified according to independent noun-based measures. However, a non-linear c-ombination of the adjective measures was shown to be equal to the noun measures in distinguishing between the [H*!H*] and [L+H* unaccented] tunes. The eye-movement data suggest that naive listeners were comparably sensitive to both lab and spontaneous prosodic cues on the adjective and made anticipatory eye-movements accordingly.
## 53                                                                                                                                                                                                                                                                                                                                                                                                                                                                                                                                                                                                                                                                                                                                                                                                                                                                                                                                                                                                                                                                                                     In tonal languages, the role of intonation in information-structuring has yet to be fully investigated. Intuitively, one would expect intonation to play only a small role in expressing communicative functions. However, experimental studies with Vietnamese native speakers show that intonation contours vary across different contexts and are used to mark certain types of information, for example, focus (Jannedy, 2007). In non-tonal languages (e.g., English), the marking of focus by intonation can influence the processing of focus alternatives (Fraundorf, Watson, & Benjamin, 2010). If Vietnamese also uses intonation to mark focus, the question arises whether the behavioral consequences of prosodic focus marking in Vietnamese are comparable to languages such as English or German. To test this, we replicate a study on memory for focus alternatives, originally carried out in German (Koch & Spalek, in progress), with Vietnamese language stimuli. In the original study, memory for focus alternatives was improved in a delayed recall task for focused elements produced with contrastive intonation in female speakers. Here, we replicate this finding with Northern Vietnamese native speakers: Contrastive intonation seems to improve later recall for focus alternatives in Northern Vietnamese, but only for female participants, in line with the findings by Koch and Spalek (in progress). These results indicate that prosodic focus marking in Vietnamese makes alternatives to the focused element more salient.
## 54                                                                                                                                                                                                                                                                                                                                                                                                                                                                                                                                                                                                                                                                                                                                                                                                                                                                                                                                                                                                                                                                                                                                                                                                                                                                                                                                                                                                                                                                                                                                                                                                                                                                                        A handful of recent experimental reports have shown that infants of 6-9 months know the meanings of some common words. Here, we replicate and extend these findings. With a new set of items, we show that when young infants (age 6-16 months, n = 49) are presented with side-by-side video clips depicting various common early words, and one clip is named in a sentence, they look at the named video at above-chance rates. We demonstrate anew that infants understand common words by 6-9 months and that performance increases substantially around 14 months. The results imply that 6-to 9-month-olds' failure to understand words not referring to objects (verbs, adjectives, performatives) in a similar prior study is not attributable to the use of dynamic video depictions. Thus, 6-to 9-month-olds' experience of spoken language includes some understanding of common words for concrete objects, but relatively impoverished comprehension of other words.
## 55                                                                                                                                                                                                                                                                                                                                                                                                                                                                                                                                                                                                                                                                                                                                                                                                                                                                                                                                                                                                                                                                                                                                                                                                                                                                                                                                                                                                                                                                                                                                                                                                                                                                                                               For adults, no and not change the truth-value of sentences they compose with. To investigate children's emerging understanding of these words, an experimenter hid a ball in a bucket or a truck, then gave an affirmative or negative clue (Experiment 1: It's not in the bucket; Experiment 2: Is it in the bucket?; No, it's not). Replicating Austin, Theakston, Lieven, & Tomasello (2014), children only understood logical no and not after age two, long after they say no but around the time they say not and use both words to deny statements. To investigate whether this simply reflects improving inhibitory control, in Experiment 3 we showed children that one container did or did not hold the ball. Twenty-month-olds now succeeded. We discuss two possible factors limiting learning both no and not-a purely linguistic difficulty learning the labels, and the possibility that negation is unavailable to thought before age two.
## 56                                                                                                                                                                                                                                                                                                                                                                                                                                                                                                                                                                                                                                                                                                                                                                                                                                                                                                                                                                                                                                                                                                                                                                                                                                                                                                                                                                         Beginning with the classic work of Shepard, Hovland, & Jenkins (1961), Type II visual patterns (e.g., exemplars are large white squares OR small black triangles) have held a special place in investigations of human learning. Recent research on Type II linguistic patterns has shown that they are relatively frequent across languages and more frequent than Type IV family resemblance patterns (e.g., exemplars have 2 out of 3 defining features). Research has also shown that human infants are adept at learning Type II patterns from very few exemplars, but adult learning appears to be more mixed. Because no study had directly compared adults and infants, Experiment 1 tested both groups on the same input and test stimuli. Adults at best showed weak learning of one of two Type II patterns, but infants showed robust learning of both patterns. Experiment 2 contrasted adults' ability to learn a Type II pattern with a Type IV pattern. Adults only showed learning of the latter, replicating previous research with different stimuli and testing procedures. Thus, adults are unable to learn a frequent linguistic pattern, one readily learned by infants. Implications for possible language learning differences between infants and adults are discussed.
## 57                                                                                                                                                                                                                                                                                                                                                                                                                                                                                                                                                                                                                                                                                                                                                                                                                                                                                                                                                                                                                                                                                                                                                                                                                                                                                                                                                                                                                                                                                                        Linguistic contexts provide useful information about verb meanings by narrowing the space of candidate concepts. Intuitively, the more information, the better. For example, ?the tall girl is fezzing,? as compared to ?the girl is fezzing,? provides more information about which event, out of multiple candidate events, is being labeled; thus, we may expect it to better facilitate verb learning. However, we find evidence to the contrary: in a verb learning study, preschoolers (N?=?60, mean age?=?38?months) only performed above chance when the subject was an unmodified determiner phase, but not when it was modified (Experiment 1). Experiment 2 replicated this pattern with a different set of stimuli and a wider age range (N?=?60, mean age?=?45?months). Further, in Experiment 2, we looked at both learning outcomes?by evaluating pointing responses at Test, and also the learning process?by tracking eye gaze during Familiarization. The results suggest that children?s limited processing abilities are to blame for poor learning outcomes, but that a nuanced understanding of how processing affects learning is required.
## 58                                                                                                                                                                                                                                                                                                                                                                                                                                                                                                                                                                                                                                                                                                                                                                                                                                                                                                                                                                                                                                                                                                                                                                                                                                                                                                                                                                                                                                                                                                                                Adults find negative sentences difficult to process, but an informative context can facilitate processing substantially, suggesting that much of this difficulty may come from the pragmatics of negation. Are children sensitive to the pragmatics of negation as well? Although children perform poorly on many tests of negation comprehension, we argue that these past findings are due to children's sensitivity to general pragmatic principles that govern communication, rather than the conceptual difficulty of negation. In Experiment 1, replicating previous work, we found that adults rated negative sentences as more felicitous in more informative contexts. In Experiment 2, we showed that children are also sensitive to the contexts of true negative sentences, with three- and four-year-olds also rating true negative sentences higher in more informative contexts. We discuss children's understanding of negation and pragmatics in light of these results, arguing that the felicity of negative sentences for both adults and children is determined by the informativeness of these sentences in context.
## 59                                                                                                                                                                                                                                                                                                                                                                                                                                                                                                                                                                                                                                                                                                                                                                                                                                                                                                                                                                                                                                                                                                                                                                                                                                                                                                                                                                                                                                                                                                                                                                                                                                                      Three experiments explored how well children recognize events from different types of visual experience: either by directly seeing an event or by indirectly experiencing it from post-event visual evidence. In Experiment 1, 4- and 5- to 6-year-old Turkish-speaking children (n = 32) successfully recognized events through either direct or indirect visual access. In Experiment 2, a new group of 4- and 5- to 6-year-olds (n = 37) reliably attributed event recognition to others who had direct or indirect visual access to events (even though performance was lower than Experiment 1). In both experiments, although children's accuracy improved with age, there was no difference between the two types of access. Experiment 3 replicated the findings from the youngest participants of Experiments 1 and 2 with a matched sample of English-speaking 4-year-olds (n = 37). Thus children can use different kinds of visual experience to support event representations in themselves and others.
## 60                                                                                                                                                                                                                                                                                                                                                                                                                                                                                                                                                                                                                                                                                                                                                                                                                                                                                                                                                                                                                                                                                                                                                                                                                                                                                                         Failing to communicate a message in everyday settings can be a frustrating experience. However, miscommunication can lead to disaster in high-stakes situations. Yet in these contexts, under pressure to perform efficiently, speakers may also find themselves with limited resources to devote to message clarity. To understand how cognitive constraint affects communication and explore a possible low-cost solution, we investigated a method for moderating ambiguity production in the face of competing attentional demands: taking the perspective of the listener. Over two experiments, speakers labeled images (Experiment 1) or provided instructions (Experiment 2) to listeners in a non-interactive communication task. In both experiments, speakers were randomly assigned to cognitive constraint and perspective-taking conditions, such that some speakers were under higher cognitive constraint and some speakers received a simple perspective-taking directive. We replicated previous findings that additional cognitive constraint impairs speakers' ability to avoid ambiguity. Additionally, we found that a simple directive can promote speaker clarity when labeling images, but not when providing instructions. These results suggest that a simple directive is likely insufficient to ensure speaker clarity in all cases.
## 61                                                                                                                                                                                                                                                                                                                                                                                                                                                                                                                                                                                                                                                                                                                                                                                                                                                                                                                                                                                                                                                                                                                                                                                                                                                                                                                                                                                                                                                                                                                      This paper explores the representations underlying lexical semantics. In particular, we test whether a word's meaning can affect a word's articulation. In Experiment 1, participants produced high-effort (e.g., yelling) and low-effort (e.g., chatting) words that are semantically related to articulation, as well as words that are semantically unrelated to articulation (e.g., kicking). We found that vocal words were produced with greater intensity than non-vocal words. In Experiment 2, we explored the specificity of this effect by investigating how words semantically related to the mouth, but unrelated to vocalization (e.g., chewing) were articulated. Analyses revealed that mouth words did not differ from controls, and we replicated the vocal effects from Experiment 1, suggesting fine-grain motor activation from lexical semantics. Experiment 3 revealed that the semantics of a verb influences the prosodic intensity of a sentence prior to the onset of the verb. Together, these data suggest aspects of lexical meaning influence prosody, and that motor representations may underlie lexical semantics.
## 62                                                                                                                                                                                                                                                                                                                                                                                                                                                                                                                                                                                                                                                                                                                                                                                                                                                                                                                                                                                                                                                                                                                                                                                                                                                                                                                                                                                                                                                                                                                                                                                                                                                                                                                     This study investigates the influence of a robot's speech rate. In human communication, slow speech is considered boring, speech at normal speed is perceived as credible, and fast speech is perceived as competent. To seek the appropriate speech rate for robots, we test whether these tendencies are replicated in human-robot interaction by conducting an experiment with four rates of speech: fast, normal, moderately slow, and slow. Our experimental results reveal a rather surprising trend. Participants prefer normal and moderately slow speech to fast speech. A robot that provides normal or moderately slow speech is perceived as competent. We further study how context affects this perception. In a situation where the robot and participants talk while walking, we found that slow speech was the most comprehensible. In addition, slow speech is subjectively perceived as good as moderately slow and normal speech.
## 63                                                                                                                                                                                                                                                                                                                                                                                                                                                                                                                                                                                                                                                                                                                                                                                                                                                                                                                                                                                                                                                                                                                                                                                                                                                                                                                                                                                                                                                                                                                                                                     We introduce a novel paradigm for studying the cognitive processes used by listeners within interactive settings. This paradigm places the talker and the listener in the same physical space, creating opportunities for investigations of attention and comprehension processes taking place during interactive discourse situations. An experiment was conducted to compare results from previous research using videotaped stimuli to those obtained within the live face-to-face task paradigm. A headworn apparatus is used to briefly display LEDs on the talker's face in four locations as the talker communicates with the participant. In addition to the primary task of comprehending speeches, participants make a secondary task light detection response. In the present experiment, the talker gave non-emotionally-expressive speeches that were used in past research with videotaped stimuli. Signal detection analysis was employed to determine which areas of the face received the greatest focus of attention. Results replicate previous findings using videotaped methods.
## 64                                                                                                                                                                                                                                                                                                                                                                                                                                                                                                                                                                                                                                                                                                                                                                                                                                                                                                                                                                                                                                                                                                                                                                                                                                                                                                                                                                                                                                                                                                                                                                 Speakers are often disfluent, for example, saying theee uh candle instead of the candle. Production data show that disfluencies occur more often during references to things that are discourse-new, rather than given. An eyetracking experiment shows that this correlation between disfluency and discourse status affects speech comprehension. Subjects viewed scenes containing four objects, including two cohort competitors (e. g., camel, candle), and followed spoken instructions to move the objects. The first instruction established one cohort as discourse-given; the other was discourse-new. The second instruction was either fluent or disfluent, and referred to either the given or new cohort. Fluent instructions led to more initial fixations on the given cohort object (replicating Dahan et al., 2002). By contrast, disfluent instructions resulted in more fixations on the new cohort. This shows that discourse-new information can be accessible under some circumstances. More generally, it suggests that disfluency affects core language comprehension processes.
## 65                                                                                                                                                                                                                                                                                                                                                                                                                                                                                                                                                                                                                                                                                                                                                                                                                                                                                                                                                                                                                                                                                                                                                                                                               Four priming experiments investigating the functional use of onsets and rimes as identification units in normally and poorly reading children, matched on reading age level, are reported. Experiments 1 (onsets) and 2 (rimes) used monosyllabic words. High- and low-frequency bigram letter clusters were primed. Primes turned out to be more effective when they coincided with the rimes of target words then when they did not. For onsets this was not the case. The effect of priming was stronger in low-frequency letter clusters. For rime units there was a significant prime by rime coincidence interaction, consistent with data presented by Bowey (1990). A differential effect of rime priming was obtained for the ability groups in high-frequency letter clusters. Only in the normal reader control group were response latencies negatively affected by noncoinciding primes in high-frequency rime units. This finding suggests that these subjects probably had better access to letter information in the final part of words. In Experiments 3 (onsets) and 4 (rimes) bisyllabic compound nouns were used. Elements of the second syllable were used as primes. Main group effects were found in both experiments, but the effects obtained with monosyllabic words were not replicated. It is concluded that onset/rime mechanisms primarily operate within the boundaries of monosyllabic words and/or in stressed syllables.
## 66                                                                                                                                                                                                                                                                                                                                                                                                                                                                                                                                                                                                                                                                                                                                                                                                                                                                                                                                                                                                                                                                                                                                                                                                                                                                                                                                                                                                                                                                                                                                            In a multiple-choice spelling recognition rest, 56 university students were more accurate on more regular than irregular words, and on lon er-case than mixed-case words, with the case mixing effect greater for irregular than regular words. In Experiment 2. the same words were presented singly in correct or incorrect spellings and distortion of word shape was achieved by case mixing (32 subjects) or by alternating the size of lower-case letters within a word (32 subjects). The main effects of regularity and distortion were replicated and the effect of distortion was greater for incorrect than correct stimuli, with correctly spelled words suffering a decrement in accuracy of less than 5 percentage points. Case mixing had a greater effect than size mixing on response latencies. In Experiment 3, with comparable rest procedures, case mixing interacted with regularity in the subjects analysis for the multiple choice format, but not the single presentation format. This result indicates that comparisons based on visual configuration may be tin artifact of multiple-choice tests.
## 67                                                                                                                                                                                                                                                                                                                                                                                                                                                                                                                                                                                                                                                                                                                                                                                                                                                                                                                                                                                                                                                                                                                                                                                                                                                                                                                                                                                                                                                                                                                                                                                                                    The Frequency (high vs. low) x Regularity (regular vs. exception) interaction found on naming response times is often taken as evidence for parallel processing of sublexical and lexical systems. Using a Go/No-go naming task, we investigated the effect of nonword versus pseudohomophone foils on sub-lexical processing and the subsequent Frequency x Regularity interaction. We ran two experiments: (1) a Go/No-go naming task with nonword foils (e. g., bint) and (2) a Go/No-go naming task with pseudohomophone foils (e. g., pynt). Experiment 1 replicated the Frequency x Regularity interaction on naming response times supporting the notion of parallel sub-lexical and lexical processing. Experiment 2 eliminated the Frequency x Regularity interaction providing evidence for the modulation of sub-lexical information. These results indicate that using pseudohomophones in the Go/No-go naming task minimized information provided from sub-lexical processing and maximized information provided from the lexical system.
## 68                                                                                                                                                                                                                                                                                                                                                                                                                                                                                                                                                                                                                                                                                                                                                                                                                                                                                                                                                                                                                                                                                                                                                                                                                                                                                                                                                                                                                                                                                                                                                                                                                                                                                                                                                                                                                                                                                                                                                                                     In three experiments, we examined lexical competition effects using the phonological priming paradigm in a shadowing task. Experiments 1A and 1B showed that an inhibitory priming effect occurred when the primes mismatched the targets on the last phoneme (/bagar/-/bagaj/). In contrast, a facilitatory priming effect was observed when the primes mismatched the targets on the medial phoneme (/viraj/-/vilaj/). Experiment 2 replicated these findings with primes presented visually rather than auditorily. The data thus indicate that the position of the mismatching phoneme is a critical factor in determining the competition effect between prime and target words.
## 69                                                                                                                                                                                                                                                                                                                                                                                                                                                                                                                                                                                                                                                                                                                                                                                                                                                                                                                                                                                                                                                                                                                                                                                                                                                                                                                                                                                                                                                                                                                                            Smith (1981) found that concrete English sentences were better recognized than abstract sentences and that this concreteness effect was potent only when the concrete sentence was also affirmative but the effect switched to an opposite end when the concrete sentence was negative. These results were partially replicated in Experiment 1 by using materials from a very different language (i.e., Chinese): concrete-affirmative sentences were better remembered than concrete-negative and abstract sentences, but no reliable difference was found between the latter two types. In Experiment 2 the task was modified by using a visual presentation instead of an oral one as in Experiment 1. Both concrete-affirmative and concrete-negative sentences were better memorized then abstract ones in Experiment 2. The findings in the two experiments are explained by a combination of the dual-coding model and Marschark's (1985) item-specific and relational processing. The differential effects of experience with different language systems on processing verbal materials in memory are also discussed.
## 70                                                                                                                                                                                                                                                                                                                                                                                                                                                                                                                                                                                                                                                                                                                                                                                                                                                                                                                                                                                                                                                                                       Two lexical decision experiments were conducted to further investigate the notion that metaphor comprehension involves the formation of a new association between topic and vehicle. Experiment 1 was essentially a replication and extension of Camac and Glucksberg (1984) demonstrating that known word associate pairs show a significant lexical decision latency advantage over their randomly paired counterparts, while topic/vehicle word pairs drawn from apr metaphors do not. The results of Experiment 1 confirm their initial findings even when printed word frequency of the two pair types is held equivalent (a factor not controlled for in the original Camac and Glucksberg study). This result suggests that preexisting topic/vehicle similarity is not an important factor in metaphor comprehension. Experiment 2 was an attempt to detect the hypothesized shift in attribute salience that results in the formation of a new association between topic and vehicle during metaphor comprehension. In Experiment 2, subjects made lexical decisions on topic/vehicle word pairs that were preceded by a paragraph designed to induce either a metaphorical or literal interpretation. For many subjects, a latency advantage was observed for topic/vehicle pairs preceded by a metaphorical context as compared to a mismatching literal context. This finding suggests that metaphor comprehension is a dynamic process which modifies preexisting topic/vehicle similarity, and that metaphorical interpretations are facilitated by extended context.
## 71                                                                                                                                                                                                                                                                                                                                                                                                                                                                                                                                                                                                                                                                                                                                                                                                                                                                                                                                                                                                                                                                                                                                                                                                                                                                                                                                                                                                                                                                                                                                                                                                                                                                                                                                     Three experiments are reported that examined whether stem complexity plays a role in inflecting polymorphemic words in language production. Experiment 1 showed that preparation effects for words with polymorphemic stems are larger when they are produced among words with constant inflectional structures compared to words with variable inflectional structures and simple stems. This replicates earlier findings for words with monomorphemic stems (Janssen et al., 2002). Experiments 2 and 3 showed that when inflectional structure is held constant, the preparation effects are equally large with simple and compound stems, and with compound and complex adjectival stems. These results indicate that inflectional encoding is blind to the complexity of the stem, which suggests that specific inflectional rather than generic morphological frames guide the generation of inflected forms in speaking words.
## 72                                                                                                                                                                                                                                                                                                                                                                                                                                                                                                                                                                                                                                                                                                                                                                                                                                                                                                                                                                                                                                                                                                                                                                                                                                                                                                                                                                                                                                                                                                                                                                                Two visual-world eyetracking experiments were conducted to investigate whether, how, and when syntactic and semantic constraints are integrated and used to predict properties of subsequent input. Experiment 1 contrasted auditory German constructions such as, The hare-nominative eats . . . (the cabbage-acc) versus The hare-accusative eats . . . (the fox-nom), presented with a picture containing a hare, fox, cabbage, and distractor. We found that the probabilities of the eye movements to the cabbage and fox before the onset of NP2 were modulated by the case-marking of NP1, indicating that the case-marking (syntactic) information and verbs' semantic constraints are integrated rapidly enough to predict the most plausible NP2 in the scene. Using English versions of the same stimuli in active/passive voice (Experiment 2), we replicated the same effect, but at a slightly earlier position in the sentence. We discuss the discrepancies in the two Germanic languages in terms of the ease of integrating information across, or within, constituents.
## 73                                                                                                                                                                                                                                                                                                                                                                                                                                                                                                                                                                                                                                                                                                                                                                                                                                                                                                                                                                                                                                                                                                                                                                                                                                                                                                                                                                                                                                                                                                                                                        The research investigated how comprehenders use verb information during syntactic parsing. Two reading experiments investigated the relationship between verb-specific variables and reading time. These experiments were close replications of prior work; however, two statistical techniques were used, rather than one. These were item-by-item correlations and participant-by-participant regression. In Experiment 1, reading time was measured using a self-paced moving window. In Experiment 2, eye movements were recorded during reading. The results of both experiments showed that the results of two types of statistical analyses support contradictory conclusions. The analyses involving participant-by-participant regression analyses provided no evidence for the early use of verb information in parsing and support syntax-first approaches to parsing. In contrast, the results of item-by-item correlation were consistent with the prior research, supporting the view that verb information can guide initial parsing decisions. Implications for theories of parsing are discussed.
## 74                                                                                                                                                                                                                                                                                                                                                                                                                                                                                                                                                                                                                                                                                                                                                                                                                                                                                                                                                                                                                                                                                                                                                                                                                                                                                               Research into typing patterns has broad applications in both psycholinguistics and biometrics (i.e., improving security of computer access via each user's unique typing patterns). We present a new software package, TypingSuite, which can be used for presenting visual and auditory stimuli, collecting typing data, and summarizing and analyzing the data. TypingSuite is a Java-based software package that is platform-independent and open-source. To validate TypingSuite as a beneficial tool for researchers who are interested in keystroke dynamics, two studies were conducted. First, a behavioural experiment based on single word typing was conducted that replicated two well-known findings in typing research, namely the lexicality and frequency effects. The results confirmed that words are typed faster than pseudowords and that high frequency words are typed faster than low frequency words. Second, in regard to biometrics, it was also shown that typing data from the same user are more similar than data from different users. Because TypingSuite allows its users to easily implement an experiment and to collect and analyze data within a single software package, it holds promise for being a valuable educational and research tool in language-related sciences such as psycholinguistics and natural language processing.
## 75                                                                                                                                                                                                                                                                                                                                                                                                                                                                                                                                                                                                                                                                                                                                                                                                                                                                                                                                                                                                                                                                                                                                                                                                                                                                                                                                                                                                                                                                                                                                                                                                                                                         The goal of the current research was to determine if conceptual metaphors are activated when people read idioms within a text. Participants read passages that included idioms that were consistent (blow your top) or inconsistent (bite his head off) with an underlying conceptual metaphor (ANGER IS HEATED FLUID IN A CONTAINER) followed by target words that were related (heat) or unrelated (lead) to the conceptual metaphor. Reading time (Experiment 1) or lexical decision time (Experiment 2) for the target words were measured. We found no evidence supporting conceptual metaphor activation. Target word reading times were unaffected by whether they were related or unrelated to underlying conceptual metaphors. Lexical decision times were facilitated for related target words in both the consistent and inconsistent idiom conditions. We suggest that the conceptual (target) domain, not a specific underlying conceptual metaphor, facilitates processing of related target words.
## 76                                                                                                                                                                                                                                                                                                                                                                                                                                                                                                                                                                                                                                                                                                                                                                                                                                                                                                                                                                                                                                                                                                                                                                                                                                                                                                                                                                                                                                                                                                                          This study investigated the effect of semantic information on artificial grammar learning (AGL). Recursive grammars of different complexity levels (regular language, mirror language, copy language) were investigated in a series of AGL experiments. In the with-semantics condition, participants acquired semantic information prior to the AGL experiment; in the without-semantics control condition, participants did not receive semantic information. It was hypothesized that semantics would generally facilitate grammar acquisition and that the learning benefit in the with-semantics conditions would increase with increasing grammar complexity. Experiment 1 showed learning effects for all grammars but no performance difference between conditions. Experiment 2 replicated the absence of a semantic benefit for all grammars even though semantic information was more prominent during grammar acquisition as compared to Experiment 1. Thus, we did not find evidence for the idea that semantics facilitates grammar acquisition, which seems to support the view of an independent syntactic processing component.
## 77                                                                                                                                                                                                                                                                                                                                                                                                                                                                                                                                                                                                                                                                                                                                                                                                                                                                                                                                                                                                                                                                                                                                                                                                                                                                                                                                                                                                                                                                                                                                                                         We examine the time-course of semantic structure formation during real-time sentence comprehension. We do this through the lens of aspectual coercion, a semantic combinatorial operation that lacks morpho-syntactic reflections, yet is indispensable for sentence interpretation. We describe two experiments. Experiment 1 replicates the results of a previously published study (Pinango, Zurif, & Jackendoff, Journal of Psycholinguistic Research, 28(4), 395-414 1999) showing that the cost of implementing aspectual coercion is detectable as late as 250 ms after the operation is licensed. Experiment 2 expands the window of observation by revealing that the implementation of aspectual coercion is not detectable immediately upon its being licensed, that is, at the point at which the syntactic representation is assumed to be fully formed. These findings suggest a dissociation in the integration of information, in which semantic composition-even mandatory and automatic semantic composition-takes time to develop after it is syntactically licensed to do so.
## 78                                                                                                                                                                                                                                                                                                                                                                                                                                                                                                                                                                                                                                                                                                                                                                                                                                                                                                                                                                                                                                                                                                                                                                                                                                                                                                                                                                                                                                                                                                                                                                                                                                                                                                                                                                                                                                                                                  This paper proposes the use of the tools of statistical meta-analysis as a method of conflict resolution with respect to experiments in cognitive linguistics. With the help of statistical meta-analysis, the effect size of similar experiments can be compared, a well-founded and robust synthesis of the experimental data can be achieved, and possible causes of any divergence(s) in the outcomes can be revealed. This application of statistical meta-analysis offers a novel method of how diverging evidence can be dealt with. The workability of this idea is exemplified by a case study dealing with a series of experiments conducted as non-exact replications of Thibodeau and Boroditsky (PLoS ONE 6(2):e16782, 2011. https://doi.org/10.1371/journal.pone.0016782).
## 79                                                                                                                                                                                                                                                                                                                                                                                                                                                                                                                                                                                                                                                                                                                                                                                                                                                                                                                                                                                                                                                                                                                                                                                                                                                                                                                                                                                                                                                                                                                                                                                                                                                                                                                                Non-exact replications are regarded as effective tools of problem solving in psycholinguistic research because they lead to more plausible experimental results; however, they are also ineffective tools of problem solving because they trigger cumulative contradictions among different replications of an experiment. This paper intends to resolve this paradox by putting forward a metatheoretical model that clarifies the criteria with the help of which various aspects of the effectiveness of the problem solving process can be differentiated and evaluated. The key point is the reconstruction of the relationship between original experiments and their non-exact replications by introducing the concept of 'experimental complex' and analysing the problem solving strategies that the researchers apply. The workability of the proposed metatheoretical model is illustrated with the help of three case studies.
## 80                                                                                                                                                                                                                                                                                                                                                                                                                                                                                                                                                                                                                                                                                                                                                                                                                                                                                                                                                                                                                                                                                                                                                                                                                                                                                                                                                                                                                                                                                                                           Recent studies about the implicit causality of inter-personal verbs showed a symmetric implicit consequentiality bias for psychological verbs. This symmetry is less clear for action verbs because the verbs assigning the implicit cause to the object argument (e.g. Peter protected John because he was in danger.) tend to assign the implicit consequence to the same argument (e.g. Peter protected John so he was not hurt.). We replicated this result by comparing continuations of inter-personal events followed by a causal connective because or a consequence connective so. Moreover, we found similar results when the consequence connective was replaced by a contrastive connective but. This result was confirmed in a second experiment where we measured the time required to imagine a consistent continuation for a fragment finishing with but s/he .... The results were consistent with a contrastive connective introducing a denial of a consequence of the previous event. The results were consistent with a model suggesting that thematic roles and connectives can predict preferred co-reference relations.
## 81                                                                                                                                                                                                                                                                                                                                                                                                                                                                                                                                                                                                                                                                                                                                                                                                                                                                                                                                                                                                                                                                                                                                                                                                                                                                                                                                                                                                                                                Four experiments investigating processing of closed-class and open-class words in isolation and in sentence contexts are reported. Taft (1990) reported that closed-class words which could not meaningfully stand alone and open-class words which could not meaningfully stand alone incurred longer lexical decision responses than did control words. Taft also reported that closed-class and open-class words which could stand alone meaningfully were not associated with longer lexical decision responses than were control words. Experiments 1 and 2 replicated Taft's effect of ability to stand alone on lexical decision responses to closed-class and open-class words presented in isolation. In Experiments 3 and 4, the same lexical decision targets were presented as part of semantically neutral context sentences in a moving window paradigm. The stand-alone effect was not present in Experiments 3 and 4. The results suggest Taft's conclusion that meaningfulness of a word influences lexical needs revision. An explanation is provided according to which support from message level and syntactic and lexical sources in sentence context influence words' perceived ''meaningfulness.''
## 82                                                                                                                                                                                                                                                                                                                                                                                                                                                                                                                                                                                                                                                                                                                                                                                                                                                                                                                                                                                                                                                                                                                                                                                                                                                                                                                                                                                                                                             Informally speaking, presuppositions are meaning components which are part of the common ground for speakers in a conversation, that is, background information which is taken for granted by interlocutors. The current literature suggests an immediate processing of presuppositions, starting directly on the word triggering the presupposition. In the present paper, we focused on two presupposition triggers in German, the definite determiner the (German der) and the iterative particle again (German wieder). Experiment 1 replicates the immediate effects which were previously observed in a self-paced reading study. Experiment 2 then investigates whether this immediate processing of presuppositions is automatic or capacity-limited by employing the psychological refractory period approach and the locus of slack-logic, which have been successfully employed for this reason in various fields of cognitive psychology. The results argue against automatic processing, but rather suggest that the immediate processing of presuppositions is capacity-limited. This potentially helps specifying the nature of the involved processes; for example, a memory search for a potential referent.
## 83                                                                                                                                                                                                                                                                                                                                                                                                                                                                                                                                                                                                                                                                                                                                                                                                                                                                                                                                                                                                                                                                                                                                                                                                                                                             In languages with flexible constituent order (so-called free word order languages), available orders are used to encode given/new distinctions; they therefore differ not only syntactically, but also in their context requirements. In Experiment 1, using a self-paced reading task, we compared Russian S V IO DO (canonical), DO S V IO and DO IO V S constructions in appropriate vs. inappropriate contexts (those that violated their context requirements). The context factor was significant, while the syntax factor was not. The less pronounced context effect evidenced in previous studies (e. g., Kaiser and Trueswell in Cognitioin 94: 113-147, 2004) might be due to the use of shorter target sentences and less extensive contexts. We also demonstrated that the slow-down starts at the first contextually inappropriate constituent, which shows that the information about context requirements is taken into account immediately, but that it develops faster on preverbal subjects and postverbal indirect objects (occupying their canonical positions) than on preverbal indirect objects (occupying a noncanonical position, or scrambled). In Experiment 2, these findings were replicated for IO S V DO and IO DO V S orders. S V IO DO orders with a continuation were used to show that there is no additional effect of inappropriate context at the end of the sentence.
## 84                                                                                                                                                                                                                                                                                                                                                                                                                                                                                                                                                                                                                                                                                                                                                                                                      Just as the false comma in this sentence, shows punctuation can influence sentence processing considerably. Pauses and other prosodic cues in spoken language serve the same function of structuring the sentence in smaller phrases, However, surprisingly little effort has been spent on the question as to whether both phenomena rest on the same mechanism and whether they are equally efficient in guiding parsing decisions. In a recent study, we showed that auditory speech boundaries evoke a specific positive shift in the listeners' event-related brain potentials (ERPs) that indicates the sentence segmentation and resulting changes in the understanding of the utterance (Steinhauer et al., 1999a). Here, ive present three ERP reading experiments demonstrating that the human brain processes commas in a similar manner and that comma perception depends crucially on the reader's individual punctuation habits. Main results of the study are: (1) Commas can determine initial parsing as efficiently as speech boundaries because they trigger the same prosodic phrasing covertly, although phonological representations seem to be activated to a lesser extent. (2) Independent of the input modality, this phrasing is reflected online by the same ERP component, namely the Closure Positive Shift (CPS). (3) Both behavioral and ERP data suggest that comma processing varies with the readers' idiosyncratic punctuation habits. (4) A combined auditory and visual ERP experiment shows that the CPS is also elicited both by delexicalized prosody and while subjects replicate prosodic boundaries during silent reading. (5) A comma-induced reversed garden path turned out to be much more difficult than the classical garden path. Implications for psycholinguistic models and future ERP research are discussed.
## 85                                                                                                                                                                                                                                                                                                                                                                                                                                                                                                                                                                                                                                                                                                                                                                                                                                                                                                                                                                                                                                                                                                                                                                                                                                                                                                                                                                                                                                                                                                                                                                                                                                                                                                                           We investigated how people produce simple and complex phrases in speaking rising a newly developed immediate recall task. People read and tried to memorize a target sentence, then read a prime sentence, then did a distractor task involving the prime sentence Despite the delay and activity between memory and recall, people could still recall the target sentence although the syntactic form of the recalled sentence was influenced by the syntactic form of the prime sentence. This result replicates the syntactic priming effect found with other experimental paradigms. Using this task, we tested how people used abstract syntactic plans to produce simple and complex noun phrases. We found syntactic priming both when targets and prime sentences matched in complexity and when they did not match, suggesting that simple and complex noun phrases are built by the same syntactic routines during speech production.
## 86                                                                                                                                                                                                                                                                                                                                                                                                                                                                                                                                                                                                                                                                                                                                                                           We investigated biases in the organization of imagery by asking participants to make stick-figure drawings of sentences containing a man, a woman and a transitive action (e.g. she kisses that guy). Previous findings show that prominent features of meaning and sentence structure are placed to the left in drawings, according to reading direction (e.g. Stroustrup and Wallentin in Lang Cogn 10(2):193-207, 2018. 10.1017/langcog.2017.19). Five hundred thirty participants listened to sentences in Danish and made eight drawings each. We replicated three findings: (1) that the first mentioned element is placed to the left more often, (2) that the agent in the sentence is placed to the left, and (3) that the grammatical subject is placed to the left of the object. We further tested hypotheses related to deixis and gender stereotypes. By adding demonstratives (e.g. Danish equivalents of this and that), that have been found to indicate attentional prominence, we tested the hypothesis that this is also translated into a left-ward bias in the produced drawings. We were unable to find support for this hypothesis. Analyses of gender biases tested the presence of a gender identification and a gender stereotype effect. According to the identification hypothesis, participants should attribute prominence to their own gender and draw it to the left, and according to the stereotype effect participants should be more prone to draw the male character to the left, regardless of own gender. We were not able to find significant support for either of the two gender effects. The combination of replications and null-findings suggest that the left-ward bias in the drawing experiment might be narrowly tied to left-to-right distribution in written language and less to overall prominence. No effect of handedness was observed.
## 87                                                                                                                                                                                                                                                                                                                                                                                                                                                                                                                                                                                                                                                                                                                                                                                                                                                                                                                                                                                                                                                                                                                                                                                                                                                                                                                                                                                                                                                                                                                 This paper enamines the question of whether there ave effects of prosody on the syntactic parsing of temporarily ambiguous sentences containing complement verbs. It reports the results of five experiments employing cross-modal response tasks where the visually presented target weld was either an 'appropriate' or an 'inappropriate' continuation in terms of the prosodic form of the preceeding auditory sentence fragment. Two experiments employing cross-modal naming only showed indications of sensitivity to syntactic and appropriateness manipulations when coupled with a simultaneous appropriateness judgment task. In contrast, the experiments employing cross-modal lexical decision showed greater sensitivity to syntactic and appropriateness effects. However, while the results from these studies replicated our earlier auditory parsing results and provided support for the suggestion that there are differences in visual and auditory parsing processes and for a 'constituent-based,' 'minimal commitment' type auditory parser, none of the studies demonstrated an effect of prosodic form on the parsing process.
## 88                                                                                                                                                                                                                                                                                                                                                                                                                                                                                                                                                                                                                                                                                                                                                                                                                                                                                                                                                                                                                                                                                                                                                                                                                                                                                                                                                                                                                                                                                                                                                                                                                                                                                                           This study examined the effects of acoustic variability on second language vocabulary learning. English native speakers learned new words in Spanish. Exposure frequency to the words was constant. Dependent measures were accuracy and latency of picture-to-Spanish and Spanish-to-English recall. Experiment 1 compared presentation formats of neutral (conversational) voice only, three voice types, and six voice types. No significant differences emerged. Experiment 2 compared presentation formats of one speaker, three speakers, and six speakers. Vocabulary learning was superior in the higher-variability conditions. Experiment 3 partially replicated Experiment 1 while rotating voice types across subjects in moderate and no-variability conditions. Vocabulary learning was superior in the higher variability conditions, These results are consistent with an exemplar-based theory of initial lexical learning and representation.
## 89                                                                                                                                                                                                                                                                                                                                                                                                                                                                                                                                                                                                                                                                                                                                                                                                                                                                                                                                                                                                                                                                                                                                                                                                                                                                                                                This study investigates associative learning explanations of the limited attainment of adult compared to child language acquisition in terms of learned attention to cues. It replicates and extends Ellis and Sagarra (2010) in demonstrating short-and long-term learned attention in the acquisition of temporal reference in Latin. In Experiment 1, salient adverbs were better learned than less salient verb inflections, early experience of adverbial cues blocked the acquisition of verbal morphology, and, contrariwise-but to a lesser degree-early experience of tense reduced later learning of adverbs. Experiment 2 demonstrated long-term transfer: Native speakers of Chinese (no tense morphology) were less able than native speakers of Spanish or Russian (rich morphology) to acquire inflectional cues from the same language experience where adverbial and verbal cues were equally available. Learned attention to tense morphology in Latin was continuous rather than discrete, ordered with regard to first language: Chinese < English < Russian < Spanish. A meta-analysis of the combined results of Ellis and Sagarra and the current study separates out positive and negative learned attention effects: The average effect size for entrenchment was large (+ 1.23), whereas that for blocking was moderate (-0.52).
## 90                                                                                                                                                                                                                                                                                                                                                                                                                                                   The role of explicit information (El) as an independent variable in instructed SLA is largely underresearched Using the framework of processing instruction, however, a series of offline studies has found no effect for El (e g, Benati, 2004; Sanz & Morgan-Short, 2004, VanPatten & Oikkenon, 1996) Fernandez (2008) presented two online experiments with mixed results She found an effect for El with processing instruction on one target structure (subjunctive in Spanish) but not the other structures (object pronouns and word order in Spanish), Thus, the effects of El could be related to the target structure or to a processing problem, or both The present study is a conceptual replication of one of Fernandez's experiments The target was German accusative case markings on articles with both subject (S)- verb (V)- object (0) and OVS word orders As shown by Jackson (2007) and LoCoco (1987), learners of German as a second language misinterpret OVS sentences as SVO, ignoring case markings as a cue of who does what to whom Thus. the goal of the instructional intervention was to push learners to process case markings and word order correctly The treatment consisted of structured input items (Farley, 2005, Lee & VanPatten, 2003) under two conditions +/-El Following Fernandez, the treatment was conducted via computer using e-Prime, and learners' responses were recorded as they made their vary through the items Whereas Fernandez did not find an effect for El for word order and object pronouns in Spanish, we found an effect for word order and case markings in German (a) Twice as many learners in the +El group reached criterion (began to process input strings correctly) compared with the -El group, and (b) learners in the +El group began processing word order and case markings sooner than in the -El group Even though the processing problem was the same in both Fernandez's and our experiments, we attribute the difference in results to the interaction of particular structures with the processing problem and call for additional research on the role of El rot just in processing instruction but in all formal interventions.
## 91                                                                                                                                                                                                                                                                                                                                                                                                                                                                                                                                                                                                                                                                                                                                                                                                                                                                                                                                                                                                                                                                                                                                                                                                                                                                                                                                                                                                                                                                                                                                                                                      I investigated the trajectory of processing variability, as measured by coefficient of variation (CV), using an intentional word learning experiment and reanalyzing published eye-tracking data of an incidental word learning study (Elgort et al., 2018). In the word learning experiment, native English speakers (N = 35) studied Swahili-English word pairs (k = 16) before performing 10 blocks of animacy judgment tasks. Results replicated the initial CV increase reported in Solovyeva and DeKeyser (2018) and, importantly, captured a roughly inverted U-shaped development in CV. In the reanalysis of eye-tracking data, I computed CVs based on reading times on the target and control words. Results did not reveal a similar inverted U-shaped development over time but suggested more stable processing of the high-frequency control words. Taken together, these results uncovered a fuller trajectory in CV development, differences in processing demands for different aspects of word knowledge, and the potential use of CV with eye-tracking research.
## 92                                                                                                                                                                                                                                                                                                                                                                                                                                                                                                                                                                                                                                                                                                                                                                                                                                                                                                                                                                                                                                              This paper reports replications of studies of implicit artificial grammar (AG) learning and explicit series-solution learning with experienced second language learners in order to examine their population and content generalizability. As found by Reber, Walkenfeld, and Hernstadt (1991), there was significantly greater variance in explicit compared to implicit learning. In contrast to Reber et al.'s findings, intelligence quotient (IQ) was significantly negatively related to implicit learning. As found by Knowlton and Squire (1996), chunks that appeared with high frequency (high chunk-strength) in AG training influenced incorrect acceptance of ungrammatical transfer test items containing them but did not affect the judgments of grammatical items. In a third experiment, learners semantically processed sentences in Samoan, a novel language for this population. This experiment found little evidence for the content generalizability of these AG findings to the incidental learning of Samoan. Implicit AG and incidental Samoan learning had different patterns of correlation with cognitive abilities (10, working memory, and aptitude) and differed in sensitivity to chunk-strength. As found for AG learning, high chunk-strength negatively affected correct rejection of ungrammatical Samoan transfer test items. Additionally, high chunk-strength negatively affected correct acceptance of grammatical items. For these grammatical items, the number of chunks they contained - not their frequency during training - positively influenced grammaticality judgments.
## 93                                                                                                                                                                                                                                                                                                                                                                                                                                                                                                                                                                                                                                                                                                                                                                                                                                                                                                                                                                                                                                                                                                                                                                                                                                                                                                                                                                                                                                                                                                                                                        This paper reports on the reanalysis of Suzuki's (2017) experiment and investigated the extent to which learning schedules influence automatization of second language (L2) morphology. Sixty participants were separated into two groups, which studied morphological rules for oral production under short-spacing (3.3-day intervals) and long-spacing learning conditions (7-day intervals). Their oral production test performance resulted in two measures of automatization: reaction time (RD as an index of speedup and coefficient of variance (CV) as an index of stability/restructuring. The results showed that, while RT of both groups declined significantly after the training, the 3.3-day group exhibited greater propensity for restructuring than the 7-day group. Furthermore, procedural learning ability measured by the Tower of London task was significantly associated with RT, but not with CV, in the 3.3-day group only. These findings suggest that learning schedules and procedural learning ability influence different stages of automatization of L2 morphological learning.
## 94                                                                                                                                                                                                                                                                                                                                                                                                                                                                                                                                                                                                                                                                                                                                                                                                                                                                         While previous research has shown that processing instruction (PI) can more effectively facilitate the acquisition of target structures than traditional drill practice, the processing mechanism of PI has not been adequately examined because most assessment tasks have been offline. Using eye-tracking, this two-experiment study compared changes in processing patterns between two types of training: PI and traditional instruction (TI) on intermediate-level L2 learners' acquisition of the French causative. Both experiments used a pretraining/posttraining design involving a dichotomous scene selection eye-tracking task to measure eye-movement patterns and accuracy in picture selection while participants processed auditory sentences. Neither group received explicit information (EI) in Experiment 1 while both experimental groups in Experiment 2 received EI before processing sentences. Results of Experiment 1 revealed the PI group had significantly higher accuracy scores than the TI group. A change in eye-movement pattern was also observed after training for the PI group but not for the TI group. In Experiment 2, when both groups received EI, PI subjects were again significantly more accurate than TI subjects, but both groups' accuracy scores were not reliably higher than subjects in the PI and TI groups in Experiment 1 who did not receive EI. Eye-movement patterns in Experiment 2 showed that both TI and PI started to shift their gaze to the correct picture at the same point as PI subjects did in Experiment 1. This suggests that EI helped the TI group start entertaining the correct picture as a possible response sooner but the EI did not help the PI group process the target structure sooner than the TI group.
## 95                                                                                                                                                                                                                                                                                                                                                                                                                                                                                                                                                                                                                                                                                                                                                                                                                                                                                                                                                                                                                                                                                                                                                                                                                                                                              Class inclusion theory asserts that one cannot reverse the topic and vehicle of a metaphor and produce a new, meaningful metaphor that is based on the same interpretive ground. In 2 experiments we test that claim. In Study I we replicate the procedures employed by Glucksberg, McGlone, & Manfredi (1997) that provided support for the assertion. However we now add experimental conditions in which the target metaphors, either with the topic and vehicle in its canonical order or reversed, are placed in discourse contexts that provide support for a meaningful interpretation based on the same ground. In contrast to the prediction of class inclusion theory, fully 72% of the cases the reversed metaphors were rated as interpretable and interpretation was based on the same ground used in interpreting the metaphors in their canonical order. In Study 2, the online processing of the metaphors in context are examined in a word-by-word reading task. We find that canonical and reversed order metaphors were read at the same rate throughout and both sets exhibited the same reading patterns: increased reading time of the noun-phrase (NP) that contains the metaphoric vehicle and of the first word in the text that follows the metaphor. We take these data to indicate that nonreversibility cannot be taken as a necessary condition of metaphor.
## 96                                                                                                                                                                                                                                                                                                                                                                                                                                                                                                                                                                                                                                                                                                                                                                                                                                                                                                                                                                                                                                                                                                                                                                                                                                                                                                                                                                                                                                                                                                                                                                               People regularly encounter metaphors in a variety of different communicative settings, but most studies of metaphor framing have relied exclusively on written materials. Across three experiments (N= 2399), we examined the relative power of metaphor framing in different communication formats. Participants read, heard, or watched someone report a series of metaphorically framed issues. They answered a target question about each issue by selecting between two response options, one of which was conceptually congruent with the metaphor frame. Results revealed a similarly-sized metaphor framing effect in each communication modality. Neither speaker gender nor race reliably moderated the effects of metaphor framing for audiovisual messages, though framing effects were stronger when the gender of the speaker and observer matched. We also replicated the finding that metaphors are more effective when they are extended into the response option language. These results provide new insights into the efficacy and generalizability of metaphor framing.
## 97                                                                                                                                                                                                                                                                                                                                                                                                                                                                                                                                                                                                                                                                                                                                                                                                                                                                                                                                                                                                                                                                                                                                                                                                                                                                                                                                                                                                                                                                             While we often talk about time using spatial terms, experimental investigation of space-time associations has focused primarily on the space in front of the participant. This has had two consequences: the disregard of the space behind the participant (exploited in language and gesture) and the creation of potential task demands produced by spatialized manual button-presses. We introduce and test a new paradigm that uses auditory stimuli and vocal responses to address these issues. Participants made temporal judgments about deictic or sequential relationships presented auditorily along a body-centered sagittal or transversal axis. Results involving the transversal axis replicated previous work while sagittal axis results were surprising. Deictic judgments did not use the sagittal axis but sequential judgments did, in a previously undocumented way. Participants associated earlier judgments with the space in front of them and later judgments with the space behind them. These findings, using a new approach, provide evidence that different time concepts recruit space differently, mediated by meaning, stimulus modality and response mode.
## 98                                                                                                                                                                                                                                                                                                                                                                                                                                                                                                                                                                                                                                                                                                                                                                                                                                                                                                                                                                                                                                                                                                                                                                                                                                                                            This paper revisits a study by Machery et al. (2004), suggesting that, in experimental versions of Kripke's (1980) fictional cases on the use of proper names, Westerners are more likely than East Asian participants to show intuitions compatible with Kripke's causal-historical (CH) theory of reference. We conducted two experiments, recruting participants from Norway and Bangladesh, either in English (experiment 1; N = 75) or in the participants' native languages (experiment 2; N = 60), using modified cases and a new approach to data analysis. We replicated the results of Machery et al. (2004), but we show that the residual finding-i.e., that participants who are not aligned with CH produce responses consistent with a definite descriptions (DD) theory of reference-does not hold. Most participants in our experiments, and nearly all those who do not provide CH answers, respond as predicted by a theory that accommodates speaker's reference in reasoning about uses of proper names, not according to DD. We suggest that cross-cultural variation in this task is real. However, explanations of variation within or across cultures need not invoke competing theories of reference (CH vs DD), and can be unified within a single, broadly Kripkean analysis that honors the basic distinction between semantic reference and speaker's reference.
## 99                                                                                                                                        When interpreting disjunctive sentences of the form 'A or B', young children have been reported to differ from adults in two ways. First, children have been reported to interpret disjunction inclusively rather than exclusively, accepting 'A or B' in contexts in which both A and B are true (Chierchia et al. 2001; Gualmini et al. 2001). Second, some children have been reported to interpret disjunction conjunctively, rejecting 'A or B' in contexts in which only one of the disjuncts is true (Paris 1973; Braine & Rumain 1981; Chierchia et al. 2004; Singh et al. 2015). In this article, we extend the investigation of children's interpretation of disjunction to include both simple and complex forms of disjunction, in two typologically unrelated languages: French and Japanese. First, given that complex disjunctions have been argued to give rise to obligatory exclusivity inferences (Spector 2014), we investigated whether the obligatoriness of the inference would play a role in the acquisition of the exclusive interpretation. Second, using a paradigm that makes the use of disjunction felicitous, we aimed to establish whether the finding of conjunctive interpretations would be replicated for both simple and complex forms of disjunction, and in languages other than English. The main findings from our experiment are that both French-and Japanese-speaking children interpreted the simple and complex disjunctions either inclusively or conjunctively; in contrast, adults generally accessed exclusive readings of both disjunctions. We argue that our results lend further support to the proposal put forth in Singh et al. (2015), according to which the reason some children compute conjunctive meanings while adults compute exclusivemeanings is that the two groups differ in their respective sets of alternatives for disjunction. Crucially, adults access conjunction as an alternative to disjunction, and compute exclusive interpretations; in contrast, children access only the individual disjuncts as alternatives, and therefore either interpret the disjunction literally or compute conjunctive inferences. More generally, our findings can be explained quite naturally within recent proposals according to which children differ from adults in the computation of scalar inferences because they are more restricted than adults in the set of scalar alternatives they can access (Barner et al. 2011; Tieu et al. 2015b, among others).
## 100                                                                                                                                                                                                                                                                                                                                                                                                                                                                                                                                                                                                                                                                                                                                                                                                                                                                                                                                                                                                                                                                                                                                                                                                                                                                                                                                                                                                                                                                                                                                                                                                                             In one task-switching experiment, we compared bilinguals and monolinguals to explore the reliability of the bilingualism effect on the n-2 repetition cost. In a second task-switching experiment, we tested another group of bilinguals and monolinguals and measured both the n-1 shift cost and the n-2 repetition cost to test the hypothesis that bilingualism should confer a general greater efficiency of the executive control functioning. According to this hypothesis, we expected a reduced n-1 shift cost and an enhanced n-2 repetition cost for bilinguals compared to monolinguals. However, we did not observe such results. Our findings suggest that previous results cannot be replicated and that the n-2 repetition cost is another index that shows no reliable bilingualism effect. Finally, we observed a negative correlation between the two switch costs among bilinguals only. This finding may suggest that the two groups employ different strategies to cope with interference in task-switching paradigms.
## 101                                                                                                                                                                                                                                                                                                                                                                                                                                                                                                                                                                                                                                                                                                                                                                                                                                                                                                                                                                                                                                                                                                                                                                                                                                                                                                                                                                                                                                                                                Do the lexical representations of the non-response language enter into lexical competition during speech production? This issue has been studied by means of the picture-word interference paradigm in which two paradoxical effects have been observed. The so-called CROSS-LANGUAGE IDENTIFY EFFECT (Costa, Miozzo and Caramazza, 1999) has been taken as evidence against cross-linguistic lexical competition. In contrast, the so-called PHONO-TRANSLATION EFFECT (Hermans, Bongaerts, De Bot and Schreuder 1998) has been interpreted as revealing lexical competition across languages. In this article, we assess the reliability of these two effects by testing Spanish-Catalan highly-proficient bilinguals performing a Stroop task. The results of the experiment are clear: while the cross-language identity facilitation effect is reliably replicated, the phono-translation interference effect is absent from the Stroop task. From these results, we conclude that we should be cautious when drawing strong conclusions about the presence of competition across languages based on the phono-translation effect observed in the picture-word interference paradigm.
## 102                                                                                                                                                                                                                                                                                                                                                                                                                                                                                                                                                                                                                                                                                                                                                                                                                                                                                                                                                                                                                                                                                                                                                                                                                                                                                                                                            This study aims to examine the influence of multiple translations of a word on bilingual processing in three translation recognition experiments during which French English bilinguals had to decide whether two words were translations of each other or not. In the Just experiment, words with only one translation were recognized as translations faster than words with multiple translations. Furthermore, when words were presented with their dominant translation, the recognition process was faster than when words were presented with their non-dominant translation. In Experiment 2, these effects were replicated in both directions of translation (L1-L2 and L2-L1). In Experiment 3, we manipulated number-of-translations and the semantic relatedness between the different translations of a word. When the two translations of a word (i.e., bateau) were related in meaning (synonyms such as the English translations boat and ship), the translation recognition process was faster than when the two translations of a word (i.e., argent) were unrelated in meaning (the two translations money and silver). The consequences of translation ambiguities are discussed in the light oldie distributed conceptual feature model of bilingual memory (De Groot, 1992b; Hell and De Groot, 1998b).
## 103                                                                                                                                                                                                                                                                                                                                                                                                                                                                                                                                                                                                                                                                                                                                                                                                                                                                                                                                                                                                                                                                                                                                                                                                                                                                                                                                                                                                                                                                                                                                                                                                                               This study examined whether L1-Mandarin learners of L2-English use verb bias and complementizer cues to process temporarily ambiguous English sentences the same way native speakers do. SVO word order places verbs early in sentences in both languages, allowing the use of verb-based knowledge to anticipate what could follow. The two languages differ, however, in whether an optional complementizer signals embedded clauses. In a self-paced reading experiment, native English speakers and L1-Mandarin learners of L2-English read sentences containing temporary ambiguity about whether a noun was the direct object of the verb preceding it or the subject of an embedded clause. Native speakers replicated previous work showing an optimally efficient interactive pattern of cue use, while non-native learners showed additive effects of the two cues, consistent with predictions of the Competition Model about learning how to use multiple cues in a second language that sometimes agree and sometimes do not.
## 104                                                                                                                                                                                                                                                                                                                                                                                                                                                                                                                                                                                                                                                                                                                                                                                                                                                                                                                                                                                                                                                                                                                                                                                                                                                                                                                                                                                                                                                                                                                                                                                                                                                                     Levy, Mc Veigh, Marful and Andreson (2007) found that naming pictures in L2 impaired subsequent recall of the L1 translation words. This was interpreted as evidence for a domain-general inhibitory mechanism (RIF) underlying first language attrition. Because this result is at odds with some previous findings and theoretical assumptions, we wanted to assess its reliability and replicate the experiment with various groups. Participants were first shown drawings along with their labels in the non-dominant language. Afterwards, they named 75% of these drawings in their first language or in their non-dominant language. Finally, participants' memory of all L1 words was tested through the presentation of a rhyme-cue. Recall of L1 words was better after naming pictures in the non-dominant language compared to when the picture was not named at all. This result suggests that speaking a second language protects rather than harms the memory of our first language.
## 105                                                                                                                                                                                                                                                                                                                                                                                                                                                                                                                                                                                                                                                                                                                                                                                                                                                                                                                                                                                                                                                                                                                                                                                                                                                                                                                                                                                                                                                                                                                                                                                                         This study newly investigates whether the functional weight of a prosodic cue in the native language predicts listeners' learning and use of that cue in second-language speech segmentation. It compares English and Dutch listeners' use of fundamental-frequency (F0) rise as a cue to word-final boundaries in French. F0 rise signals word-initial boundaries in English and Dutch, but has a weaker functional weight in English than Dutch because it is more strongly correlated with vowel quality in English than Dutch. English- and Dutch-speaking learners of French matched in French proficiency and experience, and native French listeners completed a visual-world eye-tracking experiment in French where they monitored words ending with/out an F0 rise (replication of Tremblay, Broersma, Coughlin & Choi, 2016). Dutch listeners made earlier/greater use of the F0 rise than English listeners, and in one condition they made greater use of F0 rise than French listeners, extending the cue-weighting theory to speech segmentation.
## 106                                                                                                                                                                                                                                                                                                                                                                                                                                                                                                                                                                                                                                                                                                                                                                                                                                                                                                                                                                                                                                                                                                                                                                                                                                                                                                                                                                                                                                                                                                                                                                                                                                                                                                                                                                                                         Four experiments are reported which were designed to test hypotheses concerning the asymmetry of masked translation priming Experiment I confirmed the presence of L2-L1 priming with a semantic categorization task and demonstrated that this effect was restricted to exemplars. Experiment 2 showed that the translation priming effect was not due to response congruence. Experiment 3 replicated this finding, and demonstrated that the 150 ms backward mask that had been used in earlier translation priming experiments was not essential. Finally, it was demonstrated in Experiment 4 that L2-L1 priming was not obtained for an ad hoc category indicating that priming was not obtained merely because the task required semantic interpretation. These results provide further support for the Sense Model proposed by Finkbeiner et al. (2004).
## 107                                                                                                                                                                                                                                                                                                                                                                                                                                                                                                                                                                                                                                                                                                                                                                                                                                                                                                                                                                                                                                                                                                                                                                                                                                                                                                                                                                                                                                                                                                                                    In everyday communication, speakers make use of a variety of contextual and gestural cues to modulate the meaning of an utterance. Young children have difficulty in integrating multiple communicative cues when some of them have to be interpreted differently depending on other co-occurring cues. However, bilingual children, who regularly experience communicative challenges that demand greater attention and flexibility, may be more adept in integrating multiple cues to understand a speaker's communicative intent. We replicated Nurmsoo and Bloom's (2008) procedure with three-year-old monolingual and bilingual children using a procedure in which they saw two novel objects while the experimenter could see only one. The experimenter looked at the object she could see and said either There's the [novel-word!] or Where's the [novel-word]?. Compared to monolinguals, bilingual preschoolers were better able to integrate the semantics of where, perceptual access of the experimenter, and the nonlinguistic context of the game to successfully differentiate the speaker's communicative intent.
## 108                                                                                                                                                                                                                                                                                                                                                                                                                                                                                                                                                                                                                                                                                                                                                                                                                                                                                                                                                                                                This study reports on the result of ethno-linguistic research which aims to investigate whether an emerging form of code-switching among three languages, namely Cantonese, English and Putonghua, exists in Hong Kong. This study follows the research method of Sung (2010) which the author recorded his experiences as a 'purist' in Hong Kong: during a three-day experiment - the author employed only Cantonese, English and Putonghua, respectively. Field notes and reflective diaries were used to record the incidents of communication breakdowns resulted from the use of pure-code instead of mixed codes. Because of the recent change in language policy in which Putonghua has placed more important roles in language teaching in Hong Kong, Putonghua has been added into the current study intentionally because Sung (2010) only included Cantonese and English, which largely ignored the fact the Putonghua has become an important part in the linguistic environment of Hong Kong. The difficulties of using only pure-Cantonese, pure-English and pure-Putonghua in Hong Kong will be discussed. Also, I suggest that there exists the code-switching among Cantonese, English and Putonghua in Hong Kong because of the increasing contact with China and the new policy of using Putonghua as the Medium of Instruction (PMI) in some primary schools. Moreover, the present study suggests that the use of pure-code in these languages may hinder communication in Hong Kong. Further studies are needed on code-switching among the three languages within the younger generation of Hongkongers, especially those who attend PMI schools.
## 109                                                                                                                                                                                                                                                                                                                                                                                                                                                                                                                                                                                                                                                                                                                                                                                                                                                                                         Kim, Euhee, Myung-Kwan Park, and Hye-Jin Seo. 2020. L2ers' predictions of syntactic structure and reaction Wiles during sentence processing. Linguistic Research 37(Special Edition): 189-218. This paper investigates how Korean 12 learners of English predict upcoming syntactic structure based on a newly received word during sentence processing. Studies like Linzen and Jaeger (2016) suggest that readers use their probabilistic inference developed by their experience of the language to which they have been exposed to predict the most appropriate syntactic structure. This study replicates the experiment for L2ers following Linzen and Jaeger (ibid.), which investigates the way of predicting syntactic structure by using the subcategorization frame of a verb to understand L1 language processing. We employ the information-complexity metrics such as surprisal, entropy, and entropy reduction to quantify the uncertainty/unexpectedness of a given word that reflects the processing difficulty during sentence processing. The results show that L2ers' tendency to read different regions of a sentence varies. Reading times are longer in the verb and the ambiguous regions of the structurally ambiguous than of the structurally unambiguous sentences. Likewise, reading times are longer in the disambiguating region of the unambiguous than of the ambiguous sentences. Reading firms are also longer when the surprisal increases in the disambiguating region. Overall, the findings reveal that such information-complexity metrics as entropy reduction and surprisal play an instrumental role in accounting for the aspects of sentence processing by Korean 12 learners of English. (Shinhan University . Dongguk University)
## 110                                                                                                                                                                                                                                                                                                                                                                                                                                                                                                                                                                                                                                                                                                                                                                                                                                                                                                                                                                                                                                                                                                                                                                                                                                                                                                                                                                                                                                                                                                                                                                                                                                                          This paper presents the results of a rate scaling speech production experiment which seeks to replicate and examine in greater detail the results of a set of experiments reported in Stetson (1951). Stetson observed, based on a set of pioneering articulatory experiments, that coda consonants resyllabify as onset consonants in syllables repeated at fast speech rates. In the current experiment, speakers produced repetitions of simple CV and VC syllables in time to a metronome pacer which systematically changed in period, Data indicate that, while durational patterns for CV and VC syllables are very different at slow rates, the patterns tend to converge at fast rates. However, closer examination of fast rate tokens, reveals that differences between CV and VC tokens persist at fast rates, even though such tokens are generally heard as CV tokens. These results are discussed with respect to the nature of CV and VC organization and the effect of the rate-changing task.
## 111                                                                                                                                                                                                                                                                                                                                                                                                                                                                                                                                                                                                                                                                                                                                                                                                                                                                                                                                                                                                                                                                                                                                                                                                                                                           Previous research has shown that orthography influences the learning and processing of spoken non-native words. In this paper, we examine the effect of L1 orthography on non-native sound perception. In Experiment 1, 204 Spanish learners of Dutch and a control group of 20 native speakers of Dutch were asked to classify Dutch vowel tokens by choosing from auditorily presented options, in one task, and from the orthographic representations of Dutch vowels, in a second task. The results show that vowel categorization varied across tasks: the most difficult vowels in the purely auditory task were the easiest in the orthographic task and, conversely, vowels with a relatively high success rate in the purely auditory task were poorly classified in the orthographic task. The results of Experiment 2 with 22 monolingual Peruvian Spanish listeners replicated the main results of Experiment 1 and confirmed the existence of orthographic effects. Together, the two experiments show that when listening to auditory stimuli only, native speakers of Spanish have great difficulty classifying certain Dutch vowels, regardless of the amount of experience they may have with the Dutch language. Importantly, the pairing of auditory stimuli with orthographic labels can help or hinder Spanish listeners' sound categorization, depending on the specific sound contrast.
## 112                                                                                                                                                                                                                                                                                                                                                                                                                                                                                                                                                                                                                                                                                                                                                                                                                                                                                                                                                                                                                                                                                                                                                                                                                                                                                                                                                                                                                                                                                                                                              The present study explores learning phonological alternations that contain exceptions. Participants were exposed to a back/round vowel harmony pattern in which a regular suffix obeyed a vowel harmony rule, varying between /e/ and /o/ depending on the back/round phonetic features of the stem, and a non-alternating suffix that was always /o/ regardless of the features of the stem vowel. Participants in Experiment 1 learned the behavior of both suffixes, but correct performance for the non-alternating suffix was higher when the suffix happened to be in harmony with the stem. Participants in Experiment 2 were exposed to the non-alternating affix in harmonic contexts only, and continued to show a bias towards harmony. Experiment 3 replicated Experiment 2 with minimal training on disharmonic cases of the non-alternating morpheme. However, participants were less likely to learn the alternating affix without exposure to morphological stem, stem + suffix alternations in Experiment 4, suggesting a bias towards morphophonological alternations in learning vowel harmony patterns.
## 113                                                                                                                                                                                                                                                                                                                                                                                                                                                                                                                                                                                                                                                                                                                                                                                                                                                                                                     We present the results from an experiment that tests the perception of English consonantal sequences by Korean speakers and we confirm that perceptual epenthesis in a second languge (L2) arises from syllable structure restrictions of the first language (L1), rather than linear co-occurence restrictions. Our study replicates and extends Dupoux, Kakehi, Hirose, Pallier, & Mehler's (1999) results that suggested that listeners perceive epenthetic vowels within consonantal sequences that violate the phonotactics of their L1. Korean employs at least two kinds of phonotactic restrictions: (i) syllable structure restrictions that prohibit the occurence of certain consonants in coda position (e.g., *[c.], *[g.]), while allowing others (e.g., [k.], [1.]), and (ii) consonantal contact restrictions that ban the co-occurrence of certain heterosyllabic consonants (e.g., *[k.m]; *[l.n]) due to various phonological processes that repair such sequences on the surface (i.e., /k.m/ -> [n.m]; /l.n/ -> [1.1]). The results suggest that Korean syllable structure restrictions, rather than consonantal contact restrictions, result in the perception of epenthetic vowels. Furthermore, the frequency of co-occurrence fails to explain the epenthesis effects in the percept of consonant clusters employed in the present study. We address questions regarding the interaction between speech perception and phonology and test the validity of Steriade's (2001 a,b) Perceptual-Mapping (P-Map) hypothesis for the Korean sonorant assimilation processes. Our results indicate that Steriade's hypothesis makes incorrect predictions about Korean phonology and that speech perception is not isomorphic to speech production.
## 114                                                                                                                                                                                                                                                                                                                                                                                                                                                                                                                                                                                                                                                                                                                                                                                                                                                                                                                                                                                                                                                                                                                                            As the division between emotion and emotion-laden words has been viewed as controversial by, for example, Kousta and colleagues, the current study attempted a replication and extension of findings previously described by Kazanas and Altarriba. In their findings, Kazanas and Altarriba reported significant differences in response times (RTs) and priming effects between emotion and emotion-laden words, with faster RTs and larger priming effects with emotion words than with emotion-laden words. These findings were consistent across unmasked (Experiment 1) and masked (Experiment 2) versions of a lexical decision task, where participants either explicitly or implicitly processed the prime words of each prime-target word pair. Findings from Experiment 2 have been previously replicated by Kazanas and Altarriba with a Spanish-English bilingual sample, when tested in English, the participants' functionally dominant language. The current study was designed to extend these previous findings, using a l000-ms stimulus onset asynchrony (SOA), which was longer than the 250-ms SOA originally used by Kazanas and Altarriba. Findings from the current study supported the division between emotion and emotion-laden words, as they replicated those previously described by Kazanas and Altarriba. In addition, the current study determined that negative words were processed significantly slower in this experiment, with a long SOA (replicating findings by Rossell and Nobre).
## 115                                                                                                                                                                                                                                                                                                                                                                                                                                                                                                                                                                                                                                                                                                                                                                                                                                                                                                                                                                                                                                                                                                                                                                                                                                                                                                                                                                                                                                                          The current research explores the role of lexical representations and processing in the recognition of phonological variants. Two alternative approaches for variant recognition are considered: a representational approach that posits frequency-graded lexical representations for variant forms and inferential processes that mediate between the spoken variant and the lexical representation. In a lexical decision task (Experiment 1) and in a phoneme identification task (Experiment 2) using real words, low-frequency variants, but not high-frequency variants, show improved recognition rates following additional experience with the variants. This knowledge generalized to novel variant forms. Experiment 3 replicated these results using an artificial lexicon and showed that recognition of low-frequency variants was influenced by similarity to a high-frequency variant form. Similarity to a high-frequency variant alone, however, was insufficient to explain recognition of the infrequent variants (Experiments 4 and 5). The results support a hybrid account of variant recognition that relies on both multiple frequency-graded representations and inference processes.
## 116                                                                                                                                                                                                                                                                                                                                                                                                                                                                                                                                                                                                                                                                                                                                                                                                                                       We tested for transposition effects (TEs) in Hindi (a Modern Indo-Aryan language) using unprimed lexical decision. TEs are defined as less accurate and slower responses to transposed-nonwords (e.g., < psate(sic), formed from base-word < paste(sic)) than corresponding replaced-nonwords (e.g., < pzute(sic)). In Hindi's orthography, letters map transparently to phonemes (except schwa), but the letters are arranged into akshars, (<[C-n]V(sic)) which encode open syllables. This formal characteristic makes Hindi's orthography typologically aksharic. We used TEs to determine whether the orthography's typological units, letters and akshars, are also functional units for readers. We conducted three visual word recognition experiments with adult readers whose native language was Hindi. In Experiment 1, we found TEs for consonant (< C(sic)) and matra (< M(sic), a vowel diacritic) letters, using different stimulus sets for each type of transposition. In the next two experiments, we used the same base words to form all of the transposed and replaced items. In Experiment 2, we replicated the findings of Experiment 1 in a different stimulus set; additionally, we found TEs for transpositions between a < C(sic) letter and a < CM(sic) akshar. In Experiment 3, we replicated results of the first two experiments by finding TEs for both consonants and matras in another stimulus set; additionally, we found similar TEs for < CM(sic) akshars. These results show that < C(sic) and < M(sic) letters are functional units for Hindi readers; the transposition results for < CM(sic) akshars are tentative. TEs for letters show that the aksharic grouping of letters does not prevent readers from decoding the constituent letters of akshars. Hindi is read alphabetically.
## 117                                                                                                                                                                                                                                                                                                                                                                                                                                                                                                                                                                                                                                                                                                                                                                                             Two experiments investigated the role of parallelism in strategies of pronoun assignment and tested the proposition that a parallel function strategy and the subject assignment strategy jointly contribute to the interpretation of pronouns. The parallel function strategy interprets a pronoun as coreferential with a preceding noun phrase in the same grammatical role, while the subject assignment strategy interprets a pronoun as coreferential with a preceding subject noun phrase. The results of Experiment 1 showed that the subject assignment and parallel function strategies jointly constrained assignment. When both strategies yielded the same interpretation, as was the case for subject assignments with subject pronouns, subject assignment preferences were greater than when only the subject assignment strategy was available, as was the case with nonsubject pronouns in partially parallel sentences (i.e., sentences in which the pronoun and antecedent differed grammatically). When the two strategies yielded conflicting assignments, as was the case with nonsubject pronouns in fully parallel sentences, subject assignments were reduced even further and nonsubject assignments were preferred. Experiment 2 replicated the results of Experiment 1 in isolated sentences rather than in short texts and ruled out the idea that a third ''parallel order'' strategy might be used in partially parallel sentences. We suggest that the subject assignment strategy reflects the topic status of the subject noun phrase, while parallel function reflects the use of correspondences between sentence structures in comprehension. We also suggest that the joint operation of heuristic strategies implies a model of discourse processing in which a number of constraints compete in the interpretation of noun phrases.
## 118                                                                                                                                                                                                                                                                                                                                                                                                                                                                                                                                                                                                                                                                                                                                                                                                                                                                                                                                                                                                                                                                                                                                                                                                                                                                                                    Two experiments investigated how people assign an interpretation to question phrases. In order to determine the meaning of the WH-phrase (e.g., who, what), a gap must be located and the role associated with the gap assigned to the WH-phrase. Two experiments tested the Lexical Expectation model of Fodor (1978), according to which lexical properties of the verb determine when a gap is posited, and the All Resorts model of Stowe (1984), according to which all possibilities are considered and evaluated on their pragmatic appropriateness. In Experiment 1, subjects judged the meaningfulness of full sentences. The frequency with which verbs are used transitively determined whether there was an effect of the plausibility of the WH-phrase to act as an object of the verb. Effects of plausibility of the WH-phrase as an object showed up in just those cases where the object role should be assigned to the WH-phrase according to the Lexical Expectation model, rather than as predicted by the All Resorts model. In Experiment 2, these results were replicated using the word-by-word self-paced reading paradigm. The plausibility effect showed up at the verb itself when it is normally used transitively. This evidence suggests that a gap is preferred even over a lexically filled object for transitive expectation verbs.
## 119                                                                                                                                                                                                                                                                                                                                                                                                                                                                                                                                                                                                                                                                                                                                                                                                                                                                                                                                                                                                                                                                                                                                                                                                                                                                                                                                                                                                                                                                                                                                                                                                                                                                                                                                                                            Pragmatic inferences require listeners to use alternatives to arrive at the speaker's intended meaning. Previous research has shown that intonation interacts with alternatives but not how it does so. We present two mouse tracking experiments that test how pitch accents affect the processing of ad hoc scalar implicatures in English. The first shows that L+H* accents facilitate implicatures relative to H* accents. The second replicates this finding and demonstrates that the facilitation is caused by early derivation of the implicature in the L+H* condition. We attribute the effect to a link between L+H* and pragmatic considerations, such as speaker knowledge effects, or the saliency of alternatives relevant to the computation of implicatures. More generally our findings illustrate how intonation interacts at a cognitive level with pragmatic inference.
## 120                                                                                                                                                                                                                                                                                                                                                                                                                                                                                                                                                                                                                                                                                                                                                                                                                                                                                                                                                                                                                                                                                                                                                                                                                                                                                                                                                        The ability of speakers to exaggerate speech sounds (hyperarticulation) has led to the theory that the targets themselves must be hyperarticulated. Johnson, Flemming, and Wright (1993) found that perceptual best exemplar choices for vowels were more extreme than listeners' own productions. Our first experiment, using their procedure, only partially replicated their results. Low vowels showed a higher F I, consistent with hyperspace. Front vowels also showed more frontness in F2, but back vowels were less extreme (hypoarticulated) on F2. Our second experiment used an identification and rating of each stimulus, yielding similar results of a smaller magnitude. Our results indicate that the perceptual space is calibrated to a particular (synthetic) vowel space, which is not related straightforwardly to the speakers' spaces. The original hyperspace hypothesis can be attributed to the methodology which led to extreme judgments and of the fronting of back vowels in California English. The present results indicate that no such hypothesis is needed. Vowel targets are measurable from an individual's productions, and the individual's perception of other speakers (even synthetic ones) is based on information about the vocal tract and dialect of the speaker.
## 121                                                                                                                                                                                                                                                                                                                                                                                                                                                                                                                                                                                                                                                                                                                                                                                                                                                                                                                                                                                                                                                                                                                                                                                                            In an eye-tracking study, Romero-Fresco (2016) discovered that when watching a dubbed film, Spanish viewers hardly looked at characters' mouths and focussed instead on their eyes - a phenomenon he termed 'the dubbing effect'. Our study is a conceptual replication of Romero-Fresco's study, aimed at answering the question of whether a similar effect also takes place in voice-over: do viewers avoid looking at characters' mouths to stay immersed in the film story? With this question in mind, we tested 35 Polish native speakers watching a 6-minute voiced-over excerpt from Casablanca while their eyes were monitored with an eye tracker. We also measured viewers' immersion levels as well as their enjoyment and comprehension. In this paper, we present two experiments. In Experiment 1, by analysing viewers' gaze behaviour and immersion levels, we found that Polish viewers did not avoid looking at characters' mouths. In Experiment 2, we compared our results with those obtained in the original study with Spanish and English viewers. We found that visual attention distribution in Polish voice-over resembled the one observed in English viewers, who watched the film with the original soundtrack. Both Polish and English viewers spent more time looking at characters' eyes in scenes with no dialogue compared to scenes with dialogue, as opposed to the Spanish people for whom the tendency was reversed.
## 122                                                                                                                                                                                                                                                                                                                                                                                                                                                                                                                                                                                                                                                                                                                                                              Experimental studies on AVT have grown incrementally over the past decade. This growing body of research has explored several aspects of AVT reception and production using behavioural measures such as eye tracking, as well as venturing into physiological measures such as electroencephalography (EEG), galvanic skin response, and heart rate. As a novel approach to the field of AVT, the experimental approach has borrowed heavily from other fields with established experimental traditions, such as psycholinguistics, psychology, and cognitive science. However, these methodologies are often not implemented with the same rigour as in the disciplines from which they were taken, making for highly eclectic and, at times, inconsistent practices. The absence of a common framework and best practice for experimental research in AVT poses significant risk in addition to the potential reputational damage. Some of the most important risks are: the duplication of efforts, studies that cannot be replicated due to a lack of methodological standardisation and rigour, and findings that are, at best, impossible to generalise from and, at worst, invalid. Given the growing body of work in AVT taking a quasi-experimental approach, it is time to consolidate our position and establish a common framework in order to ensure the integrity of our endeavours. This chapter analyses problems and discusses solutions specifically related to the multidisciplinary nature of experimental AVT research. In so doing, it aims to set the course for future experimental research in AVT, in order to gain credibility in the wider scientific community and contributes new insights to the fields from which AVT has been borrowing. Its conclusion lays out the foundation for a common core of measures and norms to regulate research in the growing field of AVT.
## 123                                                                                                                                                                                                                                                                                                                                                                                                                                                                                                                                                                                                                                                                                                                                                                                                                                                                        The Subregular Hypothesis (Heinz 2010) states that only patterns with specific subregular computational properties are phonologically learnable. Lai (2015) provided the initial laboratory support for this hypothesis. The current study aimed to replicate and extend the earlier findings by using a different experimental paradigm (oddball task) and a different measure of learning (sensitivity index, d'). Specifically, we compared the learnability of two phonotactic patterns that differ computationally and typologically: a simple rule (First-Last Assimilation) that requires agreement between the first and last segment of a word (predicted to be unlearnable), and a harmony rule (Sibilant Harmony) that requires the agreement of features throughout the word (predicted to be learnable). The First-Last Assimilation rule was tested under two experimental conditions: one where the training data were also consistent with the Sibilant Harmony rule, and one where the training data were only consistent with the First-Last rule. As in Lai (2015), we found that participants were significantly more sensitive to violations of the Sibilant Harmony (SH) rule than to the First-Last Assimilation (FL) rules. However, unlike Lai (2015), we also found that participants showed some residual sensitivity to the First-Last rule, but that sensitivity interacted with rule type so that participants were significantly more sensitive to SH rule violations. We conclude that participants in Artificial Grammar Learning experiments exhibit evidence of Universal Grammar constraining their learning, but patterns predicted to be unlearnable as a linguistic system can still be learned to some degree, due to non-linguistic learning mechanisms.
## 124                                                                                                                                                                                                                                                                                                                                                                                                                                                                                                                                                                                                                                                                                                                                                                                                                                                                                                                                                                                                                                                                                                                                                                                                                                                                                                                                                                                                                                                                                                                                                                                                                                                                                                                                            In a study of quantifier-scope priming, Chemla and Bott (2015) found evidence suggesting that, while representations of quantifiers' relative scope can be primed, a scope inversion operation cannot. We identify a confound in their materials. In Experiment 1, we replicate their finding with this confound intact. In Experiment 2, we remove the confound and find that all priming disappears. This confound demonstrates how structural priming paradigms can be sensitive to many dimensions of similarity, pointing to a need for task-specific controls. We conclude that the prior study does not provide evidence concerning the priming of either relative scope representations or operations. While priming of scope representations has been independently found in other paradigms, the jury is still out on Chemla and Bott's more novel finding - the absence of priming of a scope inversion operation.
## 125                                                                                                                                                                                                                                                                                                                                                                                                                                                                                                                                                                                                                                                                                                                                                                                                                                                                                                                                                                                                                                                                                                         This paper discusses data from two self-paced reading experiments as well as an acceptability rating study that shed light on the binding behaviour of demonstrative pronouns as opposed to personal pronouns. Participants read (Experiments 1 & 2) or rated (Experiment 3) single sentences that contained either a demonstrative pronoun (DPro) or a personal pronoun (PPro). Sentences contained a determiner phrase (DP) that functioned as the grammatical subject and a DP that functioned as the direct, indirect or prepositional object. The pronoun was either contained in the direct object DP or a prepositional object DP. In half of the sentences, pronouns could only be interpreted as bound by the subject DP. In the other half of sentences, they could only be interpreted as bound by the object DR Results from Experiment 1 reveal similar reading times for DPros and PPros when they were bound by the object DP, and significantly longer reading times for DPros than PPros when they were bound by the subject DP. Experiment 2 replicated the DPro effect from Experiment 1 with materials where potential subject and object binders were quantifiers. Finally, Experiment 3 shows that also in the context of quantifier binding DPros are not generally dispreferred. Sentences with a DPro were only rated as less acceptable than sentences with a PPro when the potential binder was the subject. Taken together, our data provide evidence that DPros can be bound as long as their binders are not grammatical subjects.
## 126                                                                                                                                                                                                                                                                                                                                                                                                                                                                                                                                                                                                                                                                                                                                                                                                                                                                                                                                                                                                                                                                                                                                                                                                                                                                                                                                                                                                                                                     The reliability of acceptability judgments made by individual linguists has often been called into question. Recent large-scale replication studies conducted in response to this criticism have shown that the majority of published English acceptability judgments are robust. We make two observations about these replication studies. First we raise the concern that English acceptability judgments may be more reliable than judgments in other languages. Second, we argue that it is unnecessary to replicate judgments that illustrate uncontroversial descriptive facts; rather, candidates for replication can emerge during formal or informal peer review. We present two experiments motivated by these arguments. Published Hebrew and Japanese acceptability contrasts considered questionable by the authors of the present paper were rated for acceptability by a large sample of naive participants. Approximately half of the contrasts did not replicate. We suggest that the reliability of acceptability judgments, especially in languages other than English, can be improved using a simple open review system, and that formal experiments are only necessary in controversial cases.
## 127                                                                                                                                                                                                                                                                                                                                                                                                                                                                                                                                                                                                                                                                                                                                                                                                                                                                                                                                                                                                                                                                                                                                          We used a Deese-Roediger-McDermott false memory paradigm to compare Spanish words in which the phonetic realization of /s/ can vary (word-medial positions: bu[s] to similar to bu[h] to 'chest', word-final positions: remo[s] similar to remo[h] 'oars') to words in which it cannot (word-initial positions: [s] opa similar to *[h] opa 'soup'). At study, participants listened to lists of nine words that were phonological neighbors of an unheard critical item (e.g., popa, sepa, soja, etc. for the critical item sopa). At test, participants performed free recall and yes/no recognition tasks. Replicating previous work in this paradigm, results showed robust false memory effects: that is, participants were more likely to (falsely) remember a critical item than a random intrusion. When the realization of /s/ was consistent across conditions (Experiment 1), false memory rates for varying versus non-varying words did not significantly differ. However, when the realization of /s/ varied between [s] and [h] in those positions which allow it (Experiment 2), false recognition rates for varying words like busto were significantly higher than those for non-varying words like sopa. Assuming that higher false memory rates are indicative of greater lexical activation, we interpret these results to support the predictions of exemplar theory, which claims that words with heterogeneous versus homogeneous acoustic realizations should exhibit distinct patterns of activation.
## 128                                                                                                                                                                                                                                                                                                                                                                                               Across languages, plural marking on count nouns typically gives rise to a multiplicity inference, indicating that the noun ranges over sums with a cardinality of 2 or more. Plural marking has also been observed to occur on mass nouns in Greek and a few other languages, giving rise to a parallel abundance inference, indicating that there is a lot of the relevant substance. It has been observed in the literature that both of these inferences disappear in downward-entailing environments, such as when a plural appears in the scope of negation (Tsoulas 2009; Kane et al. 2015). There are two main competing approaches in the literature that aim to account for the described pattern with respect to multiplicity inferences: the ambiguity approach (Farkas & de Swart 2010) and the implicature approach (Sauerland 2003; Spector 2007; Mayr 2015, among others). As discussed in Tieu et al. (2018), while both approaches can account for the upward-versus downward-entailing pattern of multiplicity inferences, they differ in what they predict with respect to the acquisition of these inferences and their relationship with implicatures. lieu et al. (2014; 2018) investigated multiplicity inferences in English and reported evidence for the implicature approach. In this paper, we first show how the ambiguity approach and the implicature approach to the multiplicity inference can be extended to account for the abundance inference. We then report on an experiment that tests the predictions of the two approaches for multiplicity and abundance inferences in preschool-aged children and adult native speakers of Greek. Our results replicate the patterns reported in Tieu et al. (2014; 2018) for multiplicity inferences, and crucially reveal an analogous pattern for abundance inferences. Adults computed both kinds of inferences more in upward-entailing environments than in downward-entailing ones, and children computed fewer inferences overall than adults did. These results reflect an overall pattern of implicature calculation in line with a unified implicature analysis across the three kinds of inferences. By contrast, we discuss how they pose a challenge for the ambiguity approach.
## 129                                                                                                                                                                                                                                                                                                                                                                                                                                                                                                                                                                                                                                                                                                                                                                                                                                                                                                                                                                                                        In this paper, we extend investigations of the possible effects of cross-linguistic influence at the pragmatics-syntax interface (Hulk & Muller 2000; Muller & Hulk 2001; Serratrice, Sorace & Paoli 2004), by presenting two experiments designed to probe how Spanish monolingual and Spanish-English bilingual preschool-age children approach the 'some, but not all' scalar implicature (SI) associated with algunos ('some'). We compare algunos and unos (also a 'some' indefinite, but one that is not context-linked and does not induce an SI), and algunos and todos (the universal quantifier 'every/all'). The performance of the children is compared to fluent adult Spanish heritage speakers. Experiment 1 is a variation of Noveck's (2001) statement evaluation task, also replicated by Guasti et al. (2005). Experiment 2 is a forced-choice picture selection task. Results demonstrate that adults were the only group to consistently calculate the SI associated with algunos - a finding that was expected to some extent, given that our tasks were stripped of the contextual support that could benefit children's pragmatic reasoning. While bilingual and monolingual children displayed comparable performance across tasks, bilinguals in Experiment 2 appeared to experience difficulty with judgments related to todos - a pattern we attribute (in light of independent findings) to the cognitive overload in the task, not the lexical entry of this quantifier. We conclude that young monolingual and bilingual children confront the same challenges when called upon to deploy pragmatic skills in a discourse context.
## 130                                                                                                                                                                                                                                                                                                                                                                                                                                                                                                                                                                                                                                                                                                                                                                                                                                                                                                                                                                                                                                                                                                                                                                                                                                                                                                      Greek prenuclear accents show a sharp rise that starts near the onset of the accented syllable and peaks on the following unaccented syllable (if there is one). We have presented elsewhere evidence for analyzing these accents as consisting of a L(ow) and a H(igh) target. In a first experiment exploring the factors that affect the location of the H target, we discovered that in words with antepenultimate lexical stress the H is consistently aligned just after the onset of the first postaccentual vowel. In a second experiment we replicated this finding, showing that the alignment of the H is not affected by variations in the duration of the accented syllable. A third experiment showed that for some speakers the alignment of the H may be affected by tonal crowding, when the accented syllable is close to the end of the word and/or close to the next accent. Overall, however, the results show that the L and H targets are independently aligned relative to the segmental string: the accentual rise is neither of fixed slope nor of fixed duration. This result, which replicates and extends earlier findings of Prieto, van Santen & Hirschberg (1995) for Mexican Spanish, is difficult to accommodate in a theory that views pitch movements as the primes of intonational structure. (C) 1998 Academic Press Limited.
## 131                                                                                                                                                                                                                                                                                                                                                                                                                                                                                                                                                                                                                                                                                                                                                                                                                                                                                                                                                                                                                                                                                                                                                                                                                                                                                                                                                                                                                                                                                           De Boer (2010a) used a simplified adaptation of Mermelstein's articulatory model without lips to investigate the effect of larynx position on the articulatory abilities of a human-like vocal tract. He found an optimal larynx height at which the largest range of signals [are] produced. Our replication of de Boer's simulations confirmed the peak in the F2 range function when 'the vertical and horizontal parts of the tract are approximately equally long. However, simulations show that the diminished F2 range at high larynx positions stems from an insufficiently low F2 in the /u/ region. Without an independent lip closure paired with the velodorsal constriction, the model cannot create the two cavities necessary to produce simultaneously low F1 and F2. A second series of simulations with a proper lip section essentially restored the F2 space across a large range of larynx heights. These experiments confirm that larynx height does not inordinately restrict the acoustic F1 F2 capacities. of a mammalian vocal tract, provided that lips are appropriately included in the simulations. (C) 2014 Elsevier Ltd. All rights reserved.
## 132                                                                                                                                                                                                                                                                                                                                                                                                                                                                                                                                                                                                                                                                                                                                                                                                                                                                                                                                                                                                                                                                                                                                                             Highly reduced pronunciation variants, such as something like 'yeshay' for yesterday, are abundant in conversational speech. Previous research has shown that listeners understand such pronunciation variants only in their sentential contexts. This study further investigates the roles of the acoustic properties of reduced words themselves and of semantic/syntactic and acoustic information in their contexts. We report five experiments in which participants were tested on the recognition of reduced words originating from a corpus of conversational Dutch. Experiment 1, a visual doze test, demonstrates that our set of reduced words cannot be guessed just on the basis of their semantic/syntactic context. Experiment 2 replicates the earlier finding that reduced words can only be recognised in their contexts. Experiment 3 shows that the reduced words were less well recognised if the context is presented visually in the form of orthographic transcriptions. These experiments suggest that listeners need some acoustic properties of reduced words themselves, together with semantic/syntactic and acoustic information in their contexts, to recognise reduced words. Experiments 4 and 5 confirm the importance of acoustic information for word recognition by showing that high-frequency hearing loss hinders both the interpretation of the target words' acoustic properties, and the use of neighbouring context. (C) 2011 Elsevier Ltd. All rights reserved.
## 133                                                                                                                                                                                                                                                                                                                                                                                                                                                      The experiments reported here used auditory-visual mismatches to compare three approaches to speaker normalization in speech perception: radical invariance, vocal tract normalization, and talker normalization. In contrast to the first two, the talker normalization theory assumes that listeners' subjective, abstract impressions of talkers play a role in speech perception. Experiment 1 found that the gender of a visually presented face affects the location of the phoneme boundary between [upsilon] and [Lambda] in the perceptual identification of a continuum of auditory-visual stimuli ranging from hood to hud. This effect was found for both stereotypical and non-stereotypical'' male and female voices. The experiment also found that voice stereotypicality had an effect on the phoneme boundary. The difference between male and female talkers was greater when the talkers were rated by listeners as stereotypical. Interestingly, for the two female talkers in this experiment, rated stereotypicality was correlated with voice breathiness rather than vowel fundamental frequency. Experiment 2 replicated and extended experiment 1 and tested whether the visual stimuli in experiment 1 were being perceptually integrated with the acoustic stimuli. In addition to the effects found in experiment 1, there was a boundary effect for the visually presented word: listeners responded hood more frequently when the acoustic stimulus was paired with a movie clip of a talker saying hood. Experiment 3 tested the abstractness of the talker information used in speech perception. Rather than seeing movie clips of male and female talkers, listeners were instructed to imagine a male or female talker while performing an audio-only identification task with a gender-ambiguous hood-hud continuum. The phoneme boundary differed as a function of the imagined gender of the talker. The results from these experiments suggest that listeners integrate abstract gender information with phonetic information in speech perception. This conclusion supports the talker normalization theory of perceptual speaker normalization. (C) 1999 Academic Press.
## 134                                                                                                                                                                                                                                                                                                                                                                                                                                                                                                                                                                                                                                                                                                                                                                                                                                                                                                                                                                                                                                                                                                                                                                                                                          It is common practice in the statistical analysis of phonetic data to draw conclusions on the basis of statistical significance. While p-values reflect the probability of incorrectly concluding a null effect is real, they do not provide information about other types of error that are also important for interpreting statistical results. In this paper, we focus on three measures related to these errors. The first, power, reflects the likelihood of detecting an effect that in fact exists. The second and third, Type M and Type S errors, measure the extent to which estimates of the magnitude and direction of an effect are inaccurate. We then provide an example of design analysis (Gelman & Carlin, 2014), using data from an experimental study on German incomplete neutralization, to illustrate how power, magnitude, and sign errors vary with sample and effect size. This case study shows how the informativity of research findings can vary substantially in ways that are not always, or even usually, apparent on the basis of a p-value alone. We conclude by repeating three recommendations for good statistical practice in phonetics from best practices widely recommended for the social and behavioral sciences: report all results; design studies which will produce high-precision estimates; and conduct direct replications of previous findings. (c) 2018 Elsevier Ltd. All rights reserved.
## 135                                                                                                                                                                                                                                                                                                                                                                                                                                                                                                                                                                                                                                                                                                                                                                                                                                                                                                                                                                                                                                                                                                                                                                                                                                                                                                                                                                                                                           Before a [-voice] coda, F-1 is higher for monophthongs but lower for /aI/ than before [+voice]. We test the hypothesis that this is due to hyperarticulation before voiceless consonants (Thomas, J. Phonetics 88 (2000) 1). Experiment 1, with 16 American English speakers, found the /aI/ pattern of more peripheral F-1 and F-2 in the offglides /I eI au/ as well, showing that it is part of the realization of [-voice]. The diphthong nuclei were less affected than the offglides. Experiment 2, with different 16 American English speakers, collected tight-tide judgments of a synthetic stimulus in which offglide F-1, offglide F-2, and nuclear duration were varied independently. Tight responses were facilitated by lower F-1, by higher F-2, and by shorter nuclei. Log-linear analysis showed that the three factors contributed independently, and that F-2 was a stronger cue than F-1 in terms of logits per Bark. Experiment 3, with a different 16 speakers, replicated Experiment 2 for ate-aid judgments (/eI/ having shown the smallest effects in Experiment (1)). Thus, [-voice] is correlated with, and cued by, peripheralization of diphthong offglides. (C) 2003 Elsevier Science Ltd. All rights reserved.
## 136                                                                                                                                                                                                                                                                                                                                                                                                                                                                                                                                                                                                                                                                                                                                                                                                                                                                                                                                                                                                                                                                                                                                                          Recent evidence shows that studies on perceptual recalibration and its generalization can inform us about the presence and nature of prelexical units used for speech perception. Listeners recalibrate perception when hearing an ambiguous auditory stimulus between, for example, /p/ and /t/ in unambiguous lexical context (kee[p/T]>/p/, mee[p/t]->/t/) or visual context (presence vs. absence of lip closure). A later encountered ambiguous auditory only stimulus is then perceived in line with the previously experienced context. Unlike studies using lexical context to guide learning, experiments with the visual paradigm suggested that prelexical units are rather specific and context-dependent. However, these experiments raised doubts whether lexically-guided and visually-guided recalibration are targeting the same type of units, or whether learning in the visually-guided paradigm with limited variability during exposure is task-specific. The present study shows successful visually-guided learning following exposure to a variety of different learning trials. We also show that patterns of generalization found with the visually-guided paradigm can be replicated with a lexically-guided paradigm: listeners do not generalize a recalibrated stop contrast across manner of articulation. This supports suggestions that the units of perception depend on the distribution of relevant cues in the speech signal. (C) 2015 Elsevier Ltd. All rights reserved.
## 137                                                                                                                                                                                                                                                                                                                                                                                                                                                                                                                                                                                                                                                                                                                                                                                                                                        It has been claimed that the long established neutralization of the voicing distinction in domain final position in German is phonetically incomplete. However, many studies that have advanced this claim have subsequently been criticized on methodological grounds, calling incomplete neutralization into question. In three production experiments and one perception experiment we address these methodological criticisms. In the first production study, we address the role of orthography. In a large scale auditory task using pseudowords, we confirm that neutralization is indeed incomplete and suggest that previous null results may simply be due to lack of statistical power. In two follow-up production studies (Experiments 2 and 3), we rule out a potential confound of Experiment 1, namely that the effect might be due to accommodation to the presented auditory stimuli, by manipulating the duration of the preceding vowel. While the between-items design (Experiment 2) replicated the findings of Experiment 1, the between-subjects version (Experiment 3) failed to find a statistically significant incomplete neutralization effect, although we found numerical tendencies in the expected direction. Finally, in a perception study (Experiment 4), we demonstrate that the subphonemic differences between final voiceless and devoiced stops are audible, but only barely so. Even though the present findings provide evidence for the robustness of incomplete neutralization in German, the small effect sizes highlight the challenges of investigating this phenomenon. We argue that without necessarily postulating functional relevance, incomplete neutralization can be accounted for by recent models of lexical organization. (C) 2014 Elsevier Ltd. All rights reserved.
## 138                                                                                                                                                                                                                                                                                                                                                                                                                                                                                                                                                                                                                                                                                                                                                                                                                                                          The distinction between underlying and excrescent stops in pairs like 'mints' and 'mince' was convincingly demonstrated by Fourakis and Port (1986). Several subsequent studies have been unable to replicate the result for speakers of American English, or have done so only partially. These studies have largely dealt with the acoustic signal. This study presents an approach to stop excrescence that refers to both the aerodynamics and articulation of the phenomenon. The results confirm and expand on the original findings. Using nasal flow as an indirect measure of velopharyngeal aperture and electropalatography (EPG) to estimate the moment of oral release, the presence of occlusion, as well as the duration of nasal and oral occlusion were measured. Overall contact across the palate was also measured. Disyllabic and monosyllabic tokens with /ns/ and /nts/ in final position were pronounced by four male speakers of American English. Disyllabic tokens could be either stressed or unstressed on the final syllable. In Experiment I, speakers produced tokens in a standard carrier phrase; in Experiment II, they produced one of the items in contrastive focus to its 'homophonous' counterpart, e.g., 'I said mince not mints'. Underlying stops were significantly longer than excrescent stops, including in the contrastive-focus condition. A trading relation between nasal and oral stop duration was demonstrated when the stop was excrescent, but not when it was underlying. This suggests that the nasal-oral occlusion in epenethetic stops is divided proportionally between the underlying nasal and excrescent oral stop, but that the durations of the nasal and underlying oral stops are independent. (C) 2011 Elsevier Ltd. All rights reserved.
## 139                                                                                                                                                                                                                                                                                                                                                                                                                                                                                                                                                                                                                                                                                                                                                                                                                                                                                                                                                                                                                                                                                                                                                                                                                                                                                                                                                                                         The difficulty of pinpointing a specific event within words that would correspond to the p-centre is well known. The current experiment, investigating the position of p-centres in Czech, aims to replicate the findings from English and several other languages, and substantially increase the range of phonotactic types and the number of participants. In a speech-metronome synchronization task, 24 subjects pronounced a set of 37 natural disyllabic Czech words of differing complexity at two metronome rates. The beginning of the first vowel (V1) and the moment of the fastest increase in energy within the first syllable were the most consistent synchronization points, but the p-centre occurred earlier than at the V1 initial boundary. Synchronization intervals were significantly influenced by the complexity of the syllabic onset: the p-centre was positioned earlier (further from the V1) as more consonants were included in the onset. The effects of vowel length and final coda were also present, but weaker. In addition, various aspects of human musicality were found to correlate with the ability of speakers to synchronize their articulations with an isochronous auditory sequence. (C) 2015 Elsevier Ltd. All rights reserved.
## 140                                                                                                                                                                                                                                                                                                                                                                                                                                                                                                                                                                                                                                                                                                                                                                                                                                                                                                                                                                                                                                                                                                                                                                                                                                             The efficacy of aphasia therapy has been demonstrated in group and single case studies. However, in group studies, treatment was underspecified and replication was impossible. In single case studies, treatment was clearly defined and experimentally controlled but its scope was generally limited, e.g. learning of a pre-determined and relatively low number of words or reading of irregular words. The purpose of this study was to investigate the effects of an intensive (4-5 h/day), clearly defined, and model-based treatment in a subject-DA-with multiple language disorders who was beyond the period of spontaneous recovery. DA underwent a general neuropsychological and language evaluation 8 months post-onset. He had moderate comprehension disorders, reading and writing were slow and severely damaged, and his speech was agrammatic. DA's spontaneous production was analysed according to the Quantitative Production Analysis (QPA) method. Treatment lasted 8 months and was directed at each functional impairment. Results at the first and the final evaluation were compared. Important differences were found in many language tasks and in spontaneous production, showing that intensive and model-based treatment can improve the patient's daily use of language, even in a chronic subject past the period of spontaneous recovery. (C) 2004 Elsevier Ltd. All rights reserved.
## 141                                                                                                                                                                                                                                                                                                                                                                                                                                                                                                                                                                                                                                                                                                                                                                                                                                                                                                                                                                                                                                                                                                                                                                                                                                                                                                                                                                                                                                                                                                                                                                                                                                                                                                                                                                                                                                             Patients with acquired deep dyslexia are unable to read nonwords aloud. The deficit has therefore been attributed to damage in nonlexical phonological processing. Buchanan, Hildebrandt and MacKinnon [5 Journal of Neurolingistics 8, 163-182, 1994] demonstrated that a deep dyslexic patient could process nonword phonology in two implicit tests. The generality of this claim is examined by replicating the Buchanan et al. experiments with two other deep dyslexic patients. Dissociations between Lexical and nonlexical processing are examined by manipulating nonword phonology in a task that does not require lexical analysis. The results suggest that sensitivity to nonword phonology in deep dyslexia is common and is distinct from a purely lexical analysis. Copyright (C) 1996 Elsevier Science Ltd
## 142                                                                                                                                                                                                                                                                                                                                                                                                                                                                                                                                                                                                                                                                                                                                                                                                                                                                                                                                                                                                                                                                                                                                                                                                                                                                                                                                                                                      This study examined whether the phonological substitution rule of tone sandhi modulates tone perception in the preattentive stage. Tone sandhi is commonly present in East Asian languages. An example from Mandarin is the Tone tone 3 sandhi rule: T3 is pronounced as T2 when followed by another T3 (33 -> 23). Previous mismatch negativity (MMN) studies in Mandarin have reported a smaller amplitude or longer latency in standard-deviant pair consisting of T2 and T3 (T2-T3) than in Tl-T3. The most widely accepted explanation for this is that T2 and T3 have steeper pitch slopes than Tl. This study tested an alternative account based on the phonological rule that the frequent substitution that occurs between T2 and T3 results in reduced MMN. In Experiment 1, we first tried to replicate the finding in Mandarin. In Experiment 2, using both unskilled and skilled speakers, we tested a sandhi tone pair of very different pitch slopes in Taiwanese. Delayed peak latency of sandhi pair was evident in both languages but only in skilled speakers. Our results did not support the shared-pitch-slope account and were instead consistent with the argument that a language-specific phonological rule could modulate preattentive tone processing.
## 143                                                                                                                                                                                                                                                                                                                                                                                                                                                                                                                                                                                                                                                                                                                                                                                                                                                                                                       We used event related potentials (ERPs) in order to investigate how sentences, semantically ambiguous with respect to number, are understood. Although sentences such as (i) Every kid climbed a tree lack any syntactic or lexical ambiguity, two possible meanings are available, where either many trees or just one tree was climbed. Previous behavioural studies showed a plural preference, whereas ERP and behavioural experiments conducted in our lab have not. In this work, we further investigate sentences as in (i), called quantifier scope ambiguous sentences, and compare them to unambiguous sentences, (ii) Every kid climbed the trees. Participants read sentences presented in 1- and 2-word chunks, and judged, at the target word tree(s), whether 1 or 2 words appeared on the computer screen (Berent et al., 2005). Previously, interference effects resulted for judgments that 1 word was on the screen when it was marked plural (e.g., trees) versus singular (e.g., tree). Interestingly, Patson and Warren (2010) also showed that this was the case for judgments made for singular words, e.g., tree, in quantifier ambiguous sentences, confirming the plural preference. The current ERP study did not replicate their behavioural findings. Difficulty for 1 responses was not observed for trees in (ii) nor was it observed for tree in quantifier scope sentences (i). Instead, a P300 effect was found at the target word tree(s), where amplitudes differed depending on congruency in number interpretation for subjects and direct objects. Results are discussed in terms of heuristic first sentence processing mechanisms, and relevant features of event knowledge. (C) 2016 Elsevier Ltd. All rights reserved.
## 144                                                                                                                                                                                                                                                                                                                                                                                                                                                                                                                                                                                                                                                                                                                                                                                                                                                                                                                                                                                                                                    The present article reports an ERP study with two experiments designed to assess the influence of emotional adjectives on sentence processing by means of a gender agreement (grammaticality) judgment task. Participants were shown transitive sentences in Spanish presented word-by-word, with a complex object-NP consisting of a noun + an adjective + some additional material. ERPs were recorded at the critical adjectives, which could be either emotionally unpleasant, neutral, or pleasant, and could either agree (match) or fail to agree (mismatch) in gender with the modified noun. Results replicate the typical ERP signatures of morphosyntactic errors. Specifically, larger amplitudes for LAN and P600 components were found in sentences with a gender mismatch in both experiments. However, ERP components did not show any differences between emotional and neutral adjectives in these sentences. Only in Experiment 2 was a significant interaction found between grammaticality and emotionality. Thus, in the 500-700 ms window larger amplitudes emerged for neutral words than for emotional words in match sentences. Likewise, a significant interaction between grammaticality and emotionality in the participants' performance revealed that the presence, of emotional adjectives did facilitate the detection of gender errors. Overall, these results show that during sentence reading syntactic error detection seems to occur in an encapsulated manner, although performance may indeed be facilitated by the presence of emotional words. (C) 2017 Elsevier Ltd. All rights reserved.
## 145                                                                                                                                                                                                                                                                                                                                                                                                                                                                                                                                                                                                                                                                                                                                                                                                                                                                                                                                                                                                                                                                                                                                                                                       We investigated individual differences in second language (L2) proficiency by looking at the efficiency or automaticity of semantic priming using behavioural and event-related brain potential (ERP) measures. In Experiment 1, 37 first language (L1) English speakers varying in second language (L2, French) proficiency made living/non-living judgments to English and French nouns in lists blocked by language. Sixty critical words were each presented twice, once primed by a semantic associate in the preceding trial (e.g. ADULT, CHILD) and once unprimed (e.g. RABBIT, CHILD). Measures of response time (RT) and intra-individual variability in response time (coefficient of variation, CV) were obtained. The CV provided an index of processing efficiency that has been related to automaticity. Participants performed faster and with lower CVs (i.e. with greater efficiency) in L1 than L2, and the more highly proficient bilinguals had lower CVs than the less proficient bilinguals. Experiment 2 replicated these results with 29 participants and provided an electrical brain activity measure of processing efficiency using the N400 ERP. The similar pattern of results obtained between the behavioural and N400 ERP CV measures supported the idea that the CV measure of electrical brain activity can provide useful information about the automaticity or efficiency of cognitive processing. (C) 2003 Elsevier Ltd. All rights reserved.
## 146                                                                                                                                                                                                                                                                                                                                                                                                                                                                                                                                                                                                                                                                                                                                                                                                                                                                                                                                                                                                                                                   Objects presented in categorically related contexts are typically named slower than objects presented in unrelated contexts, a phenomenon termed semantic interference. However, not all semantic relationships induce interference. In the present study, we investigated the influence of object part-relations in the blocked cyclic naming paradigm. In Experiment 1 we established that an object's parts do induce a semantic interference effect when named in context compared to unrelated parts (e.g., leaf, root, nut, bark; for tree). In Experiment 2) we replicated the effect during perfusion functional magnetic resonance imaging (fMRI) to identify the cerebral regions involved. The interference effect was associated with significant perfusion signal increases in the hippocampal formation and decreases in the dorsolateral prefrontal cortex. We failed to observe significant perfusion signal changes in the left lateral temporal lobe, a region that shows reliable activity for interference effects induced by categorical relations in the same paradigm and is proposed to mediate lexical-semantic processing. We interpret these results as supporting recent explanations of semantic interference in blocked cyclic naming that implicate working memory mechanisms. However, given the failure to observe significant perfusion signal changes in the left temporal lobe, the results provide only partial support for accounts that assume semantic interference in this paradigm arises solely due to lexical-level processes. (C) 2015 Elsevier Ltd. All rights reserved.
## 147                                                                                                                                                                                                                                                                                                                                                                                                                                                                                                                                                                                                                                                                                                                                                                                                                                                                                                                                                                                                                                                                                                                                    Most previous studies on negation have generally only focused on sentential negation (not), but the time course of processing negative meaning from different sources remains poorly understood. In an ERP study (Experiment 1), we make use of the negation sensitivity of negative polarity items (NPIs) and examine the time course of processing different kinds of negation. Four kinds of NPI-licensing environments were examined: the negative determiner no, the negative determiner few, the focus marker only, and emotive predicates (e.g., surprised). While the first three contribute a negative meaning via semantic assertion (explicit negation), the last gives rise to a pragmatic negative inference via non-asserted content (implicit negation). Under all these environments, an NPI elicited a smaller N400 compared to an unlicensed NPI, suggesting that negation, regardless of its source, is rapidly computed online. However, we also observed that explicit negative meaning (i.e., semantic, as contributed in the assertion) and implicit negative meaning (contributed by pragmatic inferences) were integrated into the grammatical representation in different ways, leading to a difference in the P600, and calling for a separation of semantic and pragmatic integration during sentence processing (and NPI licensing). The qualitative differences between these conditions were also replicated in a self-paced reading study (Experiment 2). (C) 2015 Elsevier Ltd. All rights reserved.
## 148                                                                                                                                                                                                                                                                                                                                                                                                                                                                                                                                                                                                                                                                                                                                                                                                                                                                                                                                                                                                                                                                                    The foreign language vocabulary learning research literature often attributes strong mnemonic potency to the cognitive processing of meaning when learning words. Routinely cited as support for this idea are experiments by Craik and Tulving (C&T) demonstrating superior recognition and recall of studied words following semantic tasks (deep encoding) compared to structure-related tasks (shallow encoding). However, participants in C&T were not language learners but native speakers of English studying known English nouns. These experiments have never been directly replicated using nonnatives to establish the relevance of the findings to nonnatives and learners. The present study replicated C&T Experiment 5, comparing effects of shallow and deep encoding tasks on subsequent recognition of target words by native and nonnative speakers of English with equivalent short-term memory function. The results showed depth effects similar to C&T for all participants, indicating that C&T's results do generalize to less proficient speakers of the target language. It is crucial, however, that nonnative speakers of English benefited less from semantic encoding than native speakers, suggesting an effect of preexisting knowledge representations on mnemonic effects derived from semantic processing, and hence, a limit to the relevance of C&T for learners. Results are discussed in terms of the constructs of the mental lexicon, expert knowledge, distinctiveness and levels of processing in memory research and language learning.
## 149                                                                                                                                                                                                                                                                                                                                                                                                                                                                                                                                                                                    This research examines the hypotheses about how print represents the speech that preliterate children select when they receive input compatible with several such hypotheses. In Experiment 1, preschoolers were taught to read hat and hats and book and books. Then, in generalization tests, they were probed for what they had learned about the letter s. All of the children were able to transfer to other plurals (e.g., to decide that bikes said ''bikes'' rather than ''bike,'' and that dog said ''dog'' and not ''dogs''), but only those who knew the sound of the letter s prior to the experiment were able to decide, for example, that bus said ''bus'' and not ''bug.'' The failure to detect the phonemic value of s on the part of alphabetically naive children was replicated in Experiments 2, 3, and 4, which instituted a variety of controls. In Experiment 5, it was found that, although preschoolers who had been taught to read pairs of words distinguished by the comparative affix er (such as small/smaller) were able to generalize to other comparatives (e.g., mean/meaner), they could not generalize to pairs where er had no morphemic value (e.g., corn/corner). A similar failure by alphabetically naive children to detect the syllabic, as compared with the morphemic, status of the superlative affix est was found in Experiment 6. Overall, the results indicate that most preliterate children fail to select phonologically based hypotheses, even when these are available in the input. Instead, they focus on morphophonology and/or semantic aspects of words' referents. The research is couched in terms of the Learnability Theory (LT) (Gold, 1967), which provides a convenient framework for considering a series of interrelated questions about the acquisition of literacy. In particular, it is argued that if the data available to the child includes the pronunciation of written words, the alphabetic principle may be unlearnable, given the hypothesis selection procedures identified in these experiments.
## 150                                                                                                                                                                                                                                                                                                                                                                                                                                                                                                                                                                                                                                                                                                                                                                                                                                                                                                                                                                                                                                                                                                                                                                                                                                                                                             In two semantic priming experiments, this study examined how southern French speakers process the standard French [o] variant in closed syllables in comparison to their own variant [LATIN SMALL LETTER OPEN O]. In Experiment 1, southern French speakers showed facilitation in the processing of the associated target word VIOLET whether the word prime mauve was pronounced by a standard French speaker ([mov]) or a southern French speaker ([mLATIN SMALL LETTER OPEN Ov]). More importantly, Experiment 1 has also revealed that words of type mauve, which are subject to dialectal variation, behave exactly in the same way as words of type gomme, which are pronounced with [LATIN SMALL LETTER OPEN O] by both southern and standard French speakers, and for which we also found no modulation in the magnitude of the priming effect as a function of the dialect of the speaker. Experiment 2 replicated the priming effect found with the standard French variant [mov], and failed to show a priming effect with nonwords such as [m oe v] that also differ from the southern French variant [mLATIN SMALL LETTER OPEN Ov] by only one phonetic feature. Our study thus provides further evidence for efficient processing of dialectal variants during spoken word recognition, even if these variants are not part of the speaker's own productions.
## 151                                                                                                                                                                                                                                                                                                                                                                                                                                                                                                                                                                                                                                                                                                                                                                                                                                                                                                                                                                                                                                                                                                                                                                                                                                                                                                                                                                                                                                                                                                                            This paper investigates the limited attainment of adult compared to child language acquisition in terms of learned attention to morphological cues. It replicates Ellis and Sagarra in demonstrating short-term learned attention in the acquisition of temporal reference in Latin, and it extends the investigation using eye-tracking indicators to determine the extent to which these biases are overt or covert. English native speakers learned adverbial and morphological cues to temporal reference in a small set of Latin phrases under experimental conditions. Comprehension and production data demonstrated that early experience with adverbial cues enhanced subsequent use of this cue dimension and blocked the acquisition of verbal tense morphology. Effects of early experience of verbal morphology were less pronounced. Eye-tracking measures showed that early experience of particular cue dimensions affected what participants overtly focused upon during subsequent language processing and how this overt study resulted in turn in covert attentional biases in comprehension and in productive knowledge.
## 152                                                                                                                                                                                                                                                                                                                                                                                                                                                                                                                                                                                                                                                                                                                                                                                                                                                                                                                                                                                                                                                                                                                                                                                                                                                                                                                                                                                                                                                                                                                                                                                                                                                                                         Processes involved in late bilinguals' production of morphologically complex words were studied using an event-related brain potentials (ERP) paradigm in which EEGs were recorded during participants' silent productions of English past- and present-tense forms. Twenty-three advanced second language speakers of English (first language [L1] German) were compared to a control group of 19 L1 English speakers from an earlier study. We found a frontocentral negativity for regular relative to irregular past-tense forms (e.g., asked vs. held) during (silent) production, and no difference for the present-tense condition (e.g., asks vs. holds), replicating the ERP effect obtained for the L1 group. This ERP effect suggests that combinatorial processing is involved in producing regular past-tense forms, in both late bilinguals and L1 speakers. We also suggest that this paradigm is a useful tool for future studies of online language production.
## 153                                                                                                                                                                                                                                                                                                                                                                                                                                                                                                                                                                                                                                                                                                                                                                                                                                                                                                                                                                                                                                                                                                                                                                                                                                                                                                                                                                                                                                                                                                                                                       In a series of lexical priming experiments we examined the interaction between spelling processes dedicated to spelling familiar words (lexical processes) and those dedicated to spelling unfamiliar words or nonwords (sublexical processes). Participants listened to lists of intermixed monosyllabic words and nonwords and were required to spell only the nonwords. In the priming condition. nonwords were preceded by real word primes that were phonologically related to the nonwords. In two experiments, we found that the spellings of nonwords could be influenced by previously heard rhyming words, replicating previous work. Furthermore, we examined the mechanism of this lexical/sublexical interaction and found that it is both phonologically and orthographically based and that word primes are most effective when they overlap in word body (vowel + coda) with the nonword. We conclude that lexical and sublexical processes interact in a manner that involves a dynamic updating of sound-spelling correspondences, which, at a minimum, are specified in terms of the word body.
## 154                                                                                                                                                                                                                                                                                                                                                                                                                                                                                                                                                                                                                                                                                                                                                                                                        Goswami and Bryant (1990) proposed a theory of reading development based on three causal connections. One of these causal connections was based on the relationship between rhyming skills and reading development found in English. To explain this connection, they suggested that young readers of English used analogies based on rimes as one means of deciphering the alphabetic code. This proposal has recently become the subject of some debate. The most serious critique has been advanced by Seymour and his colleagues (Duncan, Seymour, & Hill, 1997; Seymour & Duncan, 1997; Seymour & Evans, 1994). These authors reported a series of studies with Scottish schoolchildren which, they claim, show that progression in normal reading acquisition is from a small unit (phonemic) approach in the initial stage to a large unit (rime-based) approach at a later stage. Two experiments are presented which replicate those conducted by Seymour and his group with samples of English schoolchildren. Different results are found. It is argued that methodological and instructional factors may be very important for the conceptual interpretation of studies attempting to pit small units (phonemes) against large units (onsets and rimes) in reading. In particular, it is necessary to consider whether a given phonological awareness task requires the recognition of shared phonological segments (epilinguistic processing) or the identification and production of shared phonological segments (metalinguistic processing). It is also important to take into account the nature of the literacy instruction being implemented in participating schools. If the phonological aspects of this tuition focus solely on phonemes (small units), then poor rime-level (large unit) performance may be found in metalinguistic tasks.
## 155                                                                                                                                                                                                                                                                                                                                                                                                                                                                                                                                                                                                                                                                                                                                                                                                                                                                                                                                                                                                                                                                                                                                                                                                                                                                           While there is a consensus that speakers plan their utterances before they start producing them, the scope of the initial planning unit remains controversial. In subject-initial utterances, is the planning unit the whole subject phrase or a smaller functional phrase within the subject phrase? Allum and Wheeldon (2007) reported that speakers show faster onset latencies in producing utterances like The flower above the house is red, where the subject consists of two functional phrases (the flower and above the house) than in producing The flower and the house are red, where there is a single, longer functional phrase (The flower and the house), both in head-initial languages like English and head-final languages like Japanese. This has been taken to suggest that the functional phrase is a preferred unit of planning, rather than the whole subject. Experiment 1 in the present study replicates Allum and Wheeldon's study with speakers of another head-final language (Mandarin Chinese) and finds similar results. Experiments 2 and 3 investigate whether syntactic processing or visual grouping could potentially explain the pattern of responses, and find that they cannot. Together, these results provide further empirical support for the claim that the functional phrase is a primary unit of grammatical planning for speech production.
## 156                                                                                                                                                                                                                                                                                                                                                                                                                                                                                                                                                                                                                                                                                                                                                                                                                                                                                                                                                                                                                                                                                                                                                                                                                                                                                                                                                                                                                                                                                                                                                                                                                                                                                                                                                                                                                       This article reports two experiments exploring heterosexual men's use of homophobic slang in social contexts varied by sex ratio. Study 1 (N = 127) experimentally demonstrated that compared with a mixed-sex audience, heterosexual men with an all-male audience reported higher levels of hetero-identity concern (HIC) and more homophobic slang use. These men had similar levels of HIC compared with men with an all-female audience. Study 2 replicated Study 1's mean difference tests and explored whether the relationship between HIC and homophobic slang was affected by group sex ratio and social norms. Results suggest the relationship between HIC and homophobic slang was significant only in all-male and mixed-sex audiences and the norm of noninterference was associated with homophobic slang only in all-male groups.
## 157                                                                                                                                                                                                                                                                                                                                                                                                                                                                                                                                                                                                                                                                                                                                                                                                                                                                                                                                                                                                                                                                                                                                                                                                                                                                                                                                                                                                                                                                                                                                                                                                                                                                                                         Prejudice against a social group may lead to discrimination of members of this group. One very strong cue of group membership is a (non)standard accent in speech. Surprisingly, hardly any interventions against accent-based discrimination have been tested. In the current article, we introduce an intervention in which what participants experience themselves unobtrusively changes their evaluations of others. In the present experiment, participants in the experimental condition talked to a confederate in a foreign language before the experiment, whereas those in the control condition received no treatment. Replicating previous research, participants in the control condition discriminated against Turkish-accented job candidates. In contrast, those in the experimental condition evaluated Turkish- and standard-accented candidates as similarly competent. We discuss potential mediating and moderating factors of this effect.
## 158                                                                                                                                                                                                                                                                                                                                                                                                                                                                                                                                                                                                                                                                                                                                                                                                                                                                                                                                                                                                                                                                                                                                                                                                                                                                                                                                                                                                                                                                    Political campaign speeches are deemed influential in winning people's minds and votes. While the language used in such speeches has often been credited with their impact, empirical research in this area is scarce. We report on two experiments investigating how language variables such as rhetorical schemes (e.g., contrast, list of three) and valence framing (using positive vs. negative words) affect immediate attention and consecutive information processing of political radio speeches. Experiment 1 measured immediate attention for radio speeches measured through moment-to-moment, self-report measures. Negative framing, compared with positive framing, increased immediate attention. Rhetorical schemes only increased attention in positively (but not in negatively) framed speeches. No effects on recall were found. In Experiment 2, immediate attention for similar radio speeches was measured through secondary task reaction times. Experiment 2 replicated the first experiment's effects on attention, and also yielded recall effects. A multiple-mediator model showed that comprehensibility mediated effects of rhetorical schemes and framing on recall.
## 159                                                                                                                                                                                                                                                                                                                                                                                                                                                                                                                                                                                                                                                                                                                                                                                                                                                                                                                                                                                                                                                                                                                                                                                                                                                                                                                                                                                                                                                                               The integrated model of advice giving (IMA) proposes that advising in supportive interactions should be carried out in three sequential moves: emotional support-problem inquiry and analysis-advice (EPA). Prior research indicates the utility of this framework for effective advising in supportive interactions. The current project proposed and tested an extended integrated model of advice giving, adding eSteem support (S) as a fourth move in the sequence. Two experiments were conducted. Study 1 included 371 participants recruited from Amazon Mechanical Turk. Results showed that the emotional support-problem inquiry and analysis-advice-eSteem support (EPAS) sequence did not elicit significantly higher evaluations of advice quality compared with the EPA or emotional support-problem inquiry and analysis-eSteem support-advice (EPSA) sequence. Study 2 replicated Study 1 with 364 college students and found that, compared with the other two sequences, the EPAS sequence did not produce significantly higher evaluations of advice quality or intention to follow advice. Theoretical implications and directions for future research are discussed.
## 160                                                                                                                                                                                                                                                                                                                                                                                                                                                                                                                                                                                                                                                                                                                                                                                                                                                                                                                                                                                                                                                                                                                                                                                                                                                                                                                                                                                                                                                                                                                                                                                                                                                                                                                                                                                Research on the English and the Japanese languages has shown that impression formation and attribution processes can be modeled as stemming from a desire to maximize affective coherence in linguistic representations of social events. This article describes a replication in the German language, simultaneously uncovering subtle cultural distinctions. Subjects (N = 1,905) rated 376 nouns, 393 verbs, 331 adjectives, 128 combinations of nouns and adjectives, and 100 short descriptions of social interaction on the three dimensions of the semantic differential (evaluation, potency, and activity). The data were used to estimate a set of regression equations that can be used to model impression formation and attribution. Sample applications of the model demonstrate its ability to predict the outcome of textbook classics in experimental social psychology.
## 161                                                                                                                                                                                                                                                                                                                                                                                                                                                                                                                                                                                                                                                                                                                                                                                                                                                                                                                                                                                                                                                                                                                                                                                                                                                                                                                                                                                                                                                                                                                                                                                      Three experiments investigated performance for words which differ from another word only by the transposition of two letters (e.g., salt, slat). In Experiment 1, high frequency words from transposed-letter (TL) confusable pairs were responded to more slowly than carefully matched control words in both the lexical decision and word naming task. Low frequency TL words were responded to less accurately than control words in the naming but not the lexical decision task. Experiment 2 replicated the naming data of Experiment 1 and also revealed that naming accuracy for TL word targets was reduced when they were preceded by a brief masked presentation of their confusable mate. Experiment 3 provided a third replication of the impaired naming performance for TL target words and demonstrated that the effect was insensitive to concurrent dual task demands. These TL confusability effects provide strong constraints that can contribute to evaluation and specification of current models of visual word recognition. (C) 1996 Academic Press, Inc.
## 162                                                                                                                                                                                                                                                                                                                                                                                                                                                                                                                                                                                                                                                                                                                                                                                                                                                                                                                                                                                                                                                                                                                                                                                                                                                                                                                                                    Interference has been identified as a cause of processing difficulty in linguistic dependencies, such as the subject-verb relation (Van Dyke and Lewis, 2003). However, while mounting evidence implicates retrieval interference in sentence processing, the nature of the retrieval cues involved - and thus the source of difficulty - remains largely unexplored. Three experiments used self-paced reading and eyetracking to examine the ways in which the retrieval cues provided at a verb characterize subjects. Syntactic theory has identified a number of properties correlated with subjecthood, both phrase-structural and thematic. Findings replicate and extend previous findings of interference at a verb from additional subjects, but indicate that retrieval outcomes are relativized to the syntactic domain in which the retrieval occurs. One, the cues distinguish between thematic subjects in verbal and nominal domains. Two, within the verbal domain, retrieval is sensitive to abstract syntactic properties associated with subjects and their clauses. We argue that the processing at a verb requires cue-driven retrieval, and that the retrieval cues utilize abstract grammatical properties which may reflect parser expectations. (C) 2016 Elsevier Inc. All rights reserved.
## 163                                                                                                                                                                                                                                                                                                                                                                                                                                                                                                                                                                                                                                                                                                                                                                                                                                                                                                                                                                                                                                                                                                                                                                                                                                                                                                                                                                                                                The revelation effect is a change in response behavior induced by a preceding problem-solving task. Previous studies have shown a revelation effect for faces when the problem-solving task includes attractiveness ratings of the faces. Immediately after this problem-solving task participants judged faces as more familiar than without the problem-solving task. We replicated this result in Experiment 1. Based on the discrepancy-attribution hypothesis, we predicted that a problem-solving task that excludes attractiveness ratings would not elicit a revelation effect. However, we found a reversed revelation effect with a problem-solving task that required participants to solve a puzzle of each face (Experiments 2-3). In Experiments 2 and 3, participants judged faces as less familiar after the puzzle task. Our findings support the notion that the revelation effect may manifest as either an increase or a decrease of the experienced familiarity towards the recognition probe. However, our results contradict all current theories of the revelation effect. We discuss implications of our findings for revelation effect theories and provide a possible explanation. (C) 2012 Elsevier Inc. All rights reserved.
## 164                                                                                                                                                                                                                                                                                                                                                                                                                                                                                                                                                                                                                                                                                                                                                                                                                                     Spontaneous speech differs in several ways from the sentences often studied in psycholinguistics experiments. One important difference is that naturally produced utterances often contain disfluencies. In this study, we examined how the presence of uh in a spoken sentence might affect processes that assign syntactic structure (i.e., parsing). Four experiments are reported. In the first, participants judged the grammaticality of sentences that had disfluencies either right before the head noun of the ambiguous phrase or after (e.g., Sandra bumped into the busboy and the uh uh waiter told her to be careful or Sandra bumped into the busboy and the waiter uh uh told her to be careful). Sentences in the latter condition were judged grammatical less often. This result was replicated in the second experiment, in which disfluencies were replaced with environmental sounds. These findings suggest that interruptions can affect syntactic parsing, and the content of the interruption need not be speechlike. In Experiments 3 and 4 we tested whether these effects occurred because listeners use interruptions as cues to help resolve a structural ambiguity. Results from these latter two grammaticality judgment tasks suggest that when an interruption occurs before an ambiguous noun phrase, comprehenders are more likely to assume that the noun phrase is the subject of a new clause rather than the object of an old one, and furthermore, it appears that the parser is relatively insensitive to the form of the interruption. We conclude that disfluencies can influence the parser by signaling a particular structure; at the same time, for the parser, a disfluency might be any interruption to the flow of speech. (C) 2003 Elsevier Science (USA). All rights reserved.
## 165                                                                                                                                                                                                                                                                                                                                                                                                                                                                                                                                                                                                                                                                                                                                                                                                                                                                                                                                                                                                                                                                                                                           When participants name several taxonomically related objects in close succession, they display persistent interference effects. Experimental manipulations of the semantic naming context have been used in two variants, a blocked and a continuous paradigm. Counterintuitively, results from previous studies suggest that the context effects induced by these paradigms arise at distinct levels of processing, namely at the lemma level (blocked paradigm), and at the interface of conceptual and lexical representations (continuous paradigm). In five experiments, both variants of the paradigm were assessed in object naming, semantic classification, word naming, and word-plus-determiner naming tasks. Experiments 1-3 show that participants display semantic context effects only in those tasks that mandatorily require conceptual processing (semantic classification, object naming). Experiment 4 fails to replicate the finding that, in the continuous paradigm, semantic context effects can transfer from object naming to word-plus-determiner naming but not vice versa, instead yielding no transfer in either direction. Experiment 5 demonstrates that the effects seen in semantic classification and object naming influence each other, suggesting that they are causally linked and that they both originate at the conceptual level. The implications of these findings for current accounts of lexical-semantic encoding in word production are discussed. (C) 2013 Elsevier Inc. All rights reserved.
## 166                                                                                                                                                                                                                                                                                                                                                                                                                                                                                                                                                                                                                                                                                                                                                                                                                                                                                                                                                                                                                                                                   The assumption that activation is cascaded implies that the semantic properties of all neighbors of the input word are activated to varying degrees. This assumption is tested using masked priming in a semantic categorization experiment, where the prime belongs to the same category as the target (a congruent prime), or to a different category (an incongruent prime). In Experiment 1, the prime was a nonword neighbor of an exemplar or non-exemplar of the category, and a clear congruence effect was produced, even though the orthographic overlap was fairly low (e.g., lucchibi-zucchini). In Experiment 2, the prime was a word neighbor (e.g., capable-cabbage), which eliminates the possibility that the prime was simply interpreted as equivalent to the nearest task-relevant word, but a congruence effect was still obtained. Experiment 3 replicated this effect. Experiments 4-6 investigated the possible role of the category using a two-alternative forced choice discrimination task, where the task was simply to guess which of two subsequently presented words was more similar in meaning to the masked word. Despite better than chance performance when the masked word was related to one of the alternatives, performance was at chance when the masked word was a neighbor of a word that was related to one of the alternatives, indicating that semantic activation is not normally cascaded. It is concluded that the categorization task fundamentally alters the way in which a masked word is processed. (C) 2015 Elsevier Inc. All rights reserved.
## 167                                                                                                                                                                                                                                                                                                                                                                                                                                                                                                                                                                                                                                                                                                                                                                                                                                                                                                                                                                                                                                                                                                                                                                                                                                                                                                                                        Second-language (L2) acquisition is generally thought to be constrained by maturational factors that circumscribe a critical period for nativelike attainment. Consistent with the maturational view are age effects among learners who begin L2 acquisition prior to, but not after, closure of the putative critical period. Also favoring the maturational account is the scarcity of late L2 learners at asymptote who perform like natives, and weak effects of native language-target language pairings. With Korean and Chinese learners of English, the experimental study of Johnson and Newport (1989) yielded just these types of evidence. Some subsequent studies do not support the critical period account of L2 acquisition constraints, however. Accordingly, we undertook a replication of Johnson and Newport (1989), using the exact methods and materials of the original experiment, and a sample of Spanish natives (n = 61). In keeping with recent research, L2 attainment negatively correlates with age of learning even if learning commences after the presumed end of the critical period. We also find modest evidence of nativelike attainment among late learners. Our data further suggest that the outcome of L2 acquisition may depend on L1-L2 pairings and L2 use. (C) 2001 Academic Press.
## 168                                                                                                                                                                                                                                                                                                                                                                                                                                                                                                                                                                                                                                                                                                                                                                                                                                                                                                                                                                                                                                                                                                                                                                                                                                                                                                                                                                                                                                                                                                                                                                                                                                                                             Recent studies using the masked priming paradigm have reported facilitating effects of syllable primes in French and English word naming (Ferrand, Segui, Grainger, 1996; Ferrand, Segui, & Humphreys, 1997). However, other studies have not been able to replicate these effects in Dutch and English (Schiller, 1998, 1999, 2000). In Experiment 1, using the same stimuli and procedure as Ferrand et al. (1996), we did not replicate the syllable priming effect in French. In Experiments 2a and 2b, when prime duration was increased (from 30 to 45 and 60 ms), we did not obtain a syllable priming effect. In Experiment 3, with 60 participants and exactly the same procedure as Ferrand et al. (1996), we again failed to replicate the syllable priming effect. We conclude that the syllable priming effect is not a reliable effect and should be considered cautiously in the elaboration of models of word reading. (C) 2002 Elsevier Science (USA). All rights reserved.
## 169                                                                                                                                                                                                                                                                                                                                                                                                                                                                                                                                                                                                                                                                                                                                                                                                                                                                                                                                                                                                                                                                                                                                                                                                                                                              In question-answering, speakers display their metacognitive states using filled pauses and prosody (Smith & Clark, 1993). We examined whether listeners are actually sensitive to this information. Experiment 1 replicated Smith and Clark's study; respondents were tested on general knowledge questions, surveyed about their FOK (feeling-of-knowing) for these questions, and tested for recognition of answers. In Experiment 2, listeners heard spontaneous verbal responses from Experiment 1 and were tested on their feeling-of-another's-knowing (FOAK) to see if metacognitive information was reliably conveyed by the surface form of responses. For answers, rising intonation and Longer latencies led to fewer FOAK ratings by listeners. For nonanswers, longer latencies led to higher FOAK ratings. In Experiment 3, electronically edited responses with 1-s latencies led to higher FOAK ratings for answers and lower FOAK ratings for nonanswers than those with 5-s latencies. Filled pauses led to lower ratings for answers and higher ratings for nonanswers than did unfilled pauses. There was no support for a filler-as-morpheme hypothesis, that ''um'' and ''uh'' contrast in meaning. We conclude that listeners can interpret the metacognitive information that speakers display about their states of knowledge in question-answering. (C) 1995 Academic Press, Inc.
## 170                                                                                                                                                                                                                                                                                                                                                                                                                                                                                                                                                                                                                                                                                                                                                                                                                                                                                                                                                                                                                                                                                                                                                                                                                                                                                                                                                                                                                                                                                                                                                        Three experiments examined the hypothesis that it preferentially refers to the most salient entity in a discourse, whereas that preferentially refers to a conceptual composite. In Experiment I eye movements were monitored as participants followed spoken instructions such as, Put the cup on the saucer. Now put it/that.... The preferred referent was the theme (cup) for it and the composite for that (cup on the saucer) with the goal (saucer) rarely chosen. Experiment 2 demonstrated that stressing it reduces the number of theme interpretations. Experiment 3 replicated the main findings from Experiment 1, regardless of whether or not the theme was the backward-looking center. The authors conclude that entities without linguistic antecedents are sometimes preferred over entities with linguistic antecedents and a single construct such as salience is insufficient to account for differences among referential forms. Candidate reference-specific constructs include the availability of conceptual composites and syntactic role. (c) 2005 Elsevier Inc. All rights reserved.
## 171 In dialog settings, conversational partners converge on similar names for referents. These lexically entrained terms [Garrod, S., & Anderson, A. (1987). Saying what you mean in dialog: A study in conceptual and semantic co-ordination. Cognition, 27, 181-218] are part of the common ground between the particular individuals who established the entrained term [Brennan, S. E., & Clark, H. H. (1996). Conceptual pacts and lexical choice in conversation. Journal of Experimental Psychology: Learning, Memory, and Cognition, 22, 1482-1493], and are thought to be encoded in memory with a partner-specific cue. Thus far, analyses of the time-course of interpretation suggest that partner-specific information may not constrain the initial interpretation of referring expressions [Barr, D. J., & Keysar, B. (2002). Anchoring comprehension in linguistic precedents. Journal of Memory and Language, 46, 391-418; Krommuller, E., & Barr, D. J. (2007). Perspective-free pragmatics: Broken precedents and the recovery-from-preemption hypothesis. Journal of Memory and Language, 56, 436-455]. However, these studies used non-interactive paradigms, which may limit the use of partner-specific representations. This article presents the results of three eye-tracking experiments. Experiment la used an interactive conversation methodology in which the experimenter and participant jointly established entrained terms for various images. On critical trials, the same experimenter, or a new experimenter described a critical image using an entrained term, or a new term. The results demonstrated an early, online partner-specific effect for interpretation of entrained terms, as well as preliminary evidence for an early, partner-specific effect for new terms. Experiment 1b used a non-interactive paradigm in which participants completed the same task by listening to image descriptions recorded during Experiment 1a; the results showed that partner-specific effects were eliminated. Experiment 2 replicated the partner-specific findings of Experiment 1 a with an interactive paradigm and scenes that contained previously unmentioned images. The results suggest that partner-specific interpretation is most likely to occur in interactive dialog settings; the number of critical trials and stimulus characteristics may also play a role. The results are consistent with a large body of work demonstrating that the language processing system uses a rich source of contextual and pragmatic representations to guide on-line processing decisions. (C) 2009 Elsevier Inc. All rights reserved.
## 172                                                                                                                                                                                                                                                                                                                                                                                                                                                                                                                                                                                                                                                                                                                                                                                                                                                                                                                                                                                                                                                                                                                                                                                                                                            The present study investigated the role of spelling-sound consistency in naming printed disyllabic words. Participants in Experiment 1 named 1000 monomorphemic six-letter disyllabic words. Spelling-sound consistency measures for 11 orthographic segments were used to predict the naming latencies and error rates on the words. The consistency of vowel segments, particularly the one in the second syllable, contributed significantly to the prediction of naming latencies and error rates. In addition, the consistency of the BOB (body-of-the-BOSS, which is the orthographic segment containing the first vowel grapheme and as many following consonants as make an orthographically legal word ending) was also a significant predictor. The effect of the spelling-sound consistency of BOB and V-2 segments was replicated in factorial experiments. These findings suggest that readers learn spelling-sound relationships not only for individual letters of graphemes but also for larger orthographic segments in disyllabic words, likely those that provide information about pronunciations beyond that of the individual letters of which they are composed. This study provides the kind of information that is needed to extend current models of word recognition beyond their current focus on monosyllabic words to more complex words. (C) 2002 Elsevier Science (USA). All rights reserved.
## 173                                                                                                                                                                                                                                                                                                                                                                                                                                                                                                                                                                                                                                                                                                                                                                                                                                                                                                                                               Dynamic visual noise (DVN), an array of squares that randomly switch between black and white, interferes with certain tasks that involve visuo-spatial processing. Based on the assumption that the representation of concrete words includes an imagistic code whereas that of abstract words does not, Parker and Dagnall (2009) predicted that DVN should disrupt visual working memory and selectively interfere with memory for concrete words. They observed a reversal of the concreteness effect in both a delayed free recall and a delayed recognition test. In six studies, we partially replicate and extend their work. In Experiments 1 (delayed free recall) and 2 (delayed recognition), DVN abolished, but did not reverse, the concreteness effect. Experiments 3 and 4 found no effect of DVN on a prototypical working memory task, immediate serial recall: concreteness effects were observed in both the control and DVN conditions. In contrast, Experiment 5 showed that DVN abolished the concreteness effect in an immediate serial recognition test. In the final experiment, subjects did not know whether they would receive an immediate serial recall or an immediate serial recognition test until after the list had been presented. Nonetheless, DVN had no effect on immediate serial recall but once again eliminated the concreteness effect on immediate serial recognition. The results (1) extend the effects of DVN on the concreteness effect to working memory tasks, (2) suggest that immediate serial recall and immediate serial recognition are more different than similar, and (3) have implications for theories of DVN, the concreteness effect, and models of memory.
## 174                                                                                                                                                                                                                                                                                                                                                                                                                                                                                                                                                                                                                                                                                                                                                                                                                                                                                                                                                                                                                                                                                                                                                                                                                                                                                                                                                                                                                                                                                                                                                                                                                                                                                                                  Three experiments investigated the basis of false recognition errors caused by category repetition. The subjective nature of the false alarms was measured by asking participants to make remember-know decisions to each item they judged as old. Experiment 1 replicated the finding by Dewhurst and Anderson (1999) that false remember responses to nonstudied category members increased with the number of items from the same category that were presented at encoding. Participants in Experiment 2 made more false remember responses to category members of high instance frequency than to members of low instance frequency. Participants in Experiment 3 made more false remember responses to members of small categories than to members of large categories. These findings support the view that the false positive remember responses result from associative responses made to items presented at encoding. (C) 2001 Academic Press.
## 175                                                                                                                                                                                                                                                                                                                                                                                                                                                                                                                                                                                                                                                                                                                                                                                                                                                                                                                                                                                                                                                                                                                                                                                                                                                                                                                                          Recent work shows that word segmentation is influenced by distal prosodic characteristics of the input several syllables from the segmentation point (Dilley & McAuley, 2008). Here, participants heard eight-syllable sequences with a lexically ambiguous four-syllable ending (e.g., crisis turnip vs. cry sister nip). The prosodic characteristics of the initial five syllables were resynthesized in a manner predicted to favor parsing of the final syllables as either a monosyllabic or a disyllabic word; the acoustic characteristics of the final three syllables were held constant. Experiments la-c replicated earlier results showing that utterance-initial prosody influences segmentation utterance-finally, even when lexical content is removed through low-pass filtering, and even when an on-line cross-modal paradigm is used. Experiments 2 and 3 pitted distal prosody against, respectively, distal semantic context and prosodic attributes of the test words themselves. Although these factors jointly affected which words participants heard, distal prosody remained an extremely robust segmentation cue. These findings suggest that distal prosody is a powerful factor for consideration in models of word segmentation and lexical access. (C) 2010 Elsevier Inc. All rights reserved.
## 176                                                                                                                                                                                                                                                                                                                                                                                                                                                                                                                                                                                                                                                                                                                                                                                                                                                                                                                                                                                                                                                                                                                                                                                                          Baddeley, Gathercole, and Papagno (1998) proposed a model of associative word learning in which the phonological loop, as defined in Baddeley's working memory model, is primarily a language learning device, rather than a mechanism for the memorization of familiar words. Using a dual-task paradigm, Papagno, Valentine, and Baddeley (1991) found that articulatory suppression, loading verbal working memory, had an effect on the memorizing of word-nonword pairs, but not on the memorizing of word-concrete word pairs. The present work explored the potential for visual codes in unfamiliar word learning. In a first experiment, we replicated the results of Papagno et al. for both nonwords and highly imageable nouns. In addition, we found that articulatory suppression disrupted the memorizing of word-abstract word pairs, suggesting that phonological involvement may be triggered by the absence of visual representations for the abstract words. Experiment 2 showed that an artificially induced association between a nonword and a nonnameable visual image was sufficient to compensate for diminished verbal working memory resources due to articulatory suppression. In a third experiment, we demonstrated that our results generalize to other types of abstract words (i.e., function words), auditory stimulus presentation, and to word learning in children. (C) 2002 Elsevier Science (USA). All rights reserved.
## 177                                                                                                                                                                                                                                                                                                                                                                                                                                                                                                                                                                                                                                                                                                                                                                                                                                                                                                                                                                                                                                                                                                                                                                                                                                                                                                                                                                                                                                                                                                                                                                 Matching bias in conditional reasoning consists of a tendency to select as relevant cases whose lexical content marches thai referred to in the conditional statement, regardless of the presence of negatives. Evans (1983) demonstrated that use of explicit rather than implicit negative eases markedly reduced the matching bias effect on the conditional truth table task. In apparent contrast, recent studies of explicit negation on the Wason selection task have failed to iind evidence of logical facilitation. Experiment I of the present study strongly replicated the Evans (1983) findings and extended them to three forms of conditional statement. Experiments 2 and 3 showed further that the use of explicit negatives removed completely the matching bias effect on the Wason selection task. However, consistent with other recent studies, this elimination of bias did not lead to facilitation of correct responding. The findings are interpreted as providing evidence chat matching bias reflects a linguistically cued relevance effect. (C) 1996 Academic Press, Inc.
## 178                                                                                                                                                                                                                                                                                                                                                                                                                                                                                                                                                                                                                                                                                                                                                                                                                                                                                                                                                                                                                                                                                                                                Two experiments explored false recall of unstudied critical items (e.g., chair) following the presentation of 16 semantic associates to the critical word (e.g., sit, desk), 16 phonological associates to the critical word (e.g., cheer, hair), and every composition of hybrid list in between (e.g., 14 semantic and 2 phonological associates). Results replicated the over additive pattern of critical false recall from hybrid lists relative to pure lists found by Watson, Balota, and Roediger (2003) and clarified the form of the false recall function across varying degrees of hybridization. Both experiments showed that including just one or two of the other type of associate in an otherwise pure list led to a considerable increase in false recall. A within-subjects design (Experiment 1) suggested that after this initial rapid increase, false recall continued to increase gradually to an apex at the balanced hybrid list composition, whereas a between-subjects design (Experiment 2) showed that false recall plateaued after the initial rapid increase and that the overall shape of the function is a ziggurat. Furthermore, the function is roughly symmetrical; semantic and phonological associates appear to make equivalent contributions to over-additive false recall from hybrid lists. The results provide constraints on theoretical accounts of DRM false memories, and can be accommodated by a modified activation/monitoring framework. (C) 2016 Elsevier Inc. All rights reserved.
## 179                                                                                                                                                                                                                                                                                                                                                                                                                                                                                                                                                                                                                                                                                                                                                                                                                                                                                                                                                                                                                                                                                                                                                                                                                                                                                                                                                                                                                     We contrasted two hypotheses concerning how speakers determine adjective order during referential communication. The discriminatory efficiency hypotheses claims that speakers place the most discriminating adjective early to facilitate referent identification. By contrast, the availability-based ordering hypothesis assumes that speakers produce most available adjectives early to ease production. Experiment 1 showed that speakers use more pattern-before-color modifier orders (than the reversed) when pattern, not color, distinguished the referent from alternatives, providing support for the discriminatory efficiency hypothesis. Participants also overspecified color more often than pattern, and they generally favored color-before-pattern orders, in support of the availability-based ordering hypothesis. Experiments 2 and 3 replicated both effects in a dialogue setting, where speakers' adjective ordering was also primed by their partner's ordering, using conjoined and non-conjoined constructions. We propose a novel model (PASS) that explains how discriminability and availability simultaneously influence adjective selection and ordering via competition in the speaker's message representation.
## 180                                                                                                                                                                                                                                                                                                                                                                                                                                                                                                                                                                                                                                                                                                                                                                                                                                                                                                                                                                                                                                                                                                                                                                                                                                                                                                         Eyetracking and the self-paced moving-window reading paradigm were used in two experiments examining the contributions of both frequency-based verb biases and the plausibility of particular word combinations to the comprehension of temporarily ambiguous sentences. The temporary ambiguity concerned whether a noun following a verb was its direct object (The senator regretted the decision immediately.), or instead the subject of an embedded clause (The senator regretted the decision had been made public.). The experiments crossed the plausibility of the temporarily ambiguous noun as a direct object (e.g., The senator regretted the decision ... vs The senator regretted the reporter ...) with verb bias, eliminating a confound present in earlier research and allowing an examination of interactions between the two factors. Unbiased verbs were included as well to evaluate the role of plausibility in the absence of verb bias. The results generally replicated Trueswell, Tanenhaus, and Kello's (1993) finding that verb bias has rapid effects on ambiguity resolution, and showed in addition that verb bias and plausibility interact during comprehension. The results are most consistent with parallel interactive models of language comprehension such as constraint satisfaction models. (C) 1997 Academic Press.
## 181                                                                                                                                                                                                                                                                                                                                                                                                                                                                                                                                                                                                                                                                                                                                                                                                                                                                                                                                                                                                                                                                                                                                                                                                                                                                                                                                 Four experiments were conducted to examine the concreteness effect ill implicit and explicit memory measures. Experiment 1 replicated prior reports of an imagery effect on an implicit conceptual memory test. In Experiment 2. we confirmed our prediction of conceptual sensitivity of free recall, explicit general knowledge, explicit word fragment completion, and the implicit general knowledge tests with a levels of processing manipulation. Furthermore, although we obtained the concreteness effect (better memory for concrete than abstract nouns) in free recall and we explicit general knowledge test, we failed to find this effect in the implicit general knowledge test. Experiment 3 revealed that the failure to iind the concreteness effect on the implicit general knowledge test was not attributable to combining two encoding manipulations in Experiment 2. In Experiment 4. we ruled out the possibility that the failure to find the concreteness effect in conceptual implicit memory may be related to the number of meaningful associates for targets, We discuss the implications of these findings within the context of the transfer appropriate processing framework (Roediger, Weldon, & Challis, 1989) and the dual-code hypothesis (Paivio, 1971, 1991) of memory. (C) 2001 Academic Press.
## 182                                                                                                                                                                                                                                                                                                                                                                                                                                                                                                                                                                                                                                                                                                                                                                                                                                                                                                                                                                                                                                                                                                                                                                        The lexical bias effect is the tendency for phonological speech errors to result in words more often than in nonwords. This effect has been accounted for by postulating feedback from sublexical to lexical representations, but also by assuming that the self-monitor covertly repairs more nonword errors than word errors. The only evidence that appears to exclusively support a monitoring account is Baars, Motley, and MacKay's (1975) demonstration that the lexical bias is modulated by context: There was lexical bias in a mixed context of words and nonwords, but not in a pure nonword context. However, there are methodological problems with that experiment and theoretical problems with its interpretation. Additionally, a recent study failed to replicate contextual modulation (Humphreys, 2002). We therefore conducted two production experiments that solved the methodological problems. Both experiments showed there is indeed contextual modulation of the lexical bias effect. A control perception experiment excluded the possibility that the comprehension component of the task contributed to the results. In contrast to Baars et al., the production experiments suggested that lexical errors are suppressed in a nonword context. This supports a new account by which there is both feedback and self-monitoring, but in which the self-monitor sets its criteria adaptively as a function of context. (C) 2004 Elsevier Inc. All rights reserved.
## 183                                                                                                                                                                                                                                                                                                                                                                                                                                                                                                                                                                                                                                                                                                                                                                                                                                                                                                                                                                                                                                                                                                                                                                                                                                                                                                                                                                                                                                                                                                Does recognition memory rely on discrete recollection, continuous evidence, or both? Is continuous evidence sensitive to only the recency and duration of study (familiarity), or is it also sensitive to details of the study episode? Dual process theories assume recognition is based on recollection and familiarity, with only recollection providing knowledge about study details. Single process theories assume a single continuous evidence dimension that can provide information about familiarity and details. We replicated list (Yonelinas, 1994) and plural (Rotello, Macmillan, & Van Tassel, 2000) discrimination experiments requiring knowledge of details to discriminate targets from similar non-targets. We also ran modified versions of these experiments aiming to increase recollection by removing non-targets that could be discriminated by familiarity alone. Single process models provided the best trade-off between goodness-of-fit and model complexity and dual process models were able to account for the data only when they incorporated continuous evidence sensitive to details. (c) 2006 Elsevier Inc. All rights reserved.
## 184                                                                                                                                                                                                                                                                                                                                                                                                                                                                                                                                                                                                                                                                                                                                                                                                                                                                                                                                                                                                                                                                                                                                                                                                                                                                                                                                                                                                         Based on previous reports of bilinguals' reduced non-linguistic switch cost, we explored how bilingualism affects various task-switching mechanisms. We tested different groups of Spanish monolinguals and highly-proficient Catalan-Spanish bilinguals in different task-switching implementations. In Experiment 1 we disengaged the restart cost typically occurring after a cue from the switch cost itself using two cue-task versions varying in explicitness. In Experiment 2 we tested bilingualism effects on overriding conflicting response sets by including bivalency effects. In Experiment 3 we attempted to replicate the reduced switch cost of bilinguals with the same implementation as in previous studies. Relative to monolinguals, bilinguals showed a reduced restart cost in the implicit cue-task version of Experiment 1 and overall faster response latencies in Experiment 2. However, bilinguals did not show reduced switch cost in any experiment - not even in an omnibus analysis combining the standardized switch cost scores of 292 participants across the three experiments. These results qualify previous claims about bilingualism reducing non-linguistic switch costs. (C) 2013 Elsevier Inc. All rights reserved.
## 185                                                                                                                                                                                                                                                                                                                                                                                                                                                                                                                                                                                                                                                                                                                                                                                                                                                                                                                                                                                                                                                                                                                                                                                                                                  False memories arising from associatively related lists are a robust phenomenon that resists many efforts to prevent it. However, a few variables have been shown to reduce this form of false memory. Explanations for how the reduction is accomplished have focused on either output monitoring processes or constraints on access, but neither idea alone is sufficient to explain extant data. Our research was driven by a framework that distinguishes item-based and event-based distinctive processing to account for the effects of different variables on both correct recall of study list items and false recall. We report the results of three experiments examining the effect of a deep orienting task and the effect of visual presentation of study items, both of which have been shown to reduce false recall. The experiments replicate those previous findings and add important new information about the effect of the variables on a recall test that eliminates the need for monitoring. The results clearly indicate that both post-access monitoring and constraints on access contribute to reductions in false memories. The results also showed that the manipulations of study modality and orienting task had different effects on correct and false recall, a pattern that was predicted by the item-based/event-based distinctive processing framework. (C) 2011 Elsevier Inc. All rights reserved.
## 186                                                                                                                                                                                                                                                                                                                                                                                                                                                                                                                                                                                                                                                                                                                                                                                                                                                                                                                                                                                                                                                                                                                                                                                                                                                                                                                                                                                                                                                                                                                                                                                                                                          Cue-based retrieval theories in sentence processing predict two classes of interference effect: (i) Inhibitory interference is predicted when multiple items match a retrieval cue: cue-overloading leads to an overall slowdown in reading time; and (ii) Facilitatory interference arises when a retrieval target as well as a distractor only partially match the retrieval cues; this partial matching leads to an overall speedup in retrieval time. Inhibitory interference effects are widely observed, but facilitatory interference apparently has an exception: reflexives have been claimed to show no facilitatory interference effects. Because the claim is based on underpowered studies, we conducted a large-sample experiment that investigated both facilitatory and inhibitory interference. In contrast to previous studies, we find facilitatory interference effects in reflexives. We also present a quantitative evaluation of the cue-based retrieval model of Engelmann, Jager, and Vasishth (2019).
## 187                                                                                                                                                                                                                                                                                                                                                                                                                                                                                                                                                                                                                                                                                                                                                                                                                                                                                                                                                                                                                                                                                                                                                                                                                                                                                                                                                                                                                                                                                                         In three experiments we tested the predictions of two models of determiner selection in the production of Dutch noun phrases (NPs). In Experiment 1, participants named pictures using plural and unmarked determiner + noun NPs. In Experiment 2, participants named pictures using diminutive and unmarked determiner + noun NPs. In both experiments, we found that production latencies for plural and diminutive NPs relative to their unmarked baselines were affected by the gender of the base noun even though this feature of nouns is logically unnecessary in the selection of determiners in these types of NPs. In Experiment 3, we replicated the findings of Experiments 1, and generalized the observed pattern of results to a new condition: plural-diminutive NPs. This pattern of results, showing that the gender of the base noun is visible to the determiner selection process even when this information is logically superfluous, finds a ready explanation in frame-based models of determiner selection and is inconsistent with hierarchical models of determiner selection. (C) 2002 Elsevier Science (USA). All rights reserved.
## 188                                                                                                                                                                                                                                                                                                                                                                                                                                                                                                                                                                                                                                                                                                                                                                                                                                                                                                                                                                                                                                                                                                                                 Recent research has shown much evidence that sentence comprehension can be extremely predictive. However, we currently know little about the limits of predictive processing. In the two eye-tracking experiments, we examined whether predictive information in dependency formation is inevitably given priority over a well-known structural preference in syntactic ambiguity resolution. Experiment 1 used sentences including control nouns like order (e.g. After Andrew's order to wash the kids came over to the house). If predictive dependency information is given priority over disambiguation preferences, then readers could immediately interpret the kids as the ones who have been ordered to wash, thus avoiding the garden path at the main verb came. However, garden path effects were found irrespective of control information, although the garden path difficulty was reduced when the lexical control information highlighted the globally correct analysis (as in the above example), relative to when it did not. Experiment 2 replicated these results with adjunct control, where the relevant dependency is obligatory (e.g. After refusing to wash the kids came over to the house). Again, control information did not influence initial disambiguation, but did affect the difficulty of garden path recovery. Overall, the results suggest that there are limitations on the influence of predictive dependency formation on on-line structural disambiguation. (C) 2014 Published by Elsevier Inc.
## 189                                                                                                                                                                                                                                                                                                                                                                                                                                                                                                                                                                                                                                                                                                                                                                                                                                                                                                                                                                                                                                                                                                                                                                                                                                                                                                                                                                                                                                                                                                                    Do readers keep track of protagonists' common ground? If so, what role does this awareness play in reactivating distant information? Using passages in which protagonists part and later reunite, in Experiment 1 we replicated the finding (Greene, Gerrig, McKoon, & Ratcliff, 1994) that their reunion reactivates text elements that were part of their common ground. In Experiment 2, we tested two accounts of this effect: (a) readers reactivate the contents of protagonists' common ground by way of preparing for their interaction; and (b) low-level memory processes reactivate text elements associated with the protagonist returning to the reader's focus. We found that the pattern of reactivation was almost identical whether the text element was part of common ground or not, supporting the memory-based account. In Experiment 3 we obtained evidence that despite the lack of role for common ground in reactivating distant information, readers are indeed aware of who knows what about whom in short texts. The results are discussed in terms of memory-based theories of text processing. (C) 1998 Academic Press.
## 190                                                                                                                                                                                                                                                                                                                                                                                                                                                                                                                                                                                                                                                                                                                                                                                                                                                                                                                                                                                                                                                                                                                                                                                                                                                                                                                                                                                                                                Two event-related potential experiments investigated the effects of syntactic and semantic context information on the processing of noun/verb (NV) homographs (e.g., park). Experiment 1 embedded NV-homographs and matched unambiguous words in contexts that provided only syntactic Cues or both syntactic and semantic constraints. Replicating prior work, when only syntactic information was available NV-homographs elicited sustained frontal negativity relative to unambiguous words. Semantic constraints eliminated this frontal ambiguity effect. Semantic constraints also reduced N400 amplitudes, but less so for homographs than unambiguous words. Experiment 2 showed that this reduced N400 facilitation was limited to cases in which the semantic context picks out a non-dominant meaning, likely reflecting the semantic mismatch between the context and residual, automatic activation of the contextually-inappropriate dominant sense. Overall, the findings suggest that ambiguity resolution in context involves the interplay between multiple neural networks, some involving more automatic semantic processing mechanisms and others involving top-down control mechanisms. Published by Elsevier Inc.
## 191                                                                                                                                                                                                                                                                                                                                                                                                                                                                                                                                                                                                                                                                                                                                                                                                                                                                                                                                                                                                                                                                                                                                                                                                                                                                                                                                                                                                                                           Twelve experiments examined the effect of neighborhood density on repetition latency for nonwords. Previous reports have indicated that nonwords from high density neighborhoods are repeated with shorter latency than nonwords from low density neighborhoods (e.g., Vitevitch & Luce, 1998). Experiment I replicated these previously reported results; however, further analysis indicated an interaction of neighborhood density and stimulus duration in determining nonword repetition latency. Experiment 2 employed stimuli with reduced durational differences, finding that there was no effect of neighborhood density on repetition latency when stimulus duration was statistically controlled. Experiments 3 and 4 replicated these results with an alternative presentation regimen. Experiments 5-12 repeated these investigations with different stimulus sets, and obtained consistent effects of stimulus duration on repetition latency, and either no effect of neighborhood density or a latency advantage for low density rather than high density nonwords. The theoretical implications of these results for models of lexical processing are discussed. (C) 2004 Elsevier Inc. All rights reserved.
## 192                                                                                                                                                                                                                                                                                                                                                                                                                                                                                                                                                                                                                                                                                                                                                                                                                                                                                                                                                                                                                                          According to Levelt, Roelofs, and Meyer (1999) speakers generate the phonological and phonetic representations of successive syllables of a word in sequence and only begin to speak after having fully planned at least one complete phonological word. Therefore, speech onset latencies should be longer for long than for short words. We tested this prediction in four experiments in which Dutch participants named or categorized objects with monosyllabic or disyllabic names. Experiment I yielded a length effect on production latencies when objects with long and short names were tested in separate blocks, but not when they were mixed. Experiment 2 showed that the length effect was not due to a difference in the ease of object recognition. Experiment 3 replicated the results of Experiment I using a within-participants design. In Experiment 4, the long and short target words appeared in a phrasal context. In addition to the speech onset latencies, we obtained the viewing times for the target objects, which have been shown to depend on the time necessary to plan the form of the target names. We found word length effects for both dependent variables, but only when objects with short and long names were presented in separate blocks. We argue that in pure and mixed blocks speakers used different response deadlines, which they tried to meet by either generating the motor programs for one syllable or for all syllables of the word before speech onset. Computer simulations using WEAVER++ support this view. (C) 2002 Elsevier Science (USA). All rights reserved.
## 193                                                                                                                                                                                                                                                                                                                                                                                                                                                                                                                                                                                                                                                                                                                                                                                                                                                                                                                                                                                                                                                                                                                                                            Processing items for their survival relevance often produces a robust memory advantage. The current experiments assessed possible proximate mechanisms responsible for this advantage by assessing output strategies during free recall. Previous research has shown that item clustering during recall can provide diagnostic information about the structure of representations in episodic memory, particularly the encoding of temporal, semantic, and source information. Following survival processing and moving or pleasantness controls, measures of temporal and semantic clustering were generated. A robust recall advantage was found for survival processing, but no evidence for temporal clustering was detected. Above-chance levels of semantic clustering were obtained, but there were no differences between the survival and control conditions. An additional clustering measure based on scenario-based relevance ratings also failed to explain recall differences, as did absolute and relative measures of remembered temporal position. Our results indicate that neither enhanced temporal coding nor increased semantic processing among the items on the study list can easily explain the oft-replicated survival processing advantage. Our results also suggest that the ubiquitous temporal clustering patterns seen in free recall studies may be a product, in part, of using intentional learning and multiple study trials. (C) 2016 Elsevier Inc. All rights reserved.
## 194                                                                                                                                                                                                                                                                                                                                                                                                                                                                                                                                                                                                                                                                                                                                                                                                                                                                                                                                                                                                                                                                                                                                                                                                                                                                                 The present study sought to clarify whether phonological similarity of encoded information impairs free recall performance (the phonological similarity effect: PSE) for nonwords. Five experiments examined the influence of the encoding process on the PSE in a step-by-step fashion, by using lists that consisted of phonologically similar (decoy) digit-(target) nonword pairs. Experiment 1 demonstrated the PSE when participants were required to recall both digits and nonwords. Experiment 2a used the same lists, but participants were required to recall either digits or nonwords (but not both). The PSE disappeared under these conditions but appeared again in Experiment 2b, in which the procedure was the same as Experiment 2a except that participants engaged in articulatory rehearsal during an 8 s recall delay. Experiment 2c replicated Experiments 2a and 2b, using a within-participants design. Experiment 3 demonstrated the PSE when participants articulated the digits as a distractor activity. These results suggest that encoding of phonologically similar items, including the re-encoding of items that is associated with rehearsal, disrupts other to-be-stored item information. This would be evidence for interference-based forgetting of item-specific information in short-term memory. (c) 2012 Elsevier Inc. All rights reserved.
## 195                                                                                                                                                                                                                                                                                                                                                                                                                                                                                                                                                                                                                                                                                                                                                                                                                                                                                                                                                                                                                                                                                                                                                                                                                                                                                                                                                                                                                                                                                                                                                                                                                                                                                                                                                                           In six experiments we investigated priming for perceptually related word pairs (i.e., words that refer to objects with the same shape such as pizza-coin), trying to replicate earlier findings by Schreuder, Flores d'Arcais, and Glazenborg (1984) while avoiding some of the methodological problems that were present in that study. Under standard conditions no perceptual priming was obtained. However, in all experiments priming for associated pairs was found. Only after activation tasks that focused on perceptual features was priming for perceptually related word pairs found in pronunciation. Perceptual priming was also obtained in lexical decision after activation tasks, but only when strong associates were not presented in the experiment. The results show that priming for perceptually related word pairs is not a general finding, (C) 1998 Academic Press.
## 196                                                                                                                                                                                                                                                                                                                                                                                                                                                                                                                                                                                                                                                                                                                                                                                                                                                                                                                                                                                                                                                                                                                                                                                                                                                                                                                                                                                                                                                                                                                          One key issue for models of bilingual memory is to what degree the semantic representation from one of the languages is shared with the other language. In the present paper, we examine whether there is an early, automatic semantic priming effect across languages for noncognates with highly proficient (Basque/Spanish) bilinguals. Experiment I was a between-language masked semantic priming lexical decision experiment. Results showed a significant between-language semantic priming effect for both Basque-Spanish and Spanish-Basque pairs. Experiment 2 showed that the magnitude of the between-language and within-language masked semantic priming effects was quite similar. Experiment 3 replicated the findings of Experiment 2 with highly proficient bilinguals whose mother tongue was Spanish. Thus, highly fluent bilinguals develop early and automatic between-language links with noncognates at the semantic level, as predicted by the hierarchical revised model and the BIA+ model. We examine the implications of these results for models of bilingual memory. (C) 2008 Elsevier Inc. All rights reserved.
## 197                                                                                                                                                                                                                                                                                                                                                                                                                                                                                                                                                                                                                                                                                                                                                                                                                                                                                                                                                                                                                                                                                                                                                                                                                                                                                                                                                                                                                                                                                                                                    Why are testing effects on memory stronger when practice tests involve free recall versus recognition? Three experiments tested the hypothesis that relational processing is evoked to a greater extent during free recall practice than during recognition practice. Students studied a list of words from taxonomic categories and then either restudied the word list several times or alternated between practice testing (either free recall or recognition) and restudy. Two days later, all groups completed final free recall and recognition tests. Replicating prior research, final free recall was greater after free recall practice versus recognition practice. Importantly, performance on measures of relational processing (clustering and category access) was greater following free recall practice versus recognition practice. The practice testing groups also showed unexpected but theoretically interesting differences in item-specific processing. The current work sheds light on why practice test format moderates testing effects; we discuss implications for existing theories of testing effects.
## 198                                                                                                                                                                                                                                                                                                                                                                                                                                                                                                                                                                                                                                                                                                                                                                                                                                                                                                                                                                                                                                                                                        Retrieval-induced forgetting (RIF) refers to the finding that retrieval practice on a subset of studied items can induce later forgetting of related unpracticed items. Although previous studies indicated that RIF is retrieval specific - i.e., it arises after retrieval practice but not after reexposure cycles -, the results of more recent work suggest otherwise, indicating that some reexposure formats can induce RIF very similar to how retrieval practice does. Whereas this prior work employed recall at test, here we revisited retrieval specificity of RIF employing item recognition. The results of three experiments are reported, which examined the effects of retrieval practice and some of the recently suggested reexposure formats on unpracticed items' recognition. In each of these experiments, we showed RIF after retrieval practice but did not find any evidence for RIF-like forgetting after reexposure. These findings demonstrate retrieval specificity of RIF in item recognition, challenging strength-based accounts of RIF and indicating a critical role of inhibition in RIF. Together with the results from the recent recall studies, which we replicated in three further experiments, the present findings are consistent with a two-factor account of RIF, which assigns a role for both inhibition and strength-based blocking in RIF. While both inhibition and blocking may contribute to RIF in certain recall formats, only inhibition may induce RIF in item recognition. (C) 2015 Elsevier Inc. All rights reserved.
## 199                                                                                                                                                                                                                                                                                                                                                                                                                                                                                                                                                                                                                                                                                                                                                                                                                                                                                                                                                                                                                                                                                                                                                                                                                                       Four experiments examined the nature of forgetting and the processing-storage relationship during performance on a prevalent working memory task, the reading span test. Using two different presentation paradigms, Experiments I and 2 replicated Towse, Hitch, and Hutton's (1998, 2000) finding that the Short-Final lists, which presented a long sentence first and a short sentence last, led to better recall performance than the reverse-order Long-Final lists. This effect was still obtained when the retention duration for the target words was held constant and the amount of sentence processing required during that interval was varied (Experiment 3). However, the effect disappeared when the retention duration was varied while holding constant the amount of sentence processing required (Experiment 4). These results suggest that the amount of processing activities, not the sheer passage of time, may be the critical factor underlying the sentence order effect, thereby challenging purely time-based explanations of forgetting during reading span performance. In addition, the analysis of reading times (Experiment 1) revealed that the number of memory items had a subtle yet reliable negative effect on reading times, suggesting that the processing and storage requirements of the reading span test are not completely independent. (C) 2003 Elsevier Inc. All rights reserved.
## 200                                                                                                                                                                                                                                                                                                                                                                                                                                                                                                                                                                                                                                                                                                                                                                                                                                                                                                                                                                                                                                                                                                                                                                                                                                                                                                                                                                                                                                                                                                                                                                                                                                                                                         Performance on a wide variety of memory tasks can be hypothesized to be influenced by processes associated with controlling the contents of memory. In this project 328 adults ranging from 18 to 93 years of age performed six tasks (e.g., multiple trial recall with an interpolated interference list, directed forgetting, proactive interference, and retrieval inhibition) postulated to yield measures of the effectiveness of memory control. Although most of the patterns from earlier studies were replicated, only a few of the measures of memory control were reliable at the level of individual differences. Furthermore, the memory control measures had very weak relations with the age of the participant. Analyses examining the relations between established cognitive abilities and variables from the experimental tasks revealed that most of the variables were related only to episodic memory ability. (c) 2006 Elsevier Inc. All rights reserved.
## 201                                                                                                                                                                                                                                                                                                                                                                                                                                                                                                                                                                                                                                                                                                                                                                                                                                                                                                                                                                                                                                     Semantic preview benefit in reading is an elusive and controversial effect because empirical studies do not always (but sometimes) find evidence for it. Its presence seems to depend on (at least) the language being read, visual properties of the text (e.g., initial letter capitalization), the type of relationship between preview and target, and as shown here, semantic constraint generated by the prior sentence context. Schotter (2013) reported semantic preview benefit for synonyms, but not semantic associates when the preview/target was embedded in a neutral sentence context. In Experiment 1, we embedded those same previews/targets into constrained sentence contexts and in Experiment 2 we replicated the effects reported by Schotter (2013; in neutral sentence contexts) and Experiment 1 (in constrained contexts) in a within-subjects design. In both experiments, we found an early (i.e., first-pass) apparent preview benefit for semantically associated previews in constrained contexts that went away in late measures (e.g., total time). These data suggest that sentence constraint (at least as manipulated in the current study) does not operate by making a single word form expected, but rather generates expectations about what kinds of words are likely to appear. Furthermore, these data are compatible with the assumption of the E-Z Reader model that early oculomotor decisions reflect hedged bets that a word will be identifiable and, when wrong, lead the system to identify the wrong word, triggering regressions. (C) 2015 Elsevier Inc. All rights reserved.
## 202                                                                                                                                                                                                                                                                                                                                                                                                                                                                                                                                                                                                                                                                                                                                                                                                                                                                                                                                                                                                                                                                                                                                                                                                                                                       Five form-based priming experiments examined whether phonological representations are used for lexical access during speech recognition. The effects of final phonological overlap between spoken monosyllabic target words and preceding primes on the recognition of those target words were examined. In Experiment 1, single-word shadowing latencies decreased as the number of phonemes shared between primes and targets increased. The greatest facilitation was observed when primes and targets rhymed. In Experiment 2, the number of shared phonemes was controlled. There was an influence of shared rime on shadowing performance in addition to that due to overlapping segments. Experiment 3 replicated this finding and showed that there was equivalent facilitation with word and nonword primes. Facilitation in shadowing short words was observed in Experiment 4 only when primes and targets rhymed. In Experiment 5, listeners responded to the same primes and targets as in Experiment 4; the items were presented for continuous lexical decision in an undifferentiated list. Facilitation was observed only when targets immediately followed primes with which they rhymed. Final-overlap facilitation appears to reflect two processes: a bias based on the perceptual salience of rhyme, and the activation of prelexical phonological representations. (C) 2000 Academic Press.
## 203                                                                                                                                                                                                                                                                                                                                                                                                                                                                                                                                                                                                                                                                                                                                                                                                                                                                                                                                                                                                                                                                                                                                                                                                                                            A central question for psycholinguistics concerns the role of grammatical constraints in online sentence processing. Many current theories maintain that the language processing mechanism constructs a parse or parses that are grammatically consistent with the whole of the perceived input each time it processes a word. Several bottom-up, dynamical models make a contrasting prediction: partial parses which are syntactically compatible with only a proper subpart of the input are sometimes constructed, at least temporarily. Three self-paced reading experiments probed for interference from such locally coherent structures. The first tested for a distracting effect of irrelevant Subject-Predicate interpretations of Noun Phrase Verb Phrase sequences (e.g., The coach smiled at the player tossed a frisbee) on reading times. The second addressed the question of whether the interference effects can be treated as lexical interference, instead of involving the formation of locally coherent syntactic structures. The third replicated the reading time effects of the first two experiments with grammaticality judgments. We evaluate the dynamical account, comparing it to other approaches that also predict effects of local coherence, and arguing against accounts which rule out the formation of merely locally coherent structures. (C) 2004 Elsevier Inc. All rights reserved.
## 204                                                                                                                                                                                                                                                                                                                                                                                                                                                                                                                                                                                                                                                                                                                                                                                                                                                                                                                                                                                                                                                                                                                                                                                                                                                                                                                                                                                                                     Event-related brain potentials were recorded during RSVP reading to test the hypothesis that quantifier expressions are incrementally interpreted fully and immediately. In sentences tapping general knowledge (Farmers grow crops/worms as their primary source of income), Experiment 1 found larger N400s for atypical (worms) than typical objects (crops). Experiment 2 crossed object typicality with non-logical subject noun phrase quantifiers (most, few). Offline plausibility ratings exhibited the crossover interaction predicted by full quantifier interpretation: Most farmers grow crops and Few farmers grow worms were rated more plausible than Most farmers grow worms and Few farmers grow crops. Object N400s, although modulated in the expected direction, did not reverse. Experiment 3 replicated these findings with adverbial quantifiers (Farmers often/rarely grow crops/worms). Interpretation of quantifier expressions thus is neither fully immediate nor fully delayed. Furthermore, object atypicality was associated with a frontal slow positivity in few-type/rarely quantifier contexts, suggesting systematic processing differences among quantifier types. (C) 2010 Elsevier Inc. All rights reserved.
## 205                                                                                                                                                                                                                                                                                                                                                                                                                                                                                                                                                                                                                                                                                                                                                                                                                                                                                                                                                                                                                                                                                                                                                                  We examined the effects of letter-transposition in Hebrew in three masked-priming experiments. Hebrew, like English has an alphabetic orthography where sequential and contiguous letter strings represent phonemes. However, being a Semitic language it has a non-concatenated morphology that is based on root derivations. Experiment I showed that transposed-letter (TL) root primes inhibited responses to targets derived from the non-transposed root letters, and that this inhibition was unrelated to relative root frequency. Experiment 2 replicated this result and showed that if the transposed letters of the root created a nonsense-root that had no lexical representation, then no inhibition and no facilitation were obtained. Finally, Experiment 3 demonstrated that in contrast to English, French, or Spanish, TL nonword primes did not facilitate recognition of targets, and when the root letters embedded in them consisted of a legal root morpheme, they produced inhibition. These results suggest that lexical space in alphabetic orthographies may be structured very differently in different languages if their morphological structure diverges qualitatively. In Hebrew, lexical space is organized according to root families rather than simple orthographic structure, so that all words derived from the same root are interconnected or clustered together, independent of overall orthographic similarity. (C) 2009 Elsevier Inc. All rights reserved.
## 206                                                                                                                                                                                                                                                                                                                                                                                                                                                                                                                                                                                                                                                                                                                                                                                                                                                                                                                                                                                                                                                                                                                                                                                                                                                                                                                                                                                                                                                                                                                                                                                    Recognition memory tests normally include targets from a study list and comparable new test items (lures). False recognitions occur when participants incorrectly claim that lures had been studied, and they are higher for lures that are phonetically or semantically related to studied words. Brainerd, Reyna, and Kneer (1995) reported that priming a critical lure by testing its related studied word in the preceding position reduced false recognition to a level well below the control rate (false-recognition reversal). Three experiments are reported that failed to replicate false-recognition reversal at the impressive levels reported by Brainerd et al. For some critical contrasts, false recognition to primed lures was significantly higher than to new control lures. For some experimental comparisons there was evidence that priming led to false-recognition suppression, but none of the critical contrasts showed that false recognition was significantly higher to new control lures than to primed experimental lures. (C) 2000 Academic Press.
## 207                                                                                                                                                                                                                                                                                                                                                                                                                                                                                                                                                                                                                                                                                                                                                                                                                                                                                                                                                                                                                                                                                                                                                                                                                                                                                                                                                         Four experiments demonstrate effects of prosodic structure on speech production latencies. Experiments 1 to 3 exploit a modified version of the Sternbeg et al. (1978, 1980) prepared speech production paradigm to look for evidence of the generation of prosodic structure during the final stages of sentence production. Experiment 1 provides evidence that prepared sentence production latency is a function of the number of phonological words that a sentence comprises when syntactic structure, number of lexical items, and number of syllables are held constant. Experiment 2 demonstrated that production latencies in Experiment 1 were indeed determined by prosodic structure rather than the number of content words that a sentence comprised. The phonological word effect was replicated in Experiment 3 using utterances with a different intonation pattern and phrasal structure. Finally, in Experiment 4, an on-line version of the sentence production task provides evidence for the phonological word as the preferred unit of articulation during the on-line production of continuous speech. Our findings are consistent with the hypothesis that the phonological word is a unitof processing during the phonological encoding of connected speech. (C) 1997 Academic Press.
## 208                                                                                                                                                                                                                                                                                                                                                                                                                                                                                                                                                                                                                                                                                                                                                                                                                                                                                                                                                                                                                  The current study examined effects of syllable articulation on eye movements during the silent reading of Chinese sentences, which contained two types of two-character target words whose second characters were subject to dialect-specific variation. In one condition the second syllable was articulated with a neutral tone for northern-dialect Chinese speakers and with a full tone for southern-dialect Chinese speakers (neutral-tone target words) and in the other condition the second syllable was articulated with a full tone irrespective of readers' dialect type (full-tone target words). Native speakers of northern and southern Chinese dialects were recruited in Experiment 1 to examine the effect of dialect-specific articulation on silent reading. Recordings of their eye movements revealed shorter viewing durations for neutral- than for full-tone target words only for speakers of northern but not for southern dialects, indicating that dialect-specific articulation of syllabic tone influenced visual word recognition. Experiment 2 replicated the syllabic tone effect for speakers of northern dialects, and the use of gaze-contingent display changes further revealed that these readers processed an upcoming parafoveal word less effectively when a neutral- than when a full-tone target was fixated. Shorter viewing duration for neutral-tone words thus cannot be attributed to their easier lexical processing; instead, tonal effects appear to reflect Chinese readers' simulated articulation of to-be-recognized words during silent reading. (C) 2014 Elsevier Inc. All rights reserved.
## 209                                                                                                                                                                                                                                                                                                                                                                                                                                                                                                                                                                                                                                                                                                                                                                                                                                                                                                                                                                                                                                                                                                                                                                                                                                                  Three experiments addressed the role of phonological information in visual word recognition using a letter search task. Subjects were presented with a target letter (e.g., ''I'') followed by a letter string (a word, a pseudohomophone, a pseudoword control, or a nonword). Their task was to indicate whether the target letter was present in the letter string. All experiments presented pseudohomophones (e.g., TAIP or BRANE) that either contained a letter (I) that was absent in their sound-alike word (TAPE) or were missing a letter (I) that was present in their sound-alike word (BRAIN). In Experiments 1 and 2, subjects made more miss errors under data-limited conditions when the letter string was a target-present pseudohomophone (TAIP) and more false alarm errors when the letter string was a target-absent pseudohomophone (BRANE). Experiment 2 controlled for a possible confound in the data in terms of word completion strategies. In Experiment 3, we replicated the pseudohomophone disadvantage using resource-limited conditions: detection times were longer for pseudohomophones. The existence of a pseudohomophone disadvantage in a supposedly graphemic task adds further support to the accumulating evidence that phonological information generated from the printed word is an early and major constraint in visual word recognition. (C) 1995 Academic Press, Inc.
## 210                                                                                                                                                                                                                                                                                                                                                                                                                                                                                                                                                                                                                                                                                                                                                                                                                                                                                                                                                                                                                                                                                                                                                                                                                                                               Previous consistency research investigated inconsistency only in the mapping of spelling to phonology (feedforward inconsistency). The present experiments investigated whether inconsistency in the mapping of phonology to spelling (feedback inconsistency) would also affect visual word perception. In Experiment 1, we replicated the basic feedback consistency effect previously obtained by Stone, Vanhoy, and Van Orden (1997) in a lexical-decision task. Lexical-decision latencies and errors were increased when a word's phonological rhyme could be spelled in multiple ways. In Experiment 2, we showed that both feedforward and feedback consistency affected lexical-decision performance to the same extent. In Experiment 3, feedback consistency effects persisted in immediate naming, but they were smaller and less reliable than feedforward consistency effects. A portion of the feedforward consistency effects persisted in delayed naming. Our results suggest that a part of both feedforward and feedback consistency effects seem to be modulated by task specific properties (e.g., spelling check in lexical decision or ease of generating articulatory programs in naming). However, another part seems to uncover a task-independent basic principle underlying word perception: the bidirectional coupling of orthography and phenology. (C) 1997 Academic Press.
##     cit_times pub_date pub_year vol   issue start_page end_page
## 1          29              2008  23       6        866      888
## 2          73      MAY     1993   8       2        197      233
## 3          15              2011  26      10       1667     1686
## 4           7              2011  26       2        224      235
## 5          37              2012  27       4        539      571
## 6          21  AUG-NOV     1992   7 03. Apr        281      299
## 7          86      JUN     2004  19       3        369      390
## 8          28      OCT     2001  16 05. Jun        609      636
## 9          42      AUG     1999  14       4        381      391
## 10         37      AUG     1997  12       4        401      422
## 11         36              2011  26 04. Jun        509      529
## 12          8      DEC     1995  10       6        631      653
## 13         47  FEB-APR     1996  11 01. Feb        107      134
## 14         24  NOV-DEC     2006  21 07. Aug        920      944
## 15          9              2012  27       1         90      116
## 16        496      APR     2000  15       2        159      201
## 17         43              1989   4 03. Apr       SI51     SI76
## 18         33              2009  24       1        120      135
## 19         27              2010  25       4        568      588
## 20         16      AUG     2004  19       4        533      559
## 21        266      FEB     1993   8       1          1       56
## 22          3      APR     2000  15       2        203      222
## 23         32      FEB     2002  17       1          1       13
## 24         73      FEB     2000  15       1          1       44
## 25         25  JAN-APR     2006  21 01. Mrz        112      140
## 26         20  NOV-DEC     2006  21 07. Aug        945      973
## 27         31      AUG     2001  16       4        367      392
## 28         26      FEB     1999  14       1         15       45
## 29         25              2009  24 07. Aug        947      966
## 30         97      FEB     2004  19       1         57       95
## 31          7      OCT     2001  16 05. Jun        661      672
## 32         35      OCT     2002  17       5        503      535
## 33         12      AUG     2005  20       4        553      587
## 34         83      AUG     1994   9       3        393      422
## 35         11   Feb 07     2015  30 01. Feb        149      166
## 36          4              2017  32       9       1119     1132
## 37          0   Nov 26     2015  30      10       1339     1344
## 38          0                NA  NA                            
## 39         19   OCT 21     2015  30       9       1077     1095
## 40          2              2016  31       7        921      939
## 41          2   Jan 02     2020  35       1         93      105
## 42          0    DEC 1     2020  35      10       1326     1354
## 43         23              2014  29       2        147      157
## 44          4              2014  29       4        408      423
## 45          7              2017  32       4        471      487
## 46          0                NA  NA                            
## 47          0   Apr 02     2020  35       3        339      359
## 48          0   DEC 30     2020  36       1         13       24
## 49          9              2018  33       9       1128     1151
## 50          3              2016  31       9       1130     1149
## 51          0   OCT 17     2019  10       1                    
## 52          0   Apr 04     2017   8       1                    
## 53          1   OCT 26     2020  11       1                    
## 54         25              2015  11       4        369      380
## 55         13              2017  13       4        430      450
## 56          1              2019  15       4        279      294
## 57          1   Jan 02     2020  16       1         22       42
## 58          1              2018  14       4        262      278
## 59          3              2019  15       2        138      156
## 60          0      DEC     2019  11       4        621      644
## 61          0      JUN     2016   8       2        314      334
## 62          5              2012  13       3        408      435
## 63          3              2010  11       2        336      348
## 64         87      JAN     2003  32       1         25       36
## 65          1      NOV     1998  27       6        639      659
## 66          4      JUL     2000  29       4        433      451
## 67          5      DEC     2011  40 05. Jun        367      378
## 68          6      OCT     2009  38       5        475      490
## 69          1      SEP     1993  22       5        505      518
## 70          4      JUL     1995  24       4        269      287
## 71          8      SEP     2004  33       5        365      381
## 72        157      JAN     2003  32       1         37       55
## 73          2      AUG     2009  38       4        363      378
## 74          0      APR     2015  44       2        127      139
## 75          0      OCT     2020  49       5        885      913
## 76          0      OCT     2017  46       5       1285     1308
## 77         43      MAY     2006  35       3        233      244
## 78          1      AUG     2018  47       4        777      801
## 79          6      OCT     2017  46       5       1249     1271
## 80          4      AUG     2014  43       4        465      485
## 81          6      JUL     1996  25       4        443      481
## 82          3      APR     2020  49       2        247      273
## 83         13      AUG     2011  40       4        291      306
## 84        126      MAY     2001  30       3        267      295
## 85         32      JAN     1999  28       1         71       92
## 86          3      AUG     2019  48       4        843      858
## 87         28      MAR     1996  25       2        291      318
## 88         66      SEP     2005  27       3        387      414
## 89         47      DEC     2011  33       4        589      624
## 90         22      DEC     2009  31       4        559      575
## 91          0      MAY     2020  42       2        327      357
## 92         74      JUN     2005  27       2        235      268
## 93          4      DEC     2018  40       4        923      937
## 94          6      JUN     2018  40       2        241      268
## 95         14              2006  21       1          1       22
## 96          0   Apr 02     2020  35       2        136      152
## 97          8              2014  29       4        316      327
## 98          0      MAY     2020  37       2        297      309
## 99         19      FEB     2017  34       1        127      152
## 100         8      JAN     2018  21       1        195      208
## 101        20      MAR     2008  11       1        121      131
## 102        27      APR     2010  13       2        157      183
## 103         3      NOV     2019  22       5        897      911
## 104        17      APR     2012  15       2        365      377
## 105         4      MAY     2018  21       3        640      652
## 106        37      JUL     2010  13       3        327      340
## 107        20      JUL     2015  18       3        391      399
## 108         2      SPR     2018  35       1         75       95
## 109         0              2020  37                189      218
## 110        15      JUN     2001  44                197      216
## 111        59      SEP     2010  53                343      365
## 112         0              2020  NA                            
## 113        52              2007  50                 23       52
## 114        15      SEP     2016  59       3        339      352
## 115        10      MAR     2014  57       1         42       67
## 116         0              2020  NA                            
## 117        24  OCT-DEC     1995  38                393      418
## 118        35  OCT-DEC     1991  34                319      340
## 119         9      JUN     2017  60       2        200      223
## 120        12              2004  47                155      174
## 121         1      JAN     2020  NA      33         41       69
## 122        17      JUL     2018  NA      30        105      126
## 123         1   Jun 12     2020   5       1                    
## 124         0   MAR 27     2020   5       1                    
## 125         0   Jun 29     2018   3       1                    
## 126         9   Sep 13     2018   3       1                    
## 127         0   Sep 21     2017   2       1                    
## 128         2    OCT 3     2018   3       1                    
## 129         3   Apr 18     2017   2       1                    
## 130       149      JAN     1998  26       1          3       25
## 131         8      SEP     2014  46                161      167
## 132        22      JUL     2011  39       3        330      343
## 133       153      OCT     1999  27       4        359      384
## 134        10      SEP     2018  70                 70       85
## 135        22      JAN     2004  32       1          1       33
## 136        15      MAR     2016  55                 96      108
## 137        36      MAR     2014  43                 11       25
## 138         3      OCT     2011  39       4        660      667
## 139         3      MAR     2016  55                 38       52
## 140         0      SEP     2004  17       5        383      401
## 141        17      APR     1996   9       2        113      133
## 142         0      AUG     2019  51                 63       75
## 143         3      MAY     2017  42                 63       82
## 144         6      NOV     2017  44                203      222
## 145        48  MAR-MAY     2004  17 02. Mrz        237      262
## 146         3      NOV     2015  36                 56       71
## 147        16      MAY     2016  38                 71       88
## 148         8      OCT     2012  33       4        665      689
## 149        42      DEC     1996  17       4        401      426
## 150         1      MAR     2019  40       2        351      372
## 151        25      MAY     2014  35       3        547      579
## 152         5      MAR     2016  37       2        487      506
## 153        26      OCT     2004  25       4        565      585
## 154        53      MAR     2000  21       1         63       93
## 155         7      SEP     2015  36       5       1059     1075
## 156         1      JUN     2013  32       2        162      180
## 157        19      JAN     2014  33       1         68       77
## 158         2      JUN     2015  34       3        273      299
## 159         0      JUN     2020  39       3        334      348
## 160        17      MAR     2011  30       1         82      102
## 161       135      DEC     1996  35       6        775      800
## 162         5      APR     2017  93                 22       54
## 163         9      JUL     2012  67       1         86       92
## 164        51      AUG     2003  49       2        183      200
## 165        55      OCT     2013  69       3        228      256
## 166         5      NOV     2015  85                  1       14
## 167       261      FEB     2001  44       2        235      249
## 168        34      FEB     2003  48       2        435      443
## 169       195      JUN     1995  34       3        383      398
## 170        50      AUG     2005  53       2        292      313
## 171        94      AUG     2009  61       2        171      190
## 172        49      FEB     2003  48       2        255      280
## 173        11      OCT     2018 102                 97      114
## 174        39      JAN     2001  44       1        153      167
## 175        39      OCT     2010  63       3        274      294
## 176        27      APR     2003  48       3        527      541
## 177        49      JUN     1996  35       3        392      409
## 178        12      APR     2017  93                154      168
## 179         9      AUG     2018 101                 37       50
## 180       348      JUL     1997  37       1         58       93
## 181        44      JAN     2001  44       1         96      117
## 182        56      JAN     2005  52       1         58       70
## 183        35      NOV     2006  55       4        495      514
## 184        89      OCT     2013  69       3        257      276
## 185        29      NOV     2011  65       4        378      389
## 186         7      APR     2020 111                            
## 187        30      APR     2003  48       3        635      652
## 188         8      MAY     2014  73                 59       80
## 189        33      JUL     1998  39       1         70       84
## 190        44      NOV     2009  61       4        538      555
## 191        22      FEB     2005  52       2        171      192
## 192       113      JAN     2003  48       1        131      147
## 193         9      APR     2017  93                304      314
## 194         1      JAN     2013  68       1          1        9
## 195        64      MAY     1998  38       4        401      418
## 196        73      MAY     2008  58       4        916      930
## 197         2      APR     2019 105                141      152
## 198        12      JAN     2016  86                 97      118
## 199        94      MAY     2004  50       4        425      443
## 200        42      JUL     2006  55       1        102      125
## 201        45      AUG     2015  83                118      139
## 202        53      OCT     2000  43       3        530      560
## 203       112      MAY     2004  50       4        355      370
## 204        37      AUG     2010  63       2        158      179
## 205        49      OCT     2009  61       3        285      302
## 206         8      NOV     2000  43       4        561      575
## 207       105      OCT     1997  37       3        356      381
## 208        16      AUG     2014  75                 93      103
## 209       103      OCT     1995  34       5        567      593
## 210        95      NOV     1997  37       4        533      554
##                article_number                                doi experimental
## 1                                      10.1080/01690960801962372            1
## 2                                      10.1080/01690969308406954            1
## 3                                   10.1080/01690965.2010.524765            1
## 4               PII 923379546       10.1080/01690965.2010.486209            1
## 5                                   10.1080/01690965.2011.555268            1
## 6                                      10.1080/01690969208409388            1
## 7                                      10.1080/01690960344000224            1
## 8                                      10.1080/01690960143000083            1
## 9                                        10.1080/016909699386275            1
## 10                                       10.1080/016909697386790            1
## 11              PII 925484043       10.1080/01690965.2010.499215            1
## 12                                     10.1080/01690969508407116            1
## 13                                       10.1080/016909696387231            1
## 14                                       10.1080/016909600824278            1
## 15                                  10.1080/01690965.2010.544107            1
## 16                                       10.1080/016909600386084            1
## 17                                     10.1080/01690968908406363            1
## 18                                     10.1080/01690960802234177            1
## 19                                     10.1080/01690960903381174            1
## 20                                     10.1080/01690960344000260            1
## 21                                     10.1080/01690969308406948            1
## 22                                       10.1080/016909600386093            1
## 23                                     10.1080/01690960042000139            0
## 24                                       10.1080/016909600386101            1
## 25                                     10.1080/01690960400001861            1
## 26                                     10.1080/01690960600824344            1
## 27                                     10.1080/01690960042000111            1
## 28                                       10.1080/016909699386365            1
## 29                                     10.1080/01690960802154615            1
## 30                                     10.1080/01690960344000134            1
## 31                                     10.1080/01690960143000092            1
## 32                                     10.1080/01690960143000308            1
## 33                                     10.1080/01690960444000241            1
## 34                                     10.1080/01690969408402125            1
## 35                                  10.1080/01690965.2013.862285            1
## 36                                 10.1080/23273798.2017.1284340            1
## 37                                 10.1080/23273798.2015.1069363            1
## 38                                 10.1080/23273798.2020.1859568            1
## 39                                 10.1080/23273798.2015.1054844            1
## 40                                 10.1080/23273798.2016.1183798            1
## 41                                 10.1080/23273798.2019.1628284            1
## 42                                 10.1080/23273798.2020.1764600            0
## 43                                  10.1080/01690965.2012.735678            1
## 44                                  10.1080/01690965.2013.788197            1
## 45                                 10.1080/23273798.2016.1247213            1
## 46                                 10.1080/23273798.2020.1852291            1
## 47                                 10.1080/23273798.2019.1659987            1
## 48                                 10.1080/23273798.2020.1803373            1
## 49                                 10.1080/23273798.2018.1448092            1
## 50                                 10.1080/23273798.2016.1197954            1
## 51                         17                10.5334/labphon.180            1
## 52                                            10.5334/labphon.86            1
## 53                         16                10.5334/labphon.253            1
## 54                                  10.1080/15475441.2014.979387            1
## 55                                 10.1080/15475441.2017.1317253            1
## 56                                 10.1080/15475441.2019.1617149            1
## 57                                 10.1080/15475441.2019.1676751            1
## 58                                 10.1080/15475441.2018.1463850            1
## 59                                 10.1080/15475441.2018.1544075            1
## 60                                       10.1017/langcog.2019.40            1
## 61                                        10.1017/langcog.2015.3            1
## 62                                         10.1075/is.13.3.05kan            1
## 63                                         10.1075/is.11.2.22tho            1
## 64                                       10.1023/A:1021980931292            1
## 65                                       10.1023/A:1023279921733            1
## 66                                       10.1023/A:1005159329417            1
## 67                                     10.1007/s10936-011-9174-2            1
## 68                                     10.1007/s10936-009-9101-y            1
## 69                                            10.1007/BF01068251            1
## 70                                            10.1007/BF02145057            1
## 71                            10.1023/B:JOPR.0000039546.60121.a8            1
## 72                                       10.1023/A:1021933015362            1
## 73                                     10.1007/s10936-008-9096-9            1
## 74                                     10.1007/s10936-014-9283-9            1
## 75                                    10.1007/s10936-020-09728-1            1
## 76                                     10.1007/s10936-017-9494-y            1
## 77                                     10.1007/s10936-006-9013-z            1
## 78                                     10.1007/s10936-018-9558-7            0
## 79                                     10.1007/s10936-017-9492-0            0
## 80                                     10.1007/s10936-013-9265-3            1
## 81                                                                          1
## 82                                    10.1007/s10936-019-09686-3            1
## 83                                     10.1007/s10936-011-9171-5            1
## 84                                       10.1023/A:1010443001646            1
## 85                                       10.1023/A:1023239604158            1
## 86                                    10.1007/s10936-019-09634-1            1
## 87                                            10.1007/BF01708575            1
## 88                                     10.1017/S0272263105050175            1
## 89                                     10.1017/S0272263111000325            1
## 90                                     10.1017/S0272263109990027            1
## 91      PII S0272263119000603          10.1017/S0272263119000603            1
## 92                                     10.1017/S0272263105050126            0
## 93                                     10.1017/S0272263117000249            1
## 94                                     10.1017/S0272263117000274            1
## 95                                     10.1207/s15327868ms2101_1            1
## 96                                 10.1080/10926488.2020.1767336            1
## 97                                  10.1080/10926488.2014.948801            1
## 98                                           10.1093/jos/ffaa001            1
## 99                                            10.1093/jos/ffw010            1
## 100                                    10.1017/S136672891600119X            1
## 101                                    10.1017/S1366728907003252            1
## 102                                    10.1017/S1366728909990472            1
## 103                                    10.1017/S1366728918001001            1
## 104                                    10.1017/S1366728911000034            1
## 105                                    10.1017/S136672891700030X            1
## 106                                    10.1017/S1366728909990502            1
## 107                                    10.1017/S1366728914000133            1
## 108                              10.17250/khisli.35.1.201803.003            0
## 109                               10.17250/khisli.37..202009.008            1
## 110                                 10.1177/00238309010440020401            1
## 111                                     10.1177/0023830910371447            1
## 112                  2,38E+13           10.1177/0023830920978679            1
## 113                                 10.1177/00238309070500010201            1
## 114                                     10.1177/0023830915590677            1
## 115                                     10.1177/0023830913479105            1
## 116                  2,38E+13           10.1177/0023830920971315            1
## 117                                   10.1177/002383099503800404            1
## 118                                   10.1177/002383099103400402            1
## 119                                     10.1177/0023830917716101            1
## 120                                 10.1177/00238309040470020301            1
## 121                                                                         1
## 122                                                                         0
## 123                        56                   10.5334/gjgl.892            1
## 124                        35                  10.5334/gjgl.1201            1
## 125                        77                   10.5334/gjgl.150            1
## 126                       100                   10.5334/gjgl.528            0
## 127                        82                   10.5334/gjgl.240            1
## 128                       103                   10.5334/gjgl.531            1
## 129                        31                    10.5334/gjgl.76            1
## 130                                       10.1006/jpho.1997.0063            1
## 131                                   10.1016/j.wocn.2014.07.002            0
## 132                                   10.1016/j.wocn.2011.03.005            1
## 133                                       10.1006/jpho.1999.0100            1
## 134                                   10.1016/j.wocn.2018.05.005            0
## 135                                10.1016/S0095-4470(03)00004-4            1
## 136                                   10.1016/j.wocn.2015.12.004            1
## 137                                   10.1016/j.wocn.2014.01.002            1
## 138                                   10.1016/j.wocn.2011.07.001            1
## 139                                   10.1016/j.wocn.2015.11.003            1
## 140                             10.1016/j.jneuroling.2004.01.001            0
## 141                                 10.1016/0911-6044(96)00001-2            1
## 142                             10.1016/j.jneuroling.2019.01.001            1
## 143                             10.1016/j.jneuroling.2016.11.006            1
## 144                             10.1016/j.jneuroling.2017.06.002            1
## 145                                10.1016/S0911-6044(03)00055-1            1
## 146                             10.1016/j.jneuroling.2015.05.002            1
## 147                             10.1016/j.jneuroling.2015.11.001            1
## 148                                    10.1017/S014271641100052X            1
## 149                                    10.1017/S0142716400008171            1
## 150                                    10.1017/S0142716418000607            1
## 151                                    10.1017/S0142716412000501            1
## 152                                    10.1017/S0142716415000089            1
## 153                                    10.1017/s0142716404001274            1
## 154                                    10.1017/S0142716400001041            1
## 155                                    10.1017/S0142716414000046            1
## 156                                     10.1177/0261927X12456383            1
## 157                                     10.1177/0261927X13499761            1
## 158                                     10.1177/0261927X14557947            1
## 159                                     10.1177/0261927X20911991            1
## 160                                     10.1177/0261927X10387103            1
## 161                                       10.1006/jmla.1996.0040            1
## 162                                    10.1016/j.jml.2016.07.005            1
## 163                                    10.1016/j.jml.2011.12.011            1
## 164                                10.1016/S0749-596X(03)00027-5            1
## 165                                    10.1016/j.jml.2013.05.008            1
## 166                                    10.1016/j.jml.2015.06.007            1
## 167                                       10.1006/jmla.2000.2750            1
## 168 PII S0749-596X(02)00522-3      10.1016/S0749-596X(02)00522-3            1
## 169                                       10.1006/jmla.1995.1017            1
## 170                                    10.1016/j.jml.2005.03.003            1
## 171                                    10.1016/j.jml.2009.04.003            1
## 172 PII S0749-596X(02)00521-1      10.1016/S0749-596X(02)00521-1            1
## 173                                    10.1016/j.jml.2018.05.009            1
## 174                                       10.1006/jmla.2000.2738            1
## 175                                    10.1016/j.jml.2010.06.003            1
## 176                                10.1016/S0749-596X(02)00533-8            1
## 177                                       10.1006/jmla.1996.0022            1
## 178                                    10.1016/j.jml.2016.07.006            1
## 179                                    10.1016/j.jml.2018.03.003            1
## 180                                       10.1006/jmla.1997.2512            1
## 181                                       10.1006/jmla.2000.2749            1
## 182                                    10.1016/j.jml.2004.07.006            1
## 183                                    10.1016/j.jml.2006.07.001            1
## 184                                    10.1016/j.jml.2013.06.004            1
## 185                                    10.1016/j.jml.2011.06.003            1
## 186                    104063          10.1016/j.jml.2019.104063            1
## 187                                10.1016/S0749-596X(02)00531-4            1
## 188                                    10.1016/j.jml.2014.02.005            1
## 189                                       10.1006/jmla.1998.2567            1
## 190                                    10.1016/j.jml.2009.08.003            1
## 191                                    10.1016/j.jml.2004.10.004            1
## 192 PII S0749-596X(02)00509-0      10.1016/S0749-596X(02)00509-0            1
## 193                                    10.1016/j.jml.2016.10.009            1
## 194                                    10.1016/j.jml.2012.09.002            1
## 195                                       10.1006/jmla.1997.2557            1
## 196                                    10.1016/j.jml.2008.01.003            1
## 197                                    10.1016/j.jml.2019.01.002            1
## 198                                    10.1016/j.jml.2015.09.003            1
## 199                                    10.1016/j.jml.2003.12.003            1
## 200                                    10.1016/j.jml.2006.03.006            1
## 201                                    10.1016/j.jml.2015.04.005            1
## 202                                       10.1006/jmla.2000.2710            1
## 203                                    10.1016/j.jml.2004.01.001            1
## 204                                    10.1016/j.jml.2010.03.008            1
## 205                                    10.1016/j.jml.2009.05.003            1
## 206                                       10.1006/jmla.2000.2712            1
## 207                                       10.1006/jmla.1997.2517            1
## 208                                    10.1016/j.jml.2014.05.007            1
## 209                                       10.1006/jmla.1995.1026            1
## 210                                       10.1006/jmla.1997.2525            1
##     replication
## 1             1
## 2             1
## 3             1
## 4             1
## 5             0
## 6             0
## 7             0
## 8             1
## 9             0
## 10            0
## 11            1
## 12            0
## 13            1
## 14            0
## 15            1
## 16            1
## 17            1
## 18            1
## 19            1
## 20            1
## 21            1
## 22            1
## 23            0
## 24            1
## 25            0
## 26            0
## 27            0
## 28            1
## 29            1
## 30            1
## 31            0
## 32            1
## 33            1
## 34            1
## 35            0
## 36            1
## 37            1
## 38            0
## 39            0
## 40            0
## 41            1
## 42            0
## 43            1
## 44            1
## 45            1
## 46            0
## 47            0
## 48            0
## 49            1
## 50            1
## 51            0
## 52            0
## 53            1
## 54            0
## 55            1
## 56            0
## 57            1
## 58            0
## 59            1
## 60            0
## 61            1
## 62            0
## 63            0
## 64            0
## 65            0
## 66            1
## 67            0
## 68            0
## 69            1
## 70            1
## 71            1
## 72            0
## 73            1
## 74            0
## 75            1
## 76            0
## 77            1
## 78            0
## 79            0
## 80            0
## 81            1
## 82            1
## 83            1
## 84            0
## 85            0
## 86            1
## 87            1
## 88            1
## 89            1
## 90            1
## 91            0
## 92            0
## 93            0
## 94            0
## 95            1
## 96            1
## 97            0
## 98            1
## 99            0
## 100           1
## 101           0
## 102           0
## 103           0
## 104           1
## 105           1
## 106           0
## 107           1
## 108           1
## 109           1
## 110           1
## 111           0
## 112           0
## 113           1
## 114           1
## 115           0
## 116           1
## 117           0
## 118           0
## 119           0
## 120           1
## 121           1
## 122           0
## 123           1
## 124           1
## 125           0
## 126           0
## 127           0
## 128           0
## 129           0
## 130           1
## 131           1
## 132           1
## 133           1
## 134           0
## 135           0
## 136           1
## 137           1
## 138           0
## 139           0
## 140           0
## 141           1
## 142           0
## 143           0
## 144           0
## 145           1
## 146           0
## 147           1
## 148           1
## 149           1
## 150           0
## 151           1
## 152           0
## 153           0
## 154           1
## 155           1
## 156           0
## 157           0
## 158           0
## 159           1
## 160           1
## 161           1
## 162           1
## 163           1
## 164           0
## 165           1
## 166           1
## 167           1
## 168           1
## 169           1
## 170           0
## 171           1
## 172           0
## 173           1
## 174           1
## 175           1
## 176           0
## 177           1
## 178           1
## 179           0
## 180           0
## 181           0
## 182           1
## 183           1
## 184           1
## 185           0
## 186           1
## 187           1
## 188           0
## 189           1
## 190           0
## 191           1
## 192           1
## 193           1
## 194           1
## 195           1
## 196           1
## 197           1
## 198           1
## 199           1
## 200           0
## 201           1
## 202           1
## 203           0
## 204           1
## 205           0
## 206           1
## 207           0
## 208           0
## 209           0
## 210           1
##                                                                                                     comments
## 1                                                                one of several experiments is a replication
## 2                                                                one of several experiments is a replication
## 3                                                                                    inner-paper replication
## 4                                                     replication of previous experiment by the same authors
## 5                                            no clear communication of intent to replicate an original study
## 6                                            no clear communication of intent to replicate an original study
## 7                                            no clear communication of intent to replicate an original study
## 8                        replication of findings of a previous experiment by slightly different manipulation
## 9                                            no clear communication of intent to replicate an original study
## 10                                           no clear communication of intent to replicate an original study
## 11                                                               one of several experiments is a replication
## 12                                           no clear communication of intent to replicate an original study
## 13                                                               one of several experiments is a replication
## 14                       "replicate" as example stimulus word; no clear communication of intent to replicate
## 15                                                               one of several experiments is a replication
## 16                                                                                   inner-paper replication
## 17                                                               one of several experiments is a replication
## 18                                                               one of several experiments is a replication
## 19                                                               one of several experiments is a replication
## 20                                                                                   inner-paper replication
## 21                                                               one of several experiments is a replication
## 22                                                           replication of a previous study with extensions
## 23                                                                                       comment; meta-level
## 24                                                                                   inner-paper replication
## 25                                           no clear communication of intent to replicate an original study
## 26                                           no clear communication of intent to replicate an original study
## 27                                           no clear communication of intent to replicate an original study
## 28                                                               one of several experiments is a replication
## 29                                                                                   inner-paper replication
## 30                                                               one of several experiments is a replication
## 31                                           no clear communication of intent to replicate an original study
## 32                                                               one of several experiments is a replication
## 33                                                                                   inner-paper replication
## 34                                                                                   inner-paper replication
## 35                                           no clear communication of intent to replicate an original study
## 36                                                                       replication in a different language
## 37                                                                                   inner-paper replication
## 38                                           no clear communication of intent to replicate an original study
## 39                                           no clear communication of intent to replicate an original study
## 40                                           no clear communication of intent to replicate an original study
## 41                                                    replication of previous experiment by the same authors
## 42                                                                                 computational simulations
## 43                                                    replication of previous experiment by the same authors
## 44       one of several experiments is a replication; replication of previous experiment by the same authors
## 45                                                                       replication in a different language
## 46                                           no clear communication of intent to replicate an original study
## 47                                           no clear communication of intent to replicate an original study
## 48                                                                           reporting a replication failure
## 49                                                                                   inner-paper replication
## 50       one of several experiments is a replication; replication of previous experiment by the same authors
## 51                                           no clear communication of intent to replicate an original study
## 52                                           no clear communication of intent to replicate an original study
## 53                                                                       replication in a different language
## 54                                           no clear communication of intent to replicate an original study
## 55                                                                                   inner-paper replication
## 56                                           no clear communication of intent to replicate an original study
## 57                                                                                   inner-paper replication
## 58                                           no clear communication of intent to replicate an original study
## 59                                                                                   inner-paper replication
## 60                                           no clear communication of intent to replicate an original study
## 61                                                                                   inner-paper replication
## 62                                           no clear communication of intent to replicate an original study
## 63                                           no clear communication of intent to replicate an original study
## 64                                           no clear communication of intent to replicate an original study
## 65  no clear communication of intent to replicate an original study, writing about other replication studies
## 66       one of several experiments is a replication; replication of previous experiment by the same authors
## 67                                           no clear communication of intent to replicate an original study
## 68                                           no clear communication of intent to replicate an original study
## 69                          replication in a different language; one of several experiments is a replication
## 70                                                               one of several experiments is a replication
## 71                                                               one of several experiments is a replication
## 72                                           no clear communication of intent to replicate an original study
## 73                                                               one of several experiments is a replication
## 74                                           no clear communication of intent to replicate an original study
## 75                                                               one of several experiments is a replication
## 76                                           no clear communication of intent to replicate an original study
## 77                                                               one of several experiments is a replication
## 78                                                                                    meta-level; case study
## 79                                                                         meta-level; Philosophy of Science
## 80                                           no clear communication of intent to replicate an original study
## 81                          replication in a different language; one of several experiments is a replication
## 82                                                               one of several experiments is a replication
## 83                                                                                   inner-paper replication
## 84         different use of word "replicate": "subjects replicate prosodic boundaries during silent reading"
## 85                                           no clear communication of intent to replicate an original study
## 86                                                    replication of previous experiment by the same authors
## 87                                                    replication of previous experiment by the same authors
## 88                                                               one of several experiments is a replication
## 89                                                               one of several experiments is a replication
## 90                                                                       replication in a different language
## 91                                           no clear communication of intent to replicate an original study
## 92                                                                     meta-level; report about replications
## 93                                                                           reporting a replication success
## 94                                           no clear communication of intent to replicate an original study
## 95                                                               one of several experiments is a replication
## 96                                                                                   inner-paper replication
## 97                                           no clear communication of intent to replicate an original study
## 98                                                               one of several experiments is a replication
## 99                                           no clear communication of intent to replicate an original study
## 100                                                              one of several experiments is a replication
## 101                                          no clear communication of intent to replicate an original study
## 102                                          no clear communication of intent to replicate an original study
## 103                                          no clear communication of intent to replicate an original study
## 104                                                  replication of a previous study to validate the results
## 105                                                                      replication in a different language
## 106                                          no clear communication of intent to replicate an original study
## 107                                                                      replication in a different language
## 108                                                                      replication in a different language
## 109                                                                      replication in a different language
## 110                                                           replication and extension of previous findings
## 111                                          no clear communication of intent to replicate an original study
## 112                                          no clear communication of intent to replicate an original study
## 113                                                           replication and extension of previous findings
## 114                                          replication and extension of previous study by the same authors
## 115                                          no clear communication of intent to replicate an original study
## 116                                                                                  inner-paper replication
## 117                                          no clear communication of intent to replicate an original study
## 118                                          no clear communication of intent to replicate an original study
## 119                                          no clear communication of intent to replicate an original study
## 120                                                            replication and extension of a previous study
## 121                                                                      replication in a different language
## 122                                                                               meta-level; position paper
## 123                                               replication of a previous study using a different paradigm
## 124                                replication of a previous study in order to test for a potential confound
## 125                                          no clear communication of intent to replicate an original study
## 126                                                                                       review; meta-level
## 127                                          no clear communication of intent to replicate an original study
## 128                                          no clear communication of intent to replicate an original study
## 129                                          no clear communication of intent to replicate an original study
## 130                                                                                  inner-paper replication
## 131                                                                                   replication of a model
## 132                                        replication and extension of a previous study by the same authors
## 133                                                                                  inner-paper replication
## 134                                                                                   meta-level; case study
## 135                                          no clear communication of intent to replicate an original study
## 136                                                                                  inner-paper replication
## 137                                                                                  inner-paper replication
## 138                                          no clear communication of intent to replicate an original study
## 139                                          no clear communication of intent to replicate an original study
## 140                                                                                   meta-level; case study
## 141                                          replication and extension of previous study by the same authors
## 142                                          no clear communication of intent to replicate an original study
## 143                                          no clear communication of intent to replicate an original study
## 144                                          no clear communication of intent to replicate an original study
## 145                                                                                  inner-paper replication
## 146                                          no clear communication of intent to replicate an original study
## 147                                                                                  inner-paper replication
## 148                                                            replication and extension of a previous study
## 149                                                                                  inner-paper replication
## 150                                          no clear communication of intent to replicate an original study
## 151                                          replication and extension of previous study by the same authors
## 152                                          no clear communication of intent to replicate an original study
## 153                                          no clear communication of intent to replicate an original study
## 154                                                                      replication in a different language
## 155                         one of several experiments is a replication; replication in a different language
## 156                                          no clear communication of intent to replicate an original study
## 157                                          no clear communication of intent to replicate an original study
## 158                                          no clear communication of intent to replicate an original study
## 159                                                                                  inner-paper replication
## 160                                                                      replication in a different language
## 161                                                                                  inner-paper replication
## 162                                                                                  inner-paper replication
## 163                                                              one of several experiments is a replication
## 164                                          no clear communication of intent to replicate an original study
## 165                                                              one of several experiments is a replication
## 166                                                                                  inner-paper replication
## 167                                                                      replication in a different language
## 168                                            replication of a previous study in order to test its validity
## 169                                                            replication and extension of a previous study
## 170                                          no clear communication of intent to replicate an original study
## 171                                                                                  inner-paper replication
## 172                                          no clear communication of intent to replicate an original study
## 173                                                            replication and extension of a previous study
## 174                                                              one of several experiments is a replication
## 175         replication of a previous study by the same authors; one of several experiments is a replication
## 176                                          no clear communication of intent to replicate an original study
## 177                                                            replication and extension of a previous study
## 178                                                            replication and extension of a previous study
## 179                                          no clear communication of intent to replicate an original study
## 180                                          no clear communication of intent to replicate an original study
## 181                                          no clear communication of intent to replicate an original study
## 182                                                            replication and extension of a previous study
## 183                                            replication of a previous study in order to test its validity
## 184                                                                      replication in a different language
## 185                                          no clear communication of intent to replicate an original study
## 186                                            replication of a previous study in order to test its validity
## 187                                                                                  inner-paper replication
## 188                                          no clear communication of intent to replicate an original study
## 189                                                              one of several experiments is a replication
## 190                                          no clear communication of intent to replicate an original study
## 191                                                              one of several experiments is a replication
## 192                                                                                  inner-paper replication
## 193                                                                                  inner-paper replication
## 194                                                                                  inner-paper replication
## 195                                                            replication and extension of a previous study
## 196                                                                                  inner-paper replication
## 197                                                                                  inner-paper replication
## 198                                                                                  inner-paper replication
## 199                                                                      replication in a different language
## 200                                          no clear communication of intent to replicate an original study
## 201                                        replication and extension of a previous study by the same authors
## 202                                                                                  inner-paper replication
## 203                                          no clear communication of intent to replicate an original study
## 204                                                                                  inner-paper replication
## 205                                          no clear communication of intent to replicate an original study
## 206                                                              one of several experiments is a replication
## 207                                          no clear communication of intent to replicate an original study
## 208                                          no clear communication of intent to replicate an original study
## 209                                          no clear communication of intent to replicate an original study
## 210                                                                      replication in a different language
##     oa_article
## 1            0
## 2            0
## 3            0
## 4            0
## 5            0
## 6            1
## 7            0
## 8            1
## 9            0
## 10           0
## 11           0
## 12           0
## 13           0
## 14           0
## 15           1
## 16           0
## 17           0
## 18           0
## 19           0
## 20           0
## 21           0
## 22           0
## 23           0
## 24           0
## 25           0
## 26           0
## 27           0
## 28           0
## 29           0
## 30           0
## 31           0
## 32           0
## 33           1
## 34           0
## 35           1
## 36           0
## 37           0
## 38           1
## 39           0
## 40           1
## 41           1
## 42           0
## 43           0
## 44           0
## 45           1
## 46           0
## 47           0
## 48           0
## 49           0
## 50           1
## 51           1
## 52           1
## 53           1
## 54           1
## 55           0
## 56           1
## 57           1
## 58           0
## 59           0
## 60           0
## 61           0
## 62           0
## 63           1
## 64           0
## 65           0
## 66           0
## 67           0
## 68           0
## 69           0
## 70           0
## 71           1
## 72           0
## 73           0
## 74           0
## 75           0
## 76           0
## 77           0
## 78           1
## 79           1
## 80           0
## 81           0
## 82           1
## 83           1
## 84           0
## 85           0
## 86           0
## 87           0
## 88           0
## 89           0
## 90           0
## 91           0
## 92           0
## 93           0
## 94           0
## 95           0
## 96           0
## 97           0
## 98           1
## 99           0
## 100          1
## 101          0
## 102          0
## 103          0
## 104          0
## 105          0
## 106          0
## 107          0
## 108          1
## 109          1
## 110          0
## 111          0
## 112          0
## 113          0
## 114          0
## 115          0
## 116          0
## 117          0
## 118          0
## 119          1
## 120          0
## 121          1
## 122          1
## 123          1
## 124          1
## 125          1
## 126          1
## 127          1
## 128          1
## 129          1
## 130          0
## 131          0
## 132          0
## 133          0
## 134          0
## 135          0
## 136          0
## 137          0
## 138          0
## 139          0
## 140          0
## 141          0
## 142          0
## 143          0
## 144          0
## 145          0
## 146          0
## 147          0
## 148          0
## 149          0
## 150          0
## 151          1
## 152          0
## 153          0
## 154          0
## 155          0
## 156          0
## 157          0
## 158          0
## 159          0
## 160          0
## 161          0
## 162          0
## 163          0
## 164          0
## 165          0
## 166          0
## 167          0
## 168          0
## 169          0
## 170          0
## 171          1
## 172          0
## 173          1
## 174          0
## 175          0
## 176          0
## 177          0
## 178          0
## 179          1
## 180          0
## 181          0
## 182          0
## 183          0
## 184          0
## 185          1
## 186          1
## 187          0
## 188          0
## 189          0
## 190          1
## 191          0
## 192          1
## 193          0
## 194          0
## 195          0
## 196          0
## 197          0
## 198          0
## 199          0
## 200          1
## 201          1
## 202          0
## 203          0
## 204          1
## 205          1
## 206          0
## 207          0
## 208          0
## 209          0
## 210          0
##                                                                                                                                                                                                                                                                                                          cit_init_study
## 1                                                                                                                                                                Vitevitch, M. S., & Stamer, M. K. (2006). The curious case of competition in Spanish speech production. Language and Cognitive Processes, 21, 760 770.
## 2                                                                                                                                                                                               Cutler, A., Mehler, J., Norris, D. & Segui, J. (1983). A language-spodfic comprehension strategy. Nature, 304, 159-160.
## 3                                                                                                                                                                                                                                                                                                                  same
## 4                                                                                           Braun, B., & Tagliapietra, L. (2009). The role of contrastive intonation contours in the retrieval of contextual alternatives. Language and Cognitive Processes. Advance online publication.\ndoi:10.1080/01690960903036836
## 5                                                                                                                                                                                                                                                                                                                      
## 6                                                                                                                                                                                                                                                                                                                      
## 7                                                                                                                                                                                                                                                                                                                      
## 8                                                                                                                                 Mehler, J., Dommergues, J.Y., Frauenfelder, U., & Segu��, J. (1981). The syllable�s role in speech segmentation. Journal of Verbal Learning and Verbal Behavior, 20, 298�305.
## 9                                                                                                                                                                                                                                                                                                                      
## 10                                                                                                                                                                                                                                                                                                                     
## 11                                                                                                                     Dunabeitia, J. A., Perea, M., & Carreiras,M. (2007). Do transposed-letter similarity effects occur at a morpheme level? Evidence for morpho-orthographic decomposition. Cognition, 105, 691 703.
## 12                                                                                                                                                                                                                                                                                                                     
## 13                                                                                                                                                                Grosjean,  F. (1983). How  long  is  the  sentence?  Prediction  and  prosody  in  the  on-line processing of language.  Linguistics , 21, 501�529.
## 14                                                                                                                                                                                                                                                                                                                     
## 15                                                                                                                                                                          Connolly, A. C., Fodor, J. A., Gleitman, L. R., & Gleitman, H. (2007). Why stereotypes don't even make good defaults. Cognition, 103, 1-22.
## 16                                                                                                                                                                                                                                                                                                                 same
## 17                                                                                                                                                            Holmes, V. M., Kennedy. A., & Murray, W. S. (1981). Syntactic structure and the garden\npath. Quarterly Journal of Experimental Psychology, 39A, 271-294.
## 18                                                                                                                                 Navarrete, E., & Costa, A. (2005). Phonological activation of ignored pictures: Further evidence for a cascade model of lexical access. Journal of Memory and Language, 53, 359 377.
## 19                                                                                                                                               Prevor, M. D., & Diamond, A. (2005). Color-object interference in young children: A Stroop effect in children 3� 6� years old. Cognitive Development, 20, 256 274.
## 20                                                                                                                                                                                                                                                                                                                 same
## 21                                                                                                                                                                                                                  Bybee, J.L. & Modcr, C.L. (1983). Morphological classes as natural categories. Language,59,251-270.
## 22                                                                                               Fowler, C.A., & Housum, J. (1987). Talkers� signalling of ��new�� and ��old�� words in speech and listeners� perception and use of the distinction. Journal ofMemory and Language, 26, 489-�504.
## 23                                                                                                                                                                                                                                                                                                                     
## 24                                                                                                                                                                                                                                                                                                                 same
## 25                                                                                                                                                                                                                                                                                                                     
## 26                                                                                                                                                                                                                                                                                                                     
## 27                                                                                                                                                                                                                                                                                                                     
## 28                                                                                                                                                                      Collins, A.F., & Ellis, A.W. (1992). Phonological priming of lexical entries in speech production.British Journal of Psychology, 83, 375�388.
## 29                                                                                                                                                                                                                                                                                                                 same
## 30                                                                                                                                     Ziegler, J.C., & Ferrand, L. (1998). Orthography shapes the perception of speech: The consistency effect in auditory recognition. Psychonomic Bulletin and Review, 5, 683�689.
## 31                                                                                                                                                                                                                                                                                                                     
## 32                                                                                                                                                                         Wheeldon, L.R., & Levelt, W.J.M. (1995). Monitoring the time course of phonological encoding. Journal of Memory and Language, 34, 311�334.
## 33                                                                                                                                                                                                                                                                                                                 same
## 34                                                                                                                                                                                                                                                                                                                 same
## 35                                                                                                                                                                                                                                                                                                                     
## 36                                                                                                                               Laganaro, M., & Alario, F. (2006). On the locus of the syllable frequency effect in speech production. Journal of Memory and Language, 55(2), 178�196. doi:10.1016/j.jml.2006.05.001
## 37                                                                                                                                                                                                                                                                                                                 same
## 38                                                                                                                                                                                                                                                                                                                     
## 39                                                                                                                                                                                                                                                                                                                     
## 40                                                                                                                                                                                                                                                                                                                     
## 41                                                                         Lowder MW, & Ferreira F (2016b). Prediction in the processing of repair disfluencies: Evidence from the visual-world paradigm. Journal of Experimental Psychology: Learning, Memory, and Cognition, 42, 1400�1416. doi: 10.1037/xlm0000256
## 42                                                                                                                                                                                                                                                                                                                     
## 43                                                                                                             New, B., Arau`jo, V., & Nazzi, T. (2008). Differential processing of consonants and vowels in lexical access through reading. Psychological Science, 19, 1223 1227. doi:10.1111/j.1467-9280.2008.02228.x
## 44                                                               Patson, N. D., & Ferreira, F. (2009). Conceptual plural information is used to guide early parsing decisions: Evidence from garden-path sentences with reciprocal verbs. Journal of Memory and Language, 60(4), 464 486. doi:10.1016/j.jml.2009.02.003
## 45                                                                                                              Baayen, R. H., Dijkstra, T., & Schreuder, R. (1997). Singulars and plurals in Dutch: Evidence for a parallel dual-route model. Journal of Memory and Language, 37, 94�117. doi:10.1006/jmla.1997.2509
## 46                                                                                                                                                                                                                                                                                                                     
## 47                                                                                                                                                                                                                                                                                                                     
## 48                                                                                                                                                                                                                                                                                                                     
## 49                                                                                                                                                                                                                                                                                                                 same
## 50                                                                    Yang, J., Wang, S., Xu, Y., & Rayner, K. (2009). Do chinese readers obtain preview benefit from word n+2? Evidence from eye movements. Journal of Experimental Psychology: Human Perception and Performance, 35, 1192-1204. doi: 10.1037/a0013554
## 51                                                                                                                                                                                                                                                                                                                     
## 52                                                                                                                                                                                                                                                                                                                     
## 53                                                                                                                                                                                         Koch, X., & Spalek, K. (in progress). Contrastive intonation effects on word recall for information-structural alternatives.
## 54                                                                                                                                                                                                                                                                                                                     
## 55                                                                                                                                                                                                                                                                                                                 same
## 56                                                                                                                                                                                                                                                                                                                     
## 57                                                                                                                                                                                                                                                                                                                 same
## 58                                                                                                                                                                                                                                                                                                                     
## 59                                                                                                                                                                                                                                                                                                                 same
## 60                                                                                                                                                                                                                                                                                                                     
## 61                                                                                                                                                                                                                                                                                                                 same
## 62                                                                                                                                                                                                                                                                                                                     
## 63                                                                                                                                                                                                                                                                                                                     
## 64                                                                                                                                                                                                                                                                                                                     
## 65                                                                                                                                                                                                                                                                                                                     
## 66                                                                                                                                          Ehri, L. C., & Wilce, L. S. (1982). Recognition of spellings printed in lower and mixed case: Evidence for orthographic images. Journal of Reading Behavior, 14, 219�230.
## 67                                                                                                                                                                                                                                                                                                                     
## 68                                                                                                                                                                                                                                                                                                                     
## 69                                                                                                                                               Smith, C. D. (1981). Recognition memory for sentences as a function of concreteness/abstractness and affirmation/negation. British Journal of Psychology, 72, 125-129.
## 70                                                                                                                                          Camac, M. K., & Glucksberg, S. (1984). Metaphors do not use associations between concepts, they are used to create them. Journal of Psycholinguistic Research, 13, 443-455.
## 71                                                                                                                                                                 Janssen, D. P., Roelofs, A., & Levelt, W. J. M. (2002). Inflectional frames in language production. Language and Cognitive Processes, 17, 209�236.
## 72                                                                                                                                                                                                                                                                                                                     
## 73                            Trueswell, J. C., Tanenhaus, M. K., & Kello, C. (1993). Verb-specific constraints in sentence processing: Separating effects of lexical preference from garden-paths. Journal of Experimental Psychology. Learning, Memory, and Cognition, 19, 528�553. doi:10.1037/0278-7393.19.3.528.
## 74                                                                                                                                                                                                                                                                                                                     
## 75                                                                                                                                                                Gibbs, R. W., Bogdanovich, J. M., Sykes, J. R., & Barr, D. J. (1997). Metaphor in idiom comprehension. Journal of Memory and Language, 37, 141�154.
## 76                                                                                                                                                                                                                                                                                                                     
## 77                                                                                                             Pi�ango, M. M., Zurif, E., & Jackendoff, R. (1999). Real-time processing implications of aspectual coercion at the syntax-semantics interface. Journal of Psycholinguistic Research, 28(4), 395�414.
## 78                                                                                                                                                                                                                                                                                                                     
## 79                                                                                                                                                                                                                                                                                                                     
## 80                                                                                                                                                                                                                                                                                                                     
## 81                                                                                                                                                                                                  Taft, M. (1990). Lexical processing of functionally constrained words. Journal of Memory and Language, 29, 245-257.
## 82                                                                                                  Tiemann, S., Schmid, M., Bade, N., Rolke, B., Hertrich, I., Ackermann, H., et al. (2011). Psycholinguistic evidence for presuppositions: On-line and off-line data. Proceedings of Sinn & Bedeutung, 15, 581�595.
## 83                                                                                                                                                                                                                                                                                                                 same
## 84                                                                                                                                                                                                                                                                                                                     
## 85                                                                                                                                                                                                                                                                                                                     
## 86                                                                                                                      Stroustrup, S., & Wallentin, M. (2018). Grammatical category influences lateralized imagery for sentences. Language and Cognition, 10(2), 193�207. https ://doi.org/10.1017/langc og.2017.19.
## 87                                                                                                                                                                                                                       Watt, S. M. (1992). The auditory parsing of complement sentences. Ph.D., University of Dundee.
## 88                                                                                                                                                                                                                     Barcroft, J (2001) Acoustic variation and lexical acquisition. Language Learning, 51, 563�590.
## 89                                                                                                                                      Ellis , N. C. , & Sagarra , N. ( 2010 ). The bounds of adult language acquisition: Blocking and learned attention . Studies in Second Language Acquisition , 32 , 553 � 580 .
## 90                                                                                                                                                      Fern�ndez , C . ( 2008 ). Reexamining the role of explicit information in processing instruction. Studies in Second Language Acquisition , 30 , 277 � 305 .
## 91                                                                                                                                                                                                                                                                                                                     
## 92                                                                                                                                                                                                                                                                                                                     
## 93                                                                                                                                                                                                                                                                                                                     
## 94                                                                                                                                                                                                                                                                                                                     
## 95                                                                                                                                                                  Glucksberg, S., McGlone, M. S., & Manfredi, D. (1997). Property attribution in metaphor comprehension. Journal of Memory and Language, 36, 50�67.
## 96                                                                                                                                                                                                                                                                                                                 same
## 97                                                                                                                                                                                                                                                                                                                     
## 98                                                                                                                                                                                 Machery, Edouard, Ron Mallon, Shaun Nichols & Stephen P. Stich (2004), �Semantics, crosscultural style�. Cognition 92: B1�B12.
## 99                                                                                                                                                                                                                                                                                                                     
## 100                                                                                                                                                                   Prior, A. (2012). Too much of a good thing: Stronger bilingual inhibition leads to larger lag-2 task repetition costs. Cognition, 125(1), 1�12.
## 101                                                                                                                                                                                                                                                                                                                    
## 102                                                                                                                                                                                                                                                                                                                    
## 103                                                                                                                                                                                                                                                                                                                    
## 104                                                                                            Levy, B. J., Mc Veigh, N., Marful, A., & Anderson, M. C.\n(2007). Inhibiting your native language: The role of retrieval-induced forgetting during second-language 0acquisition. Psychological Science, 18 (1), 29�34.
## 105                     Tremblay, A., Broersma, M., Coughlin, C. E., & Choi, J. (2016). Effects of the native language on the learning of fundamental frequency in second-language speech segmentation. Frontiers in Psychology, 7. Retrieved from http://journal.frontiersin.org/article/10.3389/fpsyg.2016.00985/full
## 106                                                                                                                                                                                                                                                                                                                    
## 107                                                                                                                                                         Nurmsoo, E., & Bloom, P. (2008). Preschoolers� perspectivetaking in word learning: Do they blindly follow eye gaze? Psychological Science, 19, 211�215.
## 108                                                                                                                                                                               Sung, Chit Cheung Matthew. 2010. Being a �purist� in Hong Kong: to use or not to use mixed code. Changing English 17(4): 411-419.
## 109                                                                                                                                     Linzen, Tal and T. Florian Jaeger. 2016. Uncertainty and expectation in sentence processing: Evidence from subcategorization distributions. Cognitive Science 40(6): 1382-1411.
## 110                                                                                                                                                                                                                                           STETSON, R. H. (1951). Motor phonetics. 2nd ed. Amsterdam: North-Holland.
## 111                                                                                                                                                                                                                                                                                                                    
## 112                                                                                                                                                                                                                                                                                                                    
## 113                                                                                                DUPOUX, E., KAKEHI, K., HIROSE, Y., PALLIER, C., & MEHLER, J. (1999). Epenthetic vowels in Japanese: A perceptual illusion? Journal of Experimental Psychology: Human Perception and Performance, 25, 1568 � 1578.
## 114                                                                                                       Kazanas, S. A., & Altarriba, J. (2015). The automatic activation of emotion and emotion-laden words: Evidence from a masked and unmasked priming paradigm. American Journal of Psychology, 128(3), 323�336.
## 115                                                                                                                                                                                                                                                                                                                    
## 116                                                                                                                                                                                                                                                                                                                same
## 117                                                                                                                                                                                                                                                                                                                    
## 118                                                                                                                                                                                                                                                                                                                    
## 119                                                                                                                                                                                                                                                                                                                    
## 120                                                                                                                                                                            JOHNSON, K., FLEMMING, E., & WRIGHT, R. (1993). The hyperspace effect: Phonetic targets are hyperarticulated. Language, 69, 505 � 528.
## 121   Romero-Fresco, Pablo (2016). �The dubbing effect: An eye-tracking study comparing the reception of original and dubbed films.� Paper presented at Linguistic and Cultural Representation in Audiovisual Translation (Sapienza Universit� di Roma and Universit� degli Studi di Roma Tre, 11-13 February).
## 122                                                                                                                                                                                                                                                                                                                    
## 123                                                                                                                                                                       Lai, Regina. 2015. Learnable vs. unlearnable harmony patterns. Linguistic Inquiry 46(3). 425�451. DOI: https://doi.org/10.1162/LING_a_00188
## 124                                                                                                                        Chemla, Emmanuel & Lewis Bott. 2015. Using Structural Priming to Study Scopal Representations and Operations. Linguistic Inquiry 46(1). 157�172. DOI: https://doi.org/10.1162/LING_a_00178
## 125                                                                                                                                                                                                                                                                                                                    
## 126                                                                                                                                                                                                                                                                                                                    
## 127                                                                                                                                                                                                                                                                                                                    
## 128                                                                                                                                                                                                                                                                                                                    
## 129                                                                                                                                                                                                                                                                                                                    
## 130                                                                                                                                                                                                                                                                                                                same
## 131                                                                                                                                                                     de Boer, B. (2010). Investigating the acoustic effect of the descended larynx with articulatory models. Journal of Phonetics, 38(4), 679�686.
## 132                                                                                                                                                                                         Ernestus, M., Baayen, H., & Schreuder, R. (2002). The recognition of reduced word forms. Brain and Language, 81, 162�173.
## 133                                                                                                                                                                                                                                                                                                                same
## 134                                                                                                                                                                                                                                                                                                                    
## 135                                                                                                                                                                                                                                                                                                                    
## 136                                                                                                                                                                                                                                                                                                                same
## 137                                                                                                                                                                                                                                                                                                                same
## 138                                                                                                                                                                                                                                                                                                                    
## 139                                                                                                                                                                                                                                                                                                                    
## 140                                                                                                                                                                                                                                                                                                                    
## 141                                                                                                                                   BUCHANAN L., HILDEBRANDT, N. and MACKINNON, Cl. E. Implicit phonological processing in deep dyslexia: a rowse is implicitly a rose. Journal of Neurolinguistics 8, 163-182, 1994.
## 142                                                                                                                                                                                                                                                                                                                    
## 143                                                                                                                                                                                                                                                                                                                    
## 144                                                                                                                                                                                                                                                                                                                    
## 145                                                                                                                                                                                                                                                                                                                same
## 146                                                                                                                                                                                                                                                                                                                    
## 147                                                                                                                                                                                                                                                                                                                same
## 148                                                                                                                                              Craik, F. I. M., & Tulving, E. (1975). Depth of processing and the retention of words in episodic memory. Journal of Experimental Psychology: General, 104, 268�294.
## 149                                                                                                                                                                                                                                                                                                                same
## 150                                                                                                                                                                                                                                                                                                                    
## 151                                                                                                                                                  Ellis, N. C., & Sagarra, N. (2010b). The bounds of adult language acquisition: Blocking and learned attention. Studies in Second Language Acquisition, 32, 1�28.
## 152                                                                                                                                                                                                                                                                                                                    
## 153                                                                                                                                                                                                                                                                                                                    
## 154                                                                                                                                                                            Duncan, L. G., Seymour, P. H. K., & Hill, S. (1997). How important are rhyme and analogy in beginning reading? Cognition, 63, 171�208.
## 155                                                                                                            Allum, P. H., & Wheeldon, L. R. (2007). Planning scope in spoken sentence production: The role of grammatical units. Journal of Experimental Psychology: Learning, Memory, and Cognition, 33, 791�810.
## 156                                                                                                                                                                                                                                                                                                                    
## 157                                                                                                                                                                                                                                                                                                                    
## 158                                                                                                                                                                                                                                                                                                                    
## 159                                                                                                                                                                                                                                                                                                                same
## 160                                                                                                                                               Smith, H. W., Matsuno, T., & Umino, M. (1994). How similar are impression formation processes among Japanese and Americans? Social Psychology Quarterly, 57, 124-139.
## 161                                                                                                                                                                                                                                                                                                                same
## 162                                                                                                                                                                                                                                                                                                                same
## 163                                                                                                                                                                                 Bornstein, B. H., & Wilson, J. R. (2004). Extending the revelation effect to faces: Haven�t we met before? Memory, 12, 140�146.
## 164                                                                                                                                                                                                                                                                                                                    
## 165                                                                                                                                                                 Damian, M. F., Vigliocco, G., & Levelt, W. J. M. (2001). Effects of semantic context in the naming of pictures and words. Cognition, 81, B77�B86.
## 166                                                                                                                                                                                                                                                                                                                same
## 167                                                                                            Johnson, J. S., & Newport, E. L. (1989). Critical period effects in second language learning: The influence of maturational state on the acquisition of English as a second language. Cognitive Psychology, 21, 60�99.
## 168                                                                                                                                                Ferrand, L., Segui, J., & Grainger, J. (1996). Masked priming of word and picture naming: the role of syllabic units. Journal of Memory and Language, 35, 708�723.
## 169                                                                                                                                                                                               Smith, V. L., & Clark, H. H. (1993). On the course of answering questions. Journal of Memory and Language, 32, 25-38.
## 170                                                                                                                                                                                                                                                                                                                    
## 171                                                                                                                                                                                                                                                                                                                same
## 172                                                                                                                                                                                                                                                                                                                    
## 173                                                                                                               Parker, A., & Dagnall, N. (2009). Concreteness effects revisited: The influence of dynamic visual noise on memory for concrete and abstract words. Memory, 17, 397-410. doi:10.1080/09658210902802967
## 174                                                                                                                                                        Dewhurst, S. A., & Anderson, S. J. (1999). Effects of exact and category repetition in true and false recognition memory. Memory & Cognition, 27, 665�673.
## 175                                                                                                                                                    Dilley, L. C., & McAuley, J. D. (2008). Distal prosodic context affects word segmentation and lexical processing. Journal of Memory and Language, 59, 294�311.
## 176                                                                                                                                                                                                                                                                                                                    
## 177                                                                                                                                                                 EVANS, J. ST. B. T. (1983). Linguistic determinants of bias in conditional reasoning. Quarterly Journal of Experimental Psychology, 35A, 635�644.
## 178 Watson, J. M., Balota, D. A., & Roediger, H. L. (2003). Creating false memories with hybrid lists of semantic and phonological associates: Over-additive false memories produced by converging associative networks. Journal of Memory and Language, 49, 95�118. http://dx.doi.org/10.1016/S0749-596X(03)00019-6.
## 179                                                                                                                                                                                                                                                                                                                    
## 180                                                                                                                                                                                                                                                                                                                    
## 181                                                                                                                                                                                                                                                                                                                    
## 182                                                                                                                 Baars, B. J., Motley, M. T., & MacKay, D. G. (1975). Output editing for lexical status in artificially elicited slips of the tongue. Journal of Verbal Learning and Verbal Behavior, 14, 382�391.
## 183                                                                                                         Yonelinas, A. P. (1994). Receiver-operating characteristics in recognition memory: Evidence for a dual process model. Journal of Experimental Psychology: Learning, Memory and Cognition, 20(6), 1341-1354.
## 184                                                                                           Prior, A., & Gollan, T. H. (2011). Good language-switchers are good taskswitchers: Evidence from Spanish�English and Mandarin�English bilinguals. Journal of International Neuropsychological Society, 17, 682�691.
## 185                                                                                                                                                                                                                                                                                                                    
## 186                                                                                                         Dillon, B., Mishler, A., Sloggett, S., & Phillips, C. (2013). Contrasting intrusion profiles for agreement and anaphora: Experimental and modeling evidence. Journal of Memory and Language, 69 , 85�103.
## 187                                                                                                                                                                                                                                                                                                                same
## 188                                                                                                                                                                                                                                                                                                                    
## 189                                                                                                                                                Greene, S. B., Gerrig, R. J., McKoon, G., & Ratcliff, R. (1994). Unheralded pronouns and management by common ground. Journal of Memory and Language, 33, 511�526.
## 190                                                                                                                                                                                                                                                                                                                    
## 191                                                                                                                                                                Vitevitch, M. S., & Luce, P. A. (1998). When words compete: Levels of processing in perception of spoken words. Psychological Science, 9, 325�329.
## 192                                                                                                                                                                                                                                                                                                                same
## 193                                                                                                                                                                                                                                                                                                                same
## 194                                                                                                                                                                                                                                                                                                                same
## 195                                                                                                                                           SCHREUDER, R., FLORES D'ARCAIS, G. B., & GLAZENBORG, G. (1984). Effects of perceptual and conceptual similarity in semantic priming. Psychological Research, 45, 339-354.
## 196                                                                                                                                                                                                                                                                                                                same
## 197                                                                                                                                                                                                                                                                                                                same
## 198                                                                                                                                                                                                                                                                                                                same
## 199                                                                                                                                                                         Towse, J. N., Hitch, G. J., & Hutton, U. (2000). On the interpretation of working memory span in adults. Memory & Cognition, 28, 341�348.
## 200                                                                                                                                                                                                                                                                                                                    
## 201                                                                                                                                                                                              Schotter ER. Synonyms provide semantic preview benefit in English. Journal of Memory and Language. 2013; 69:619�633.
## 202                                                                                                                                                                                                                                                                                                                same
## 203                                                                                                                                                                                                                                                                                                                    
## 204                                                                                                                                                                                                                                                                                                                same
## 205                                                                                                                                                                                                                                                                                                                    
## 206                                                                                                                                                        Brainerd, C. J., Reyna, V. F., & Kneer, R. (1995). Falserecognition reversal: When similarity is distinctive. Journal of Memory and Language, 34, 157�185.
## 207                                                                                                                                                                                                                                                                                                                    
## 208                                                                                                                                                                                                                                                                                                                    
## 209                                                                                                                                                                                                                                                                                                                    
## 210                                                                                                               Stone, G. O., Vanhoy, M. D., & Van Orden, G. C. (1997). Perception is a two-way street: Feedforward and feedback phonology in visual word recognition. Journal of Memory and Language, 36, 337�359.
##                                     journal_init_study auth_overlap
## 1                     Language and Cognitive Processes            0
## 2                                               Nature            0
## 3                                                 same            1
## 4                     Language and Cognitive Processes            1
## 5                                                                NA
## 6                                                                NA
## 7                                                                NA
## 8       Journal of Verbal Learning and Verbal Behavior            1
## 9                                                                NA
## 10                                                               NA
## 11                                           Cognition            1
## 12                                                               NA
## 13                                         Linguistics            1
## 14                                                               NA
## 15                                           Cognition            0
## 16                                                same            1
## 17        Quarterly Journal of Experimental Psychology            1
## 18                      Journal of Memory and Language            0
## 19                               Cognitive Development            0
## 20                                                same            1
## 21                    Language and Cognitive Processes            0
## 22                      Journal of Memory and Language            0
## 23                                                               NA
## 24                                                same            1
## 25                                                               NA
## 26                                                               NA
## 27                                                               NA
## 28                       British Journal of Psychology            0
## 29                                                same            1
## 30                     Psychonomic Bulletin and Review            1
## 31                                                               NA
## 32                      Journal of Memory and Language            1
## 33                                                same            1
## 34                                                same            1
## 35                                                               NA
## 36                      Journal of Memory and Language            0
## 37                                                same            1
## 38                                                               NA
## 39                                                               NA
## 40                                                               NA
## 41                  Journal of Experimental Psychology            1
## 42                                                               NA
## 43                               Psychological Science            1
## 44                      Journal of Memory and Language            1
## 45                      Journal of Memory and Language            0
## 46                                                               NA
## 47                                                               NA
## 48                                                               NA
## 49                                                same            1
## 50                  Journal of Experimental Psychology            0
## 51                                                               NA
## 52                                                               NA
## 53                                  Memory & Cognition           NA
## 54                                                               NA
## 55                                                same            1
## 56                                                               NA
## 57                                                same            1
## 58                                                               NA
## 59                                                same            1
## 60                                                               NA
## 61                                                same            1
## 62                                                               NA
## 63                                                               NA
## 64                                                               NA
## 65                                                               NA
## 66                         Journal of Reading Behavior            0
## 67                                                               NA
## 68                                                               NA
## 69                       British Journal of Psychology            0
## 70                Journal of Psycholinguistic Research            0
## 71                    Language and Cognitive Processes            1
## 72                                                               NA
## 73                  Journal of Experimental Psychology            0
## 74                                                               NA
## 75                      Journal of Memory and Language            0
## 76                                                               NA
## 77                Journal of Psycholinguistic Research            1
## 78                                                               NA
## 79                                                               NA
## 80                                                               NA
## 81                      Journal of Memory and Language            0
## 82                     Proceedings of Sinn & Bedeutung            1
## 83                                                same            1
## 84                                                               NA
## 85                                                               NA
## 86                              Language and Cognition            1
## 87                                                <NA>            1
## 88                                   Language Learning            1
## 89              Studies in Second Language Acquisition            1
## 90              Studies in Second Language Acquisition            0
## 91                                                               NA
## 92                                                               NA
## 93                                                               NA
## 94                                                               NA
## 95                      Journal of Memory and Language            0
## 96                                                same            1
## 97                                                               NA
## 98                                           Cognition            0
## 99                                                               NA
## 100                                          Cognition            0
## 101                                                              NA
## 102                                                              NA
## 103                                                              NA
## 104                              Psychological Science            0
## 105                            Frontiers in Psychology            1
## 106                                                              NA
## 107                              Psychological Science            0
## 108                                   Changing English            0
## 109                                  Cognitive Science            0
## 110                                               <NA>            0
## 111                                                              NA
## 112                                                              NA
## 113                 Journal of Experimental Psychology            0
## 114                     American Journal of Psychology            1
## 115                                                              NA
## 116                                               same            1
## 117                                                              NA
## 118                                                              NA
## 119                                                              NA
## 120                                           Language            0
## 121                                               <NA>            0
## 122                                                              NA
## 123                                 Linguistic inquiry            0
## 124                                 Linguistic inquiry            0
## 125                                                              NA
## 126                                                              NA
## 127                                                              NA
## 128                                                              NA
## 129                                                              NA
## 130                                               same            1
## 131                               Journal of Phonetics            0
## 132                                 Brain and Language            1
## 133                                               same            1
## 134                                                              NA
## 135                                                              NA
## 136                                               same            1
## 137                                               same            1
## 138                                                              NA
## 139                                                              NA
## 140                                                              NA
## 141                        Journal of Neurolinguistics            1
## 142                                                              NA
## 143                                                              NA
## 144                                                              NA
## 145                                               same            1
## 146                                                              NA
## 147                                               same            1
## 148                 Journal of Experimental Psychology            0
## 149                                               same            1
## 150                                                              NA
## 151             Studies in Second Language Acquisition            1
## 152                                                              NA
## 153                                                              NA
## 154                                          Cognition            0
## 155                 Journal of Experimental Psychology            0
## 156                                                              NA
## 157                                                              NA
## 158                                                              NA
## 159                                               same            1
## 160                        Social Psychology Quarterly            0
## 161                                               same            1
## 162                                               same            1
## 163                                             Memory            0
## 164                                                              NA
## 165                                          Cognition            0
## 166                                               same            1
## 167                               Cognitive Psychology            0
## 168                     Journal of Memory and Language            0
## 169                     Journal of Memory and Language            0
## 170                                                              NA
## 171                                               same            1
## 172                                                              NA
## 173                                             Memory            0
## 174                                 Memory & Cognition            1
## 175                     Journal of Memory and Language            1
## 176                                                              NA
## 177       Quarterly Journal of Experimental Psychology            1
## 178                     Journal of Memory and Language            1
## 179                                                              NA
## 180                                                              NA
## 181                                                              NA
## 182     Journal of Verbal Learning and Verbal Behavior            0
## 183                 Journal of Experimental Psychology            0
## 184 Journal of International Neuropsychologial Society            0
## 185                                                              NA
## 186                     Journal of Memory and Language            0
## 187                                               same            1
## 188                                                              NA
## 189                     Journal of Memory and Language            0
## 190                                                              NA
## 191                              Psychological Science            0
## 192                                               same            1
## 193                                               same            1
## 194                                               same            1
## 195                             Psychological Research            0
## 196                                               same            1
## 197                                               same            1
## 198                                               same            1
## 199                                 Memory & Cognition            0
## 200                                                              NA
## 201                     Journal of Memory and Language            1
## 202                                               same            1
## 203                                                              NA
## 204                                               same            1
## 205                                                              NA
## 206                     Journal of Memory and Language            0
## 207                                                              NA
## 208                                                              NA
## 209                                                              NA
## 210                     Journal of Memory and Language            0
##     year_init_study years_between exp_paradigm sample materials_setup
## 1              2006             2            0      0               0
## 2              1983            10            0      1               1
## 3              2011             0            0      0               0
## 4              2009             2            0      1               0
## 5                NA            NA           NA     NA              NA
## 6                NA            NA           NA     NA              NA
## 7                NA            NA           NA     NA              NA
## 8              1981            20            0      0               1
## 9                NA            NA           NA     NA              NA
## 10               NA            NA           NA     NA              NA
## 11             2007             4            0      0               0
## 12               NA            NA           NA     NA              NA
## 13             1983            13            0      1               1
## 14               NA            NA           NA     NA              NA
## 15             2007             5            0      0               1
## 16             2000             0            0      0               1
## 17             1981             8            0      0               1
## 18             2005             4            0      1               1
## 19             2005             5            0      0               0
## 20             2004             0            0      0               0
## 21             1983            10            0      0               1
## 22             1987            13            0      0               1
## 23               NA            NA           NA     NA              NA
## 24             2000             0            0      0               1
## 25               NA            NA           NA     NA              NA
## 26               NA            NA           NA     NA              NA
## 27               NA            NA           NA     NA              NA
## 28             1992             7            1      0               0
## 29             2009             0            0      0               1
## 30             1998             6            0      1               1
## 31               NA            NA           NA     NA              NA
## 32             1995             7            0      1               1
## 33             2005             0            0      0               0
## 34             1994             0            0      0               0
## 35               NA            NA           NA     NA              NA
## 36             2006            11            0      1               1
## 37             2015             0            0      0               0
## 38               NA            NA           NA     NA              NA
## 39               NA            NA           NA     NA              NA
## 40               NA            NA           NA     NA              NA
## 41             2016             4            0      0               0
## 42               NA            NA           NA     NA              NA
## 43             2008             6            0      0               0
## 44             2009             5            0      0               0
## 45             1997            20            0      1               0
## 46               NA            NA           NA     NA              NA
## 47               NA            NA           NA     NA              NA
## 48               NA            NA           NA     NA              NA
## 49             2018             0            0      1               0
## 50             2009             7            0      0               0
## 51               NA            NA           NA     NA              NA
## 52               NA            NA           NA     NA              NA
## 53               NA            NA            0      1               1
## 54               NA            NA           NA     NA              NA
## 55             2017             0            0      0               0
## 56               NA            NA           NA     NA              NA
## 57             2020             0            0      1               1
## 58               NA            NA           NA     NA              NA
## 59             2019             0            0      1               1
## 60               NA            NA           NA     NA              NA
## 61             2016             0            0      0               1
## 62               NA            NA           NA     NA              NA
## 63               NA            NA           NA     NA              NA
## 64               NA            NA           NA     NA              NA
## 65               NA            NA           NA     NA              NA
## 66             1982            18            0      0               0
## 67               NA            NA           NA     NA              NA
## 68               NA            NA           NA     NA              NA
## 69             1981            12            0      1               1
## 70             1984            11            0      0               1
## 71             2002             2            0      0               1
## 72               NA            NA           NA     NA              NA
## 73             1993            16            0      0               1
## 74               NA            NA           NA     NA              NA
## 75             1997            23            0      0               1
## 76               NA            NA           NA     NA              NA
## 77             1999             7            0      0               0
## 78               NA            NA           NA     NA              NA
## 79               NA            NA           NA     NA              NA
## 80               NA            NA           NA     NA              NA
## 81             1990             6            0      1               0
## 82             2011             9            0      0               1
## 83             2011             0            0      0               1
## 84               NA            NA           NA     NA              NA
## 85               NA            NA           NA     NA              NA
## 86             2018             1            0      0               1
## 87             1992             4            0      0               1
## 88             2001             4            0      0               0
## 89             2010             1            0      0               1
## 90             2008             1            0      1               1
## 91               NA            NA           NA     NA              NA
## 92               NA            NA           NA     NA              NA
## 93               NA            NA           NA     NA              NA
## 94               NA            NA           NA     NA              NA
## 95             1997             9            0      0               1
## 96             2020             0            0      0               1
## 97               NA            NA           NA     NA              NA
## 98             2004            16            0      0               1
## 99               NA            NA           NA     NA              NA
## 100            2012             6            0      1               1
## 101              NA            NA           NA     NA              NA
## 102              NA            NA           NA     NA              NA
## 103              NA            NA           NA     NA              NA
## 104            2007             5            0      0               1
## 105            2016             2            0      1               0
## 106              NA            NA           NA     NA              NA
## 107            2008             7            0      1               0
## 108            2010             8            0      0               0
## 109            2016             4            0      1               1
## 110            1951            69            0      0               1
## 111              NA            NA           NA     NA              NA
## 112              NA            NA           NA     NA              NA
## 113            1999             8            0      1               1
## 114            2015             1            0      0               0
## 115              NA            NA           NA     NA              NA
## 116            2020             0            0      0               1
## 117              NA            NA           NA     NA              NA
## 118              NA            NA           NA     NA              NA
## 119              NA            NA           NA     NA              NA
## 120            1993            11            0      1               1
## 121            2016             4            0      1               1
## 122              NA            NA           NA     NA              NA
## 123            2015             5            1      0               0
## 124            2015             5            0      0               0
## 125              NA            NA           NA     NA              NA
## 126              NA            NA           NA     NA              NA
## 127              NA            NA           NA     NA              NA
## 128              NA            NA           NA     NA              NA
## 129              NA            NA           NA     NA              NA
## 130            1998             0            0      0               1
## 131            2010             4           NA     NA              NA
## 132            2002             9            0      0               0
## 133            1999             0            0      0               0
## 134              NA            NA           NA     NA              NA
## 135              NA            NA           NA     NA              NA
## 136            2016             0            0      0               1
## 137            2014             0            0      0               1
## 138              NA            NA           NA     NA              NA
## 139              NA            NA           NA     NA              NA
## 140              NA            NA           NA     NA              NA
## 141            1994             2            0      0               0
## 142              NA            NA           NA     NA              NA
## 143              NA            NA           NA     NA              NA
## 144              NA            NA           NA     NA              NA
## 145            2004             0            0      0               1
## 146              NA            NA           NA     NA              NA
## 147            2016             0            1      0               1
## 148            1975            37            0      1               0
## 149            1996             0            0      0               1
## 150              NA            NA           NA     NA              NA
## 151            2010             4            1      0               1
## 152              NA            NA           NA     NA              NA
## 153              NA            NA           NA     NA              NA
## 154            1997             3            0      1               1
## 155            2007             8            0      1               1
## 156              NA            NA           NA     NA              NA
## 157              NA            NA           NA     NA              NA
## 158              NA            NA           NA     NA              NA
## 159            2020             0            0      1               0
## 160            1994            17            0      1               1
## 161            1996             0            1      0               0
## 162            2017             0            1      0               1
## 163            2004             8            0      0               1
## 164              NA            NA           NA     NA              NA
## 165            2001            12            0      0               1
## 166            2015             0            0      0               1
## 167            1989            12            0      1               1
## 168            1996             7            0      0               0
## 169            1993             2            0      0               1
## 170              NA            NA           NA     NA              NA
## 171            2005             0            1      0               1
## 172              NA            NA           NA     NA              NA
## 173            2009             9            0      0               1
## 174            1999             2            0      0               0
## 175            2008             2            0      0               1
## 176              NA            NA           NA     NA              NA
## 177            1983            13            0      0               0
## 178            2003            14            0      0               0
## 179              NA            NA           NA     NA              NA
## 180              NA            NA           NA     NA              NA
## 181              NA            NA           NA     NA              NA
## 182            1975            30            0      0               0
## 183            1994            12            0      0               1
## 184            2011             2            0      1               1
## 185              NA            NA           NA     NA              NA
## 186            2013             7            0      0               1
## 187            2003             0            0      0               1
## 188              NA            NA           NA     NA              NA
## 189            1994             4            0      0               1
## 190              NA            NA           NA     NA              NA
## 191            1998             7            0      0               0
## 192            2003             0            0      0               0
## 193            2017             0            0      0               0
## 194            2013             0            0      0               0
## 195            1984            14            0      0               1
## 196            2008             0            0      1               0
## 197            2019             0            0      0               1
## 198            2016             0            0      0               1
## 199            2000             4            0      1               1
## 200              NA            NA           NA     NA              NA
## 201            2013             2            0      0               0
## 202            2000             0            0      0               1
## 203              NA            NA           NA     NA              NA
## 204            2010             0            0      0               1
## 205              NA            NA           NA     NA              NA
## 206            1995             5            0      0               1
## 207              NA            NA           NA     NA              NA
## 208              NA            NA           NA     NA              NA
## 209              NA            NA           NA     NA              NA
## 210            1997             0            0      1               1
##     measurement manipulation control type_replication
## 1             0            0       1          partial
## 2             0            0       0       conceptual
## 3             0            1       0          partial
## 4             0            0       0          partial
## 5            NA           NA      NA             <NA>
## 6            NA           NA      NA             <NA>
## 7            NA           NA      NA             <NA>
## 8             0            1       0       conceptual
## 9            NA           NA      NA             <NA>
## 10           NA           NA      NA             <NA>
## 11            0            1       0          partial
## 12           NA           NA      NA             <NA>
## 13            0            0       0       conceptual
## 14           NA           NA      NA             <NA>
## 15            0            0       0          partial
## 16            1            1       0       conceptual
## 17            0            0       0          partial
## 18            0            0       0       conceptual
## 19            0            1       0          partial
## 20            0            1       0          partial
## 21            0            1       0       conceptual
## 22            0            1       0       conceptual
## 23           NA           NA      NA             <NA>
## 24            0            0       0          partial
## 25           NA           NA      NA             <NA>
## 26           NA           NA      NA             <NA>
## 27           NA           NA      NA             <NA>
## 28            0            0       1       conceptual
## 29            0            1       0       conceptual
## 30            0            0       0       conceptual
## 31           NA           NA      NA             <NA>
## 32            0            1       1       conceptual
## 33            0            1       1       conceptual
## 34            1            1       0       conceptual
## 35           NA           NA      NA             <NA>
## 36            0            0       0       conceptual
## 37            0            1       0          partial
## 38           NA           NA      NA             <NA>
## 39           NA           NA      NA             <NA>
## 40           NA           NA      NA             <NA>
## 41            0            0       0           direct
## 42           NA           NA      NA             <NA>
## 43            0            1       0          partial
## 44            0            0       0           direct
## 45            0            0       0          partial
## 46           NA           NA      NA             <NA>
## 47           NA           NA      NA             <NA>
## 48           NA           NA      NA             <NA>
## 49            0            1       0       conceptual
## 50            0            0       0           direct
## 51           NA           NA      NA             <NA>
## 52           NA           NA      NA             <NA>
## 53            0            0       1       conceptual
## 54           NA           NA      NA             <NA>
## 55            0            0       1          partial
## 56           NA           NA      NA             <NA>
## 57            1            1       0       conceptual
## 58           NA           NA      NA             <NA>
## 59            0            0       0       conceptual
## 60           NA           NA      NA             <NA>
## 61            0            1       0       conceptual
## 62           NA           NA      NA             <NA>
## 63           NA           NA      NA             <NA>
## 64           NA           NA      NA             <NA>
## 65           NA           NA      NA             <NA>
## 66            0            1       1       conceptual
## 67           NA           NA      NA             <NA>
## 68           NA           NA      NA             <NA>
## 69            0            0       0       conceptual
## 70            0            0       1       conceptual
## 71            0            1       0       conceptual
## 72           NA           NA      NA             <NA>
## 73            0            0       0          partial
## 74           NA           NA      NA             <NA>
## 75            0            1       0       conceptual
## 76           NA           NA      NA             <NA>
## 77            0            0       0           direct
## 78           NA           NA      NA             <NA>
## 79           NA           NA      NA             <NA>
## 80           NA           NA      NA             <NA>
## 81            0            0       0          partial
## 82            0            0       1       conceptual
## 83            0            1       0       conceptual
## 84           NA           NA      NA             <NA>
## 85           NA           NA      NA             <NA>
## 86            0            0       0          partial
## 87            0            0       0          partial
## 88            1            0       1       conceptual
## 89            0            0       0          partial
## 90            0            0       0       conceptual
## 91           NA           NA      NA             <NA>
## 92           NA           NA      NA             <NA>
## 93           NA           NA      NA             <NA>
## 94           NA           NA      NA             <NA>
## 95            0            1       0       conceptual
## 96            0            0       0          partial
## 97           NA           NA      NA             <NA>
## 98            0            0       0          partial
## 99           NA           NA      NA             <NA>
## 100           0            1       0       conceptual
## 101          NA           NA      NA             <NA>
## 102          NA           NA      NA             <NA>
## 103          NA           NA      NA             <NA>
## 104           0            1       0       conceptual
## 105           0            0       0          partial
## 106          NA           NA      NA             <NA>
## 107           0            0       0          partial
## 108           0            1       0          partial
## 109           0            0       0       conceptual
## 110           0            1       1       conceptual
## 111          NA           NA      NA             <NA>
## 112          NA           NA      NA             <NA>
## 113           0            0       1       conceptual
## 114           0            1       0          partial
## 115          NA           NA      NA             <NA>
## 116           0            1       0       conceptual
## 117          NA           NA      NA             <NA>
## 118          NA           NA      NA             <NA>
## 119          NA           NA      NA             <NA>
## 120           0            0       0       conceptual
## 121           0            0       0       conceptual
## 122          NA           NA      NA             <NA>
## 123           0            0       1       conceptual
## 124           0            0       1          partial
## 125          NA           NA      NA             <NA>
## 126          NA           NA      NA             <NA>
## 127          NA           NA      NA             <NA>
## 128          NA           NA      NA             <NA>
## 129          NA           NA      NA             <NA>
## 130           1            0       0       conceptual
## 131          NA           NA      NA             <NA>
## 132           0            1       1       conceptual
## 133           0            1       0          partial
## 134          NA           NA      NA             <NA>
## 135          NA           NA      NA             <NA>
## 136           0            1       0       conceptual
## 137           0            1       0       conceptual
## 138          NA           NA      NA             <NA>
## 139          NA           NA      NA             <NA>
## 140          NA           NA      NA             <NA>
## 141           0            0       0           direct
## 142          NA           NA      NA             <NA>
## 143          NA           NA      NA             <NA>
## 144          NA           NA      NA             <NA>
## 145           1            0       1       conceptual
## 146          NA           NA      NA             <NA>
## 147           0            1       0       conceptual
## 148           0            1       0       conceptual
## 149           0            0       1       conceptual
## 150          NA           NA      NA             <NA>
## 151           0            0       0       conceptual
## 152          NA           NA      NA             <NA>
## 153          NA           NA      NA             <NA>
## 154           0            0       1       conceptual
## 155           0            0       0       conceptual
## 156          NA           NA      NA             <NA>
## 157          NA           NA      NA             <NA>
## 158          NA           NA      NA             <NA>
## 159           0            1       0       conceptual
## 160           0            0       0       conceptual
## 161           0            1       0       conceptual
## 162           1            1       0       conceptual
## 163           0            0       0          partial
## 164          NA           NA      NA             <NA>
## 165           0            0       0          partial
## 166           0            0       0          partial
## 167           0            0       0       conceptual
## 168           0            0       0           direct
## 169           0            0       0          partial
## 170          NA           NA      NA                 
## 171           0            0       1       conceptual
## 172          NA           NA      NA             <NA>
## 173           0            0       0          partial
## 174           0            0       0           direct
## 175           0            0       1       conceptual
## 176          NA           NA      NA             <NA>
## 177           0            1       1       conceptual
## 178           0            1       0          partial
## 179          NA           NA      NA             <NA>
## 180          NA           NA      NA             <NA>
## 181          NA           NA      NA             <NA>
## 182           0            0       1          partial
## 183           0            1       0       conceptual
## 184           0            0       0       conceptual
## 185          NA           NA      NA             <NA>
## 186           0            0       0          partial
## 187           0            1       0       conceptual
## 188          NA           NA      NA             <NA>
## 189           0            0       0          partial
## 190          NA           NA      NA             <NA>
## 191           0            0       0           direct
## 192           0            1       0          partial
## 193           0            0       1          partial
## 194           0            1       0          partial
## 195           0            0       0          partial
## 196           0            0       0          partial
## 197           0            0       0          partial
## 198           1            0       0       conceptual
## 199           0            0       0       conceptual
## 200          NA           NA      NA             <NA>
## 201           0            0       1          partial
## 202           0            1       0       conceptual
## 203          NA           NA      NA             <NA>
## 204           0            1       1       conceptual
## 205          NA           NA      NA             <NA>
## 206           0            1       0       conceptual
## 207          NA           NA      NA             <NA>
## 208          NA           NA      NA             <NA>
## 209          NA           NA      NA             <NA>
## 210           0            0       1       conceptual
##                                                                                    language
## 1                                                                        Spanish (& German)
## 2             Spanish (Castillian) & Australian English (initial: French & British English)
## 3                                                                                   English
## 4                                                                                     Dutch
## 5                                                                                          
## 6                                                                                          
## 7                                                                                          
## 8                                                                                    French
## 9                                                                                          
## 10                                                                                         
## 11                                                                                  Spanish
## 12                                                                                         
## 13                                                                French (initial: English)
## 14                                                                                         
## 15                                                                                  English
## 16                                                                                  English
## 17                                                                                  English
## 18                                                                 Dutch (initial: Spanish)
## 19                                                                                    Dutch
## 20                                                                                   French
## 21                                                                                  English
## 22                                                                                  English
## 23                                                                                         
## 24                                                                                  English
## 25                                                                                         
## 26                                                                                         
## 27                                                                                         
## 28                                                                                  English
## 29                                                                                Cantonese
## 30                                                             Portuguese (initial: French)
## 31                                                                                         
## 32                                                                 English (initial: Dutch)
## 33                                                                                  Spanish
## 34                                                                         Mandarin Chinese
## 35                                                                                         
## 36                                                                English (initial: French)
## 37                                                                                      ASL
## 38                                                                                         
## 39                                                                                         
## 40                                                                                         
## 41                                                                                  English
## 42                                                                                         
## 43                                                                                   French
## 44                                                                                  English
## 45                                                                            Dutch, German
## 46                                                                                         
## 47                                                                                         
## 48                                                                                         
## 49                                                 bilinguals with different main languages
## 50                                                                                  Chinese
## 51                                                                                         
## 52                                                                                         
## 53                                                           Vietnamese (previously German)
## 54                                                                                         
## 55                                                                                  English
## 56                                                                                         
## 57                                                                                  English
## 58                                                                                         
## 59                                                             English (previously Turkish)
## 60                                                                                         
## 61                                                                                  English
## 62                                                                                         
## 63                                                                                         
## 64                                                                                         
## 65                                                                                         
## 66                                                                                  English
## 67                                                                                         
## 68                                                                                         
## 69                                                             Chinese (previously English)
## 70                                                                                  English
## 71                                                                                    Dutch
## 72                                                                                         
## 73                                                                                  English
## 74                                                                                         
## 75                                                                                  English
## 76                                                                                         
## 77                                                                                  English
## 78                                                                                         
## 79                                                                                         
## 80                                                                                         
## 81                                           American English (initial: Australian English)
## 82                                                                                   German
## 83                                                                                  Russian
## 84                                                                                         
## 85                                                                                         
## 86                                                                                   Danish
## 87                                                                                  English
## 88                                                       Spanish (with speakers of English)
## 89                                                         Latin (with speakers of English)
## 90                                      German (with speakers of English; initial: Spanish)
## 91                                                                                         
## 92                                                                                         
## 93                                                                                         
## 94                                                                                         
## 95                                                                                  English
## 96                                                                                  English
## 97                                                                                         
## 98                                                                                  English
## 99                                                                                         
## 100                                                            Spanish; Catalan; bilinguals
## 101                                                                                        
## 102                                                                                        
## 103                                                                                        
## 104                                                   Spanish; English; Catalan; bilinguals
## 105                       French (with speakers of Dutch; initial: with speakers of Korean)
## 106                                                                                        
## 107               English; bilinguals with different main languages (initial: monolinguals)
## 108                            Cantonese; English; Putonghua (initial: Cantonese & English)
## 109                     English (with speakers of Korean; initial: English native speakers)
## 110                                                                                 English
## 111                                                                                        
## 112                                                                                        
## 113                                           Korean; English (initial: Japanese & English)
## 114                                                                                 English
## 115                                                                                        
## 116                                                                                   Hindi
## 117                                                                                        
## 118                                                                                        
## 119                                                                                        
## 120                                                    English (initial: different dialect)
## 121                                                                                  Polish
## 122                                                                                        
## 123                                                                                 English
## 124                                                                                 English
## 125                                                                                        
## 126                                                                                        
## 127                                                                                        
## 128                                                                                        
## 129                                                                                        
## 130                                                                                   Greek
## 131                                                                                        
## 132                                                                                   Dutch
## 133                                                                                 English
## 134                                                                                        
## 135                                                                                        
## 136                                                                                  German
## 137                                                                                  German
## 138                                                                                        
## 139                                                                                        
## 140                                                                                        
## 141                                                                                 English
## 142                                                                                        
## 143                                                                                        
## 144                                                                                        
## 145                                                                                 English
## 146                                                                                        
## 147                                                                                 English
## 148               English (native & nonnative speakers; previously nonnative speakers only)
## 149                                                                                 English
## 150                                                                                        
## 151                                                                                 English
## 152                                                                                        
## 153                                                                                        
## 154                                                   English (previously Scottish English)
## 155                                                Chinese (previously Japanese (&English))
## 156                                                                                        
## 157                                                                                        
## 158                                                                                        
## 159                                                                                 English
## 160                                                            German (previously Japanese)
## 161                                                                                 English
## 162                                                                                 English
## 163                                                                                 English
## 164                                                                                        
## 165                                                                                  German
## 166                                                                                 English
## 167                     English (with Spanish natives; initial: Korean and Chinese natives)
## 168                                                                                  French
## 169                                                                                 English
## 170                                                                                        
## 171                                                                                 English
## 172                                                                                        
## 173                                                                                 English
## 174                                                                                 English
## 175                                                                                 English
## 176                                                                                        
## 177                                                                                 English
## 178                                                                                 English
## 179                                                                                        
## 180                                                                                        
## 181                                                                                        
## 182                                                                                 English
## 183                                                                                 English
## 184 Spanish-Catalan bilinguals (previously English-Spanish and English-Mandarin bilinguals)
## 185                                                                                        
## 186                                                                                 English
## 187                                                                                   Dutch
## 188                                                                                        
## 189                                                                                 English
## 190                                                                                        
## 191                                                                                 English
## 192                                                                                   Dutch
## 193                                                                                 English
## 194                                                                                Japanese
## 195                                                                                   Dutch
## 196                                                     Spanish (previously Spanish-Basque)
## 197                                                                                 English
## 198                                                                                  German
## 199                                                           Japanese (previously English)
## 200                                                                                        
## 201                                                                                 English
## 202                                                                                 English
## 203                                                                                        
## 204                                                                                 English
## 205                                                                                        
## 206                                                                                 English
## 207                                                                                        
## 208                                                                                        
## 209                                                                                        
## 210                                                             French (previously English)
##     success rep_citation init_citation init_cit_til_rep
## 1        NA           29            61                3
## 2        NA           73            92               26
## 3        NA           15            NA               NA
## 4        NA            7            43                2
## 5        NA           37            NA               NA
## 6        NA           21            NA               NA
## 7        NA           86            NA               NA
## 8        NA           28           312              137
## 9        NA           42            NA               NA
## 10       NA           37            NA               NA
## 11       NA           36            97               32
## 12       NA            8            NA               NA
## 13       NA           47            48                8
## 14       NA           24            NA               NA
## 15       NA            9            57               33
## 16       NA          496            NA               NA
## 17       NA           43            28                5
## 18       NA           33           124               34
## 19       NA           27            56               15
## 20       NA           16            NA               NA
## 21       NA          266           173               17
## 22       NA            3           241               55
## 23       NA           32            NA               NA
## 24       NA           73            NA               NA
## 25       NA           25            NA               NA
## 26       NA           20            NA               NA
## 27       NA           31            NA               NA
## 28       NA           26            22                7
## 29       NA           25            NA               NA
## 30       NA           97           216               27
## 31       NA            7            NA               NA
## 32       NA           35           174               43
## 33       NA           12            NA               NA
## 34       NA           83            NA               NA
## 35       NA           11            NA               NA
## 36       NA            4            64               53
## 37       NA            0            NA               NA
## 38       NA            0            NA               NA
## 39       NA           19            NA               NA
## 40       NA            2            NA               NA
## 41        1            2            11               11
## 42       NA            0            NA               NA
## 43       NA           23            82               36
## 44        1            4            21               11
## 45       NA            7           345              294
## 46       NA            0            NA               NA
## 47       NA            0            NA               NA
## 48       NA            0            NA               NA
## 49       NA            9            NA               NA
## 50        1            3            42               36
## 51       NA            0            NA               NA
## 52       NA            0            NA               NA
## 53       NA            1             0                0
## 54       NA           25            NA               NA
## 55       NA           13            NA               NA
## 56       NA            1            NA               NA
## 57       NA            1            NA               NA
## 58       NA            1            NA               NA
## 59       NA            3            NA               NA
## 60       NA            0            NA               NA
## 61       NA            0            NA               NA
## 62       NA            5            NA               NA
## 63       NA            3            NA               NA
## 64       NA           87            NA               NA
## 65       NA            1            NA               NA
## 66       NA            4            22               10
## 67       NA            5            NA               NA
## 68       NA            6            NA               NA
## 69       NA            1             6                4
## 70       NA            4            59               22
## 71       NA            8            38                9
## 72       NA          157            NA               NA
## 73       NA            2           420              233
## 74       NA            0            NA               NA
## 75       NA            0           118              118
## 76       NA            0            NA               NA
## 77        1           43            66               21
## 78       NA            1            NA               NA
## 79       NA            6            NA               NA
## 80       NA            4            NA               NA
## 81       NA            6            14                4
## 82       NA            3            NA               NA
## 83       NA           13            NA               NA
## 84       NA          126            NA               NA
## 85       NA           32            NA               NA
## 86       NA            3             3                2
## 87       NA           28            NA               NA
## 88       NA           66             9                3
## 89       NA           47            54                1
## 90       NA           22            59                3
## 91       NA            0            NA               NA
## 92       NA           74            NA               NA
## 93       NA            4            NA               NA
## 94       NA            6            NA               NA
## 95       NA           14           149               51
## 96       NA            0            NA               NA
## 97       NA            8            NA               NA
## 98       NA            0           293              293
## 99       NA           19            NA               NA
## 100      NA            8            18               15
## 101      NA           20            NA               NA
## 102      NA           27            NA               NA
## 103      NA            3            NA               NA
## 104      NA           17           130               51
## 105      NA            4             8                4
## 106      NA           37            NA               NA
## 107      NA           20            23               16
## 108      NA            2             3                0
## 109      NA            0            32               32
## 110      NA           15            NA               NA
## 111      NA           59            NA               NA
## 112      NA            0            NA               NA
## 113      NA           52           224               40
## 114      NA           15            21                2
## 115      NA           10            NA               NA
## 116      NA            0            NA               NA
## 117      NA           24            NA               NA
## 118      NA           35            NA               NA
## 119      NA            9            NA               NA
## 120      NA           12           108               35
## 121      NA            1            NA               NA
## 122      NA           17            NA               NA
## 123      NA            1            10               10
## 124      NA            0             5               NA
## 125      NA            0            NA               NA
## 126      NA            9            NA               NA
## 127      NA            0            NA               NA
## 128      NA            2            NA               NA
## 129      NA            3            NA               NA
## 130      NA          149            NA               NA
## 131      NA            8            16                4
## 132      NA           22            98               32
## 133      NA          153            NA               NA
## 134      NA           10            NA               NA
## 135      NA           22            NA               NA
## 136      NA           15            NA               NA
## 137      NA           36            NA               NA
## 138      NA            3            NA               NA
## 139      NA            3            NA               NA
## 140      NA            0            NA               NA
## 141       1           17            30                1
## 142      NA            0            NA               NA
## 143      NA            3            NA               NA
## 144      NA            6            NA               NA
## 145      NA           48            NA               NA
## 146      NA            3            NA               NA
## 147      NA           16            NA               NA
## 148      NA            8            NA               NA
## 149      NA           42            NA               NA
## 150      NA            1            NA               NA
## 151      NA           25            54               11
## 152      NA            5            NA               NA
## 153      NA           26            NA               NA
## 154      NA           53           134               21
## 155      NA            7            59               33
## 156      NA            1            NA               NA
## 157      NA           19            NA               NA
## 158      NA            2            NA               NA
## 159      NA            0            NA               NA
## 160      NA           17            32               22
## 161      NA          135            NA               NA
## 162      NA            5            NA               NA
## 163      NA            9            12                7
## 164      NA           51            NA               NA
## 165      NA           55           216              110
## 166      NA            5            NA               NA
## 167      NA          261          1081              131
## 168       0           34           150               38
## 169      NA          195           159                4
## 170      NA           50            NA               NA
## 171      NA           94            NA               NA
## 172      NA           49            NA               NA
## 173      NA           11            17               11
## 174       1           39            78                8
## 175      NA           39            88                9
## 176      NA           27            NA               NA
## 177      NA           49            44               19
## 178      NA           12            63               49
## 179      NA            9            NA               NA
## 180      NA          348            NA               NA
## 181      NA           44            NA               NA
## 182      NA           56           231              136
## 183      NA           35           732              202
## 184      NA           89           237               15
## 185      NA           29            NA               NA
## 186      NA            7           104              104
## 187      NA           30            NA               NA
## 188      NA            8            NA               NA
## 189      NA           33            60                9
## 190      NA           44            NA               NA
## 191       1           22           355               87
## 192      NA          113            NA               NA
## 193      NA            9            NA               NA
## 194      NA            1            NA               NA
## 195      NA           64            81               20
## 196      NA           73            NA               NA
## 197      NA            2            NA               NA
## 198      NA           12            NA               NA
## 199      NA           94           103               20
## 200      NA           42            NA               NA
## 201      NA           45            67               18
## 202      NA           53            NA               NA
## 203      NA          112            NA               NA
## 204      NA           37            NA               NA
## 205      NA           49            NA               NA
## 206      NA            8           171               51
## 207      NA          105            NA               NA
## 208      NA           16            NA               NA
## 209      NA          103            NA               NA
## 210      NA           95           206                4
\end{verbatim}

\hypertarget{results-1}{%
\subsection{Results}\label{results-1}}

\hypertarget{discussion}{%
\section{Discussion}\label{discussion}}

\hypertarget{conclusion}{%
\section{Conclusion}\label{conclusion}}

\newpage

\hypertarget{references}{%
\section{References}\label{references}}

\begingroup
\setlength{\parindent}{-0.5in}
\setlength{\leftskip}{0.5in}

\hypertarget{refs}{}
\begin{CSLReferences}{0}{0}
\end{CSLReferences}

\endgroup


\end{document}
